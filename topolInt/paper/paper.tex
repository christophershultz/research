\documentclass[]{article}
\setlength\parindent{0pt}
\usepackage[a4paper,bindingoffset=0.2in,%
left=1in,right=1in,top=1in,bottom=1in,%
footskip=.25in]{geometry}
%opening
\title{Topological Data Analysis of Treasury Yield Curve Rates}
\author{Christopher Shultz}

\begin{document}

\maketitle

\begin{abstract}
\noindent
We explore the use of Topological Data Analysis (TDA) to study the evolution of US Treasury Yield Curve Rates from 2015 - 2020. We particularly focus on the use of TDA to understand the shape and structure of our multidimensional rates series and whether useful inference can be derived from topological tools.

\end{abstract}

\section{Introduction}
% WHAT IS TDA
Topological Data Analysis (TDA) is an emerging multidisciplinary field that uses tools from topology, statistics, and scientific computing to extract insights about complex data. Many existing papers introduce the technique to an unfamiliar reader \cite{bubenik_2015, chazal_2017, munch_2017, wasserman_2018}. From a practical perspective, TDA can be used primarily to explore the structure of multidimensional data with regard to shape and connectivity. \\

% WHERE HAS IT BEEN USED
TDA has shown significant promise in a variety of areas, including cosmology \cite{chen_et_al_2015, van_de_weygaert_et_al_2010, van_de_weygaert_et_al_2011, sousbie_2011}, image analysis \cite{bonis_et_al_2016, li_et_al_2014, carriere_et_al_2015, singh_et_al_2014, adcock_et_al_2014}, finance \cite{gidea_katz_2018}, and neuroscience \cite{arai_et_al_2014, babichev_dabaghian_2016, basso_et_al_2016, bendich_et_al_2016}. Across all of its prior uses, the general theme of extracting information about the shape and structure of complex multidimensional data persists. \\

%WHAT DO I AIM TO USE IT FOR? 
With this paper, I aim to demonstrate the utility of TDA for exploring economic and financial questions, with a specific focus on treasury yield curve rates. A similar exploration takes place in \cite{gidea_katz_2018} with attention to the equity markets, and we follow their example in our study of treasury rates. \\

% WHY SHOULD YOU CARE? 

% HOW DO WE DO THIS AND WHAT DID WE LEARN? WHY DOES THIS MATTER? 

% PAPER OUTLINE

\section{Background of TDA}

% HOW DOES TDA WORK? WHAT IS IT AND WHAT ARE ITS MATH? 

% RELEVANT PROPERTIES

% DETAILS OF PREVIOUS APPLICATIONS 

% PROS / CONS of USING TDA
 
\section{Background of Treasury Curve Yield Rates} 

% WHAT DO WE NEED TO SHARE ABOUT TREASURY CURVE YIELD RATES? 
% WHAT ARE THEY AND WHAT DO THEY MEAN

% WHAT PREVIOUS LITERATURE EXPLORES TCYRS? 
% WHAT DID THEY FIND AND WHAT KINDS OF METHODS DID THEY USE? 

% HAS TDA EVER BEEN USED? 

\section{Methodological Description and Simulation} 

% DETAILS OF THE METHODOLOGY / PIPELINE USED IN THIS PAPER

% WHAT SHOULD THIS HELP US DO? 

\section{Data Description}

% DESCRIPTIVE STATS

% DESCRIPTION OF DATA / SOURCES

% VARIETY OF VISUALS EXPLAINING ITS STRUCTURE

% TIME SERIES FEATURES: AUTOCORRELATION, STATIONARITY, ETC. 

\section{Results}

% DESCRIPTION OF A WALKTHROUGH OF THE PIPELINE PROCESS
% DISCUSS FINDINGS ALONG THE WAY

% SUMMARY OF KEY FINDINGS WITH EVIDENCE

\section{Conclusion}

% WHAT DID WE LEARN? 

% WHAT ARE THE IMPLICATIONS? 

% WHAT ARE DIRECTIONS FOR FUTURE RESEARCH? 

\clearpage
\bibliographystyle{plain}
\bibliography{refs}

\end{document}




% \textbf{Data Source}: https://www.treasury.gov/resource-center/data-chart-center/interest-rates/pages/textview.aspx?data=yield
