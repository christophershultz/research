\documentclass[]{article}
\setlength\parindent{0pt}
\usepackage[a4paper,bindingoffset=0.2in,%
left=1in,right=1in,top=1in,bottom=1in,%
footskip=.25in]{geometry}
\usepackage{hyperref}
\usepackage{graphicx}
\usepackage{amssymb}
\usepackage{pgfplots}
\usepackage{layouts}
\usepackage{float}
\usepackage[skip=0pt]{caption}

%opening
\title{Topological Data Analysis of Treasury Yield Curve Rates}
\author{Christopher Shultz}

\begin{document}

\maketitle

\begin{abstract}
\noindent
We explore the use of Topological Data Analysis (TDA) to study the evolution of US Treasury Yield Curve Rates from 2015 - 2020. We particularly focus on the use of TDA to understand the shape and structure of our multidimensional rates series and whether useful economic inference can be derived from topological tools.

\end{abstract}

\section{Introduction}

% WHAT IS TDA
Topological Data Analysis (TDA) is an emerging multidisciplinary field that uses tools from topology, statistics, and scientific computing to extract insights about complex data. Many existing papers introduce the technique to an unfamiliar reader \cite{bubenik_2015, chazal_2017, munch_2017, wasserman_2018}. From a practical perspective, TDA can be used primarily to explore the structure of multidimensional data with regard to shape and connectivity. \\

% WHERE HAS IT BEEN USED
TDA has shown significant promise in a variety of fields, including cosmology \cite{chen_et_al_2015, van_de_weygaert_et_al_2010, van_de_weygaert_et_al_2011, sousbie_2011}, image analysis \cite{bonis_et_al_2016, li_et_al_2014, carriere_et_al_2015, singh_et_al_2014, adcock_et_al_2014}, finance \cite{gidea_katz_2018, gidea_2016, guo_et_al_2020, majumdar_laha_2020, qiu_et_al_2020, basu_and_dlotko_2020, dlotko_et_al_2019, kim_et_al_2020}, and neuroscience \cite{arai_et_al_2014, babichev_dabaghian_2016, basso_et_al_2016, bendich_et_al_2016}, among many others. This paper does not attempt to present a comprehensive overview of all existing applications of TDA in the scientific literature. However, across all examined papers, the primary theme involves the extraction of insights about the shape and structure of complex multidimensional datasets. \\

%WHAT DO I AIM TO USE IT FOR? 
With this paper, I aim to demonstrate the utility of TDA for exploring economic and financial questions, with a specific focus on treasury yield curve rates. A similar exploration takes place in \cite{gidea_katz_2018} with attention to the equity markets, and we follow their example in our study of rates. Their findings are particularly interesting as they present one of few existing analyses utilizing TDA to explore economic and financial research. Gidea and Katz find a strong rising trend for trading days prior to financial downturns, which implies that TDA may be useful as a new econometric analysis to complement traditional methods. \\

% WHY SHOULD YOU CARE? 
This analysis is meaningful for two main reasons. First, it is (to our knowledge) the first implementation of TDA to examine the shape and structure of treasury yield curve rates. Second, it provides an introduction of TDA to the economics community, who may not have been previously exposed. Based on our survey of the literature, applications of TDA are common in finance, but not much was found in the economics space. Given its utilization across a wide number of fields and the increasing complexity of economic data, TDA has the potential to add significant value to researchers in economics and finance. \\

To complete our analysis, we utilize daily treasury yield curve rates data from 2015 - 2020 across multiple maturity periods. \textbf{\textit{description of method here}. \textit{description of results here}. \textit{description of why this matters here}}. \\

% PAPER OUTLINE
The remainder of this paper is structured as follows: section \ref{tda} describes the mathematical and theoretical details of TDA, including a detailed view of some existing applications; section \ref{yield} provides a summary of the existing literature on yield curve rates and the methodologies that have been used to date. Sections \ref{meth} and \ref{data} explore our methodology and dataset, respectively, providing a detailed outline of the analysis we undertake and the nuances of the underlying dataset. Finally, sections \ref{res} and \ref{conc} present our results, conclusions, and thoughts on future research. \\

\section{Topological Data Analysis}
\label{tda}

% HOW DOES TDA WORK? WHAT IS IT AND WHAT ARE ITS MATH? 
TDA allows for the extraction of new information from high-dimensional, incomplete, and/or complex datasets. It shows promise as an emerging method for providing empirically sound methods for analysis and understanding of the underlying data that can be represented as point clouds in Euclidean space. TDA packages are implemented in a number of languages, including C++, Python, and R. \\ 

Chazel and Michel \cite{chazal_2017} provide a generalized TDA pipeline that most studies implementing TDA employ: 
\begin{enumerate}
	\item The input data is a finite set of points with some distance or similarity metric. In many cases, including this paper, the distance can be induced by the underlying metric space. Typically, data is represented as a series of points in a Euclidean n-dimensional space $\mathbb{E}^n$ \cite{ghrist_2007}. 
	\item Using the point cloud from [1], a "continuous" shape is built on top of the underlying dataset to highlight its shape. This typically takes place through the use of simplicial complexes. 
	\item Topological or geometric information is inferred from the structures built on the data. 
	\item The extracted information provides new families of features and descriptions of the data that can be used to better understand the data or be combined with other existing features for modeling.
\end{enumerate}

% What is a metric space and why does it matter for TDA? 

\subsection{Persistence Diagrams} 

The persistence diagram can show a great deal of information about a given point cloud as well as describing more complicated structure, loops, and voids that are not visible with other methods \cite{munch_2017}. Homology in degree 0 describes the connectedness of the data; degree 1 detects holes or tunnels; degree 2 captures voids, etc.  \cite{bubenik_2015}. Features that \textit{persist} when the resolution changes are referred to as \textbf{persistent homology}. \\

% What is a persistence diagram? 

\subsection{Simplicial Complexes} 

With our collection of data in $\mathbb{E}$, we use the point cloud as the vertices of a combinatorial graph whose edges are determined by proximity. While this approach allows for useful clustering, a number of additional higher order features are ignored under this structure. Additional features can be accurately derived by thinking of the graph as a scaffold for a higher-dimensional object. Specifically, one completes the graph to a simplicial complex - a space built from simple pieces (simplicies) identified combinatorially along faces. 

The next task is to build a useful simplicial complex that represents the structure of the data and which uses the original data as the vertex set. The Vietoris-Rips complex for parameter $t$ is constructed as follows: the vertex set is given by the data itself; For each pair of points $x,y$ in the dataset, we include the edge $xy$ if the distance between them is at most $t:d(x,y)\leq t$. For a higher dimensional simplex given by vertices $x_0, \cdots x_d$, we include the simplex if the complex has all possible edges; explicitly, this means that every vertex $x_0, \cdots x_d$ is within distance $t$ of every other vertex in the simplex. 

\textbf{Definition 1}: Given a collection of points ${x_\alpha} \in \mathbb{E}^n$, the \textbf{Rips Complex}, $\mathcal{R_\epsilon}$ is the abstract simplicial complex whose k-simplices correspond to unordered $(k+1)$-tuples of points ${x_\alpha}$ $k \choose 0$, which are pairwise within distance $\epsilon$. 

The Rips complex is particularly useful for seeing structure in the data as long as the connectivity parameter $t$ is chosen well. The best way to choose $t$ is to look at the continuum of possible $t$ values and explore what appears in terms of structure. 

\subsection{Barcodes} 

After conversion of our data set into a family of simplicial complexes, we view these topological objects via a theory of persistent homology that is encoded in the form of a Betti number: a \textbf{barcode} \cite{ghrist_2007}. 

\section{Treasury Yield Curve Rates} 
\label{yield}

% WHAT DO WE NEED TO SHARE ABOUT TREASURY CURVE YIELD RATES? 
% WHAT ARE THEY AND WHAT DO THEY MEAN

% WHAT PREVIOUS LITERATURE EXPLORES TCYRS? 
% WHAT DID THEY FIND AND WHAT KINDS OF METHODS DID THEY USE? 

% HAS TDA EVER BEEN USED? 

\section{Methodology} 
\label{meth}

% DETAILS OF THE METHODOLOGY / PIPELINE USED IN THIS PAPER

% WHAT SHOULD THIS HELP US DO? 

\section{Data}
\label{data}

% DESCRIPTIVE STATS
The dataset analyzed in this study was retrieved from the US Treasury \footnote{https://www.treasury.gov/resource-center/data-chart-center/interest-rates} and contains daily yield curve rates data from January 2, 2015 to December 31, 2020 for a number of different maturities: 1 mo, 3 mo, 6 mo, 1 year, 2 yr, 3 yr, 5 yr, 7 yr, 10 yr, 20 yr, and 30 yr. The time series are visualized below in Figure \ref{fig:linePlotRates}
, with our shortest and longest maturities highlighted. 

\begin{figure}[H]
	\centering
	\resizebox{0.5\textwidth}{!}{%% Creator: Matplotlib, PGF backend
%%
%% To include the figure in your LaTeX document, write
%%   \input{<filename>.pgf}
%%
%% Make sure the required packages are loaded in your preamble
%%   \usepackage{pgf}
%%
%% and, on pdftex
%%   \usepackage[utf8]{inputenc}\DeclareUnicodeCharacter{2212}{-}
%%
%% or, on luatex and xetex
%%   \usepackage{unicode-math}
%%
%% Figures using additional raster images can only be included by \input if
%% they are in the same directory as the main LaTeX file. For loading figures
%% from other directories you can use the `import` package
%%   \usepackage{import}
%%
%% and then include the figures with
%%   \import{<path to file>}{<filename>.pgf}
%%
%% Matplotlib used the following preamble
%%
\begingroup%
\makeatletter%
\begin{pgfpicture}%
\pgfpathrectangle{\pgfpointorigin}{\pgfqpoint{6.400000in}{4.800000in}}%
\pgfusepath{use as bounding box, clip}%
\begin{pgfscope}%
\pgfsetbuttcap%
\pgfsetmiterjoin%
\definecolor{currentfill}{rgb}{1.000000,1.000000,1.000000}%
\pgfsetfillcolor{currentfill}%
\pgfsetlinewidth{0.000000pt}%
\definecolor{currentstroke}{rgb}{1.000000,1.000000,1.000000}%
\pgfsetstrokecolor{currentstroke}%
\pgfsetdash{}{0pt}%
\pgfpathmoveto{\pgfqpoint{0.000000in}{0.000000in}}%
\pgfpathlineto{\pgfqpoint{6.400000in}{0.000000in}}%
\pgfpathlineto{\pgfqpoint{6.400000in}{4.800000in}}%
\pgfpathlineto{\pgfqpoint{0.000000in}{4.800000in}}%
\pgfpathclose%
\pgfusepath{fill}%
\end{pgfscope}%
\begin{pgfscope}%
\pgfsetbuttcap%
\pgfsetmiterjoin%
\definecolor{currentfill}{rgb}{1.000000,1.000000,1.000000}%
\pgfsetfillcolor{currentfill}%
\pgfsetlinewidth{0.000000pt}%
\definecolor{currentstroke}{rgb}{0.000000,0.000000,0.000000}%
\pgfsetstrokecolor{currentstroke}%
\pgfsetstrokeopacity{0.000000}%
\pgfsetdash{}{0pt}%
\pgfpathmoveto{\pgfqpoint{0.800000in}{0.528000in}}%
\pgfpathlineto{\pgfqpoint{5.760000in}{0.528000in}}%
\pgfpathlineto{\pgfqpoint{5.760000in}{4.224000in}}%
\pgfpathlineto{\pgfqpoint{0.800000in}{4.224000in}}%
\pgfpathclose%
\pgfusepath{fill}%
\end{pgfscope}%
\begin{pgfscope}%
\pgfsetbuttcap%
\pgfsetroundjoin%
\definecolor{currentfill}{rgb}{0.000000,0.000000,0.000000}%
\pgfsetfillcolor{currentfill}%
\pgfsetlinewidth{0.803000pt}%
\definecolor{currentstroke}{rgb}{0.000000,0.000000,0.000000}%
\pgfsetstrokecolor{currentstroke}%
\pgfsetdash{}{0pt}%
\pgfsys@defobject{currentmarker}{\pgfqpoint{0.000000in}{-0.048611in}}{\pgfqpoint{0.000000in}{0.000000in}}{%
\pgfpathmoveto{\pgfqpoint{0.000000in}{0.000000in}}%
\pgfpathlineto{\pgfqpoint{0.000000in}{-0.048611in}}%
\pgfusepath{stroke,fill}%
}%
\begin{pgfscope}%
\pgfsys@transformshift{1.023396in}{0.528000in}%
\pgfsys@useobject{currentmarker}{}%
\end{pgfscope}%
\end{pgfscope}%
\begin{pgfscope}%
\definecolor{textcolor}{rgb}{0.000000,0.000000,0.000000}%
\pgfsetstrokecolor{textcolor}%
\pgfsetfillcolor{textcolor}%
\pgftext[x=1.023396in,y=0.430778in,,top]{\color{textcolor}\rmfamily\fontsize{10.000000}{12.000000}\selectfont 2015}%
\end{pgfscope}%
\begin{pgfscope}%
\pgfsetbuttcap%
\pgfsetroundjoin%
\definecolor{currentfill}{rgb}{0.000000,0.000000,0.000000}%
\pgfsetfillcolor{currentfill}%
\pgfsetlinewidth{0.803000pt}%
\definecolor{currentstroke}{rgb}{0.000000,0.000000,0.000000}%
\pgfsetstrokecolor{currentstroke}%
\pgfsetdash{}{0pt}%
\pgfsys@defobject{currentmarker}{\pgfqpoint{0.000000in}{-0.048611in}}{\pgfqpoint{0.000000in}{0.000000in}}{%
\pgfpathmoveto{\pgfqpoint{0.000000in}{0.000000in}}%
\pgfpathlineto{\pgfqpoint{0.000000in}{-0.048611in}}%
\pgfusepath{stroke,fill}%
}%
\begin{pgfscope}%
\pgfsys@transformshift{1.774911in}{0.528000in}%
\pgfsys@useobject{currentmarker}{}%
\end{pgfscope}%
\end{pgfscope}%
\begin{pgfscope}%
\definecolor{textcolor}{rgb}{0.000000,0.000000,0.000000}%
\pgfsetstrokecolor{textcolor}%
\pgfsetfillcolor{textcolor}%
\pgftext[x=1.774911in,y=0.430778in,,top]{\color{textcolor}\rmfamily\fontsize{10.000000}{12.000000}\selectfont 2016}%
\end{pgfscope}%
\begin{pgfscope}%
\pgfsetbuttcap%
\pgfsetroundjoin%
\definecolor{currentfill}{rgb}{0.000000,0.000000,0.000000}%
\pgfsetfillcolor{currentfill}%
\pgfsetlinewidth{0.803000pt}%
\definecolor{currentstroke}{rgb}{0.000000,0.000000,0.000000}%
\pgfsetstrokecolor{currentstroke}%
\pgfsetdash{}{0pt}%
\pgfsys@defobject{currentmarker}{\pgfqpoint{0.000000in}{-0.048611in}}{\pgfqpoint{0.000000in}{0.000000in}}{%
\pgfpathmoveto{\pgfqpoint{0.000000in}{0.000000in}}%
\pgfpathlineto{\pgfqpoint{0.000000in}{-0.048611in}}%
\pgfusepath{stroke,fill}%
}%
\begin{pgfscope}%
\pgfsys@transformshift{2.528485in}{0.528000in}%
\pgfsys@useobject{currentmarker}{}%
\end{pgfscope}%
\end{pgfscope}%
\begin{pgfscope}%
\definecolor{textcolor}{rgb}{0.000000,0.000000,0.000000}%
\pgfsetstrokecolor{textcolor}%
\pgfsetfillcolor{textcolor}%
\pgftext[x=2.528485in,y=0.430778in,,top]{\color{textcolor}\rmfamily\fontsize{10.000000}{12.000000}\selectfont 2017}%
\end{pgfscope}%
\begin{pgfscope}%
\pgfsetbuttcap%
\pgfsetroundjoin%
\definecolor{currentfill}{rgb}{0.000000,0.000000,0.000000}%
\pgfsetfillcolor{currentfill}%
\pgfsetlinewidth{0.803000pt}%
\definecolor{currentstroke}{rgb}{0.000000,0.000000,0.000000}%
\pgfsetstrokecolor{currentstroke}%
\pgfsetdash{}{0pt}%
\pgfsys@defobject{currentmarker}{\pgfqpoint{0.000000in}{-0.048611in}}{\pgfqpoint{0.000000in}{0.000000in}}{%
\pgfpathmoveto{\pgfqpoint{0.000000in}{0.000000in}}%
\pgfpathlineto{\pgfqpoint{0.000000in}{-0.048611in}}%
\pgfusepath{stroke,fill}%
}%
\begin{pgfscope}%
\pgfsys@transformshift{3.280000in}{0.528000in}%
\pgfsys@useobject{currentmarker}{}%
\end{pgfscope}%
\end{pgfscope}%
\begin{pgfscope}%
\definecolor{textcolor}{rgb}{0.000000,0.000000,0.000000}%
\pgfsetstrokecolor{textcolor}%
\pgfsetfillcolor{textcolor}%
\pgftext[x=3.280000in,y=0.430778in,,top]{\color{textcolor}\rmfamily\fontsize{10.000000}{12.000000}\selectfont 2018}%
\end{pgfscope}%
\begin{pgfscope}%
\pgfsetbuttcap%
\pgfsetroundjoin%
\definecolor{currentfill}{rgb}{0.000000,0.000000,0.000000}%
\pgfsetfillcolor{currentfill}%
\pgfsetlinewidth{0.803000pt}%
\definecolor{currentstroke}{rgb}{0.000000,0.000000,0.000000}%
\pgfsetstrokecolor{currentstroke}%
\pgfsetdash{}{0pt}%
\pgfsys@defobject{currentmarker}{\pgfqpoint{0.000000in}{-0.048611in}}{\pgfqpoint{0.000000in}{0.000000in}}{%
\pgfpathmoveto{\pgfqpoint{0.000000in}{0.000000in}}%
\pgfpathlineto{\pgfqpoint{0.000000in}{-0.048611in}}%
\pgfusepath{stroke,fill}%
}%
\begin{pgfscope}%
\pgfsys@transformshift{4.031515in}{0.528000in}%
\pgfsys@useobject{currentmarker}{}%
\end{pgfscope}%
\end{pgfscope}%
\begin{pgfscope}%
\definecolor{textcolor}{rgb}{0.000000,0.000000,0.000000}%
\pgfsetstrokecolor{textcolor}%
\pgfsetfillcolor{textcolor}%
\pgftext[x=4.031515in,y=0.430778in,,top]{\color{textcolor}\rmfamily\fontsize{10.000000}{12.000000}\selectfont 2019}%
\end{pgfscope}%
\begin{pgfscope}%
\pgfsetbuttcap%
\pgfsetroundjoin%
\definecolor{currentfill}{rgb}{0.000000,0.000000,0.000000}%
\pgfsetfillcolor{currentfill}%
\pgfsetlinewidth{0.803000pt}%
\definecolor{currentstroke}{rgb}{0.000000,0.000000,0.000000}%
\pgfsetstrokecolor{currentstroke}%
\pgfsetdash{}{0pt}%
\pgfsys@defobject{currentmarker}{\pgfqpoint{0.000000in}{-0.048611in}}{\pgfqpoint{0.000000in}{0.000000in}}{%
\pgfpathmoveto{\pgfqpoint{0.000000in}{0.000000in}}%
\pgfpathlineto{\pgfqpoint{0.000000in}{-0.048611in}}%
\pgfusepath{stroke,fill}%
}%
\begin{pgfscope}%
\pgfsys@transformshift{4.783030in}{0.528000in}%
\pgfsys@useobject{currentmarker}{}%
\end{pgfscope}%
\end{pgfscope}%
\begin{pgfscope}%
\definecolor{textcolor}{rgb}{0.000000,0.000000,0.000000}%
\pgfsetstrokecolor{textcolor}%
\pgfsetfillcolor{textcolor}%
\pgftext[x=4.783030in,y=0.430778in,,top]{\color{textcolor}\rmfamily\fontsize{10.000000}{12.000000}\selectfont 2020}%
\end{pgfscope}%
\begin{pgfscope}%
\pgfsetbuttcap%
\pgfsetroundjoin%
\definecolor{currentfill}{rgb}{0.000000,0.000000,0.000000}%
\pgfsetfillcolor{currentfill}%
\pgfsetlinewidth{0.803000pt}%
\definecolor{currentstroke}{rgb}{0.000000,0.000000,0.000000}%
\pgfsetstrokecolor{currentstroke}%
\pgfsetdash{}{0pt}%
\pgfsys@defobject{currentmarker}{\pgfqpoint{0.000000in}{-0.048611in}}{\pgfqpoint{0.000000in}{0.000000in}}{%
\pgfpathmoveto{\pgfqpoint{0.000000in}{0.000000in}}%
\pgfpathlineto{\pgfqpoint{0.000000in}{-0.048611in}}%
\pgfusepath{stroke,fill}%
}%
\begin{pgfscope}%
\pgfsys@transformshift{5.536604in}{0.528000in}%
\pgfsys@useobject{currentmarker}{}%
\end{pgfscope}%
\end{pgfscope}%
\begin{pgfscope}%
\definecolor{textcolor}{rgb}{0.000000,0.000000,0.000000}%
\pgfsetstrokecolor{textcolor}%
\pgfsetfillcolor{textcolor}%
\pgftext[x=5.536604in,y=0.430778in,,top]{\color{textcolor}\rmfamily\fontsize{10.000000}{12.000000}\selectfont 2021}%
\end{pgfscope}%
\begin{pgfscope}%
\definecolor{textcolor}{rgb}{0.000000,0.000000,0.000000}%
\pgfsetstrokecolor{textcolor}%
\pgfsetfillcolor{textcolor}%
\pgftext[x=3.280000in,y=0.251766in,,top]{\color{textcolor}\rmfamily\fontsize{10.000000}{12.000000}\selectfont Date}%
\end{pgfscope}%
\begin{pgfscope}%
\pgfsetbuttcap%
\pgfsetroundjoin%
\definecolor{currentfill}{rgb}{0.000000,0.000000,0.000000}%
\pgfsetfillcolor{currentfill}%
\pgfsetlinewidth{0.803000pt}%
\definecolor{currentstroke}{rgb}{0.000000,0.000000,0.000000}%
\pgfsetstrokecolor{currentstroke}%
\pgfsetdash{}{0pt}%
\pgfsys@defobject{currentmarker}{\pgfqpoint{-0.048611in}{0.000000in}}{\pgfqpoint{-0.000000in}{0.000000in}}{%
\pgfpathmoveto{\pgfqpoint{-0.000000in}{0.000000in}}%
\pgfpathlineto{\pgfqpoint{-0.048611in}{0.000000in}}%
\pgfusepath{stroke,fill}%
}%
\begin{pgfscope}%
\pgfsys@transformshift{0.800000in}{0.696000in}%
\pgfsys@useobject{currentmarker}{}%
\end{pgfscope}%
\end{pgfscope}%
\begin{pgfscope}%
\definecolor{textcolor}{rgb}{0.000000,0.000000,0.000000}%
\pgfsetstrokecolor{textcolor}%
\pgfsetfillcolor{textcolor}%
\pgftext[x=0.525308in, y=0.647775in, left, base]{\color{textcolor}\rmfamily\fontsize{10.000000}{12.000000}\selectfont \(\displaystyle {0.0}\)}%
\end{pgfscope}%
\begin{pgfscope}%
\pgfsetbuttcap%
\pgfsetroundjoin%
\definecolor{currentfill}{rgb}{0.000000,0.000000,0.000000}%
\pgfsetfillcolor{currentfill}%
\pgfsetlinewidth{0.803000pt}%
\definecolor{currentstroke}{rgb}{0.000000,0.000000,0.000000}%
\pgfsetstrokecolor{currentstroke}%
\pgfsetdash{}{0pt}%
\pgfsys@defobject{currentmarker}{\pgfqpoint{-0.048611in}{0.000000in}}{\pgfqpoint{-0.000000in}{0.000000in}}{%
\pgfpathmoveto{\pgfqpoint{-0.000000in}{0.000000in}}%
\pgfpathlineto{\pgfqpoint{-0.048611in}{0.000000in}}%
\pgfusepath{stroke,fill}%
}%
\begin{pgfscope}%
\pgfsys@transformshift{0.800000in}{1.181549in}%
\pgfsys@useobject{currentmarker}{}%
\end{pgfscope}%
\end{pgfscope}%
\begin{pgfscope}%
\definecolor{textcolor}{rgb}{0.000000,0.000000,0.000000}%
\pgfsetstrokecolor{textcolor}%
\pgfsetfillcolor{textcolor}%
\pgftext[x=0.525308in, y=1.133324in, left, base]{\color{textcolor}\rmfamily\fontsize{10.000000}{12.000000}\selectfont \(\displaystyle {0.5}\)}%
\end{pgfscope}%
\begin{pgfscope}%
\pgfsetbuttcap%
\pgfsetroundjoin%
\definecolor{currentfill}{rgb}{0.000000,0.000000,0.000000}%
\pgfsetfillcolor{currentfill}%
\pgfsetlinewidth{0.803000pt}%
\definecolor{currentstroke}{rgb}{0.000000,0.000000,0.000000}%
\pgfsetstrokecolor{currentstroke}%
\pgfsetdash{}{0pt}%
\pgfsys@defobject{currentmarker}{\pgfqpoint{-0.048611in}{0.000000in}}{\pgfqpoint{-0.000000in}{0.000000in}}{%
\pgfpathmoveto{\pgfqpoint{-0.000000in}{0.000000in}}%
\pgfpathlineto{\pgfqpoint{-0.048611in}{0.000000in}}%
\pgfusepath{stroke,fill}%
}%
\begin{pgfscope}%
\pgfsys@transformshift{0.800000in}{1.667098in}%
\pgfsys@useobject{currentmarker}{}%
\end{pgfscope}%
\end{pgfscope}%
\begin{pgfscope}%
\definecolor{textcolor}{rgb}{0.000000,0.000000,0.000000}%
\pgfsetstrokecolor{textcolor}%
\pgfsetfillcolor{textcolor}%
\pgftext[x=0.525308in, y=1.618873in, left, base]{\color{textcolor}\rmfamily\fontsize{10.000000}{12.000000}\selectfont \(\displaystyle {1.0}\)}%
\end{pgfscope}%
\begin{pgfscope}%
\pgfsetbuttcap%
\pgfsetroundjoin%
\definecolor{currentfill}{rgb}{0.000000,0.000000,0.000000}%
\pgfsetfillcolor{currentfill}%
\pgfsetlinewidth{0.803000pt}%
\definecolor{currentstroke}{rgb}{0.000000,0.000000,0.000000}%
\pgfsetstrokecolor{currentstroke}%
\pgfsetdash{}{0pt}%
\pgfsys@defobject{currentmarker}{\pgfqpoint{-0.048611in}{0.000000in}}{\pgfqpoint{-0.000000in}{0.000000in}}{%
\pgfpathmoveto{\pgfqpoint{-0.000000in}{0.000000in}}%
\pgfpathlineto{\pgfqpoint{-0.048611in}{0.000000in}}%
\pgfusepath{stroke,fill}%
}%
\begin{pgfscope}%
\pgfsys@transformshift{0.800000in}{2.152647in}%
\pgfsys@useobject{currentmarker}{}%
\end{pgfscope}%
\end{pgfscope}%
\begin{pgfscope}%
\definecolor{textcolor}{rgb}{0.000000,0.000000,0.000000}%
\pgfsetstrokecolor{textcolor}%
\pgfsetfillcolor{textcolor}%
\pgftext[x=0.525308in, y=2.104422in, left, base]{\color{textcolor}\rmfamily\fontsize{10.000000}{12.000000}\selectfont \(\displaystyle {1.5}\)}%
\end{pgfscope}%
\begin{pgfscope}%
\pgfsetbuttcap%
\pgfsetroundjoin%
\definecolor{currentfill}{rgb}{0.000000,0.000000,0.000000}%
\pgfsetfillcolor{currentfill}%
\pgfsetlinewidth{0.803000pt}%
\definecolor{currentstroke}{rgb}{0.000000,0.000000,0.000000}%
\pgfsetstrokecolor{currentstroke}%
\pgfsetdash{}{0pt}%
\pgfsys@defobject{currentmarker}{\pgfqpoint{-0.048611in}{0.000000in}}{\pgfqpoint{-0.000000in}{0.000000in}}{%
\pgfpathmoveto{\pgfqpoint{-0.000000in}{0.000000in}}%
\pgfpathlineto{\pgfqpoint{-0.048611in}{0.000000in}}%
\pgfusepath{stroke,fill}%
}%
\begin{pgfscope}%
\pgfsys@transformshift{0.800000in}{2.638197in}%
\pgfsys@useobject{currentmarker}{}%
\end{pgfscope}%
\end{pgfscope}%
\begin{pgfscope}%
\definecolor{textcolor}{rgb}{0.000000,0.000000,0.000000}%
\pgfsetstrokecolor{textcolor}%
\pgfsetfillcolor{textcolor}%
\pgftext[x=0.525308in, y=2.589971in, left, base]{\color{textcolor}\rmfamily\fontsize{10.000000}{12.000000}\selectfont \(\displaystyle {2.0}\)}%
\end{pgfscope}%
\begin{pgfscope}%
\pgfsetbuttcap%
\pgfsetroundjoin%
\definecolor{currentfill}{rgb}{0.000000,0.000000,0.000000}%
\pgfsetfillcolor{currentfill}%
\pgfsetlinewidth{0.803000pt}%
\definecolor{currentstroke}{rgb}{0.000000,0.000000,0.000000}%
\pgfsetstrokecolor{currentstroke}%
\pgfsetdash{}{0pt}%
\pgfsys@defobject{currentmarker}{\pgfqpoint{-0.048611in}{0.000000in}}{\pgfqpoint{-0.000000in}{0.000000in}}{%
\pgfpathmoveto{\pgfqpoint{-0.000000in}{0.000000in}}%
\pgfpathlineto{\pgfqpoint{-0.048611in}{0.000000in}}%
\pgfusepath{stroke,fill}%
}%
\begin{pgfscope}%
\pgfsys@transformshift{0.800000in}{3.123746in}%
\pgfsys@useobject{currentmarker}{}%
\end{pgfscope}%
\end{pgfscope}%
\begin{pgfscope}%
\definecolor{textcolor}{rgb}{0.000000,0.000000,0.000000}%
\pgfsetstrokecolor{textcolor}%
\pgfsetfillcolor{textcolor}%
\pgftext[x=0.525308in, y=3.075520in, left, base]{\color{textcolor}\rmfamily\fontsize{10.000000}{12.000000}\selectfont \(\displaystyle {2.5}\)}%
\end{pgfscope}%
\begin{pgfscope}%
\pgfsetbuttcap%
\pgfsetroundjoin%
\definecolor{currentfill}{rgb}{0.000000,0.000000,0.000000}%
\pgfsetfillcolor{currentfill}%
\pgfsetlinewidth{0.803000pt}%
\definecolor{currentstroke}{rgb}{0.000000,0.000000,0.000000}%
\pgfsetstrokecolor{currentstroke}%
\pgfsetdash{}{0pt}%
\pgfsys@defobject{currentmarker}{\pgfqpoint{-0.048611in}{0.000000in}}{\pgfqpoint{-0.000000in}{0.000000in}}{%
\pgfpathmoveto{\pgfqpoint{-0.000000in}{0.000000in}}%
\pgfpathlineto{\pgfqpoint{-0.048611in}{0.000000in}}%
\pgfusepath{stroke,fill}%
}%
\begin{pgfscope}%
\pgfsys@transformshift{0.800000in}{3.609295in}%
\pgfsys@useobject{currentmarker}{}%
\end{pgfscope}%
\end{pgfscope}%
\begin{pgfscope}%
\definecolor{textcolor}{rgb}{0.000000,0.000000,0.000000}%
\pgfsetstrokecolor{textcolor}%
\pgfsetfillcolor{textcolor}%
\pgftext[x=0.525308in, y=3.561070in, left, base]{\color{textcolor}\rmfamily\fontsize{10.000000}{12.000000}\selectfont \(\displaystyle {3.0}\)}%
\end{pgfscope}%
\begin{pgfscope}%
\pgfsetbuttcap%
\pgfsetroundjoin%
\definecolor{currentfill}{rgb}{0.000000,0.000000,0.000000}%
\pgfsetfillcolor{currentfill}%
\pgfsetlinewidth{0.803000pt}%
\definecolor{currentstroke}{rgb}{0.000000,0.000000,0.000000}%
\pgfsetstrokecolor{currentstroke}%
\pgfsetdash{}{0pt}%
\pgfsys@defobject{currentmarker}{\pgfqpoint{-0.048611in}{0.000000in}}{\pgfqpoint{-0.000000in}{0.000000in}}{%
\pgfpathmoveto{\pgfqpoint{-0.000000in}{0.000000in}}%
\pgfpathlineto{\pgfqpoint{-0.048611in}{0.000000in}}%
\pgfusepath{stroke,fill}%
}%
\begin{pgfscope}%
\pgfsys@transformshift{0.800000in}{4.094844in}%
\pgfsys@useobject{currentmarker}{}%
\end{pgfscope}%
\end{pgfscope}%
\begin{pgfscope}%
\definecolor{textcolor}{rgb}{0.000000,0.000000,0.000000}%
\pgfsetstrokecolor{textcolor}%
\pgfsetfillcolor{textcolor}%
\pgftext[x=0.525308in, y=4.046619in, left, base]{\color{textcolor}\rmfamily\fontsize{10.000000}{12.000000}\selectfont \(\displaystyle {3.5}\)}%
\end{pgfscope}%
\begin{pgfscope}%
\definecolor{textcolor}{rgb}{0.000000,0.000000,0.000000}%
\pgfsetstrokecolor{textcolor}%
\pgfsetfillcolor{textcolor}%
\pgftext[x=0.469752in,y=2.376000in,,bottom,rotate=90.000000]{\color{textcolor}\rmfamily\fontsize{10.000000}{12.000000}\selectfont Rate}%
\end{pgfscope}%
\begin{pgfscope}%
\pgfpathrectangle{\pgfqpoint{0.800000in}{0.528000in}}{\pgfqpoint{4.960000in}{3.696000in}}%
\pgfusepath{clip}%
\pgfsetrectcap%
\pgfsetroundjoin%
\pgfsetlinewidth{1.003750pt}%
\definecolor{currentstroke}{rgb}{0.501961,0.501961,0.501961}%
\pgfsetstrokecolor{currentstroke}%
\pgfsetstrokeopacity{0.900000}%
\pgfsetdash{}{0pt}%
\pgfpathmoveto{\pgfqpoint{1.025455in}{0.715422in}}%
\pgfpathlineto{\pgfqpoint{1.031631in}{0.725133in}}%
\pgfpathlineto{\pgfqpoint{1.037808in}{0.725133in}}%
\pgfpathlineto{\pgfqpoint{1.039867in}{0.715422in}}%
\pgfpathlineto{\pgfqpoint{1.046044in}{0.725133in}}%
\pgfpathlineto{\pgfqpoint{1.048103in}{0.725133in}}%
\pgfpathlineto{\pgfqpoint{1.050162in}{0.734844in}}%
\pgfpathlineto{\pgfqpoint{1.052221in}{0.725133in}}%
\pgfpathlineto{\pgfqpoint{1.066633in}{0.725133in}}%
\pgfpathlineto{\pgfqpoint{1.068692in}{0.715422in}}%
\pgfpathlineto{\pgfqpoint{1.074869in}{0.725133in}}%
\pgfpathlineto{\pgfqpoint{1.076928in}{0.715422in}}%
\pgfpathlineto{\pgfqpoint{1.078987in}{0.715422in}}%
\pgfpathlineto{\pgfqpoint{1.081046in}{0.725133in}}%
\pgfpathlineto{\pgfqpoint{1.083105in}{0.715422in}}%
\pgfpathlineto{\pgfqpoint{1.091341in}{0.715422in}}%
\pgfpathlineto{\pgfqpoint{1.093400in}{0.705711in}}%
\pgfpathlineto{\pgfqpoint{1.095459in}{0.715422in}}%
\pgfpathlineto{\pgfqpoint{1.097518in}{0.715422in}}%
\pgfpathlineto{\pgfqpoint{1.103694in}{0.705711in}}%
\pgfpathlineto{\pgfqpoint{1.107812in}{0.705711in}}%
\pgfpathlineto{\pgfqpoint{1.109871in}{0.715422in}}%
\pgfpathlineto{\pgfqpoint{1.111930in}{0.705711in}}%
\pgfpathlineto{\pgfqpoint{1.120166in}{0.715422in}}%
\pgfpathlineto{\pgfqpoint{1.136638in}{0.715422in}}%
\pgfpathlineto{\pgfqpoint{1.138697in}{0.725133in}}%
\pgfpathlineto{\pgfqpoint{1.140756in}{0.715422in}}%
\pgfpathlineto{\pgfqpoint{1.148991in}{0.715422in}}%
\pgfpathlineto{\pgfqpoint{1.151050in}{0.705711in}}%
\pgfpathlineto{\pgfqpoint{1.153109in}{0.715422in}}%
\pgfpathlineto{\pgfqpoint{1.155168in}{0.705711in}}%
\pgfpathlineto{\pgfqpoint{1.161345in}{0.715422in}}%
\pgfpathlineto{\pgfqpoint{1.163404in}{0.715422in}}%
\pgfpathlineto{\pgfqpoint{1.165463in}{0.725133in}}%
\pgfpathlineto{\pgfqpoint{1.169581in}{0.725133in}}%
\pgfpathlineto{\pgfqpoint{1.175758in}{0.744555in}}%
\pgfpathlineto{\pgfqpoint{1.177817in}{0.744555in}}%
\pgfpathlineto{\pgfqpoint{1.179875in}{0.725133in}}%
\pgfpathlineto{\pgfqpoint{1.181934in}{0.725133in}}%
\pgfpathlineto{\pgfqpoint{1.183993in}{0.705711in}}%
\pgfpathlineto{\pgfqpoint{1.190170in}{0.725133in}}%
\pgfpathlineto{\pgfqpoint{1.192229in}{0.715422in}}%
\pgfpathlineto{\pgfqpoint{1.194288in}{0.734844in}}%
\pgfpathlineto{\pgfqpoint{1.196347in}{0.725133in}}%
\pgfpathlineto{\pgfqpoint{1.198406in}{0.734844in}}%
\pgfpathlineto{\pgfqpoint{1.204583in}{0.734844in}}%
\pgfpathlineto{\pgfqpoint{1.206642in}{0.725133in}}%
\pgfpathlineto{\pgfqpoint{1.208701in}{0.725133in}}%
\pgfpathlineto{\pgfqpoint{1.210760in}{0.715422in}}%
\pgfpathlineto{\pgfqpoint{1.212819in}{0.715422in}}%
\pgfpathlineto{\pgfqpoint{1.218995in}{0.725133in}}%
\pgfpathlineto{\pgfqpoint{1.221054in}{0.715422in}}%
\pgfpathlineto{\pgfqpoint{1.223113in}{0.725133in}}%
\pgfpathlineto{\pgfqpoint{1.225172in}{0.725133in}}%
\pgfpathlineto{\pgfqpoint{1.227231in}{0.715422in}}%
\pgfpathlineto{\pgfqpoint{1.233408in}{0.725133in}}%
\pgfpathlineto{\pgfqpoint{1.235467in}{0.715422in}}%
\pgfpathlineto{\pgfqpoint{1.239585in}{0.715422in}}%
\pgfpathlineto{\pgfqpoint{1.241644in}{0.705711in}}%
\pgfpathlineto{\pgfqpoint{1.247821in}{0.725133in}}%
\pgfpathlineto{\pgfqpoint{1.256056in}{0.725133in}}%
\pgfpathlineto{\pgfqpoint{1.262233in}{0.715422in}}%
\pgfpathlineto{\pgfqpoint{1.264292in}{0.715422in}}%
\pgfpathlineto{\pgfqpoint{1.266351in}{0.705711in}}%
\pgfpathlineto{\pgfqpoint{1.270469in}{0.705711in}}%
\pgfpathlineto{\pgfqpoint{1.276646in}{0.715422in}}%
\pgfpathlineto{\pgfqpoint{1.278705in}{0.705711in}}%
\pgfpathlineto{\pgfqpoint{1.280764in}{0.715422in}}%
\pgfpathlineto{\pgfqpoint{1.282823in}{0.705711in}}%
\pgfpathlineto{\pgfqpoint{1.284882in}{0.705711in}}%
\pgfpathlineto{\pgfqpoint{1.291059in}{0.715422in}}%
\pgfpathlineto{\pgfqpoint{1.293117in}{0.725133in}}%
\pgfpathlineto{\pgfqpoint{1.297235in}{0.705711in}}%
\pgfpathlineto{\pgfqpoint{1.299294in}{0.715422in}}%
\pgfpathlineto{\pgfqpoint{1.321943in}{0.715422in}}%
\pgfpathlineto{\pgfqpoint{1.324002in}{0.705711in}}%
\pgfpathlineto{\pgfqpoint{1.328120in}{0.705711in}}%
\pgfpathlineto{\pgfqpoint{1.334296in}{0.715422in}}%
\pgfpathlineto{\pgfqpoint{1.336355in}{0.705711in}}%
\pgfpathlineto{\pgfqpoint{1.338414in}{0.715422in}}%
\pgfpathlineto{\pgfqpoint{1.340473in}{0.715422in}}%
\pgfpathlineto{\pgfqpoint{1.342532in}{0.725133in}}%
\pgfpathlineto{\pgfqpoint{1.348709in}{0.715422in}}%
\pgfpathlineto{\pgfqpoint{1.352827in}{0.715422in}}%
\pgfpathlineto{\pgfqpoint{1.354886in}{0.705711in}}%
\pgfpathlineto{\pgfqpoint{1.356945in}{0.715422in}}%
\pgfpathlineto{\pgfqpoint{1.363122in}{0.715422in}}%
\pgfpathlineto{\pgfqpoint{1.365181in}{0.705711in}}%
\pgfpathlineto{\pgfqpoint{1.385770in}{0.705711in}}%
\pgfpathlineto{\pgfqpoint{1.391947in}{0.715422in}}%
\pgfpathlineto{\pgfqpoint{1.394006in}{0.705711in}}%
\pgfpathlineto{\pgfqpoint{1.398124in}{0.705711in}}%
\pgfpathlineto{\pgfqpoint{1.406359in}{0.715422in}}%
\pgfpathlineto{\pgfqpoint{1.410477in}{0.715422in}}%
\pgfpathlineto{\pgfqpoint{1.412536in}{0.725133in}}%
\pgfpathlineto{\pgfqpoint{1.414595in}{0.705711in}}%
\pgfpathlineto{\pgfqpoint{1.420772in}{0.715422in}}%
\pgfpathlineto{\pgfqpoint{1.422831in}{0.705711in}}%
\pgfpathlineto{\pgfqpoint{1.424890in}{0.715422in}}%
\pgfpathlineto{\pgfqpoint{1.426949in}{0.715422in}}%
\pgfpathlineto{\pgfqpoint{1.429008in}{0.725133in}}%
\pgfpathlineto{\pgfqpoint{1.435185in}{0.734844in}}%
\pgfpathlineto{\pgfqpoint{1.437244in}{0.725133in}}%
\pgfpathlineto{\pgfqpoint{1.439303in}{0.734844in}}%
\pgfpathlineto{\pgfqpoint{1.441362in}{0.725133in}}%
\pgfpathlineto{\pgfqpoint{1.443421in}{0.734844in}}%
\pgfpathlineto{\pgfqpoint{1.449597in}{0.744555in}}%
\pgfpathlineto{\pgfqpoint{1.451656in}{0.744555in}}%
\pgfpathlineto{\pgfqpoint{1.457833in}{0.773688in}}%
\pgfpathlineto{\pgfqpoint{1.468128in}{0.773688in}}%
\pgfpathlineto{\pgfqpoint{1.470187in}{0.734844in}}%
\pgfpathlineto{\pgfqpoint{1.478423in}{0.812532in}}%
\pgfpathlineto{\pgfqpoint{1.480482in}{0.793110in}}%
\pgfpathlineto{\pgfqpoint{1.484599in}{0.793110in}}%
\pgfpathlineto{\pgfqpoint{1.486658in}{0.783399in}}%
\pgfpathlineto{\pgfqpoint{1.492835in}{0.793110in}}%
\pgfpathlineto{\pgfqpoint{1.499012in}{0.715422in}}%
\pgfpathlineto{\pgfqpoint{1.509307in}{0.763977in}}%
\pgfpathlineto{\pgfqpoint{1.511366in}{0.754266in}}%
\pgfpathlineto{\pgfqpoint{1.515484in}{0.754266in}}%
\pgfpathlineto{\pgfqpoint{1.521660in}{0.773688in}}%
\pgfpathlineto{\pgfqpoint{1.523719in}{0.725133in}}%
\pgfpathlineto{\pgfqpoint{1.525778in}{0.725133in}}%
\pgfpathlineto{\pgfqpoint{1.527837in}{0.715422in}}%
\pgfpathlineto{\pgfqpoint{1.529896in}{0.715422in}}%
\pgfpathlineto{\pgfqpoint{1.538132in}{0.754266in}}%
\pgfpathlineto{\pgfqpoint{1.540191in}{0.725133in}}%
\pgfpathlineto{\pgfqpoint{1.542250in}{0.715422in}}%
\pgfpathlineto{\pgfqpoint{1.544309in}{0.734844in}}%
\pgfpathlineto{\pgfqpoint{1.550486in}{0.763977in}}%
\pgfpathlineto{\pgfqpoint{1.552545in}{0.763977in}}%
\pgfpathlineto{\pgfqpoint{1.554604in}{0.754266in}}%
\pgfpathlineto{\pgfqpoint{1.556663in}{0.705711in}}%
\pgfpathlineto{\pgfqpoint{1.558721in}{0.696000in}}%
\pgfpathlineto{\pgfqpoint{1.564898in}{0.705711in}}%
\pgfpathlineto{\pgfqpoint{1.566957in}{0.696000in}}%
\pgfpathlineto{\pgfqpoint{1.569016in}{0.705711in}}%
\pgfpathlineto{\pgfqpoint{1.571075in}{0.705711in}}%
\pgfpathlineto{\pgfqpoint{1.573134in}{0.696000in}}%
\pgfpathlineto{\pgfqpoint{1.579311in}{0.705711in}}%
\pgfpathlineto{\pgfqpoint{1.581370in}{0.705711in}}%
\pgfpathlineto{\pgfqpoint{1.583429in}{0.696000in}}%
\pgfpathlineto{\pgfqpoint{1.587547in}{0.696000in}}%
\pgfpathlineto{\pgfqpoint{1.593724in}{0.705711in}}%
\pgfpathlineto{\pgfqpoint{1.595782in}{0.696000in}}%
\pgfpathlineto{\pgfqpoint{1.599900in}{0.696000in}}%
\pgfpathlineto{\pgfqpoint{1.601959in}{0.705711in}}%
\pgfpathlineto{\pgfqpoint{1.610195in}{0.705711in}}%
\pgfpathlineto{\pgfqpoint{1.612254in}{0.696000in}}%
\pgfpathlineto{\pgfqpoint{1.614313in}{0.705711in}}%
\pgfpathlineto{\pgfqpoint{1.616372in}{0.705711in}}%
\pgfpathlineto{\pgfqpoint{1.622549in}{0.715422in}}%
\pgfpathlineto{\pgfqpoint{1.624608in}{0.715422in}}%
\pgfpathlineto{\pgfqpoint{1.628726in}{0.696000in}}%
\pgfpathlineto{\pgfqpoint{1.630785in}{0.705711in}}%
\pgfpathlineto{\pgfqpoint{1.636961in}{0.715422in}}%
\pgfpathlineto{\pgfqpoint{1.641079in}{0.734844in}}%
\pgfpathlineto{\pgfqpoint{1.643138in}{0.763977in}}%
\pgfpathlineto{\pgfqpoint{1.645197in}{0.773688in}}%
\pgfpathlineto{\pgfqpoint{1.651374in}{0.773688in}}%
\pgfpathlineto{\pgfqpoint{1.655492in}{0.744555in}}%
\pgfpathlineto{\pgfqpoint{1.657551in}{0.744555in}}%
\pgfpathlineto{\pgfqpoint{1.659610in}{0.773688in}}%
\pgfpathlineto{\pgfqpoint{1.665787in}{0.831954in}}%
\pgfpathlineto{\pgfqpoint{1.667846in}{0.822243in}}%
\pgfpathlineto{\pgfqpoint{1.671963in}{0.831954in}}%
\pgfpathlineto{\pgfqpoint{1.674022in}{0.831954in}}%
\pgfpathlineto{\pgfqpoint{1.680199in}{0.841665in}}%
\pgfpathlineto{\pgfqpoint{1.682258in}{0.831954in}}%
\pgfpathlineto{\pgfqpoint{1.686376in}{0.802821in}}%
\pgfpathlineto{\pgfqpoint{1.688435in}{0.812532in}}%
\pgfpathlineto{\pgfqpoint{1.694612in}{0.831954in}}%
\pgfpathlineto{\pgfqpoint{1.698730in}{0.880509in}}%
\pgfpathlineto{\pgfqpoint{1.702848in}{0.870798in}}%
\pgfpathlineto{\pgfqpoint{1.709024in}{0.909642in}}%
\pgfpathlineto{\pgfqpoint{1.711083in}{0.899931in}}%
\pgfpathlineto{\pgfqpoint{1.715201in}{0.899931in}}%
\pgfpathlineto{\pgfqpoint{1.723437in}{0.977618in}}%
\pgfpathlineto{\pgfqpoint{1.729614in}{0.929064in}}%
\pgfpathlineto{\pgfqpoint{1.731673in}{0.919353in}}%
\pgfpathlineto{\pgfqpoint{1.737850in}{0.948486in}}%
\pgfpathlineto{\pgfqpoint{1.739909in}{0.938775in}}%
\pgfpathlineto{\pgfqpoint{1.741968in}{0.958197in}}%
\pgfpathlineto{\pgfqpoint{1.746086in}{0.880509in}}%
\pgfpathlineto{\pgfqpoint{1.752262in}{0.929064in}}%
\pgfpathlineto{\pgfqpoint{1.754321in}{0.899931in}}%
\pgfpathlineto{\pgfqpoint{1.756380in}{0.890220in}}%
\pgfpathlineto{\pgfqpoint{1.758439in}{0.890220in}}%
\pgfpathlineto{\pgfqpoint{1.766675in}{0.919353in}}%
\pgfpathlineto{\pgfqpoint{1.768734in}{0.919353in}}%
\pgfpathlineto{\pgfqpoint{1.770793in}{0.899931in}}%
\pgfpathlineto{\pgfqpoint{1.772852in}{0.851376in}}%
\pgfpathlineto{\pgfqpoint{1.781088in}{0.909642in}}%
\pgfpathlineto{\pgfqpoint{1.783147in}{0.890220in}}%
\pgfpathlineto{\pgfqpoint{1.785205in}{0.899931in}}%
\pgfpathlineto{\pgfqpoint{1.787264in}{0.890220in}}%
\pgfpathlineto{\pgfqpoint{1.789323in}{0.890220in}}%
\pgfpathlineto{\pgfqpoint{1.795500in}{0.899931in}}%
\pgfpathlineto{\pgfqpoint{1.797559in}{0.899931in}}%
\pgfpathlineto{\pgfqpoint{1.799618in}{0.909642in}}%
\pgfpathlineto{\pgfqpoint{1.801677in}{0.938775in}}%
\pgfpathlineto{\pgfqpoint{1.803736in}{0.929064in}}%
\pgfpathlineto{\pgfqpoint{1.811972in}{0.948486in}}%
\pgfpathlineto{\pgfqpoint{1.814031in}{0.948486in}}%
\pgfpathlineto{\pgfqpoint{1.818149in}{0.997040in}}%
\pgfpathlineto{\pgfqpoint{1.826384in}{0.997040in}}%
\pgfpathlineto{\pgfqpoint{1.828443in}{1.006751in}}%
\pgfpathlineto{\pgfqpoint{1.830502in}{1.035884in}}%
\pgfpathlineto{\pgfqpoint{1.832561in}{1.016462in}}%
\pgfpathlineto{\pgfqpoint{1.838738in}{1.035884in}}%
\pgfpathlineto{\pgfqpoint{1.842856in}{1.016462in}}%
\pgfpathlineto{\pgfqpoint{1.844915in}{0.977618in}}%
\pgfpathlineto{\pgfqpoint{1.846974in}{0.987329in}}%
\pgfpathlineto{\pgfqpoint{1.853151in}{1.006751in}}%
\pgfpathlineto{\pgfqpoint{1.855210in}{0.987329in}}%
\pgfpathlineto{\pgfqpoint{1.857269in}{0.997040in}}%
\pgfpathlineto{\pgfqpoint{1.859328in}{0.967908in}}%
\pgfpathlineto{\pgfqpoint{1.861386in}{0.987329in}}%
\pgfpathlineto{\pgfqpoint{1.873740in}{0.987329in}}%
\pgfpathlineto{\pgfqpoint{1.875799in}{0.997040in}}%
\pgfpathlineto{\pgfqpoint{1.881976in}{1.016462in}}%
\pgfpathlineto{\pgfqpoint{1.884035in}{1.006751in}}%
\pgfpathlineto{\pgfqpoint{1.886094in}{1.016462in}}%
\pgfpathlineto{\pgfqpoint{1.888153in}{1.006751in}}%
\pgfpathlineto{\pgfqpoint{1.890212in}{1.016462in}}%
\pgfpathlineto{\pgfqpoint{1.898447in}{1.016462in}}%
\pgfpathlineto{\pgfqpoint{1.900506in}{1.045595in}}%
\pgfpathlineto{\pgfqpoint{1.902565in}{0.967908in}}%
\pgfpathlineto{\pgfqpoint{1.910801in}{1.006751in}}%
\pgfpathlineto{\pgfqpoint{1.912860in}{0.977618in}}%
\pgfpathlineto{\pgfqpoint{1.914919in}{0.987329in}}%
\pgfpathlineto{\pgfqpoint{1.919037in}{1.016462in}}%
\pgfpathlineto{\pgfqpoint{1.925214in}{1.026173in}}%
\pgfpathlineto{\pgfqpoint{1.927273in}{1.026173in}}%
\pgfpathlineto{\pgfqpoint{1.931391in}{0.977618in}}%
\pgfpathlineto{\pgfqpoint{1.933450in}{0.987329in}}%
\pgfpathlineto{\pgfqpoint{1.939626in}{0.997040in}}%
\pgfpathlineto{\pgfqpoint{1.941685in}{0.987329in}}%
\pgfpathlineto{\pgfqpoint{1.945803in}{0.987329in}}%
\pgfpathlineto{\pgfqpoint{1.954039in}{0.977618in}}%
\pgfpathlineto{\pgfqpoint{1.958157in}{0.890220in}}%
\pgfpathlineto{\pgfqpoint{1.960216in}{0.899931in}}%
\pgfpathlineto{\pgfqpoint{1.962275in}{0.919353in}}%
\pgfpathlineto{\pgfqpoint{1.982864in}{0.919353in}}%
\pgfpathlineto{\pgfqpoint{1.984923in}{0.909642in}}%
\pgfpathlineto{\pgfqpoint{1.986982in}{0.919353in}}%
\pgfpathlineto{\pgfqpoint{1.989041in}{0.909642in}}%
\pgfpathlineto{\pgfqpoint{1.997277in}{0.909642in}}%
\pgfpathlineto{\pgfqpoint{1.999336in}{0.899931in}}%
\pgfpathlineto{\pgfqpoint{2.001395in}{0.919353in}}%
\pgfpathlineto{\pgfqpoint{2.005513in}{0.919353in}}%
\pgfpathlineto{\pgfqpoint{2.011689in}{0.938775in}}%
\pgfpathlineto{\pgfqpoint{2.013748in}{0.929064in}}%
\pgfpathlineto{\pgfqpoint{2.015807in}{0.929064in}}%
\pgfpathlineto{\pgfqpoint{2.017866in}{0.909642in}}%
\pgfpathlineto{\pgfqpoint{2.026102in}{0.909642in}}%
\pgfpathlineto{\pgfqpoint{2.028161in}{0.899931in}}%
\pgfpathlineto{\pgfqpoint{2.030220in}{0.880509in}}%
\pgfpathlineto{\pgfqpoint{2.032279in}{0.890220in}}%
\pgfpathlineto{\pgfqpoint{2.034338in}{0.880509in}}%
\pgfpathlineto{\pgfqpoint{2.040515in}{0.929064in}}%
\pgfpathlineto{\pgfqpoint{2.042574in}{0.929064in}}%
\pgfpathlineto{\pgfqpoint{2.048751in}{0.977618in}}%
\pgfpathlineto{\pgfqpoint{2.054927in}{0.967908in}}%
\pgfpathlineto{\pgfqpoint{2.056986in}{0.967908in}}%
\pgfpathlineto{\pgfqpoint{2.063163in}{1.016462in}}%
\pgfpathlineto{\pgfqpoint{2.069340in}{1.035884in}}%
\pgfpathlineto{\pgfqpoint{2.071399in}{1.035884in}}%
\pgfpathlineto{\pgfqpoint{2.075517in}{0.997040in}}%
\pgfpathlineto{\pgfqpoint{2.077576in}{1.006751in}}%
\pgfpathlineto{\pgfqpoint{2.085812in}{1.026173in}}%
\pgfpathlineto{\pgfqpoint{2.087870in}{0.987329in}}%
\pgfpathlineto{\pgfqpoint{2.089929in}{0.977618in}}%
\pgfpathlineto{\pgfqpoint{2.091988in}{0.987329in}}%
\pgfpathlineto{\pgfqpoint{2.098165in}{0.967908in}}%
\pgfpathlineto{\pgfqpoint{2.100224in}{0.967908in}}%
\pgfpathlineto{\pgfqpoint{2.102283in}{0.929064in}}%
\pgfpathlineto{\pgfqpoint{2.104342in}{0.948486in}}%
\pgfpathlineto{\pgfqpoint{2.106401in}{0.948486in}}%
\pgfpathlineto{\pgfqpoint{2.112578in}{0.958197in}}%
\pgfpathlineto{\pgfqpoint{2.114637in}{0.958197in}}%
\pgfpathlineto{\pgfqpoint{2.116696in}{0.948486in}}%
\pgfpathlineto{\pgfqpoint{2.118755in}{0.958197in}}%
\pgfpathlineto{\pgfqpoint{2.120814in}{0.958197in}}%
\pgfpathlineto{\pgfqpoint{2.126990in}{0.967908in}}%
\pgfpathlineto{\pgfqpoint{2.129049in}{0.958197in}}%
\pgfpathlineto{\pgfqpoint{2.131108in}{0.958197in}}%
\pgfpathlineto{\pgfqpoint{2.133167in}{0.997040in}}%
\pgfpathlineto{\pgfqpoint{2.135226in}{0.958197in}}%
\pgfpathlineto{\pgfqpoint{2.141403in}{0.958197in}}%
\pgfpathlineto{\pgfqpoint{2.143462in}{0.948486in}}%
\pgfpathlineto{\pgfqpoint{2.147580in}{0.948486in}}%
\pgfpathlineto{\pgfqpoint{2.149639in}{0.967908in}}%
\pgfpathlineto{\pgfqpoint{2.157875in}{0.967908in}}%
\pgfpathlineto{\pgfqpoint{2.159934in}{0.958197in}}%
\pgfpathlineto{\pgfqpoint{2.161993in}{0.977618in}}%
\pgfpathlineto{\pgfqpoint{2.164051in}{0.967908in}}%
\pgfpathlineto{\pgfqpoint{2.170228in}{0.997040in}}%
\pgfpathlineto{\pgfqpoint{2.172287in}{0.977618in}}%
\pgfpathlineto{\pgfqpoint{2.176405in}{1.006751in}}%
\pgfpathlineto{\pgfqpoint{2.184641in}{1.006751in}}%
\pgfpathlineto{\pgfqpoint{2.186700in}{0.997040in}}%
\pgfpathlineto{\pgfqpoint{2.188759in}{1.006751in}}%
\pgfpathlineto{\pgfqpoint{2.190818in}{1.006751in}}%
\pgfpathlineto{\pgfqpoint{2.192877in}{1.016462in}}%
\pgfpathlineto{\pgfqpoint{2.199054in}{1.006751in}}%
\pgfpathlineto{\pgfqpoint{2.201112in}{0.997040in}}%
\pgfpathlineto{\pgfqpoint{2.203171in}{0.997040in}}%
\pgfpathlineto{\pgfqpoint{2.205230in}{0.938775in}}%
\pgfpathlineto{\pgfqpoint{2.207289in}{0.967908in}}%
\pgfpathlineto{\pgfqpoint{2.213466in}{0.977618in}}%
\pgfpathlineto{\pgfqpoint{2.215525in}{0.977618in}}%
\pgfpathlineto{\pgfqpoint{2.217584in}{0.967908in}}%
\pgfpathlineto{\pgfqpoint{2.219643in}{0.948486in}}%
\pgfpathlineto{\pgfqpoint{2.221702in}{0.967908in}}%
\pgfpathlineto{\pgfqpoint{2.227879in}{0.997040in}}%
\pgfpathlineto{\pgfqpoint{2.231997in}{0.967908in}}%
\pgfpathlineto{\pgfqpoint{2.234056in}{0.967908in}}%
\pgfpathlineto{\pgfqpoint{2.236115in}{0.977618in}}%
\pgfpathlineto{\pgfqpoint{2.242291in}{0.997040in}}%
\pgfpathlineto{\pgfqpoint{2.244350in}{0.958197in}}%
\pgfpathlineto{\pgfqpoint{2.246409in}{0.987329in}}%
\pgfpathlineto{\pgfqpoint{2.250527in}{0.987329in}}%
\pgfpathlineto{\pgfqpoint{2.256704in}{0.977618in}}%
\pgfpathlineto{\pgfqpoint{2.260822in}{0.997040in}}%
\pgfpathlineto{\pgfqpoint{2.264940in}{1.026173in}}%
\pgfpathlineto{\pgfqpoint{2.271117in}{1.016462in}}%
\pgfpathlineto{\pgfqpoint{2.279352in}{1.016462in}}%
\pgfpathlineto{\pgfqpoint{2.287588in}{1.006751in}}%
\pgfpathlineto{\pgfqpoint{2.291706in}{1.035884in}}%
\pgfpathlineto{\pgfqpoint{2.293765in}{1.035884in}}%
\pgfpathlineto{\pgfqpoint{2.299942in}{1.055306in}}%
\pgfpathlineto{\pgfqpoint{2.302001in}{1.045595in}}%
\pgfpathlineto{\pgfqpoint{2.306119in}{0.977618in}}%
\pgfpathlineto{\pgfqpoint{2.308178in}{0.987329in}}%
\pgfpathlineto{\pgfqpoint{2.316413in}{0.987329in}}%
\pgfpathlineto{\pgfqpoint{2.320531in}{0.870798in}}%
\pgfpathlineto{\pgfqpoint{2.322590in}{0.870798in}}%
\pgfpathlineto{\pgfqpoint{2.328767in}{0.938775in}}%
\pgfpathlineto{\pgfqpoint{2.332885in}{0.958197in}}%
\pgfpathlineto{\pgfqpoint{2.334944in}{0.948486in}}%
\pgfpathlineto{\pgfqpoint{2.337003in}{0.977618in}}%
\pgfpathlineto{\pgfqpoint{2.345239in}{1.016462in}}%
\pgfpathlineto{\pgfqpoint{2.347298in}{1.006751in}}%
\pgfpathlineto{\pgfqpoint{2.349357in}{1.016462in}}%
\pgfpathlineto{\pgfqpoint{2.351416in}{1.016462in}}%
\pgfpathlineto{\pgfqpoint{2.359651in}{1.035884in}}%
\pgfpathlineto{\pgfqpoint{2.361710in}{1.055306in}}%
\pgfpathlineto{\pgfqpoint{2.363769in}{0.987329in}}%
\pgfpathlineto{\pgfqpoint{2.365828in}{1.006751in}}%
\pgfpathlineto{\pgfqpoint{2.372005in}{1.026173in}}%
\pgfpathlineto{\pgfqpoint{2.374064in}{1.026173in}}%
\pgfpathlineto{\pgfqpoint{2.376123in}{1.035884in}}%
\pgfpathlineto{\pgfqpoint{2.378182in}{1.035884in}}%
\pgfpathlineto{\pgfqpoint{2.380241in}{1.026173in}}%
\pgfpathlineto{\pgfqpoint{2.386418in}{1.016462in}}%
\pgfpathlineto{\pgfqpoint{2.388477in}{1.026173in}}%
\pgfpathlineto{\pgfqpoint{2.390535in}{1.016462in}}%
\pgfpathlineto{\pgfqpoint{2.392594in}{0.987329in}}%
\pgfpathlineto{\pgfqpoint{2.394653in}{0.987329in}}%
\pgfpathlineto{\pgfqpoint{2.407007in}{1.065017in}}%
\pgfpathlineto{\pgfqpoint{2.409066in}{1.065017in}}%
\pgfpathlineto{\pgfqpoint{2.415243in}{1.094150in}}%
\pgfpathlineto{\pgfqpoint{2.429655in}{1.230104in}}%
\pgfpathlineto{\pgfqpoint{2.435832in}{1.123283in}}%
\pgfpathlineto{\pgfqpoint{2.437891in}{1.123283in}}%
\pgfpathlineto{\pgfqpoint{2.444068in}{1.142705in}}%
\pgfpathlineto{\pgfqpoint{2.448186in}{1.191260in}}%
\pgfpathlineto{\pgfqpoint{2.452304in}{1.171838in}}%
\pgfpathlineto{\pgfqpoint{2.458481in}{1.162127in}}%
\pgfpathlineto{\pgfqpoint{2.464658in}{1.162127in}}%
\pgfpathlineto{\pgfqpoint{2.466716in}{1.171838in}}%
\pgfpathlineto{\pgfqpoint{2.474952in}{1.171838in}}%
\pgfpathlineto{\pgfqpoint{2.477011in}{1.200971in}}%
\pgfpathlineto{\pgfqpoint{2.479070in}{1.191260in}}%
\pgfpathlineto{\pgfqpoint{2.481129in}{1.220393in}}%
\pgfpathlineto{\pgfqpoint{2.487306in}{1.191260in}}%
\pgfpathlineto{\pgfqpoint{2.489365in}{1.220393in}}%
\pgfpathlineto{\pgfqpoint{2.491424in}{1.230104in}}%
\pgfpathlineto{\pgfqpoint{2.493483in}{1.191260in}}%
\pgfpathlineto{\pgfqpoint{2.495542in}{1.191260in}}%
\pgfpathlineto{\pgfqpoint{2.501719in}{1.200971in}}%
\pgfpathlineto{\pgfqpoint{2.505836in}{1.200971in}}%
\pgfpathlineto{\pgfqpoint{2.507895in}{1.191260in}}%
\pgfpathlineto{\pgfqpoint{2.509954in}{1.200971in}}%
\pgfpathlineto{\pgfqpoint{2.518190in}{1.191260in}}%
\pgfpathlineto{\pgfqpoint{2.520249in}{1.210682in}}%
\pgfpathlineto{\pgfqpoint{2.522308in}{1.152416in}}%
\pgfpathlineto{\pgfqpoint{2.524367in}{1.191260in}}%
\pgfpathlineto{\pgfqpoint{2.532603in}{1.210682in}}%
\pgfpathlineto{\pgfqpoint{2.534662in}{1.210682in}}%
\pgfpathlineto{\pgfqpoint{2.536721in}{1.200971in}}%
\pgfpathlineto{\pgfqpoint{2.538780in}{1.210682in}}%
\pgfpathlineto{\pgfqpoint{2.544956in}{1.181549in}}%
\pgfpathlineto{\pgfqpoint{2.547015in}{1.200971in}}%
\pgfpathlineto{\pgfqpoint{2.551133in}{1.200971in}}%
\pgfpathlineto{\pgfqpoint{2.553192in}{1.210682in}}%
\pgfpathlineto{\pgfqpoint{2.561428in}{1.230104in}}%
\pgfpathlineto{\pgfqpoint{2.567605in}{1.181549in}}%
\pgfpathlineto{\pgfqpoint{2.573782in}{1.191260in}}%
\pgfpathlineto{\pgfqpoint{2.575841in}{1.191260in}}%
\pgfpathlineto{\pgfqpoint{2.577900in}{1.181549in}}%
\pgfpathlineto{\pgfqpoint{2.582017in}{1.200971in}}%
\pgfpathlineto{\pgfqpoint{2.588194in}{1.191260in}}%
\pgfpathlineto{\pgfqpoint{2.590253in}{1.200971in}}%
\pgfpathlineto{\pgfqpoint{2.592312in}{1.191260in}}%
\pgfpathlineto{\pgfqpoint{2.594371in}{1.200971in}}%
\pgfpathlineto{\pgfqpoint{2.596430in}{1.191260in}}%
\pgfpathlineto{\pgfqpoint{2.602607in}{1.210682in}}%
\pgfpathlineto{\pgfqpoint{2.604666in}{1.210682in}}%
\pgfpathlineto{\pgfqpoint{2.606725in}{1.220393in}}%
\pgfpathlineto{\pgfqpoint{2.608784in}{1.220393in}}%
\pgfpathlineto{\pgfqpoint{2.610843in}{1.230104in}}%
\pgfpathlineto{\pgfqpoint{2.617020in}{1.200971in}}%
\pgfpathlineto{\pgfqpoint{2.619078in}{1.220393in}}%
\pgfpathlineto{\pgfqpoint{2.621137in}{1.220393in}}%
\pgfpathlineto{\pgfqpoint{2.623196in}{1.210682in}}%
\pgfpathlineto{\pgfqpoint{2.633491in}{1.210682in}}%
\pgfpathlineto{\pgfqpoint{2.637609in}{1.191260in}}%
\pgfpathlineto{\pgfqpoint{2.639668in}{1.200971in}}%
\pgfpathlineto{\pgfqpoint{2.645845in}{1.181549in}}%
\pgfpathlineto{\pgfqpoint{2.647904in}{1.210682in}}%
\pgfpathlineto{\pgfqpoint{2.649963in}{1.307792in}}%
\pgfpathlineto{\pgfqpoint{2.654081in}{1.385480in}}%
\pgfpathlineto{\pgfqpoint{2.660257in}{1.414613in}}%
\pgfpathlineto{\pgfqpoint{2.662316in}{1.434035in}}%
\pgfpathlineto{\pgfqpoint{2.664375in}{1.404902in}}%
\pgfpathlineto{\pgfqpoint{2.666434in}{1.404902in}}%
\pgfpathlineto{\pgfqpoint{2.668493in}{1.424324in}}%
\pgfpathlineto{\pgfqpoint{2.674670in}{1.463168in}}%
\pgfpathlineto{\pgfqpoint{2.676729in}{1.453457in}}%
\pgfpathlineto{\pgfqpoint{2.678788in}{1.404902in}}%
\pgfpathlineto{\pgfqpoint{2.682906in}{1.404902in}}%
\pgfpathlineto{\pgfqpoint{2.691142in}{1.443746in}}%
\pgfpathlineto{\pgfqpoint{2.693200in}{1.443746in}}%
\pgfpathlineto{\pgfqpoint{2.695259in}{1.434035in}}%
\pgfpathlineto{\pgfqpoint{2.697318in}{1.453457in}}%
\pgfpathlineto{\pgfqpoint{2.709672in}{1.453457in}}%
\pgfpathlineto{\pgfqpoint{2.711731in}{1.434035in}}%
\pgfpathlineto{\pgfqpoint{2.717908in}{1.463168in}}%
\pgfpathlineto{\pgfqpoint{2.719967in}{1.463168in}}%
\pgfpathlineto{\pgfqpoint{2.722026in}{1.472879in}}%
\pgfpathlineto{\pgfqpoint{2.724085in}{1.463168in}}%
\pgfpathlineto{\pgfqpoint{2.726144in}{1.492301in}}%
\pgfpathlineto{\pgfqpoint{2.734379in}{1.492301in}}%
\pgfpathlineto{\pgfqpoint{2.736438in}{1.482590in}}%
\pgfpathlineto{\pgfqpoint{2.738497in}{1.482590in}}%
\pgfpathlineto{\pgfqpoint{2.746733in}{1.502012in}}%
\pgfpathlineto{\pgfqpoint{2.750851in}{1.482590in}}%
\pgfpathlineto{\pgfqpoint{2.752910in}{1.463168in}}%
\pgfpathlineto{\pgfqpoint{2.754969in}{1.463168in}}%
\pgfpathlineto{\pgfqpoint{2.763205in}{1.492301in}}%
\pgfpathlineto{\pgfqpoint{2.765264in}{1.502012in}}%
\pgfpathlineto{\pgfqpoint{2.769381in}{1.472879in}}%
\pgfpathlineto{\pgfqpoint{2.775558in}{1.502012in}}%
\pgfpathlineto{\pgfqpoint{2.777617in}{1.492301in}}%
\pgfpathlineto{\pgfqpoint{2.779676in}{1.521434in}}%
\pgfpathlineto{\pgfqpoint{2.781735in}{1.531145in}}%
\pgfpathlineto{\pgfqpoint{2.783794in}{1.569988in}}%
\pgfpathlineto{\pgfqpoint{2.789971in}{1.579699in}}%
\pgfpathlineto{\pgfqpoint{2.792030in}{1.579699in}}%
\pgfpathlineto{\pgfqpoint{2.798207in}{1.550566in}}%
\pgfpathlineto{\pgfqpoint{2.804384in}{1.569988in}}%
\pgfpathlineto{\pgfqpoint{2.808501in}{1.569988in}}%
\pgfpathlineto{\pgfqpoint{2.810560in}{1.599121in}}%
\pgfpathlineto{\pgfqpoint{2.812619in}{1.589410in}}%
\pgfpathlineto{\pgfqpoint{2.818796in}{1.599121in}}%
\pgfpathlineto{\pgfqpoint{2.820855in}{1.589410in}}%
\pgfpathlineto{\pgfqpoint{2.824973in}{1.608832in}}%
\pgfpathlineto{\pgfqpoint{2.827032in}{1.608832in}}%
\pgfpathlineto{\pgfqpoint{2.835268in}{1.599121in}}%
\pgfpathlineto{\pgfqpoint{2.837327in}{1.647676in}}%
\pgfpathlineto{\pgfqpoint{2.841445in}{1.647676in}}%
\pgfpathlineto{\pgfqpoint{2.847621in}{1.628254in}}%
\pgfpathlineto{\pgfqpoint{2.849680in}{1.637965in}}%
\pgfpathlineto{\pgfqpoint{2.851739in}{1.667098in}}%
\pgfpathlineto{\pgfqpoint{2.853798in}{1.676809in}}%
\pgfpathlineto{\pgfqpoint{2.855857in}{1.676809in}}%
\pgfpathlineto{\pgfqpoint{2.862034in}{1.647676in}}%
\pgfpathlineto{\pgfqpoint{2.866152in}{1.676809in}}%
\pgfpathlineto{\pgfqpoint{2.870270in}{1.696231in}}%
\pgfpathlineto{\pgfqpoint{2.876447in}{1.686520in}}%
\pgfpathlineto{\pgfqpoint{2.878506in}{1.676809in}}%
\pgfpathlineto{\pgfqpoint{2.882623in}{1.628254in}}%
\pgfpathlineto{\pgfqpoint{2.884682in}{1.637965in}}%
\pgfpathlineto{\pgfqpoint{2.892918in}{1.667098in}}%
\pgfpathlineto{\pgfqpoint{2.897036in}{1.705942in}}%
\pgfpathlineto{\pgfqpoint{2.899095in}{1.696231in}}%
\pgfpathlineto{\pgfqpoint{2.905272in}{1.725364in}}%
\pgfpathlineto{\pgfqpoint{2.909390in}{1.715653in}}%
\pgfpathlineto{\pgfqpoint{2.911449in}{1.705942in}}%
\pgfpathlineto{\pgfqpoint{2.913508in}{1.715653in}}%
\pgfpathlineto{\pgfqpoint{2.919685in}{1.705942in}}%
\pgfpathlineto{\pgfqpoint{2.921743in}{1.715653in}}%
\pgfpathlineto{\pgfqpoint{2.925861in}{1.715653in}}%
\pgfpathlineto{\pgfqpoint{2.927920in}{1.705942in}}%
\pgfpathlineto{\pgfqpoint{2.934097in}{1.735075in}}%
\pgfpathlineto{\pgfqpoint{2.936156in}{1.735075in}}%
\pgfpathlineto{\pgfqpoint{2.940274in}{1.812763in}}%
\pgfpathlineto{\pgfqpoint{2.942333in}{1.822474in}}%
\pgfpathlineto{\pgfqpoint{2.948510in}{1.832185in}}%
\pgfpathlineto{\pgfqpoint{2.950569in}{1.841896in}}%
\pgfpathlineto{\pgfqpoint{2.952628in}{1.793341in}}%
\pgfpathlineto{\pgfqpoint{2.956746in}{1.744786in}}%
\pgfpathlineto{\pgfqpoint{2.962922in}{1.735075in}}%
\pgfpathlineto{\pgfqpoint{2.964981in}{1.744786in}}%
\pgfpathlineto{\pgfqpoint{2.971158in}{1.744786in}}%
\pgfpathlineto{\pgfqpoint{2.977335in}{1.686520in}}%
\pgfpathlineto{\pgfqpoint{2.979394in}{1.725364in}}%
\pgfpathlineto{\pgfqpoint{2.981453in}{1.725364in}}%
\pgfpathlineto{\pgfqpoint{2.983512in}{1.715653in}}%
\pgfpathlineto{\pgfqpoint{2.985571in}{1.696231in}}%
\pgfpathlineto{\pgfqpoint{2.991748in}{1.686520in}}%
\pgfpathlineto{\pgfqpoint{2.993807in}{1.705942in}}%
\pgfpathlineto{\pgfqpoint{2.997924in}{1.667098in}}%
\pgfpathlineto{\pgfqpoint{2.999983in}{1.686520in}}%
\pgfpathlineto{\pgfqpoint{3.006160in}{1.667098in}}%
\pgfpathlineto{\pgfqpoint{3.010278in}{1.667098in}}%
\pgfpathlineto{\pgfqpoint{3.014396in}{1.696231in}}%
\pgfpathlineto{\pgfqpoint{3.020573in}{1.647676in}}%
\pgfpathlineto{\pgfqpoint{3.022632in}{1.696231in}}%
\pgfpathlineto{\pgfqpoint{3.024691in}{1.696231in}}%
\pgfpathlineto{\pgfqpoint{3.026750in}{1.676809in}}%
\pgfpathlineto{\pgfqpoint{3.028809in}{1.686520in}}%
\pgfpathlineto{\pgfqpoint{3.037044in}{1.696231in}}%
\pgfpathlineto{\pgfqpoint{3.039103in}{1.735075in}}%
\pgfpathlineto{\pgfqpoint{3.043221in}{1.705942in}}%
\pgfpathlineto{\pgfqpoint{3.049398in}{1.715653in}}%
\pgfpathlineto{\pgfqpoint{3.051457in}{1.696231in}}%
\pgfpathlineto{\pgfqpoint{3.055575in}{1.715653in}}%
\pgfpathlineto{\pgfqpoint{3.063811in}{1.715653in}}%
\pgfpathlineto{\pgfqpoint{3.065870in}{1.705942in}}%
\pgfpathlineto{\pgfqpoint{3.069988in}{1.705942in}}%
\pgfpathlineto{\pgfqpoint{3.072046in}{1.696231in}}%
\pgfpathlineto{\pgfqpoint{3.080282in}{1.725364in}}%
\pgfpathlineto{\pgfqpoint{3.082341in}{1.735075in}}%
\pgfpathlineto{\pgfqpoint{3.084400in}{1.725364in}}%
\pgfpathlineto{\pgfqpoint{3.086459in}{1.725364in}}%
\pgfpathlineto{\pgfqpoint{3.092636in}{1.676809in}}%
\pgfpathlineto{\pgfqpoint{3.094695in}{1.735075in}}%
\pgfpathlineto{\pgfqpoint{3.096754in}{1.744786in}}%
\pgfpathlineto{\pgfqpoint{3.098813in}{1.735075in}}%
\pgfpathlineto{\pgfqpoint{3.100872in}{1.735075in}}%
\pgfpathlineto{\pgfqpoint{3.109108in}{1.744786in}}%
\pgfpathlineto{\pgfqpoint{3.111166in}{1.764208in}}%
\pgfpathlineto{\pgfqpoint{3.113225in}{1.754497in}}%
\pgfpathlineto{\pgfqpoint{3.115284in}{1.754497in}}%
\pgfpathlineto{\pgfqpoint{3.121461in}{1.764208in}}%
\pgfpathlineto{\pgfqpoint{3.123520in}{1.754497in}}%
\pgfpathlineto{\pgfqpoint{3.125579in}{1.754497in}}%
\pgfpathlineto{\pgfqpoint{3.129697in}{1.773919in}}%
\pgfpathlineto{\pgfqpoint{3.135874in}{1.754497in}}%
\pgfpathlineto{\pgfqpoint{3.137933in}{1.783630in}}%
\pgfpathlineto{\pgfqpoint{3.139992in}{1.783630in}}%
\pgfpathlineto{\pgfqpoint{3.144110in}{1.764208in}}%
\pgfpathlineto{\pgfqpoint{3.150286in}{1.783630in}}%
\pgfpathlineto{\pgfqpoint{3.154404in}{1.841896in}}%
\pgfpathlineto{\pgfqpoint{3.156463in}{1.832185in}}%
\pgfpathlineto{\pgfqpoint{3.158522in}{1.841896in}}%
\pgfpathlineto{\pgfqpoint{3.164699in}{1.851607in}}%
\pgfpathlineto{\pgfqpoint{3.166758in}{1.880740in}}%
\pgfpathlineto{\pgfqpoint{3.170876in}{1.900162in}}%
\pgfpathlineto{\pgfqpoint{3.172935in}{1.890451in}}%
\pgfpathlineto{\pgfqpoint{3.179112in}{1.900162in}}%
\pgfpathlineto{\pgfqpoint{3.181171in}{1.919584in}}%
\pgfpathlineto{\pgfqpoint{3.183230in}{1.909873in}}%
\pgfpathlineto{\pgfqpoint{3.187347in}{1.948717in}}%
\pgfpathlineto{\pgfqpoint{3.193524in}{1.958428in}}%
\pgfpathlineto{\pgfqpoint{3.195583in}{1.958428in}}%
\pgfpathlineto{\pgfqpoint{3.197642in}{1.948717in}}%
\pgfpathlineto{\pgfqpoint{3.201760in}{1.948717in}}%
\pgfpathlineto{\pgfqpoint{3.207937in}{1.929295in}}%
\pgfpathlineto{\pgfqpoint{3.209996in}{1.958428in}}%
\pgfpathlineto{\pgfqpoint{3.212055in}{1.948717in}}%
\pgfpathlineto{\pgfqpoint{3.214114in}{1.929295in}}%
\pgfpathlineto{\pgfqpoint{3.216173in}{1.929295in}}%
\pgfpathlineto{\pgfqpoint{3.226467in}{1.958428in}}%
\pgfpathlineto{\pgfqpoint{3.230585in}{1.939006in}}%
\pgfpathlineto{\pgfqpoint{3.238821in}{1.997272in}}%
\pgfpathlineto{\pgfqpoint{3.240880in}{1.958428in}}%
\pgfpathlineto{\pgfqpoint{3.242939in}{1.977850in}}%
\pgfpathlineto{\pgfqpoint{3.244998in}{1.968139in}}%
\pgfpathlineto{\pgfqpoint{3.251175in}{2.036116in}}%
\pgfpathlineto{\pgfqpoint{3.253234in}{2.026405in}}%
\pgfpathlineto{\pgfqpoint{3.255293in}{2.036116in}}%
\pgfpathlineto{\pgfqpoint{3.259411in}{1.987561in}}%
\pgfpathlineto{\pgfqpoint{3.267646in}{2.123514in}}%
\pgfpathlineto{\pgfqpoint{3.269705in}{2.094382in}}%
\pgfpathlineto{\pgfqpoint{3.271764in}{2.045827in}}%
\pgfpathlineto{\pgfqpoint{3.273823in}{2.045827in}}%
\pgfpathlineto{\pgfqpoint{3.282059in}{2.094382in}}%
\pgfpathlineto{\pgfqpoint{3.284118in}{2.065249in}}%
\pgfpathlineto{\pgfqpoint{3.286177in}{2.065249in}}%
\pgfpathlineto{\pgfqpoint{3.288236in}{2.045827in}}%
\pgfpathlineto{\pgfqpoint{3.294413in}{2.104092in}}%
\pgfpathlineto{\pgfqpoint{3.296472in}{2.094382in}}%
\pgfpathlineto{\pgfqpoint{3.298531in}{2.074960in}}%
\pgfpathlineto{\pgfqpoint{3.300589in}{2.084671in}}%
\pgfpathlineto{\pgfqpoint{3.302648in}{2.084671in}}%
\pgfpathlineto{\pgfqpoint{3.310884in}{2.104092in}}%
\pgfpathlineto{\pgfqpoint{3.312943in}{2.094382in}}%
\pgfpathlineto{\pgfqpoint{3.315002in}{2.104092in}}%
\pgfpathlineto{\pgfqpoint{3.317061in}{2.094382in}}%
\pgfpathlineto{\pgfqpoint{3.325297in}{2.094382in}}%
\pgfpathlineto{\pgfqpoint{3.331474in}{2.065249in}}%
\pgfpathlineto{\pgfqpoint{3.337650in}{2.094382in}}%
\pgfpathlineto{\pgfqpoint{3.339709in}{2.094382in}}%
\pgfpathlineto{\pgfqpoint{3.343827in}{2.133225in}}%
\pgfpathlineto{\pgfqpoint{3.345886in}{2.133225in}}%
\pgfpathlineto{\pgfqpoint{3.354122in}{2.172069in}}%
\pgfpathlineto{\pgfqpoint{3.356181in}{2.201202in}}%
\pgfpathlineto{\pgfqpoint{3.360299in}{2.201202in}}%
\pgfpathlineto{\pgfqpoint{3.366476in}{2.269179in}}%
\pgfpathlineto{\pgfqpoint{3.368535in}{2.240046in}}%
\pgfpathlineto{\pgfqpoint{3.370594in}{2.230335in}}%
\pgfpathlineto{\pgfqpoint{3.372653in}{2.230335in}}%
\pgfpathlineto{\pgfqpoint{3.374711in}{2.269179in}}%
\pgfpathlineto{\pgfqpoint{3.382947in}{2.308023in}}%
\pgfpathlineto{\pgfqpoint{3.387065in}{2.278890in}}%
\pgfpathlineto{\pgfqpoint{3.389124in}{2.288601in}}%
\pgfpathlineto{\pgfqpoint{3.395301in}{2.308023in}}%
\pgfpathlineto{\pgfqpoint{3.397360in}{2.308023in}}%
\pgfpathlineto{\pgfqpoint{3.399419in}{2.298312in}}%
\pgfpathlineto{\pgfqpoint{3.401478in}{2.278890in}}%
\pgfpathlineto{\pgfqpoint{3.409714in}{2.346867in}}%
\pgfpathlineto{\pgfqpoint{3.411773in}{2.327445in}}%
\pgfpathlineto{\pgfqpoint{3.413831in}{2.327445in}}%
\pgfpathlineto{\pgfqpoint{3.415890in}{2.317734in}}%
\pgfpathlineto{\pgfqpoint{3.417949in}{2.317734in}}%
\pgfpathlineto{\pgfqpoint{3.426185in}{2.376000in}}%
\pgfpathlineto{\pgfqpoint{3.428244in}{2.405133in}}%
\pgfpathlineto{\pgfqpoint{3.432362in}{2.424555in}}%
\pgfpathlineto{\pgfqpoint{3.440598in}{2.453688in}}%
\pgfpathlineto{\pgfqpoint{3.442657in}{2.385711in}}%
\pgfpathlineto{\pgfqpoint{3.444716in}{2.366289in}}%
\pgfpathlineto{\pgfqpoint{3.452951in}{2.434266in}}%
\pgfpathlineto{\pgfqpoint{3.455010in}{2.414844in}}%
\pgfpathlineto{\pgfqpoint{3.457069in}{2.376000in}}%
\pgfpathlineto{\pgfqpoint{3.459128in}{2.376000in}}%
\pgfpathlineto{\pgfqpoint{3.467364in}{2.414844in}}%
\pgfpathlineto{\pgfqpoint{3.469423in}{2.395422in}}%
\pgfpathlineto{\pgfqpoint{3.471482in}{2.356578in}}%
\pgfpathlineto{\pgfqpoint{3.481777in}{2.405133in}}%
\pgfpathlineto{\pgfqpoint{3.485895in}{2.376000in}}%
\pgfpathlineto{\pgfqpoint{3.490012in}{2.405133in}}%
\pgfpathlineto{\pgfqpoint{3.502366in}{2.463399in}}%
\pgfpathlineto{\pgfqpoint{3.504425in}{2.453688in}}%
\pgfpathlineto{\pgfqpoint{3.510602in}{2.511954in}}%
\pgfpathlineto{\pgfqpoint{3.512661in}{2.511954in}}%
\pgfpathlineto{\pgfqpoint{3.518838in}{2.463399in}}%
\pgfpathlineto{\pgfqpoint{3.525015in}{2.511954in}}%
\pgfpathlineto{\pgfqpoint{3.529132in}{2.482821in}}%
\pgfpathlineto{\pgfqpoint{3.533250in}{2.482821in}}%
\pgfpathlineto{\pgfqpoint{3.541486in}{2.511954in}}%
\pgfpathlineto{\pgfqpoint{3.543545in}{2.521665in}}%
\pgfpathlineto{\pgfqpoint{3.547663in}{2.560509in}}%
\pgfpathlineto{\pgfqpoint{3.553840in}{2.570220in}}%
\pgfpathlineto{\pgfqpoint{3.555899in}{2.560509in}}%
\pgfpathlineto{\pgfqpoint{3.560017in}{2.560509in}}%
\pgfpathlineto{\pgfqpoint{3.562076in}{2.550798in}}%
\pgfpathlineto{\pgfqpoint{3.568252in}{2.570220in}}%
\pgfpathlineto{\pgfqpoint{3.570311in}{2.570220in}}%
\pgfpathlineto{\pgfqpoint{3.576488in}{2.541087in}}%
\pgfpathlineto{\pgfqpoint{3.586783in}{2.579931in}}%
\pgfpathlineto{\pgfqpoint{3.590901in}{2.560509in}}%
\pgfpathlineto{\pgfqpoint{3.597078in}{2.579931in}}%
\pgfpathlineto{\pgfqpoint{3.599137in}{2.579931in}}%
\pgfpathlineto{\pgfqpoint{3.601196in}{2.589642in}}%
\pgfpathlineto{\pgfqpoint{3.605313in}{2.570220in}}%
\pgfpathlineto{\pgfqpoint{3.611490in}{2.579931in}}%
\pgfpathlineto{\pgfqpoint{3.613549in}{2.560509in}}%
\pgfpathlineto{\pgfqpoint{3.615608in}{2.579931in}}%
\pgfpathlineto{\pgfqpoint{3.632080in}{2.579931in}}%
\pgfpathlineto{\pgfqpoint{3.634139in}{2.570220in}}%
\pgfpathlineto{\pgfqpoint{3.648551in}{2.570220in}}%
\pgfpathlineto{\pgfqpoint{3.654728in}{2.618775in}}%
\pgfpathlineto{\pgfqpoint{3.656787in}{2.618775in}}%
\pgfpathlineto{\pgfqpoint{3.660905in}{2.599353in}}%
\pgfpathlineto{\pgfqpoint{3.662964in}{2.609064in}}%
\pgfpathlineto{\pgfqpoint{3.669141in}{2.618775in}}%
\pgfpathlineto{\pgfqpoint{3.671200in}{2.628486in}}%
\pgfpathlineto{\pgfqpoint{3.673259in}{2.609064in}}%
\pgfpathlineto{\pgfqpoint{3.675318in}{2.618775in}}%
\pgfpathlineto{\pgfqpoint{3.677377in}{2.618775in}}%
\pgfpathlineto{\pgfqpoint{3.685612in}{2.657618in}}%
\pgfpathlineto{\pgfqpoint{3.687671in}{2.638197in}}%
\pgfpathlineto{\pgfqpoint{3.689730in}{2.638197in}}%
\pgfpathlineto{\pgfqpoint{3.691789in}{2.628486in}}%
\pgfpathlineto{\pgfqpoint{3.697966in}{2.628486in}}%
\pgfpathlineto{\pgfqpoint{3.700025in}{2.657618in}}%
\pgfpathlineto{\pgfqpoint{3.702084in}{2.647908in}}%
\pgfpathlineto{\pgfqpoint{3.704143in}{2.628486in}}%
\pgfpathlineto{\pgfqpoint{3.706202in}{2.638197in}}%
\pgfpathlineto{\pgfqpoint{3.712379in}{2.677040in}}%
\pgfpathlineto{\pgfqpoint{3.714438in}{2.667329in}}%
\pgfpathlineto{\pgfqpoint{3.716496in}{2.667329in}}%
\pgfpathlineto{\pgfqpoint{3.720614in}{2.647908in}}%
\pgfpathlineto{\pgfqpoint{3.728850in}{2.696462in}}%
\pgfpathlineto{\pgfqpoint{3.732968in}{2.696462in}}%
\pgfpathlineto{\pgfqpoint{3.735027in}{2.686751in}}%
\pgfpathlineto{\pgfqpoint{3.741204in}{2.696462in}}%
\pgfpathlineto{\pgfqpoint{3.743263in}{2.715884in}}%
\pgfpathlineto{\pgfqpoint{3.745322in}{2.706173in}}%
\pgfpathlineto{\pgfqpoint{3.747381in}{2.706173in}}%
\pgfpathlineto{\pgfqpoint{3.749440in}{2.686751in}}%
\pgfpathlineto{\pgfqpoint{3.755616in}{2.696462in}}%
\pgfpathlineto{\pgfqpoint{3.759734in}{2.725595in}}%
\pgfpathlineto{\pgfqpoint{3.761793in}{2.715884in}}%
\pgfpathlineto{\pgfqpoint{3.772088in}{2.764439in}}%
\pgfpathlineto{\pgfqpoint{3.774147in}{2.764439in}}%
\pgfpathlineto{\pgfqpoint{3.776206in}{2.745017in}}%
\pgfpathlineto{\pgfqpoint{3.778265in}{2.745017in}}%
\pgfpathlineto{\pgfqpoint{3.786501in}{2.764439in}}%
\pgfpathlineto{\pgfqpoint{3.788560in}{2.774150in}}%
\pgfpathlineto{\pgfqpoint{3.790619in}{2.764439in}}%
\pgfpathlineto{\pgfqpoint{3.792677in}{2.774150in}}%
\pgfpathlineto{\pgfqpoint{3.798854in}{2.774150in}}%
\pgfpathlineto{\pgfqpoint{3.802972in}{2.793572in}}%
\pgfpathlineto{\pgfqpoint{3.805031in}{2.783861in}}%
\pgfpathlineto{\pgfqpoint{3.807090in}{2.793572in}}%
\pgfpathlineto{\pgfqpoint{3.813267in}{2.793572in}}%
\pgfpathlineto{\pgfqpoint{3.815326in}{2.803283in}}%
\pgfpathlineto{\pgfqpoint{3.817385in}{2.793572in}}%
\pgfpathlineto{\pgfqpoint{3.821503in}{2.812994in}}%
\pgfpathlineto{\pgfqpoint{3.827680in}{2.851838in}}%
\pgfpathlineto{\pgfqpoint{3.831797in}{2.832416in}}%
\pgfpathlineto{\pgfqpoint{3.833856in}{2.812994in}}%
\pgfpathlineto{\pgfqpoint{3.835915in}{2.822705in}}%
\pgfpathlineto{\pgfqpoint{3.842092in}{2.861549in}}%
\pgfpathlineto{\pgfqpoint{3.846210in}{2.861549in}}%
\pgfpathlineto{\pgfqpoint{3.848269in}{2.851838in}}%
\pgfpathlineto{\pgfqpoint{3.850328in}{2.861549in}}%
\pgfpathlineto{\pgfqpoint{3.858564in}{2.880971in}}%
\pgfpathlineto{\pgfqpoint{3.860623in}{2.900393in}}%
\pgfpathlineto{\pgfqpoint{3.862682in}{2.900393in}}%
\pgfpathlineto{\pgfqpoint{3.870917in}{2.939237in}}%
\pgfpathlineto{\pgfqpoint{3.872976in}{2.929526in}}%
\pgfpathlineto{\pgfqpoint{3.877094in}{2.948948in}}%
\pgfpathlineto{\pgfqpoint{3.879153in}{2.939237in}}%
\pgfpathlineto{\pgfqpoint{3.885330in}{2.968370in}}%
\pgfpathlineto{\pgfqpoint{3.887389in}{2.958659in}}%
\pgfpathlineto{\pgfqpoint{3.889448in}{2.968370in}}%
\pgfpathlineto{\pgfqpoint{3.891507in}{2.968370in}}%
\pgfpathlineto{\pgfqpoint{3.893566in}{2.958659in}}%
\pgfpathlineto{\pgfqpoint{3.899743in}{2.968370in}}%
\pgfpathlineto{\pgfqpoint{3.901802in}{2.958659in}}%
\pgfpathlineto{\pgfqpoint{3.903861in}{2.968370in}}%
\pgfpathlineto{\pgfqpoint{3.905919in}{2.948948in}}%
\pgfpathlineto{\pgfqpoint{3.914155in}{2.987792in}}%
\pgfpathlineto{\pgfqpoint{3.916214in}{2.978081in}}%
\pgfpathlineto{\pgfqpoint{3.918273in}{2.997503in}}%
\pgfpathlineto{\pgfqpoint{3.920332in}{2.978081in}}%
\pgfpathlineto{\pgfqpoint{3.922391in}{2.987792in}}%
\pgfpathlineto{\pgfqpoint{3.930627in}{3.007214in}}%
\pgfpathlineto{\pgfqpoint{3.932686in}{3.007214in}}%
\pgfpathlineto{\pgfqpoint{3.936804in}{2.987792in}}%
\pgfpathlineto{\pgfqpoint{3.945039in}{3.016925in}}%
\pgfpathlineto{\pgfqpoint{3.947098in}{3.036347in}}%
\pgfpathlineto{\pgfqpoint{3.959452in}{3.036347in}}%
\pgfpathlineto{\pgfqpoint{3.961511in}{3.026636in}}%
\pgfpathlineto{\pgfqpoint{3.963570in}{2.997503in}}%
\pgfpathlineto{\pgfqpoint{3.965629in}{2.997503in}}%
\pgfpathlineto{\pgfqpoint{3.971806in}{3.007214in}}%
\pgfpathlineto{\pgfqpoint{3.973865in}{3.046058in}}%
\pgfpathlineto{\pgfqpoint{3.977983in}{3.036347in}}%
\pgfpathlineto{\pgfqpoint{3.980042in}{3.026636in}}%
\pgfpathlineto{\pgfqpoint{3.986218in}{3.036347in}}%
\pgfpathlineto{\pgfqpoint{3.988277in}{3.036347in}}%
\pgfpathlineto{\pgfqpoint{3.990336in}{3.055769in}}%
\pgfpathlineto{\pgfqpoint{3.992395in}{3.055769in}}%
\pgfpathlineto{\pgfqpoint{3.994454in}{3.046058in}}%
\pgfpathlineto{\pgfqpoint{4.002690in}{3.016925in}}%
\pgfpathlineto{\pgfqpoint{4.004749in}{3.026636in}}%
\pgfpathlineto{\pgfqpoint{4.006808in}{3.016925in}}%
\pgfpathlineto{\pgfqpoint{4.008867in}{3.016925in}}%
\pgfpathlineto{\pgfqpoint{4.015044in}{3.075191in}}%
\pgfpathlineto{\pgfqpoint{4.019161in}{3.065480in}}%
\pgfpathlineto{\pgfqpoint{4.021220in}{3.036347in}}%
\pgfpathlineto{\pgfqpoint{4.023279in}{3.026636in}}%
\pgfpathlineto{\pgfqpoint{4.029456in}{3.075191in}}%
\pgfpathlineto{\pgfqpoint{4.035633in}{3.036347in}}%
\pgfpathlineto{\pgfqpoint{4.045928in}{3.084902in}}%
\pgfpathlineto{\pgfqpoint{4.047987in}{3.075191in}}%
\pgfpathlineto{\pgfqpoint{4.050046in}{3.055769in}}%
\pgfpathlineto{\pgfqpoint{4.052105in}{3.055769in}}%
\pgfpathlineto{\pgfqpoint{4.058281in}{3.075191in}}%
\pgfpathlineto{\pgfqpoint{4.060340in}{3.075191in}}%
\pgfpathlineto{\pgfqpoint{4.064458in}{3.046058in}}%
\pgfpathlineto{\pgfqpoint{4.066517in}{3.036347in}}%
\pgfpathlineto{\pgfqpoint{4.074753in}{3.055769in}}%
\pgfpathlineto{\pgfqpoint{4.076812in}{3.036347in}}%
\pgfpathlineto{\pgfqpoint{4.078871in}{2.997503in}}%
\pgfpathlineto{\pgfqpoint{4.080930in}{3.016925in}}%
\pgfpathlineto{\pgfqpoint{4.087107in}{3.046058in}}%
\pgfpathlineto{\pgfqpoint{4.091225in}{3.046058in}}%
\pgfpathlineto{\pgfqpoint{4.095342in}{3.026636in}}%
\pgfpathlineto{\pgfqpoint{4.101519in}{3.046058in}}%
\pgfpathlineto{\pgfqpoint{4.107696in}{3.046058in}}%
\pgfpathlineto{\pgfqpoint{4.109755in}{3.055769in}}%
\pgfpathlineto{\pgfqpoint{4.115932in}{3.075191in}}%
\pgfpathlineto{\pgfqpoint{4.117991in}{3.055769in}}%
\pgfpathlineto{\pgfqpoint{4.120050in}{3.065480in}}%
\pgfpathlineto{\pgfqpoint{4.122109in}{3.055769in}}%
\pgfpathlineto{\pgfqpoint{4.124168in}{3.055769in}}%
\pgfpathlineto{\pgfqpoint{4.132403in}{3.075191in}}%
\pgfpathlineto{\pgfqpoint{4.136521in}{3.075191in}}%
\pgfpathlineto{\pgfqpoint{4.138580in}{3.084902in}}%
\pgfpathlineto{\pgfqpoint{4.144757in}{3.094613in}}%
\pgfpathlineto{\pgfqpoint{4.146816in}{3.075191in}}%
\pgfpathlineto{\pgfqpoint{4.150934in}{3.075191in}}%
\pgfpathlineto{\pgfqpoint{4.152993in}{3.065480in}}%
\pgfpathlineto{\pgfqpoint{4.159170in}{3.084902in}}%
\pgfpathlineto{\pgfqpoint{4.161229in}{3.084902in}}%
\pgfpathlineto{\pgfqpoint{4.163288in}{3.094613in}}%
\pgfpathlineto{\pgfqpoint{4.165347in}{3.075191in}}%
\pgfpathlineto{\pgfqpoint{4.167406in}{3.084902in}}%
\pgfpathlineto{\pgfqpoint{4.175641in}{3.084902in}}%
\pgfpathlineto{\pgfqpoint{4.177700in}{3.075191in}}%
\pgfpathlineto{\pgfqpoint{4.181818in}{3.075191in}}%
\pgfpathlineto{\pgfqpoint{4.187995in}{3.065480in}}%
\pgfpathlineto{\pgfqpoint{4.194172in}{3.114035in}}%
\pgfpathlineto{\pgfqpoint{4.196231in}{3.084902in}}%
\pgfpathlineto{\pgfqpoint{4.204467in}{3.084902in}}%
\pgfpathlineto{\pgfqpoint{4.210643in}{3.026636in}}%
\pgfpathlineto{\pgfqpoint{4.216820in}{3.055769in}}%
\pgfpathlineto{\pgfqpoint{4.218879in}{3.046058in}}%
\pgfpathlineto{\pgfqpoint{4.220938in}{3.065480in}}%
\pgfpathlineto{\pgfqpoint{4.225056in}{3.065480in}}%
\pgfpathlineto{\pgfqpoint{4.231233in}{3.055769in}}%
\pgfpathlineto{\pgfqpoint{4.233292in}{3.046058in}}%
\pgfpathlineto{\pgfqpoint{4.235351in}{3.055769in}}%
\pgfpathlineto{\pgfqpoint{4.237410in}{3.055769in}}%
\pgfpathlineto{\pgfqpoint{4.239469in}{3.065480in}}%
\pgfpathlineto{\pgfqpoint{4.245645in}{3.055769in}}%
\pgfpathlineto{\pgfqpoint{4.247704in}{3.055769in}}%
\pgfpathlineto{\pgfqpoint{4.249763in}{3.065480in}}%
\pgfpathlineto{\pgfqpoint{4.251822in}{3.046058in}}%
\pgfpathlineto{\pgfqpoint{4.260058in}{3.065480in}}%
\pgfpathlineto{\pgfqpoint{4.262117in}{3.075191in}}%
\pgfpathlineto{\pgfqpoint{4.268294in}{3.046058in}}%
\pgfpathlineto{\pgfqpoint{4.274471in}{3.065480in}}%
\pgfpathlineto{\pgfqpoint{4.276530in}{3.055769in}}%
\pgfpathlineto{\pgfqpoint{4.278589in}{3.055769in}}%
\pgfpathlineto{\pgfqpoint{4.280648in}{3.094613in}}%
\pgfpathlineto{\pgfqpoint{4.282707in}{3.055769in}}%
\pgfpathlineto{\pgfqpoint{4.288883in}{3.065480in}}%
\pgfpathlineto{\pgfqpoint{4.290942in}{3.055769in}}%
\pgfpathlineto{\pgfqpoint{4.297119in}{3.055769in}}%
\pgfpathlineto{\pgfqpoint{4.303296in}{3.036347in}}%
\pgfpathlineto{\pgfqpoint{4.305355in}{3.036347in}}%
\pgfpathlineto{\pgfqpoint{4.307414in}{3.046058in}}%
\pgfpathlineto{\pgfqpoint{4.311532in}{3.016925in}}%
\pgfpathlineto{\pgfqpoint{4.319768in}{3.016925in}}%
\pgfpathlineto{\pgfqpoint{4.323885in}{2.997503in}}%
\pgfpathlineto{\pgfqpoint{4.325944in}{2.978081in}}%
\pgfpathlineto{\pgfqpoint{4.334180in}{2.997503in}}%
\pgfpathlineto{\pgfqpoint{4.336239in}{2.997503in}}%
\pgfpathlineto{\pgfqpoint{4.338298in}{3.007214in}}%
\pgfpathlineto{\pgfqpoint{4.340357in}{2.978081in}}%
\pgfpathlineto{\pgfqpoint{4.350652in}{2.978081in}}%
\pgfpathlineto{\pgfqpoint{4.352711in}{2.958659in}}%
\pgfpathlineto{\pgfqpoint{4.354770in}{2.910104in}}%
\pgfpathlineto{\pgfqpoint{4.360946in}{2.919815in}}%
\pgfpathlineto{\pgfqpoint{4.365064in}{2.871260in}}%
\pgfpathlineto{\pgfqpoint{4.367123in}{2.822705in}}%
\pgfpathlineto{\pgfqpoint{4.375359in}{2.861549in}}%
\pgfpathlineto{\pgfqpoint{4.377418in}{2.851838in}}%
\pgfpathlineto{\pgfqpoint{4.383595in}{2.745017in}}%
\pgfpathlineto{\pgfqpoint{4.389772in}{2.764439in}}%
\pgfpathlineto{\pgfqpoint{4.391831in}{2.754728in}}%
\pgfpathlineto{\pgfqpoint{4.393890in}{2.783861in}}%
\pgfpathlineto{\pgfqpoint{4.395949in}{2.774150in}}%
\pgfpathlineto{\pgfqpoint{4.398007in}{2.754728in}}%
\pgfpathlineto{\pgfqpoint{4.404184in}{2.842127in}}%
\pgfpathlineto{\pgfqpoint{4.406243in}{2.832416in}}%
\pgfpathlineto{\pgfqpoint{4.418597in}{2.890682in}}%
\pgfpathlineto{\pgfqpoint{4.420656in}{2.890682in}}%
\pgfpathlineto{\pgfqpoint{4.424774in}{2.803283in}}%
\pgfpathlineto{\pgfqpoint{4.426833in}{2.774150in}}%
\pgfpathlineto{\pgfqpoint{4.433010in}{2.793572in}}%
\pgfpathlineto{\pgfqpoint{4.437127in}{2.774150in}}%
\pgfpathlineto{\pgfqpoint{4.439186in}{2.686751in}}%
\pgfpathlineto{\pgfqpoint{4.447422in}{2.725595in}}%
\pgfpathlineto{\pgfqpoint{4.449481in}{2.696462in}}%
\pgfpathlineto{\pgfqpoint{4.451540in}{2.735306in}}%
\pgfpathlineto{\pgfqpoint{4.453599in}{2.735306in}}%
\pgfpathlineto{\pgfqpoint{4.455658in}{2.754728in}}%
\pgfpathlineto{\pgfqpoint{4.461835in}{2.754728in}}%
\pgfpathlineto{\pgfqpoint{4.463894in}{2.715884in}}%
\pgfpathlineto{\pgfqpoint{4.465953in}{2.715884in}}%
\pgfpathlineto{\pgfqpoint{4.470071in}{2.696462in}}%
\pgfpathlineto{\pgfqpoint{4.476247in}{2.686751in}}%
\pgfpathlineto{\pgfqpoint{4.478306in}{2.686751in}}%
\pgfpathlineto{\pgfqpoint{4.480365in}{2.657618in}}%
\pgfpathlineto{\pgfqpoint{4.482424in}{2.657618in}}%
\pgfpathlineto{\pgfqpoint{4.484483in}{2.638197in}}%
\pgfpathlineto{\pgfqpoint{4.492719in}{2.638197in}}%
\pgfpathlineto{\pgfqpoint{4.498896in}{2.511954in}}%
\pgfpathlineto{\pgfqpoint{4.505073in}{2.579931in}}%
\pgfpathlineto{\pgfqpoint{4.507132in}{2.579931in}}%
\pgfpathlineto{\pgfqpoint{4.511249in}{2.638197in}}%
\pgfpathlineto{\pgfqpoint{4.513308in}{2.609064in}}%
\pgfpathlineto{\pgfqpoint{4.519485in}{2.647908in}}%
\pgfpathlineto{\pgfqpoint{4.521544in}{2.618775in}}%
\pgfpathlineto{\pgfqpoint{4.523603in}{2.628486in}}%
\pgfpathlineto{\pgfqpoint{4.527721in}{2.628486in}}%
\pgfpathlineto{\pgfqpoint{4.535957in}{2.618775in}}%
\pgfpathlineto{\pgfqpoint{4.538016in}{2.609064in}}%
\pgfpathlineto{\pgfqpoint{4.540075in}{2.609064in}}%
\pgfpathlineto{\pgfqpoint{4.542134in}{2.599353in}}%
\pgfpathlineto{\pgfqpoint{4.548311in}{2.599353in}}%
\pgfpathlineto{\pgfqpoint{4.550369in}{2.589642in}}%
\pgfpathlineto{\pgfqpoint{4.552428in}{2.599353in}}%
\pgfpathlineto{\pgfqpoint{4.554487in}{2.589642in}}%
\pgfpathlineto{\pgfqpoint{4.562723in}{2.628486in}}%
\pgfpathlineto{\pgfqpoint{4.564782in}{2.628486in}}%
\pgfpathlineto{\pgfqpoint{4.568900in}{2.570220in}}%
\pgfpathlineto{\pgfqpoint{4.570959in}{2.550798in}}%
\pgfpathlineto{\pgfqpoint{4.577136in}{2.579931in}}%
\pgfpathlineto{\pgfqpoint{4.581254in}{2.531376in}}%
\pgfpathlineto{\pgfqpoint{4.585372in}{2.443977in}}%
\pgfpathlineto{\pgfqpoint{4.591548in}{2.521665in}}%
\pgfpathlineto{\pgfqpoint{4.597725in}{2.346867in}}%
\pgfpathlineto{\pgfqpoint{4.599784in}{2.356578in}}%
\pgfpathlineto{\pgfqpoint{4.605961in}{2.395422in}}%
\pgfpathlineto{\pgfqpoint{4.610079in}{2.337156in}}%
\pgfpathlineto{\pgfqpoint{4.612138in}{2.327445in}}%
\pgfpathlineto{\pgfqpoint{4.614197in}{2.327445in}}%
\pgfpathlineto{\pgfqpoint{4.622433in}{2.317734in}}%
\pgfpathlineto{\pgfqpoint{4.624491in}{2.308023in}}%
\pgfpathlineto{\pgfqpoint{4.628609in}{2.308023in}}%
\pgfpathlineto{\pgfqpoint{4.634786in}{2.317734in}}%
\pgfpathlineto{\pgfqpoint{4.636845in}{2.298312in}}%
\pgfpathlineto{\pgfqpoint{4.638904in}{2.298312in}}%
\pgfpathlineto{\pgfqpoint{4.640963in}{2.317734in}}%
\pgfpathlineto{\pgfqpoint{4.643022in}{2.308023in}}%
\pgfpathlineto{\pgfqpoint{4.649199in}{2.298312in}}%
\pgfpathlineto{\pgfqpoint{4.653317in}{2.269179in}}%
\pgfpathlineto{\pgfqpoint{4.655376in}{2.191491in}}%
\pgfpathlineto{\pgfqpoint{4.657435in}{2.172069in}}%
\pgfpathlineto{\pgfqpoint{4.663611in}{2.181780in}}%
\pgfpathlineto{\pgfqpoint{4.665670in}{2.210913in}}%
\pgfpathlineto{\pgfqpoint{4.669788in}{2.210913in}}%
\pgfpathlineto{\pgfqpoint{4.671847in}{2.201202in}}%
\pgfpathlineto{\pgfqpoint{4.680083in}{2.240046in}}%
\pgfpathlineto{\pgfqpoint{4.682142in}{2.220624in}}%
\pgfpathlineto{\pgfqpoint{4.696555in}{2.220624in}}%
\pgfpathlineto{\pgfqpoint{4.698614in}{2.230335in}}%
\pgfpathlineto{\pgfqpoint{4.700672in}{2.230335in}}%
\pgfpathlineto{\pgfqpoint{4.706849in}{2.259468in}}%
\pgfpathlineto{\pgfqpoint{4.708908in}{2.249757in}}%
\pgfpathlineto{\pgfqpoint{4.710967in}{2.269179in}}%
\pgfpathlineto{\pgfqpoint{4.715085in}{2.240046in}}%
\pgfpathlineto{\pgfqpoint{4.721262in}{2.249757in}}%
\pgfpathlineto{\pgfqpoint{4.725380in}{2.201202in}}%
\pgfpathlineto{\pgfqpoint{4.729498in}{2.181780in}}%
\pgfpathlineto{\pgfqpoint{4.735675in}{2.191491in}}%
\pgfpathlineto{\pgfqpoint{4.739792in}{2.220624in}}%
\pgfpathlineto{\pgfqpoint{4.741851in}{2.210913in}}%
\pgfpathlineto{\pgfqpoint{4.743910in}{2.220624in}}%
\pgfpathlineto{\pgfqpoint{4.750087in}{2.220624in}}%
\pgfpathlineto{\pgfqpoint{4.752146in}{2.210913in}}%
\pgfpathlineto{\pgfqpoint{4.754205in}{2.210913in}}%
\pgfpathlineto{\pgfqpoint{4.758323in}{2.230335in}}%
\pgfpathlineto{\pgfqpoint{4.764500in}{2.240046in}}%
\pgfpathlineto{\pgfqpoint{4.766559in}{2.230335in}}%
\pgfpathlineto{\pgfqpoint{4.770677in}{2.230335in}}%
\pgfpathlineto{\pgfqpoint{4.772736in}{2.220624in}}%
\pgfpathlineto{\pgfqpoint{4.778912in}{2.220624in}}%
\pgfpathlineto{\pgfqpoint{4.780971in}{2.201202in}}%
\pgfpathlineto{\pgfqpoint{4.785089in}{2.191491in}}%
\pgfpathlineto{\pgfqpoint{4.787148in}{2.172069in}}%
\pgfpathlineto{\pgfqpoint{4.793325in}{2.210913in}}%
\pgfpathlineto{\pgfqpoint{4.795384in}{2.191491in}}%
\pgfpathlineto{\pgfqpoint{4.801561in}{2.191491in}}%
\pgfpathlineto{\pgfqpoint{4.807738in}{2.220624in}}%
\pgfpathlineto{\pgfqpoint{4.811856in}{2.220624in}}%
\pgfpathlineto{\pgfqpoint{4.813914in}{2.201202in}}%
\pgfpathlineto{\pgfqpoint{4.815973in}{2.210913in}}%
\pgfpathlineto{\pgfqpoint{4.824209in}{2.210913in}}%
\pgfpathlineto{\pgfqpoint{4.826268in}{2.201202in}}%
\pgfpathlineto{\pgfqpoint{4.828327in}{2.201202in}}%
\pgfpathlineto{\pgfqpoint{4.830386in}{2.191491in}}%
\pgfpathlineto{\pgfqpoint{4.836563in}{2.201202in}}%
\pgfpathlineto{\pgfqpoint{4.838622in}{2.220624in}}%
\pgfpathlineto{\pgfqpoint{4.840681in}{2.210913in}}%
\pgfpathlineto{\pgfqpoint{4.842740in}{2.220624in}}%
\pgfpathlineto{\pgfqpoint{4.844799in}{2.201202in}}%
\pgfpathlineto{\pgfqpoint{4.850976in}{2.220624in}}%
\pgfpathlineto{\pgfqpoint{4.857152in}{2.220624in}}%
\pgfpathlineto{\pgfqpoint{4.859211in}{2.210913in}}%
\pgfpathlineto{\pgfqpoint{4.865388in}{2.230335in}}%
\pgfpathlineto{\pgfqpoint{4.867447in}{2.220624in}}%
\pgfpathlineto{\pgfqpoint{4.871565in}{2.240046in}}%
\pgfpathlineto{\pgfqpoint{4.873624in}{2.230335in}}%
\pgfpathlineto{\pgfqpoint{4.885978in}{2.230335in}}%
\pgfpathlineto{\pgfqpoint{4.888037in}{2.210913in}}%
\pgfpathlineto{\pgfqpoint{4.894213in}{2.181780in}}%
\pgfpathlineto{\pgfqpoint{4.898331in}{2.181780in}}%
\pgfpathlineto{\pgfqpoint{4.900390in}{2.104092in}}%
\pgfpathlineto{\pgfqpoint{4.902449in}{1.929295in}}%
\pgfpathlineto{\pgfqpoint{4.908626in}{1.793341in}}%
\pgfpathlineto{\pgfqpoint{4.916862in}{1.132994in}}%
\pgfpathlineto{\pgfqpoint{4.923039in}{1.016462in}}%
\pgfpathlineto{\pgfqpoint{4.925098in}{1.123283in}}%
\pgfpathlineto{\pgfqpoint{4.927156in}{1.103861in}}%
\pgfpathlineto{\pgfqpoint{4.931274in}{0.967908in}}%
\pgfpathlineto{\pgfqpoint{4.937451in}{0.929064in}}%
\pgfpathlineto{\pgfqpoint{4.939510in}{0.880509in}}%
\pgfpathlineto{\pgfqpoint{4.941569in}{0.715422in}}%
\pgfpathlineto{\pgfqpoint{4.945687in}{0.744555in}}%
\pgfpathlineto{\pgfqpoint{4.955982in}{0.696000in}}%
\pgfpathlineto{\pgfqpoint{4.958041in}{0.696000in}}%
\pgfpathlineto{\pgfqpoint{4.966276in}{0.812532in}}%
\pgfpathlineto{\pgfqpoint{4.968335in}{0.802821in}}%
\pgfpathlineto{\pgfqpoint{4.970394in}{0.783399in}}%
\pgfpathlineto{\pgfqpoint{4.972453in}{0.783399in}}%
\pgfpathlineto{\pgfqpoint{4.974512in}{0.793110in}}%
\pgfpathlineto{\pgfqpoint{4.980689in}{0.841665in}}%
\pgfpathlineto{\pgfqpoint{4.982748in}{0.831954in}}%
\pgfpathlineto{\pgfqpoint{4.984807in}{0.909642in}}%
\pgfpathlineto{\pgfqpoint{4.986866in}{0.938775in}}%
\pgfpathlineto{\pgfqpoint{4.995102in}{0.948486in}}%
\pgfpathlineto{\pgfqpoint{4.999220in}{0.831954in}}%
\pgfpathlineto{\pgfqpoint{5.001279in}{0.831954in}}%
\pgfpathlineto{\pgfqpoint{5.003337in}{0.812532in}}%
\pgfpathlineto{\pgfqpoint{5.009514in}{0.812532in}}%
\pgfpathlineto{\pgfqpoint{5.011573in}{0.802821in}}%
\pgfpathlineto{\pgfqpoint{5.013632in}{0.812532in}}%
\pgfpathlineto{\pgfqpoint{5.015691in}{0.802821in}}%
\pgfpathlineto{\pgfqpoint{5.017750in}{0.812532in}}%
\pgfpathlineto{\pgfqpoint{5.023927in}{0.812532in}}%
\pgfpathlineto{\pgfqpoint{5.030104in}{0.783399in}}%
\pgfpathlineto{\pgfqpoint{5.032163in}{0.812532in}}%
\pgfpathlineto{\pgfqpoint{5.038340in}{0.822243in}}%
\pgfpathlineto{\pgfqpoint{5.040399in}{0.822243in}}%
\pgfpathlineto{\pgfqpoint{5.044516in}{0.802821in}}%
\pgfpathlineto{\pgfqpoint{5.046575in}{0.812532in}}%
\pgfpathlineto{\pgfqpoint{5.052752in}{0.812532in}}%
\pgfpathlineto{\pgfqpoint{5.054811in}{0.822243in}}%
\pgfpathlineto{\pgfqpoint{5.056870in}{0.822243in}}%
\pgfpathlineto{\pgfqpoint{5.058929in}{0.812532in}}%
\pgfpathlineto{\pgfqpoint{5.060988in}{0.812532in}}%
\pgfpathlineto{\pgfqpoint{5.067165in}{0.822243in}}%
\pgfpathlineto{\pgfqpoint{5.069224in}{0.822243in}}%
\pgfpathlineto{\pgfqpoint{5.071283in}{0.812532in}}%
\pgfpathlineto{\pgfqpoint{5.075401in}{0.812532in}}%
\pgfpathlineto{\pgfqpoint{5.083636in}{0.831954in}}%
\pgfpathlineto{\pgfqpoint{5.085695in}{0.841665in}}%
\pgfpathlineto{\pgfqpoint{5.087754in}{0.841665in}}%
\pgfpathlineto{\pgfqpoint{5.089813in}{0.831954in}}%
\pgfpathlineto{\pgfqpoint{5.095990in}{0.831954in}}%
\pgfpathlineto{\pgfqpoint{5.100108in}{0.851376in}}%
\pgfpathlineto{\pgfqpoint{5.102167in}{0.841665in}}%
\pgfpathlineto{\pgfqpoint{5.104226in}{0.841665in}}%
\pgfpathlineto{\pgfqpoint{5.110403in}{0.861087in}}%
\pgfpathlineto{\pgfqpoint{5.112462in}{0.880509in}}%
\pgfpathlineto{\pgfqpoint{5.114521in}{0.861087in}}%
\pgfpathlineto{\pgfqpoint{5.116579in}{0.861087in}}%
\pgfpathlineto{\pgfqpoint{5.118638in}{0.851376in}}%
\pgfpathlineto{\pgfqpoint{5.124815in}{0.870798in}}%
\pgfpathlineto{\pgfqpoint{5.126874in}{0.861087in}}%
\pgfpathlineto{\pgfqpoint{5.128933in}{0.861087in}}%
\pgfpathlineto{\pgfqpoint{5.133051in}{0.841665in}}%
\pgfpathlineto{\pgfqpoint{5.139228in}{0.851376in}}%
\pgfpathlineto{\pgfqpoint{5.141287in}{0.851376in}}%
\pgfpathlineto{\pgfqpoint{5.143346in}{0.841665in}}%
\pgfpathlineto{\pgfqpoint{5.145405in}{0.851376in}}%
\pgfpathlineto{\pgfqpoint{5.147464in}{0.831954in}}%
\pgfpathlineto{\pgfqpoint{5.153641in}{0.831954in}}%
\pgfpathlineto{\pgfqpoint{5.155699in}{0.851376in}}%
\pgfpathlineto{\pgfqpoint{5.157758in}{0.831954in}}%
\pgfpathlineto{\pgfqpoint{5.159817in}{0.831954in}}%
\pgfpathlineto{\pgfqpoint{5.168053in}{0.841665in}}%
\pgfpathlineto{\pgfqpoint{5.172171in}{0.841665in}}%
\pgfpathlineto{\pgfqpoint{5.174230in}{0.822243in}}%
\pgfpathlineto{\pgfqpoint{5.176289in}{0.822243in}}%
\pgfpathlineto{\pgfqpoint{5.182466in}{0.831954in}}%
\pgfpathlineto{\pgfqpoint{5.186584in}{0.851376in}}%
\pgfpathlineto{\pgfqpoint{5.188643in}{0.802821in}}%
\pgfpathlineto{\pgfqpoint{5.190702in}{0.802821in}}%
\pgfpathlineto{\pgfqpoint{5.196878in}{0.822243in}}%
\pgfpathlineto{\pgfqpoint{5.200996in}{0.822243in}}%
\pgfpathlineto{\pgfqpoint{5.205114in}{0.802821in}}%
\pgfpathlineto{\pgfqpoint{5.215409in}{0.802821in}}%
\pgfpathlineto{\pgfqpoint{5.217468in}{0.783399in}}%
\pgfpathlineto{\pgfqpoint{5.219527in}{0.783399in}}%
\pgfpathlineto{\pgfqpoint{5.225704in}{0.793110in}}%
\pgfpathlineto{\pgfqpoint{5.227763in}{0.783399in}}%
\pgfpathlineto{\pgfqpoint{5.229822in}{0.793110in}}%
\pgfpathlineto{\pgfqpoint{5.233939in}{0.793110in}}%
\pgfpathlineto{\pgfqpoint{5.240116in}{0.802821in}}%
\pgfpathlineto{\pgfqpoint{5.244234in}{0.802821in}}%
\pgfpathlineto{\pgfqpoint{5.246293in}{0.793110in}}%
\pgfpathlineto{\pgfqpoint{5.254529in}{0.793110in}}%
\pgfpathlineto{\pgfqpoint{5.256588in}{0.783399in}}%
\pgfpathlineto{\pgfqpoint{5.258647in}{0.802821in}}%
\pgfpathlineto{\pgfqpoint{5.260706in}{0.802821in}}%
\pgfpathlineto{\pgfqpoint{5.262765in}{0.793110in}}%
\pgfpathlineto{\pgfqpoint{5.268941in}{0.812532in}}%
\pgfpathlineto{\pgfqpoint{5.271000in}{0.802821in}}%
\pgfpathlineto{\pgfqpoint{5.275118in}{0.802821in}}%
\pgfpathlineto{\pgfqpoint{5.277177in}{0.793110in}}%
\pgfpathlineto{\pgfqpoint{5.283354in}{0.802821in}}%
\pgfpathlineto{\pgfqpoint{5.285413in}{0.812532in}}%
\pgfpathlineto{\pgfqpoint{5.287472in}{0.812532in}}%
\pgfpathlineto{\pgfqpoint{5.289531in}{0.802821in}}%
\pgfpathlineto{\pgfqpoint{5.291590in}{0.802821in}}%
\pgfpathlineto{\pgfqpoint{5.299826in}{0.822243in}}%
\pgfpathlineto{\pgfqpoint{5.301885in}{0.812532in}}%
\pgfpathlineto{\pgfqpoint{5.303944in}{0.812532in}}%
\pgfpathlineto{\pgfqpoint{5.306002in}{0.802821in}}%
\pgfpathlineto{\pgfqpoint{5.314238in}{0.802821in}}%
\pgfpathlineto{\pgfqpoint{5.316297in}{0.812532in}}%
\pgfpathlineto{\pgfqpoint{5.318356in}{0.783399in}}%
\pgfpathlineto{\pgfqpoint{5.320415in}{0.793110in}}%
\pgfpathlineto{\pgfqpoint{5.328651in}{0.793110in}}%
\pgfpathlineto{\pgfqpoint{5.330710in}{0.802821in}}%
\pgfpathlineto{\pgfqpoint{5.332769in}{0.793110in}}%
\pgfpathlineto{\pgfqpoint{5.334828in}{0.793110in}}%
\pgfpathlineto{\pgfqpoint{5.341005in}{0.802821in}}%
\pgfpathlineto{\pgfqpoint{5.343064in}{0.783399in}}%
\pgfpathlineto{\pgfqpoint{5.345122in}{0.793110in}}%
\pgfpathlineto{\pgfqpoint{5.347181in}{0.783399in}}%
\pgfpathlineto{\pgfqpoint{5.349240in}{0.783399in}}%
\pgfpathlineto{\pgfqpoint{5.355417in}{0.793110in}}%
\pgfpathlineto{\pgfqpoint{5.359535in}{0.793110in}}%
\pgfpathlineto{\pgfqpoint{5.361594in}{0.783399in}}%
\pgfpathlineto{\pgfqpoint{5.363653in}{0.793110in}}%
\pgfpathlineto{\pgfqpoint{5.371889in}{0.802821in}}%
\pgfpathlineto{\pgfqpoint{5.373948in}{0.812532in}}%
\pgfpathlineto{\pgfqpoint{5.376007in}{0.802821in}}%
\pgfpathlineto{\pgfqpoint{5.384242in}{0.802821in}}%
\pgfpathlineto{\pgfqpoint{5.386301in}{0.793110in}}%
\pgfpathlineto{\pgfqpoint{5.388360in}{0.793110in}}%
\pgfpathlineto{\pgfqpoint{5.390419in}{0.783399in}}%
\pgfpathlineto{\pgfqpoint{5.392478in}{0.793110in}}%
\pgfpathlineto{\pgfqpoint{5.398655in}{0.802821in}}%
\pgfpathlineto{\pgfqpoint{5.400714in}{0.793110in}}%
\pgfpathlineto{\pgfqpoint{5.402773in}{0.793110in}}%
\pgfpathlineto{\pgfqpoint{5.404832in}{0.783399in}}%
\pgfpathlineto{\pgfqpoint{5.413068in}{0.783399in}}%
\pgfpathlineto{\pgfqpoint{5.415127in}{0.793110in}}%
\pgfpathlineto{\pgfqpoint{5.421303in}{0.793110in}}%
\pgfpathlineto{\pgfqpoint{5.427480in}{0.802821in}}%
\pgfpathlineto{\pgfqpoint{5.429539in}{0.793110in}}%
\pgfpathlineto{\pgfqpoint{5.433657in}{0.793110in}}%
\pgfpathlineto{\pgfqpoint{5.435716in}{0.783399in}}%
\pgfpathlineto{\pgfqpoint{5.446011in}{0.783399in}}%
\pgfpathlineto{\pgfqpoint{5.448070in}{0.763977in}}%
\pgfpathlineto{\pgfqpoint{5.450129in}{0.763977in}}%
\pgfpathlineto{\pgfqpoint{5.456306in}{0.773688in}}%
\pgfpathlineto{\pgfqpoint{5.458364in}{0.783399in}}%
\pgfpathlineto{\pgfqpoint{5.464541in}{0.783399in}}%
\pgfpathlineto{\pgfqpoint{5.470718in}{0.773688in}}%
\pgfpathlineto{\pgfqpoint{5.472777in}{0.783399in}}%
\pgfpathlineto{\pgfqpoint{5.474836in}{0.783399in}}%
\pgfpathlineto{\pgfqpoint{5.476895in}{0.773688in}}%
\pgfpathlineto{\pgfqpoint{5.478954in}{0.783399in}}%
\pgfpathlineto{\pgfqpoint{5.485131in}{0.773688in}}%
\pgfpathlineto{\pgfqpoint{5.487190in}{0.783399in}}%
\pgfpathlineto{\pgfqpoint{5.489249in}{0.773688in}}%
\pgfpathlineto{\pgfqpoint{5.493367in}{0.773688in}}%
\pgfpathlineto{\pgfqpoint{5.499543in}{0.783399in}}%
\pgfpathlineto{\pgfqpoint{5.501602in}{0.773688in}}%
\pgfpathlineto{\pgfqpoint{5.503661in}{0.783399in}}%
\pgfpathlineto{\pgfqpoint{5.505720in}{0.773688in}}%
\pgfpathlineto{\pgfqpoint{5.507779in}{0.773688in}}%
\pgfpathlineto{\pgfqpoint{5.513956in}{0.783399in}}%
\pgfpathlineto{\pgfqpoint{5.520133in}{0.783399in}}%
\pgfpathlineto{\pgfqpoint{5.528369in}{0.802821in}}%
\pgfpathlineto{\pgfqpoint{5.530428in}{0.793110in}}%
\pgfpathlineto{\pgfqpoint{5.532487in}{0.773688in}}%
\pgfpathlineto{\pgfqpoint{5.534545in}{0.783399in}}%
\pgfpathlineto{\pgfqpoint{5.534545in}{0.783399in}}%
\pgfusepath{stroke}%
\end{pgfscope}%
\begin{pgfscope}%
\pgfpathrectangle{\pgfqpoint{0.800000in}{0.528000in}}{\pgfqpoint{4.960000in}{3.696000in}}%
\pgfusepath{clip}%
\pgfsetrectcap%
\pgfsetroundjoin%
\pgfsetlinewidth{1.003750pt}%
\definecolor{currentstroke}{rgb}{0.501961,0.501961,0.501961}%
\pgfsetstrokecolor{currentstroke}%
\pgfsetstrokeopacity{0.900000}%
\pgfsetdash{}{0pt}%
\pgfpathmoveto{\pgfqpoint{1.025455in}{0.802821in}}%
\pgfpathlineto{\pgfqpoint{1.031631in}{0.793110in}}%
\pgfpathlineto{\pgfqpoint{1.033690in}{0.793110in}}%
\pgfpathlineto{\pgfqpoint{1.037808in}{0.773688in}}%
\pgfpathlineto{\pgfqpoint{1.039867in}{0.773688in}}%
\pgfpathlineto{\pgfqpoint{1.046044in}{0.783399in}}%
\pgfpathlineto{\pgfqpoint{1.048103in}{0.773688in}}%
\pgfpathlineto{\pgfqpoint{1.050162in}{0.783399in}}%
\pgfpathlineto{\pgfqpoint{1.054280in}{0.763977in}}%
\pgfpathlineto{\pgfqpoint{1.062516in}{0.773688in}}%
\pgfpathlineto{\pgfqpoint{1.066633in}{0.773688in}}%
\pgfpathlineto{\pgfqpoint{1.068692in}{0.763977in}}%
\pgfpathlineto{\pgfqpoint{1.074869in}{0.773688in}}%
\pgfpathlineto{\pgfqpoint{1.078987in}{0.773688in}}%
\pgfpathlineto{\pgfqpoint{1.081046in}{0.763977in}}%
\pgfpathlineto{\pgfqpoint{1.091341in}{0.763977in}}%
\pgfpathlineto{\pgfqpoint{1.093400in}{0.754266in}}%
\pgfpathlineto{\pgfqpoint{1.095459in}{0.754266in}}%
\pgfpathlineto{\pgfqpoint{1.097518in}{0.763977in}}%
\pgfpathlineto{\pgfqpoint{1.103694in}{0.773688in}}%
\pgfpathlineto{\pgfqpoint{1.105753in}{0.773688in}}%
\pgfpathlineto{\pgfqpoint{1.107812in}{0.763977in}}%
\pgfpathlineto{\pgfqpoint{1.126343in}{0.763977in}}%
\pgfpathlineto{\pgfqpoint{1.132520in}{0.773688in}}%
\pgfpathlineto{\pgfqpoint{1.134579in}{0.773688in}}%
\pgfpathlineto{\pgfqpoint{1.136638in}{0.763977in}}%
\pgfpathlineto{\pgfqpoint{1.140756in}{0.763977in}}%
\pgfpathlineto{\pgfqpoint{1.146932in}{0.773688in}}%
\pgfpathlineto{\pgfqpoint{1.155168in}{0.773688in}}%
\pgfpathlineto{\pgfqpoint{1.161345in}{0.793110in}}%
\pgfpathlineto{\pgfqpoint{1.167522in}{0.793110in}}%
\pgfpathlineto{\pgfqpoint{1.169581in}{0.802821in}}%
\pgfpathlineto{\pgfqpoint{1.175758in}{0.841665in}}%
\pgfpathlineto{\pgfqpoint{1.177817in}{0.841665in}}%
\pgfpathlineto{\pgfqpoint{1.179875in}{0.812532in}}%
\pgfpathlineto{\pgfqpoint{1.181934in}{0.812532in}}%
\pgfpathlineto{\pgfqpoint{1.183993in}{0.802821in}}%
\pgfpathlineto{\pgfqpoint{1.194288in}{0.802821in}}%
\pgfpathlineto{\pgfqpoint{1.196347in}{0.822243in}}%
\pgfpathlineto{\pgfqpoint{1.198406in}{0.812532in}}%
\pgfpathlineto{\pgfqpoint{1.204583in}{0.831954in}}%
\pgfpathlineto{\pgfqpoint{1.206642in}{0.831954in}}%
\pgfpathlineto{\pgfqpoint{1.210760in}{0.793110in}}%
\pgfpathlineto{\pgfqpoint{1.225172in}{0.793110in}}%
\pgfpathlineto{\pgfqpoint{1.227231in}{0.783399in}}%
\pgfpathlineto{\pgfqpoint{1.233408in}{0.802821in}}%
\pgfpathlineto{\pgfqpoint{1.235467in}{0.793110in}}%
\pgfpathlineto{\pgfqpoint{1.237526in}{0.773688in}}%
\pgfpathlineto{\pgfqpoint{1.241644in}{0.773688in}}%
\pgfpathlineto{\pgfqpoint{1.247821in}{0.793110in}}%
\pgfpathlineto{\pgfqpoint{1.249880in}{0.783399in}}%
\pgfpathlineto{\pgfqpoint{1.251939in}{0.793110in}}%
\pgfpathlineto{\pgfqpoint{1.253998in}{0.783399in}}%
\pgfpathlineto{\pgfqpoint{1.256056in}{0.793110in}}%
\pgfpathlineto{\pgfqpoint{1.262233in}{0.793110in}}%
\pgfpathlineto{\pgfqpoint{1.264292in}{0.783399in}}%
\pgfpathlineto{\pgfqpoint{1.268410in}{0.754266in}}%
\pgfpathlineto{\pgfqpoint{1.270469in}{0.744555in}}%
\pgfpathlineto{\pgfqpoint{1.276646in}{0.773688in}}%
\pgfpathlineto{\pgfqpoint{1.284882in}{0.773688in}}%
\pgfpathlineto{\pgfqpoint{1.291059in}{0.783399in}}%
\pgfpathlineto{\pgfqpoint{1.295176in}{0.783399in}}%
\pgfpathlineto{\pgfqpoint{1.297235in}{0.773688in}}%
\pgfpathlineto{\pgfqpoint{1.299294in}{0.783399in}}%
\pgfpathlineto{\pgfqpoint{1.305471in}{0.773688in}}%
\pgfpathlineto{\pgfqpoint{1.307530in}{0.763977in}}%
\pgfpathlineto{\pgfqpoint{1.309589in}{0.773688in}}%
\pgfpathlineto{\pgfqpoint{1.311648in}{0.763977in}}%
\pgfpathlineto{\pgfqpoint{1.313707in}{0.773688in}}%
\pgfpathlineto{\pgfqpoint{1.321943in}{0.783399in}}%
\pgfpathlineto{\pgfqpoint{1.324002in}{0.783399in}}%
\pgfpathlineto{\pgfqpoint{1.326061in}{0.773688in}}%
\pgfpathlineto{\pgfqpoint{1.328120in}{0.754266in}}%
\pgfpathlineto{\pgfqpoint{1.334296in}{0.763977in}}%
\pgfpathlineto{\pgfqpoint{1.338414in}{0.763977in}}%
\pgfpathlineto{\pgfqpoint{1.340473in}{0.773688in}}%
\pgfpathlineto{\pgfqpoint{1.342532in}{0.773688in}}%
\pgfpathlineto{\pgfqpoint{1.348709in}{0.783399in}}%
\pgfpathlineto{\pgfqpoint{1.350768in}{0.773688in}}%
\pgfpathlineto{\pgfqpoint{1.352827in}{0.793110in}}%
\pgfpathlineto{\pgfqpoint{1.356945in}{0.793110in}}%
\pgfpathlineto{\pgfqpoint{1.363122in}{0.802821in}}%
\pgfpathlineto{\pgfqpoint{1.365181in}{0.802821in}}%
\pgfpathlineto{\pgfqpoint{1.367240in}{0.793110in}}%
\pgfpathlineto{\pgfqpoint{1.371357in}{0.744555in}}%
\pgfpathlineto{\pgfqpoint{1.379593in}{0.783399in}}%
\pgfpathlineto{\pgfqpoint{1.383711in}{0.763977in}}%
\pgfpathlineto{\pgfqpoint{1.391947in}{0.802821in}}%
\pgfpathlineto{\pgfqpoint{1.394006in}{0.802821in}}%
\pgfpathlineto{\pgfqpoint{1.396065in}{0.822243in}}%
\pgfpathlineto{\pgfqpoint{1.398124in}{0.793110in}}%
\pgfpathlineto{\pgfqpoint{1.406359in}{0.783399in}}%
\pgfpathlineto{\pgfqpoint{1.408418in}{0.773688in}}%
\pgfpathlineto{\pgfqpoint{1.412536in}{0.773688in}}%
\pgfpathlineto{\pgfqpoint{1.414595in}{0.783399in}}%
\pgfpathlineto{\pgfqpoint{1.420772in}{0.793110in}}%
\pgfpathlineto{\pgfqpoint{1.422831in}{0.793110in}}%
\pgfpathlineto{\pgfqpoint{1.424890in}{0.802821in}}%
\pgfpathlineto{\pgfqpoint{1.426949in}{0.793110in}}%
\pgfpathlineto{\pgfqpoint{1.435185in}{0.831954in}}%
\pgfpathlineto{\pgfqpoint{1.437244in}{0.822243in}}%
\pgfpathlineto{\pgfqpoint{1.441362in}{0.822243in}}%
\pgfpathlineto{\pgfqpoint{1.443421in}{0.831954in}}%
\pgfpathlineto{\pgfqpoint{1.449597in}{0.841665in}}%
\pgfpathlineto{\pgfqpoint{1.451656in}{0.831954in}}%
\pgfpathlineto{\pgfqpoint{1.453715in}{0.831954in}}%
\pgfpathlineto{\pgfqpoint{1.455774in}{0.841665in}}%
\pgfpathlineto{\pgfqpoint{1.457833in}{0.831954in}}%
\pgfpathlineto{\pgfqpoint{1.470187in}{0.890220in}}%
\pgfpathlineto{\pgfqpoint{1.472246in}{0.919353in}}%
\pgfpathlineto{\pgfqpoint{1.478423in}{0.929064in}}%
\pgfpathlineto{\pgfqpoint{1.482540in}{0.909642in}}%
\pgfpathlineto{\pgfqpoint{1.486658in}{0.938775in}}%
\pgfpathlineto{\pgfqpoint{1.492835in}{0.929064in}}%
\pgfpathlineto{\pgfqpoint{1.494894in}{0.938775in}}%
\pgfpathlineto{\pgfqpoint{1.496953in}{0.909642in}}%
\pgfpathlineto{\pgfqpoint{1.499012in}{0.909642in}}%
\pgfpathlineto{\pgfqpoint{1.501071in}{0.899931in}}%
\pgfpathlineto{\pgfqpoint{1.507248in}{0.890220in}}%
\pgfpathlineto{\pgfqpoint{1.509307in}{0.880509in}}%
\pgfpathlineto{\pgfqpoint{1.511366in}{0.890220in}}%
\pgfpathlineto{\pgfqpoint{1.515484in}{0.938775in}}%
\pgfpathlineto{\pgfqpoint{1.521660in}{0.958197in}}%
\pgfpathlineto{\pgfqpoint{1.529896in}{0.919353in}}%
\pgfpathlineto{\pgfqpoint{1.538132in}{0.958197in}}%
\pgfpathlineto{\pgfqpoint{1.542250in}{0.938775in}}%
\pgfpathlineto{\pgfqpoint{1.544309in}{0.938775in}}%
\pgfpathlineto{\pgfqpoint{1.550486in}{0.948486in}}%
\pgfpathlineto{\pgfqpoint{1.552545in}{0.958197in}}%
\pgfpathlineto{\pgfqpoint{1.554604in}{0.929064in}}%
\pgfpathlineto{\pgfqpoint{1.558721in}{0.793110in}}%
\pgfpathlineto{\pgfqpoint{1.564898in}{0.802821in}}%
\pgfpathlineto{\pgfqpoint{1.569016in}{0.783399in}}%
\pgfpathlineto{\pgfqpoint{1.571075in}{0.783399in}}%
\pgfpathlineto{\pgfqpoint{1.573134in}{0.763977in}}%
\pgfpathlineto{\pgfqpoint{1.579311in}{0.793110in}}%
\pgfpathlineto{\pgfqpoint{1.583429in}{0.773688in}}%
\pgfpathlineto{\pgfqpoint{1.585488in}{0.773688in}}%
\pgfpathlineto{\pgfqpoint{1.587547in}{0.754266in}}%
\pgfpathlineto{\pgfqpoint{1.593724in}{0.754266in}}%
\pgfpathlineto{\pgfqpoint{1.597841in}{0.773688in}}%
\pgfpathlineto{\pgfqpoint{1.599900in}{0.763977in}}%
\pgfpathlineto{\pgfqpoint{1.601959in}{0.763977in}}%
\pgfpathlineto{\pgfqpoint{1.610195in}{0.773688in}}%
\pgfpathlineto{\pgfqpoint{1.614313in}{0.773688in}}%
\pgfpathlineto{\pgfqpoint{1.616372in}{0.763977in}}%
\pgfpathlineto{\pgfqpoint{1.622549in}{0.812532in}}%
\pgfpathlineto{\pgfqpoint{1.626667in}{0.812532in}}%
\pgfpathlineto{\pgfqpoint{1.628726in}{0.802821in}}%
\pgfpathlineto{\pgfqpoint{1.630785in}{0.822243in}}%
\pgfpathlineto{\pgfqpoint{1.636961in}{0.851376in}}%
\pgfpathlineto{\pgfqpoint{1.645197in}{0.919353in}}%
\pgfpathlineto{\pgfqpoint{1.651374in}{0.958197in}}%
\pgfpathlineto{\pgfqpoint{1.653433in}{0.948486in}}%
\pgfpathlineto{\pgfqpoint{1.655492in}{0.958197in}}%
\pgfpathlineto{\pgfqpoint{1.657551in}{0.958197in}}%
\pgfpathlineto{\pgfqpoint{1.659610in}{1.006751in}}%
\pgfpathlineto{\pgfqpoint{1.665787in}{1.026173in}}%
\pgfpathlineto{\pgfqpoint{1.667846in}{1.026173in}}%
\pgfpathlineto{\pgfqpoint{1.671963in}{1.035884in}}%
\pgfpathlineto{\pgfqpoint{1.674022in}{0.997040in}}%
\pgfpathlineto{\pgfqpoint{1.682258in}{1.026173in}}%
\pgfpathlineto{\pgfqpoint{1.684317in}{0.997040in}}%
\pgfpathlineto{\pgfqpoint{1.688435in}{0.997040in}}%
\pgfpathlineto{\pgfqpoint{1.696671in}{1.045595in}}%
\pgfpathlineto{\pgfqpoint{1.698730in}{1.074728in}}%
\pgfpathlineto{\pgfqpoint{1.702848in}{1.065017in}}%
\pgfpathlineto{\pgfqpoint{1.709024in}{1.103861in}}%
\pgfpathlineto{\pgfqpoint{1.713142in}{1.103861in}}%
\pgfpathlineto{\pgfqpoint{1.723437in}{1.249526in}}%
\pgfpathlineto{\pgfqpoint{1.725496in}{1.259237in}}%
\pgfpathlineto{\pgfqpoint{1.727555in}{1.210682in}}%
\pgfpathlineto{\pgfqpoint{1.729614in}{1.230104in}}%
\pgfpathlineto{\pgfqpoint{1.731673in}{1.200971in}}%
\pgfpathlineto{\pgfqpoint{1.737850in}{1.220393in}}%
\pgfpathlineto{\pgfqpoint{1.739909in}{1.220393in}}%
\pgfpathlineto{\pgfqpoint{1.744027in}{1.162127in}}%
\pgfpathlineto{\pgfqpoint{1.746086in}{1.152416in}}%
\pgfpathlineto{\pgfqpoint{1.752262in}{1.191260in}}%
\pgfpathlineto{\pgfqpoint{1.754321in}{1.152416in}}%
\pgfpathlineto{\pgfqpoint{1.756380in}{1.152416in}}%
\pgfpathlineto{\pgfqpoint{1.758439in}{1.171838in}}%
\pgfpathlineto{\pgfqpoint{1.766675in}{1.191260in}}%
\pgfpathlineto{\pgfqpoint{1.768734in}{1.181549in}}%
\pgfpathlineto{\pgfqpoint{1.770793in}{1.152416in}}%
\pgfpathlineto{\pgfqpoint{1.772852in}{1.171838in}}%
\pgfpathlineto{\pgfqpoint{1.783147in}{1.171838in}}%
\pgfpathlineto{\pgfqpoint{1.787264in}{1.142705in}}%
\pgfpathlineto{\pgfqpoint{1.789323in}{1.132994in}}%
\pgfpathlineto{\pgfqpoint{1.795500in}{1.162127in}}%
\pgfpathlineto{\pgfqpoint{1.799618in}{1.142705in}}%
\pgfpathlineto{\pgfqpoint{1.801677in}{1.113572in}}%
\pgfpathlineto{\pgfqpoint{1.803736in}{1.055306in}}%
\pgfpathlineto{\pgfqpoint{1.811972in}{1.055306in}}%
\pgfpathlineto{\pgfqpoint{1.814031in}{1.035884in}}%
\pgfpathlineto{\pgfqpoint{1.818149in}{1.094150in}}%
\pgfpathlineto{\pgfqpoint{1.824325in}{1.103861in}}%
\pgfpathlineto{\pgfqpoint{1.826384in}{1.132994in}}%
\pgfpathlineto{\pgfqpoint{1.828443in}{1.113572in}}%
\pgfpathlineto{\pgfqpoint{1.830502in}{1.132994in}}%
\pgfpathlineto{\pgfqpoint{1.832561in}{1.113572in}}%
\pgfpathlineto{\pgfqpoint{1.838738in}{1.152416in}}%
\pgfpathlineto{\pgfqpoint{1.840797in}{1.152416in}}%
\pgfpathlineto{\pgfqpoint{1.842856in}{1.142705in}}%
\pgfpathlineto{\pgfqpoint{1.844915in}{1.113572in}}%
\pgfpathlineto{\pgfqpoint{1.846974in}{1.132994in}}%
\pgfpathlineto{\pgfqpoint{1.853151in}{1.103861in}}%
\pgfpathlineto{\pgfqpoint{1.855210in}{1.113572in}}%
\pgfpathlineto{\pgfqpoint{1.857269in}{1.103861in}}%
\pgfpathlineto{\pgfqpoint{1.859328in}{1.074728in}}%
\pgfpathlineto{\pgfqpoint{1.861386in}{1.074728in}}%
\pgfpathlineto{\pgfqpoint{1.871681in}{1.113572in}}%
\pgfpathlineto{\pgfqpoint{1.875799in}{1.142705in}}%
\pgfpathlineto{\pgfqpoint{1.881976in}{1.142705in}}%
\pgfpathlineto{\pgfqpoint{1.884035in}{1.152416in}}%
\pgfpathlineto{\pgfqpoint{1.886094in}{1.142705in}}%
\pgfpathlineto{\pgfqpoint{1.888153in}{1.142705in}}%
\pgfpathlineto{\pgfqpoint{1.890212in}{1.152416in}}%
\pgfpathlineto{\pgfqpoint{1.898447in}{1.181549in}}%
\pgfpathlineto{\pgfqpoint{1.902565in}{1.142705in}}%
\pgfpathlineto{\pgfqpoint{1.904624in}{1.152416in}}%
\pgfpathlineto{\pgfqpoint{1.910801in}{1.171838in}}%
\pgfpathlineto{\pgfqpoint{1.914919in}{1.152416in}}%
\pgfpathlineto{\pgfqpoint{1.916978in}{1.181549in}}%
\pgfpathlineto{\pgfqpoint{1.919037in}{1.191260in}}%
\pgfpathlineto{\pgfqpoint{1.925214in}{1.200971in}}%
\pgfpathlineto{\pgfqpoint{1.927273in}{1.200971in}}%
\pgfpathlineto{\pgfqpoint{1.929332in}{1.152416in}}%
\pgfpathlineto{\pgfqpoint{1.931391in}{1.152416in}}%
\pgfpathlineto{\pgfqpoint{1.933450in}{1.123283in}}%
\pgfpathlineto{\pgfqpoint{1.939626in}{1.142705in}}%
\pgfpathlineto{\pgfqpoint{1.945803in}{1.142705in}}%
\pgfpathlineto{\pgfqpoint{1.954039in}{1.171838in}}%
\pgfpathlineto{\pgfqpoint{1.960216in}{1.074728in}}%
\pgfpathlineto{\pgfqpoint{1.962275in}{1.084439in}}%
\pgfpathlineto{\pgfqpoint{1.968452in}{1.065017in}}%
\pgfpathlineto{\pgfqpoint{1.970511in}{1.045595in}}%
\pgfpathlineto{\pgfqpoint{1.974628in}{1.045595in}}%
\pgfpathlineto{\pgfqpoint{1.976687in}{1.026173in}}%
\pgfpathlineto{\pgfqpoint{1.984923in}{1.026173in}}%
\pgfpathlineto{\pgfqpoint{1.989041in}{1.055306in}}%
\pgfpathlineto{\pgfqpoint{1.991100in}{1.055306in}}%
\pgfpathlineto{\pgfqpoint{1.997277in}{1.035884in}}%
\pgfpathlineto{\pgfqpoint{1.999336in}{1.045595in}}%
\pgfpathlineto{\pgfqpoint{2.001395in}{1.045595in}}%
\pgfpathlineto{\pgfqpoint{2.005513in}{1.065017in}}%
\pgfpathlineto{\pgfqpoint{2.011689in}{1.084439in}}%
\pgfpathlineto{\pgfqpoint{2.013748in}{1.113572in}}%
\pgfpathlineto{\pgfqpoint{2.015807in}{1.084439in}}%
\pgfpathlineto{\pgfqpoint{2.017866in}{1.074728in}}%
\pgfpathlineto{\pgfqpoint{2.019925in}{1.084439in}}%
\pgfpathlineto{\pgfqpoint{2.026102in}{1.094150in}}%
\pgfpathlineto{\pgfqpoint{2.030220in}{1.074728in}}%
\pgfpathlineto{\pgfqpoint{2.034338in}{1.074728in}}%
\pgfpathlineto{\pgfqpoint{2.040515in}{1.065017in}}%
\pgfpathlineto{\pgfqpoint{2.042574in}{1.045595in}}%
\pgfpathlineto{\pgfqpoint{2.044633in}{1.055306in}}%
\pgfpathlineto{\pgfqpoint{2.046692in}{1.055306in}}%
\pgfpathlineto{\pgfqpoint{2.048751in}{1.065017in}}%
\pgfpathlineto{\pgfqpoint{2.054927in}{1.065017in}}%
\pgfpathlineto{\pgfqpoint{2.063163in}{1.142705in}}%
\pgfpathlineto{\pgfqpoint{2.069340in}{1.162127in}}%
\pgfpathlineto{\pgfqpoint{2.071399in}{1.162127in}}%
\pgfpathlineto{\pgfqpoint{2.075517in}{1.142705in}}%
\pgfpathlineto{\pgfqpoint{2.077576in}{1.152416in}}%
\pgfpathlineto{\pgfqpoint{2.085812in}{1.171838in}}%
\pgfpathlineto{\pgfqpoint{2.087870in}{1.171838in}}%
\pgfpathlineto{\pgfqpoint{2.089929in}{1.162127in}}%
\pgfpathlineto{\pgfqpoint{2.091988in}{1.113572in}}%
\pgfpathlineto{\pgfqpoint{2.104342in}{1.113572in}}%
\pgfpathlineto{\pgfqpoint{2.106401in}{1.103861in}}%
\pgfpathlineto{\pgfqpoint{2.112578in}{1.084439in}}%
\pgfpathlineto{\pgfqpoint{2.114637in}{1.094150in}}%
\pgfpathlineto{\pgfqpoint{2.116696in}{1.055306in}}%
\pgfpathlineto{\pgfqpoint{2.118755in}{1.045595in}}%
\pgfpathlineto{\pgfqpoint{2.120814in}{1.055306in}}%
\pgfpathlineto{\pgfqpoint{2.126990in}{1.094150in}}%
\pgfpathlineto{\pgfqpoint{2.129049in}{1.094150in}}%
\pgfpathlineto{\pgfqpoint{2.131108in}{1.084439in}}%
\pgfpathlineto{\pgfqpoint{2.133167in}{1.113572in}}%
\pgfpathlineto{\pgfqpoint{2.135226in}{1.065017in}}%
\pgfpathlineto{\pgfqpoint{2.141403in}{1.035884in}}%
\pgfpathlineto{\pgfqpoint{2.145521in}{1.035884in}}%
\pgfpathlineto{\pgfqpoint{2.149639in}{1.055306in}}%
\pgfpathlineto{\pgfqpoint{2.157875in}{1.035884in}}%
\pgfpathlineto{\pgfqpoint{2.161993in}{1.055306in}}%
\pgfpathlineto{\pgfqpoint{2.164051in}{1.045595in}}%
\pgfpathlineto{\pgfqpoint{2.170228in}{1.084439in}}%
\pgfpathlineto{\pgfqpoint{2.174346in}{1.084439in}}%
\pgfpathlineto{\pgfqpoint{2.178464in}{1.103861in}}%
\pgfpathlineto{\pgfqpoint{2.184641in}{1.123283in}}%
\pgfpathlineto{\pgfqpoint{2.199054in}{1.123283in}}%
\pgfpathlineto{\pgfqpoint{2.201112in}{1.113572in}}%
\pgfpathlineto{\pgfqpoint{2.203171in}{1.084439in}}%
\pgfpathlineto{\pgfqpoint{2.207289in}{1.065017in}}%
\pgfpathlineto{\pgfqpoint{2.215525in}{1.094150in}}%
\pgfpathlineto{\pgfqpoint{2.217584in}{1.103861in}}%
\pgfpathlineto{\pgfqpoint{2.219643in}{1.094150in}}%
\pgfpathlineto{\pgfqpoint{2.221702in}{1.132994in}}%
\pgfpathlineto{\pgfqpoint{2.227879in}{1.132994in}}%
\pgfpathlineto{\pgfqpoint{2.231997in}{1.113572in}}%
\pgfpathlineto{\pgfqpoint{2.234056in}{1.132994in}}%
\pgfpathlineto{\pgfqpoint{2.236115in}{1.113572in}}%
\pgfpathlineto{\pgfqpoint{2.242291in}{1.142705in}}%
\pgfpathlineto{\pgfqpoint{2.244350in}{1.132994in}}%
\pgfpathlineto{\pgfqpoint{2.246409in}{1.142705in}}%
\pgfpathlineto{\pgfqpoint{2.248468in}{1.123283in}}%
\pgfpathlineto{\pgfqpoint{2.250527in}{1.123283in}}%
\pgfpathlineto{\pgfqpoint{2.256704in}{1.132994in}}%
\pgfpathlineto{\pgfqpoint{2.258763in}{1.132994in}}%
\pgfpathlineto{\pgfqpoint{2.260822in}{1.142705in}}%
\pgfpathlineto{\pgfqpoint{2.262881in}{1.142705in}}%
\pgfpathlineto{\pgfqpoint{2.264940in}{1.152416in}}%
\pgfpathlineto{\pgfqpoint{2.271117in}{1.171838in}}%
\pgfpathlineto{\pgfqpoint{2.273176in}{1.152416in}}%
\pgfpathlineto{\pgfqpoint{2.277293in}{1.152416in}}%
\pgfpathlineto{\pgfqpoint{2.279352in}{1.132994in}}%
\pgfpathlineto{\pgfqpoint{2.287588in}{1.132994in}}%
\pgfpathlineto{\pgfqpoint{2.289647in}{1.171838in}}%
\pgfpathlineto{\pgfqpoint{2.293765in}{1.191260in}}%
\pgfpathlineto{\pgfqpoint{2.302001in}{1.220393in}}%
\pgfpathlineto{\pgfqpoint{2.306119in}{1.171838in}}%
\pgfpathlineto{\pgfqpoint{2.308178in}{1.181549in}}%
\pgfpathlineto{\pgfqpoint{2.314355in}{1.162127in}}%
\pgfpathlineto{\pgfqpoint{2.316413in}{1.171838in}}%
\pgfpathlineto{\pgfqpoint{2.320531in}{1.074728in}}%
\pgfpathlineto{\pgfqpoint{2.322590in}{1.084439in}}%
\pgfpathlineto{\pgfqpoint{2.328767in}{1.103861in}}%
\pgfpathlineto{\pgfqpoint{2.330826in}{1.103861in}}%
\pgfpathlineto{\pgfqpoint{2.332885in}{1.123283in}}%
\pgfpathlineto{\pgfqpoint{2.334944in}{1.113572in}}%
\pgfpathlineto{\pgfqpoint{2.337003in}{1.132994in}}%
\pgfpathlineto{\pgfqpoint{2.343180in}{1.162127in}}%
\pgfpathlineto{\pgfqpoint{2.347298in}{1.162127in}}%
\pgfpathlineto{\pgfqpoint{2.349357in}{1.142705in}}%
\pgfpathlineto{\pgfqpoint{2.351416in}{1.142705in}}%
\pgfpathlineto{\pgfqpoint{2.359651in}{1.162127in}}%
\pgfpathlineto{\pgfqpoint{2.361710in}{1.171838in}}%
\pgfpathlineto{\pgfqpoint{2.363769in}{1.132994in}}%
\pgfpathlineto{\pgfqpoint{2.365828in}{1.142705in}}%
\pgfpathlineto{\pgfqpoint{2.372005in}{1.152416in}}%
\pgfpathlineto{\pgfqpoint{2.374064in}{1.152416in}}%
\pgfpathlineto{\pgfqpoint{2.376123in}{1.162127in}}%
\pgfpathlineto{\pgfqpoint{2.378182in}{1.162127in}}%
\pgfpathlineto{\pgfqpoint{2.380241in}{1.152416in}}%
\pgfpathlineto{\pgfqpoint{2.386418in}{1.142705in}}%
\pgfpathlineto{\pgfqpoint{2.388477in}{1.171838in}}%
\pgfpathlineto{\pgfqpoint{2.394653in}{1.171838in}}%
\pgfpathlineto{\pgfqpoint{2.400830in}{1.191260in}}%
\pgfpathlineto{\pgfqpoint{2.402889in}{1.181549in}}%
\pgfpathlineto{\pgfqpoint{2.407007in}{1.200971in}}%
\pgfpathlineto{\pgfqpoint{2.409066in}{1.200971in}}%
\pgfpathlineto{\pgfqpoint{2.415243in}{1.220393in}}%
\pgfpathlineto{\pgfqpoint{2.417302in}{1.239815in}}%
\pgfpathlineto{\pgfqpoint{2.419361in}{1.239815in}}%
\pgfpathlineto{\pgfqpoint{2.421420in}{1.268948in}}%
\pgfpathlineto{\pgfqpoint{2.429655in}{1.327214in}}%
\pgfpathlineto{\pgfqpoint{2.431714in}{1.288370in}}%
\pgfpathlineto{\pgfqpoint{2.433773in}{1.298081in}}%
\pgfpathlineto{\pgfqpoint{2.437891in}{1.278659in}}%
\pgfpathlineto{\pgfqpoint{2.444068in}{1.278659in}}%
\pgfpathlineto{\pgfqpoint{2.446127in}{1.288370in}}%
\pgfpathlineto{\pgfqpoint{2.448186in}{1.307792in}}%
\pgfpathlineto{\pgfqpoint{2.452304in}{1.298081in}}%
\pgfpathlineto{\pgfqpoint{2.458481in}{1.278659in}}%
\pgfpathlineto{\pgfqpoint{2.460540in}{1.278659in}}%
\pgfpathlineto{\pgfqpoint{2.462599in}{1.298081in}}%
\pgfpathlineto{\pgfqpoint{2.464658in}{1.278659in}}%
\pgfpathlineto{\pgfqpoint{2.466716in}{1.288370in}}%
\pgfpathlineto{\pgfqpoint{2.472893in}{1.307792in}}%
\pgfpathlineto{\pgfqpoint{2.477011in}{1.307792in}}%
\pgfpathlineto{\pgfqpoint{2.479070in}{1.298081in}}%
\pgfpathlineto{\pgfqpoint{2.481129in}{1.317503in}}%
\pgfpathlineto{\pgfqpoint{2.487306in}{1.317503in}}%
\pgfpathlineto{\pgfqpoint{2.489365in}{1.336925in}}%
\pgfpathlineto{\pgfqpoint{2.491424in}{1.336925in}}%
\pgfpathlineto{\pgfqpoint{2.493483in}{1.327214in}}%
\pgfpathlineto{\pgfqpoint{2.501719in}{1.327214in}}%
\pgfpathlineto{\pgfqpoint{2.503778in}{1.336925in}}%
\pgfpathlineto{\pgfqpoint{2.505836in}{1.327214in}}%
\pgfpathlineto{\pgfqpoint{2.509954in}{1.327214in}}%
\pgfpathlineto{\pgfqpoint{2.518190in}{1.336925in}}%
\pgfpathlineto{\pgfqpoint{2.520249in}{1.298081in}}%
\pgfpathlineto{\pgfqpoint{2.524367in}{1.298081in}}%
\pgfpathlineto{\pgfqpoint{2.532603in}{1.327214in}}%
\pgfpathlineto{\pgfqpoint{2.536721in}{1.298081in}}%
\pgfpathlineto{\pgfqpoint{2.538780in}{1.288370in}}%
\pgfpathlineto{\pgfqpoint{2.544956in}{1.278659in}}%
\pgfpathlineto{\pgfqpoint{2.549074in}{1.278659in}}%
\pgfpathlineto{\pgfqpoint{2.551133in}{1.268948in}}%
\pgfpathlineto{\pgfqpoint{2.553192in}{1.288370in}}%
\pgfpathlineto{\pgfqpoint{2.561428in}{1.298081in}}%
\pgfpathlineto{\pgfqpoint{2.563487in}{1.307792in}}%
\pgfpathlineto{\pgfqpoint{2.565546in}{1.298081in}}%
\pgfpathlineto{\pgfqpoint{2.567605in}{1.298081in}}%
\pgfpathlineto{\pgfqpoint{2.573782in}{1.268948in}}%
\pgfpathlineto{\pgfqpoint{2.575841in}{1.298081in}}%
\pgfpathlineto{\pgfqpoint{2.577900in}{1.288370in}}%
\pgfpathlineto{\pgfqpoint{2.582017in}{1.307792in}}%
\pgfpathlineto{\pgfqpoint{2.588194in}{1.307792in}}%
\pgfpathlineto{\pgfqpoint{2.592312in}{1.327214in}}%
\pgfpathlineto{\pgfqpoint{2.596430in}{1.307792in}}%
\pgfpathlineto{\pgfqpoint{2.602607in}{1.298081in}}%
\pgfpathlineto{\pgfqpoint{2.604666in}{1.307792in}}%
\pgfpathlineto{\pgfqpoint{2.606725in}{1.307792in}}%
\pgfpathlineto{\pgfqpoint{2.608784in}{1.317503in}}%
\pgfpathlineto{\pgfqpoint{2.610843in}{1.317503in}}%
\pgfpathlineto{\pgfqpoint{2.617020in}{1.307792in}}%
\pgfpathlineto{\pgfqpoint{2.619078in}{1.336925in}}%
\pgfpathlineto{\pgfqpoint{2.621137in}{1.346636in}}%
\pgfpathlineto{\pgfqpoint{2.623196in}{1.336925in}}%
\pgfpathlineto{\pgfqpoint{2.625255in}{1.336925in}}%
\pgfpathlineto{\pgfqpoint{2.633491in}{1.366058in}}%
\pgfpathlineto{\pgfqpoint{2.635550in}{1.356347in}}%
\pgfpathlineto{\pgfqpoint{2.639668in}{1.327214in}}%
\pgfpathlineto{\pgfqpoint{2.647904in}{1.366058in}}%
\pgfpathlineto{\pgfqpoint{2.652022in}{1.511723in}}%
\pgfpathlineto{\pgfqpoint{2.654081in}{1.511723in}}%
\pgfpathlineto{\pgfqpoint{2.660257in}{1.502012in}}%
\pgfpathlineto{\pgfqpoint{2.662316in}{1.540855in}}%
\pgfpathlineto{\pgfqpoint{2.664375in}{1.531145in}}%
\pgfpathlineto{\pgfqpoint{2.668493in}{1.560277in}}%
\pgfpathlineto{\pgfqpoint{2.674670in}{1.599121in}}%
\pgfpathlineto{\pgfqpoint{2.676729in}{1.599121in}}%
\pgfpathlineto{\pgfqpoint{2.678788in}{1.560277in}}%
\pgfpathlineto{\pgfqpoint{2.680847in}{1.560277in}}%
\pgfpathlineto{\pgfqpoint{2.682906in}{1.540855in}}%
\pgfpathlineto{\pgfqpoint{2.689083in}{1.560277in}}%
\pgfpathlineto{\pgfqpoint{2.691142in}{1.579699in}}%
\pgfpathlineto{\pgfqpoint{2.693200in}{1.569988in}}%
\pgfpathlineto{\pgfqpoint{2.695259in}{1.569988in}}%
\pgfpathlineto{\pgfqpoint{2.697318in}{1.560277in}}%
\pgfpathlineto{\pgfqpoint{2.707613in}{1.589410in}}%
\pgfpathlineto{\pgfqpoint{2.709672in}{1.579699in}}%
\pgfpathlineto{\pgfqpoint{2.711731in}{1.579699in}}%
\pgfpathlineto{\pgfqpoint{2.717908in}{1.589410in}}%
\pgfpathlineto{\pgfqpoint{2.719967in}{1.589410in}}%
\pgfpathlineto{\pgfqpoint{2.726144in}{1.618543in}}%
\pgfpathlineto{\pgfqpoint{2.732320in}{1.637965in}}%
\pgfpathlineto{\pgfqpoint{2.734379in}{1.608832in}}%
\pgfpathlineto{\pgfqpoint{2.736438in}{1.618543in}}%
\pgfpathlineto{\pgfqpoint{2.738497in}{1.608832in}}%
\pgfpathlineto{\pgfqpoint{2.750851in}{1.608832in}}%
\pgfpathlineto{\pgfqpoint{2.754969in}{1.589410in}}%
\pgfpathlineto{\pgfqpoint{2.765264in}{1.657387in}}%
\pgfpathlineto{\pgfqpoint{2.767323in}{1.647676in}}%
\pgfpathlineto{\pgfqpoint{2.769381in}{1.657387in}}%
\pgfpathlineto{\pgfqpoint{2.775558in}{1.647676in}}%
\pgfpathlineto{\pgfqpoint{2.779676in}{1.667098in}}%
\pgfpathlineto{\pgfqpoint{2.781735in}{1.667098in}}%
\pgfpathlineto{\pgfqpoint{2.783794in}{1.676809in}}%
\pgfpathlineto{\pgfqpoint{2.789971in}{1.686520in}}%
\pgfpathlineto{\pgfqpoint{2.792030in}{1.705942in}}%
\pgfpathlineto{\pgfqpoint{2.796148in}{1.705942in}}%
\pgfpathlineto{\pgfqpoint{2.798207in}{1.696231in}}%
\pgfpathlineto{\pgfqpoint{2.804384in}{1.686520in}}%
\pgfpathlineto{\pgfqpoint{2.806443in}{1.705942in}}%
\pgfpathlineto{\pgfqpoint{2.808501in}{1.667098in}}%
\pgfpathlineto{\pgfqpoint{2.812619in}{1.696231in}}%
\pgfpathlineto{\pgfqpoint{2.818796in}{1.715653in}}%
\pgfpathlineto{\pgfqpoint{2.820855in}{1.744786in}}%
\pgfpathlineto{\pgfqpoint{2.822914in}{1.735075in}}%
\pgfpathlineto{\pgfqpoint{2.824973in}{1.744786in}}%
\pgfpathlineto{\pgfqpoint{2.827032in}{1.744786in}}%
\pgfpathlineto{\pgfqpoint{2.835268in}{1.735075in}}%
\pgfpathlineto{\pgfqpoint{2.837327in}{1.744786in}}%
\pgfpathlineto{\pgfqpoint{2.841445in}{1.725364in}}%
\pgfpathlineto{\pgfqpoint{2.847621in}{1.725364in}}%
\pgfpathlineto{\pgfqpoint{2.855857in}{1.793341in}}%
\pgfpathlineto{\pgfqpoint{2.862034in}{1.754497in}}%
\pgfpathlineto{\pgfqpoint{2.864093in}{1.783630in}}%
\pgfpathlineto{\pgfqpoint{2.866152in}{1.783630in}}%
\pgfpathlineto{\pgfqpoint{2.868211in}{1.793341in}}%
\pgfpathlineto{\pgfqpoint{2.876447in}{1.793341in}}%
\pgfpathlineto{\pgfqpoint{2.878506in}{1.803052in}}%
\pgfpathlineto{\pgfqpoint{2.882623in}{1.764208in}}%
\pgfpathlineto{\pgfqpoint{2.890859in}{1.764208in}}%
\pgfpathlineto{\pgfqpoint{2.892918in}{1.793341in}}%
\pgfpathlineto{\pgfqpoint{2.894977in}{1.783630in}}%
\pgfpathlineto{\pgfqpoint{2.897036in}{1.803052in}}%
\pgfpathlineto{\pgfqpoint{2.899095in}{1.803052in}}%
\pgfpathlineto{\pgfqpoint{2.905272in}{1.793341in}}%
\pgfpathlineto{\pgfqpoint{2.909390in}{1.812763in}}%
\pgfpathlineto{\pgfqpoint{2.911449in}{1.803052in}}%
\pgfpathlineto{\pgfqpoint{2.913508in}{1.803052in}}%
\pgfpathlineto{\pgfqpoint{2.919685in}{1.793341in}}%
\pgfpathlineto{\pgfqpoint{2.921743in}{1.803052in}}%
\pgfpathlineto{\pgfqpoint{2.923802in}{1.793341in}}%
\pgfpathlineto{\pgfqpoint{2.925861in}{1.803052in}}%
\pgfpathlineto{\pgfqpoint{2.927920in}{1.783630in}}%
\pgfpathlineto{\pgfqpoint{2.934097in}{1.764208in}}%
\pgfpathlineto{\pgfqpoint{2.938215in}{1.783630in}}%
\pgfpathlineto{\pgfqpoint{2.940274in}{1.783630in}}%
\pgfpathlineto{\pgfqpoint{2.942333in}{1.764208in}}%
\pgfpathlineto{\pgfqpoint{2.948510in}{1.783630in}}%
\pgfpathlineto{\pgfqpoint{2.950569in}{1.812763in}}%
\pgfpathlineto{\pgfqpoint{2.954687in}{1.793341in}}%
\pgfpathlineto{\pgfqpoint{2.962922in}{1.793341in}}%
\pgfpathlineto{\pgfqpoint{2.964981in}{1.812763in}}%
\pgfpathlineto{\pgfqpoint{2.967040in}{1.812763in}}%
\pgfpathlineto{\pgfqpoint{2.969099in}{1.793341in}}%
\pgfpathlineto{\pgfqpoint{2.971158in}{1.803052in}}%
\pgfpathlineto{\pgfqpoint{2.977335in}{1.803052in}}%
\pgfpathlineto{\pgfqpoint{2.979394in}{1.822474in}}%
\pgfpathlineto{\pgfqpoint{2.983512in}{1.803052in}}%
\pgfpathlineto{\pgfqpoint{2.985571in}{1.803052in}}%
\pgfpathlineto{\pgfqpoint{2.991748in}{1.793341in}}%
\pgfpathlineto{\pgfqpoint{2.993807in}{1.822474in}}%
\pgfpathlineto{\pgfqpoint{2.997924in}{1.773919in}}%
\pgfpathlineto{\pgfqpoint{2.999983in}{1.793341in}}%
\pgfpathlineto{\pgfqpoint{3.006160in}{1.773919in}}%
\pgfpathlineto{\pgfqpoint{3.008219in}{1.793341in}}%
\pgfpathlineto{\pgfqpoint{3.010278in}{1.773919in}}%
\pgfpathlineto{\pgfqpoint{3.014396in}{1.773919in}}%
\pgfpathlineto{\pgfqpoint{3.020573in}{1.783630in}}%
\pgfpathlineto{\pgfqpoint{3.022632in}{1.793341in}}%
\pgfpathlineto{\pgfqpoint{3.026750in}{1.744786in}}%
\pgfpathlineto{\pgfqpoint{3.028809in}{1.764208in}}%
\pgfpathlineto{\pgfqpoint{3.037044in}{1.793341in}}%
\pgfpathlineto{\pgfqpoint{3.039103in}{1.832185in}}%
\pgfpathlineto{\pgfqpoint{3.043221in}{1.803052in}}%
\pgfpathlineto{\pgfqpoint{3.049398in}{1.822474in}}%
\pgfpathlineto{\pgfqpoint{3.053516in}{1.822474in}}%
\pgfpathlineto{\pgfqpoint{3.055575in}{1.832185in}}%
\pgfpathlineto{\pgfqpoint{3.057634in}{1.832185in}}%
\pgfpathlineto{\pgfqpoint{3.063811in}{1.841896in}}%
\pgfpathlineto{\pgfqpoint{3.067929in}{1.861318in}}%
\pgfpathlineto{\pgfqpoint{3.069988in}{1.851607in}}%
\pgfpathlineto{\pgfqpoint{3.080282in}{1.851607in}}%
\pgfpathlineto{\pgfqpoint{3.082341in}{1.861318in}}%
\pgfpathlineto{\pgfqpoint{3.084400in}{1.841896in}}%
\pgfpathlineto{\pgfqpoint{3.086459in}{1.861318in}}%
\pgfpathlineto{\pgfqpoint{3.092636in}{1.880740in}}%
\pgfpathlineto{\pgfqpoint{3.094695in}{1.871029in}}%
\pgfpathlineto{\pgfqpoint{3.098813in}{1.871029in}}%
\pgfpathlineto{\pgfqpoint{3.109108in}{1.919584in}}%
\pgfpathlineto{\pgfqpoint{3.111166in}{1.909873in}}%
\pgfpathlineto{\pgfqpoint{3.113225in}{1.929295in}}%
\pgfpathlineto{\pgfqpoint{3.115284in}{1.919584in}}%
\pgfpathlineto{\pgfqpoint{3.121461in}{1.900162in}}%
\pgfpathlineto{\pgfqpoint{3.123520in}{1.909873in}}%
\pgfpathlineto{\pgfqpoint{3.125579in}{1.900162in}}%
\pgfpathlineto{\pgfqpoint{3.127638in}{1.909873in}}%
\pgfpathlineto{\pgfqpoint{3.129697in}{1.929295in}}%
\pgfpathlineto{\pgfqpoint{3.135874in}{1.909873in}}%
\pgfpathlineto{\pgfqpoint{3.137933in}{1.929295in}}%
\pgfpathlineto{\pgfqpoint{3.139992in}{1.929295in}}%
\pgfpathlineto{\pgfqpoint{3.142051in}{1.948717in}}%
\pgfpathlineto{\pgfqpoint{3.144110in}{1.939006in}}%
\pgfpathlineto{\pgfqpoint{3.150286in}{1.900162in}}%
\pgfpathlineto{\pgfqpoint{3.154404in}{1.958428in}}%
\pgfpathlineto{\pgfqpoint{3.156463in}{1.948717in}}%
\pgfpathlineto{\pgfqpoint{3.158522in}{1.968139in}}%
\pgfpathlineto{\pgfqpoint{3.164699in}{1.958428in}}%
\pgfpathlineto{\pgfqpoint{3.168817in}{2.006983in}}%
\pgfpathlineto{\pgfqpoint{3.172935in}{2.026405in}}%
\pgfpathlineto{\pgfqpoint{3.179112in}{2.026405in}}%
\pgfpathlineto{\pgfqpoint{3.181171in}{2.055538in}}%
\pgfpathlineto{\pgfqpoint{3.183230in}{2.045827in}}%
\pgfpathlineto{\pgfqpoint{3.185289in}{2.074960in}}%
\pgfpathlineto{\pgfqpoint{3.187347in}{2.074960in}}%
\pgfpathlineto{\pgfqpoint{3.193524in}{2.113803in}}%
\pgfpathlineto{\pgfqpoint{3.195583in}{2.104092in}}%
\pgfpathlineto{\pgfqpoint{3.201760in}{2.104092in}}%
\pgfpathlineto{\pgfqpoint{3.207937in}{2.065249in}}%
\pgfpathlineto{\pgfqpoint{3.209996in}{2.113803in}}%
\pgfpathlineto{\pgfqpoint{3.214114in}{2.094382in}}%
\pgfpathlineto{\pgfqpoint{3.216173in}{2.104092in}}%
\pgfpathlineto{\pgfqpoint{3.222350in}{2.104092in}}%
\pgfpathlineto{\pgfqpoint{3.224408in}{2.133225in}}%
\pgfpathlineto{\pgfqpoint{3.226467in}{2.133225in}}%
\pgfpathlineto{\pgfqpoint{3.228526in}{2.123514in}}%
\pgfpathlineto{\pgfqpoint{3.230585in}{2.104092in}}%
\pgfpathlineto{\pgfqpoint{3.236762in}{2.123514in}}%
\pgfpathlineto{\pgfqpoint{3.238821in}{2.142936in}}%
\pgfpathlineto{\pgfqpoint{3.240880in}{2.123514in}}%
\pgfpathlineto{\pgfqpoint{3.242939in}{2.133225in}}%
\pgfpathlineto{\pgfqpoint{3.244998in}{2.133225in}}%
\pgfpathlineto{\pgfqpoint{3.251175in}{2.162358in}}%
\pgfpathlineto{\pgfqpoint{3.255293in}{2.162358in}}%
\pgfpathlineto{\pgfqpoint{3.257352in}{2.191491in}}%
\pgfpathlineto{\pgfqpoint{3.259411in}{2.191491in}}%
\pgfpathlineto{\pgfqpoint{3.267646in}{2.172069in}}%
\pgfpathlineto{\pgfqpoint{3.271764in}{2.191491in}}%
\pgfpathlineto{\pgfqpoint{3.273823in}{2.181780in}}%
\pgfpathlineto{\pgfqpoint{3.282059in}{2.259468in}}%
\pgfpathlineto{\pgfqpoint{3.284118in}{2.240046in}}%
\pgfpathlineto{\pgfqpoint{3.286177in}{2.249757in}}%
\pgfpathlineto{\pgfqpoint{3.288236in}{2.230335in}}%
\pgfpathlineto{\pgfqpoint{3.294413in}{2.249757in}}%
\pgfpathlineto{\pgfqpoint{3.296472in}{2.249757in}}%
\pgfpathlineto{\pgfqpoint{3.300589in}{2.230335in}}%
\pgfpathlineto{\pgfqpoint{3.310884in}{2.278890in}}%
\pgfpathlineto{\pgfqpoint{3.315002in}{2.278890in}}%
\pgfpathlineto{\pgfqpoint{3.317061in}{2.269179in}}%
\pgfpathlineto{\pgfqpoint{3.323238in}{2.298312in}}%
\pgfpathlineto{\pgfqpoint{3.325297in}{2.278890in}}%
\pgfpathlineto{\pgfqpoint{3.327356in}{2.278890in}}%
\pgfpathlineto{\pgfqpoint{3.329415in}{2.288601in}}%
\pgfpathlineto{\pgfqpoint{3.331474in}{2.288601in}}%
\pgfpathlineto{\pgfqpoint{3.337650in}{2.308023in}}%
\pgfpathlineto{\pgfqpoint{3.341768in}{2.308023in}}%
\pgfpathlineto{\pgfqpoint{3.343827in}{2.288601in}}%
\pgfpathlineto{\pgfqpoint{3.345886in}{2.298312in}}%
\pgfpathlineto{\pgfqpoint{3.352063in}{2.317734in}}%
\pgfpathlineto{\pgfqpoint{3.354122in}{2.337156in}}%
\pgfpathlineto{\pgfqpoint{3.356181in}{2.376000in}}%
\pgfpathlineto{\pgfqpoint{3.360299in}{2.376000in}}%
\pgfpathlineto{\pgfqpoint{3.366476in}{2.463399in}}%
\pgfpathlineto{\pgfqpoint{3.368535in}{2.443977in}}%
\pgfpathlineto{\pgfqpoint{3.382947in}{2.511954in}}%
\pgfpathlineto{\pgfqpoint{3.387065in}{2.482821in}}%
\pgfpathlineto{\pgfqpoint{3.389124in}{2.492532in}}%
\pgfpathlineto{\pgfqpoint{3.395301in}{2.511954in}}%
\pgfpathlineto{\pgfqpoint{3.397360in}{2.511954in}}%
\pgfpathlineto{\pgfqpoint{3.401478in}{2.492532in}}%
\pgfpathlineto{\pgfqpoint{3.403537in}{2.502243in}}%
\pgfpathlineto{\pgfqpoint{3.409714in}{2.502243in}}%
\pgfpathlineto{\pgfqpoint{3.411773in}{2.511954in}}%
\pgfpathlineto{\pgfqpoint{3.413831in}{2.511954in}}%
\pgfpathlineto{\pgfqpoint{3.415890in}{2.531376in}}%
\pgfpathlineto{\pgfqpoint{3.424126in}{2.531376in}}%
\pgfpathlineto{\pgfqpoint{3.426185in}{2.541087in}}%
\pgfpathlineto{\pgfqpoint{3.428244in}{2.579931in}}%
\pgfpathlineto{\pgfqpoint{3.438539in}{2.628486in}}%
\pgfpathlineto{\pgfqpoint{3.442657in}{2.589642in}}%
\pgfpathlineto{\pgfqpoint{3.444716in}{2.589642in}}%
\pgfpathlineto{\pgfqpoint{3.446775in}{2.560509in}}%
\pgfpathlineto{\pgfqpoint{3.452951in}{2.579931in}}%
\pgfpathlineto{\pgfqpoint{3.455010in}{2.570220in}}%
\pgfpathlineto{\pgfqpoint{3.457069in}{2.589642in}}%
\pgfpathlineto{\pgfqpoint{3.459128in}{2.570220in}}%
\pgfpathlineto{\pgfqpoint{3.467364in}{2.560509in}}%
\pgfpathlineto{\pgfqpoint{3.469423in}{2.560509in}}%
\pgfpathlineto{\pgfqpoint{3.471482in}{2.541087in}}%
\pgfpathlineto{\pgfqpoint{3.473541in}{2.570220in}}%
\pgfpathlineto{\pgfqpoint{3.475600in}{2.550798in}}%
\pgfpathlineto{\pgfqpoint{3.481777in}{2.570220in}}%
\pgfpathlineto{\pgfqpoint{3.483836in}{2.570220in}}%
\pgfpathlineto{\pgfqpoint{3.485895in}{2.589642in}}%
\pgfpathlineto{\pgfqpoint{3.487954in}{2.589642in}}%
\pgfpathlineto{\pgfqpoint{3.490012in}{2.609064in}}%
\pgfpathlineto{\pgfqpoint{3.496189in}{2.618775in}}%
\pgfpathlineto{\pgfqpoint{3.498248in}{2.657618in}}%
\pgfpathlineto{\pgfqpoint{3.500307in}{2.647908in}}%
\pgfpathlineto{\pgfqpoint{3.504425in}{2.647908in}}%
\pgfpathlineto{\pgfqpoint{3.512661in}{2.686751in}}%
\pgfpathlineto{\pgfqpoint{3.516779in}{2.657618in}}%
\pgfpathlineto{\pgfqpoint{3.518838in}{2.657618in}}%
\pgfpathlineto{\pgfqpoint{3.527073in}{2.686751in}}%
\pgfpathlineto{\pgfqpoint{3.531191in}{2.657618in}}%
\pgfpathlineto{\pgfqpoint{3.533250in}{2.667329in}}%
\pgfpathlineto{\pgfqpoint{3.539427in}{2.686751in}}%
\pgfpathlineto{\pgfqpoint{3.545604in}{2.686751in}}%
\pgfpathlineto{\pgfqpoint{3.553840in}{2.725595in}}%
\pgfpathlineto{\pgfqpoint{3.557958in}{2.725595in}}%
\pgfpathlineto{\pgfqpoint{3.560017in}{2.735306in}}%
\pgfpathlineto{\pgfqpoint{3.562076in}{2.725595in}}%
\pgfpathlineto{\pgfqpoint{3.568252in}{2.774150in}}%
\pgfpathlineto{\pgfqpoint{3.570311in}{2.764439in}}%
\pgfpathlineto{\pgfqpoint{3.576488in}{2.706173in}}%
\pgfpathlineto{\pgfqpoint{3.584724in}{2.696462in}}%
\pgfpathlineto{\pgfqpoint{3.586783in}{2.715884in}}%
\pgfpathlineto{\pgfqpoint{3.588842in}{2.715884in}}%
\pgfpathlineto{\pgfqpoint{3.590901in}{2.735306in}}%
\pgfpathlineto{\pgfqpoint{3.597078in}{2.764439in}}%
\pgfpathlineto{\pgfqpoint{3.601196in}{2.764439in}}%
\pgfpathlineto{\pgfqpoint{3.603254in}{2.754728in}}%
\pgfpathlineto{\pgfqpoint{3.605313in}{2.754728in}}%
\pgfpathlineto{\pgfqpoint{3.611490in}{2.745017in}}%
\pgfpathlineto{\pgfqpoint{3.615608in}{2.725595in}}%
\pgfpathlineto{\pgfqpoint{3.617667in}{2.706173in}}%
\pgfpathlineto{\pgfqpoint{3.619726in}{2.706173in}}%
\pgfpathlineto{\pgfqpoint{3.625903in}{2.764439in}}%
\pgfpathlineto{\pgfqpoint{3.627962in}{2.764439in}}%
\pgfpathlineto{\pgfqpoint{3.630021in}{2.774150in}}%
\pgfpathlineto{\pgfqpoint{3.634139in}{2.745017in}}%
\pgfpathlineto{\pgfqpoint{3.642374in}{2.774150in}}%
\pgfpathlineto{\pgfqpoint{3.644433in}{2.735306in}}%
\pgfpathlineto{\pgfqpoint{3.646492in}{2.745017in}}%
\pgfpathlineto{\pgfqpoint{3.648551in}{2.745017in}}%
\pgfpathlineto{\pgfqpoint{3.654728in}{2.774150in}}%
\pgfpathlineto{\pgfqpoint{3.656787in}{2.754728in}}%
\pgfpathlineto{\pgfqpoint{3.660905in}{2.745017in}}%
\pgfpathlineto{\pgfqpoint{3.662964in}{2.764439in}}%
\pgfpathlineto{\pgfqpoint{3.669141in}{2.783861in}}%
\pgfpathlineto{\pgfqpoint{3.671200in}{2.783861in}}%
\pgfpathlineto{\pgfqpoint{3.673259in}{2.774150in}}%
\pgfpathlineto{\pgfqpoint{3.675318in}{2.803283in}}%
\pgfpathlineto{\pgfqpoint{3.677377in}{2.793572in}}%
\pgfpathlineto{\pgfqpoint{3.683553in}{2.822705in}}%
\pgfpathlineto{\pgfqpoint{3.685612in}{2.822705in}}%
\pgfpathlineto{\pgfqpoint{3.689730in}{2.793572in}}%
\pgfpathlineto{\pgfqpoint{3.691789in}{2.793572in}}%
\pgfpathlineto{\pgfqpoint{3.697966in}{2.822705in}}%
\pgfpathlineto{\pgfqpoint{3.700025in}{2.822705in}}%
\pgfpathlineto{\pgfqpoint{3.702084in}{2.832416in}}%
\pgfpathlineto{\pgfqpoint{3.704143in}{2.822705in}}%
\pgfpathlineto{\pgfqpoint{3.706202in}{2.832416in}}%
\pgfpathlineto{\pgfqpoint{3.712379in}{2.842127in}}%
\pgfpathlineto{\pgfqpoint{3.714438in}{2.842127in}}%
\pgfpathlineto{\pgfqpoint{3.716496in}{2.851838in}}%
\pgfpathlineto{\pgfqpoint{3.718555in}{2.851838in}}%
\pgfpathlineto{\pgfqpoint{3.720614in}{2.861549in}}%
\pgfpathlineto{\pgfqpoint{3.728850in}{2.861549in}}%
\pgfpathlineto{\pgfqpoint{3.732968in}{2.880971in}}%
\pgfpathlineto{\pgfqpoint{3.735027in}{2.861549in}}%
\pgfpathlineto{\pgfqpoint{3.741204in}{2.851838in}}%
\pgfpathlineto{\pgfqpoint{3.743263in}{2.880971in}}%
\pgfpathlineto{\pgfqpoint{3.745322in}{2.861549in}}%
\pgfpathlineto{\pgfqpoint{3.747381in}{2.871260in}}%
\pgfpathlineto{\pgfqpoint{3.749440in}{2.871260in}}%
\pgfpathlineto{\pgfqpoint{3.755616in}{2.880971in}}%
\pgfpathlineto{\pgfqpoint{3.757675in}{2.880971in}}%
\pgfpathlineto{\pgfqpoint{3.761793in}{2.861549in}}%
\pgfpathlineto{\pgfqpoint{3.763852in}{2.880971in}}%
\pgfpathlineto{\pgfqpoint{3.770029in}{2.880971in}}%
\pgfpathlineto{\pgfqpoint{3.772088in}{2.910104in}}%
\pgfpathlineto{\pgfqpoint{3.778265in}{2.910104in}}%
\pgfpathlineto{\pgfqpoint{3.786501in}{2.919815in}}%
\pgfpathlineto{\pgfqpoint{3.788560in}{2.929526in}}%
\pgfpathlineto{\pgfqpoint{3.792677in}{2.929526in}}%
\pgfpathlineto{\pgfqpoint{3.798854in}{2.948948in}}%
\pgfpathlineto{\pgfqpoint{3.800913in}{2.939237in}}%
\pgfpathlineto{\pgfqpoint{3.802972in}{2.958659in}}%
\pgfpathlineto{\pgfqpoint{3.807090in}{2.958659in}}%
\pgfpathlineto{\pgfqpoint{3.821503in}{3.007214in}}%
\pgfpathlineto{\pgfqpoint{3.829738in}{3.007214in}}%
\pgfpathlineto{\pgfqpoint{3.831797in}{2.997503in}}%
\pgfpathlineto{\pgfqpoint{3.833856in}{2.997503in}}%
\pgfpathlineto{\pgfqpoint{3.835915in}{2.987792in}}%
\pgfpathlineto{\pgfqpoint{3.844151in}{3.036347in}}%
\pgfpathlineto{\pgfqpoint{3.846210in}{3.036347in}}%
\pgfpathlineto{\pgfqpoint{3.848269in}{3.046058in}}%
\pgfpathlineto{\pgfqpoint{3.850328in}{3.036347in}}%
\pgfpathlineto{\pgfqpoint{3.858564in}{3.084902in}}%
\pgfpathlineto{\pgfqpoint{3.862682in}{3.065480in}}%
\pgfpathlineto{\pgfqpoint{3.864741in}{3.065480in}}%
\pgfpathlineto{\pgfqpoint{3.870917in}{3.094613in}}%
\pgfpathlineto{\pgfqpoint{3.872976in}{3.084902in}}%
\pgfpathlineto{\pgfqpoint{3.875035in}{3.094613in}}%
\pgfpathlineto{\pgfqpoint{3.877094in}{3.094613in}}%
\pgfpathlineto{\pgfqpoint{3.879153in}{3.104324in}}%
\pgfpathlineto{\pgfqpoint{3.885330in}{3.114035in}}%
\pgfpathlineto{\pgfqpoint{3.889448in}{3.094613in}}%
\pgfpathlineto{\pgfqpoint{3.893566in}{3.094613in}}%
\pgfpathlineto{\pgfqpoint{3.899743in}{3.114035in}}%
\pgfpathlineto{\pgfqpoint{3.901802in}{3.104324in}}%
\pgfpathlineto{\pgfqpoint{3.903861in}{3.114035in}}%
\pgfpathlineto{\pgfqpoint{3.905919in}{3.114035in}}%
\pgfpathlineto{\pgfqpoint{3.907978in}{3.123746in}}%
\pgfpathlineto{\pgfqpoint{3.914155in}{3.133457in}}%
\pgfpathlineto{\pgfqpoint{3.916214in}{3.143168in}}%
\pgfpathlineto{\pgfqpoint{3.918273in}{3.133457in}}%
\pgfpathlineto{\pgfqpoint{3.920332in}{3.143168in}}%
\pgfpathlineto{\pgfqpoint{3.922391in}{3.143168in}}%
\pgfpathlineto{\pgfqpoint{3.930627in}{3.152879in}}%
\pgfpathlineto{\pgfqpoint{3.936804in}{3.123746in}}%
\pgfpathlineto{\pgfqpoint{3.942980in}{3.143168in}}%
\pgfpathlineto{\pgfqpoint{3.945039in}{3.133457in}}%
\pgfpathlineto{\pgfqpoint{3.947098in}{3.143168in}}%
\pgfpathlineto{\pgfqpoint{3.951216in}{3.143168in}}%
\pgfpathlineto{\pgfqpoint{3.957393in}{3.162590in}}%
\pgfpathlineto{\pgfqpoint{3.959452in}{3.152879in}}%
\pgfpathlineto{\pgfqpoint{3.961511in}{3.152879in}}%
\pgfpathlineto{\pgfqpoint{3.963570in}{3.143168in}}%
\pgfpathlineto{\pgfqpoint{3.965629in}{3.143168in}}%
\pgfpathlineto{\pgfqpoint{3.973865in}{3.201434in}}%
\pgfpathlineto{\pgfqpoint{3.977983in}{3.182012in}}%
\pgfpathlineto{\pgfqpoint{3.980042in}{3.162590in}}%
\pgfpathlineto{\pgfqpoint{3.986218in}{3.162590in}}%
\pgfpathlineto{\pgfqpoint{3.990336in}{3.182012in}}%
\pgfpathlineto{\pgfqpoint{3.994454in}{3.182012in}}%
\pgfpathlineto{\pgfqpoint{4.002690in}{3.152879in}}%
\pgfpathlineto{\pgfqpoint{4.006808in}{3.172301in}}%
\pgfpathlineto{\pgfqpoint{4.008867in}{3.162590in}}%
\pgfpathlineto{\pgfqpoint{4.015044in}{3.143168in}}%
\pgfpathlineto{\pgfqpoint{4.019161in}{3.162590in}}%
\pgfpathlineto{\pgfqpoint{4.021220in}{3.114035in}}%
\pgfpathlineto{\pgfqpoint{4.023279in}{3.104324in}}%
\pgfpathlineto{\pgfqpoint{4.029456in}{3.182012in}}%
\pgfpathlineto{\pgfqpoint{4.033574in}{3.133457in}}%
\pgfpathlineto{\pgfqpoint{4.035633in}{3.094613in}}%
\pgfpathlineto{\pgfqpoint{4.037692in}{3.133457in}}%
\pgfpathlineto{\pgfqpoint{4.043869in}{3.162590in}}%
\pgfpathlineto{\pgfqpoint{4.045928in}{3.162590in}}%
\pgfpathlineto{\pgfqpoint{4.050046in}{3.133457in}}%
\pgfpathlineto{\pgfqpoint{4.052105in}{3.123746in}}%
\pgfpathlineto{\pgfqpoint{4.058281in}{3.143168in}}%
\pgfpathlineto{\pgfqpoint{4.060340in}{3.143168in}}%
\pgfpathlineto{\pgfqpoint{4.062399in}{3.114035in}}%
\pgfpathlineto{\pgfqpoint{4.064458in}{3.123746in}}%
\pgfpathlineto{\pgfqpoint{4.066517in}{3.123746in}}%
\pgfpathlineto{\pgfqpoint{4.074753in}{3.133457in}}%
\pgfpathlineto{\pgfqpoint{4.076812in}{3.133457in}}%
\pgfpathlineto{\pgfqpoint{4.078871in}{3.123746in}}%
\pgfpathlineto{\pgfqpoint{4.080930in}{3.133457in}}%
\pgfpathlineto{\pgfqpoint{4.089166in}{3.133457in}}%
\pgfpathlineto{\pgfqpoint{4.091225in}{3.123746in}}%
\pgfpathlineto{\pgfqpoint{4.093284in}{3.084902in}}%
\pgfpathlineto{\pgfqpoint{4.095342in}{3.084902in}}%
\pgfpathlineto{\pgfqpoint{4.103578in}{3.123746in}}%
\pgfpathlineto{\pgfqpoint{4.105637in}{3.123746in}}%
\pgfpathlineto{\pgfqpoint{4.107696in}{3.114035in}}%
\pgfpathlineto{\pgfqpoint{4.109755in}{3.114035in}}%
\pgfpathlineto{\pgfqpoint{4.115932in}{3.133457in}}%
\pgfpathlineto{\pgfqpoint{4.117991in}{3.123746in}}%
\pgfpathlineto{\pgfqpoint{4.120050in}{3.133457in}}%
\pgfpathlineto{\pgfqpoint{4.122109in}{3.133457in}}%
\pgfpathlineto{\pgfqpoint{4.124168in}{3.123746in}}%
\pgfpathlineto{\pgfqpoint{4.132403in}{3.143168in}}%
\pgfpathlineto{\pgfqpoint{4.134462in}{3.133457in}}%
\pgfpathlineto{\pgfqpoint{4.146816in}{3.133457in}}%
\pgfpathlineto{\pgfqpoint{4.148875in}{3.143168in}}%
\pgfpathlineto{\pgfqpoint{4.150934in}{3.123746in}}%
\pgfpathlineto{\pgfqpoint{4.152993in}{3.143168in}}%
\pgfpathlineto{\pgfqpoint{4.159170in}{3.162590in}}%
\pgfpathlineto{\pgfqpoint{4.161229in}{3.152879in}}%
\pgfpathlineto{\pgfqpoint{4.163288in}{3.152879in}}%
\pgfpathlineto{\pgfqpoint{4.165347in}{3.143168in}}%
\pgfpathlineto{\pgfqpoint{4.167406in}{3.143168in}}%
\pgfpathlineto{\pgfqpoint{4.173582in}{3.162590in}}%
\pgfpathlineto{\pgfqpoint{4.175641in}{3.152879in}}%
\pgfpathlineto{\pgfqpoint{4.177700in}{3.152879in}}%
\pgfpathlineto{\pgfqpoint{4.179759in}{3.143168in}}%
\pgfpathlineto{\pgfqpoint{4.181818in}{3.143168in}}%
\pgfpathlineto{\pgfqpoint{4.187995in}{3.133457in}}%
\pgfpathlineto{\pgfqpoint{4.190054in}{3.143168in}}%
\pgfpathlineto{\pgfqpoint{4.192113in}{3.114035in}}%
\pgfpathlineto{\pgfqpoint{4.194172in}{3.123746in}}%
\pgfpathlineto{\pgfqpoint{4.196231in}{3.104324in}}%
\pgfpathlineto{\pgfqpoint{4.202408in}{3.114035in}}%
\pgfpathlineto{\pgfqpoint{4.204467in}{3.114035in}}%
\pgfpathlineto{\pgfqpoint{4.208584in}{3.065480in}}%
\pgfpathlineto{\pgfqpoint{4.210643in}{3.065480in}}%
\pgfpathlineto{\pgfqpoint{4.216820in}{3.084902in}}%
\pgfpathlineto{\pgfqpoint{4.218879in}{3.075191in}}%
\pgfpathlineto{\pgfqpoint{4.220938in}{3.075191in}}%
\pgfpathlineto{\pgfqpoint{4.222997in}{3.084902in}}%
\pgfpathlineto{\pgfqpoint{4.225056in}{3.084902in}}%
\pgfpathlineto{\pgfqpoint{4.231233in}{3.104324in}}%
\pgfpathlineto{\pgfqpoint{4.233292in}{3.084902in}}%
\pgfpathlineto{\pgfqpoint{4.237410in}{3.084902in}}%
\pgfpathlineto{\pgfqpoint{4.239469in}{3.094613in}}%
\pgfpathlineto{\pgfqpoint{4.245645in}{3.084902in}}%
\pgfpathlineto{\pgfqpoint{4.247704in}{3.094613in}}%
\pgfpathlineto{\pgfqpoint{4.260058in}{3.094613in}}%
\pgfpathlineto{\pgfqpoint{4.262117in}{3.084902in}}%
\pgfpathlineto{\pgfqpoint{4.268294in}{3.084902in}}%
\pgfpathlineto{\pgfqpoint{4.274471in}{3.094613in}}%
\pgfpathlineto{\pgfqpoint{4.276530in}{3.084902in}}%
\pgfpathlineto{\pgfqpoint{4.278589in}{3.065480in}}%
\pgfpathlineto{\pgfqpoint{4.280648in}{3.084902in}}%
\pgfpathlineto{\pgfqpoint{4.290942in}{3.084902in}}%
\pgfpathlineto{\pgfqpoint{4.293001in}{3.075191in}}%
\pgfpathlineto{\pgfqpoint{4.295060in}{3.084902in}}%
\pgfpathlineto{\pgfqpoint{4.303296in}{3.046058in}}%
\pgfpathlineto{\pgfqpoint{4.305355in}{3.055769in}}%
\pgfpathlineto{\pgfqpoint{4.309473in}{3.055769in}}%
\pgfpathlineto{\pgfqpoint{4.311532in}{3.046058in}}%
\pgfpathlineto{\pgfqpoint{4.319768in}{3.046058in}}%
\pgfpathlineto{\pgfqpoint{4.325944in}{3.016925in}}%
\pgfpathlineto{\pgfqpoint{4.334180in}{3.007214in}}%
\pgfpathlineto{\pgfqpoint{4.336239in}{3.007214in}}%
\pgfpathlineto{\pgfqpoint{4.338298in}{3.026636in}}%
\pgfpathlineto{\pgfqpoint{4.340357in}{2.978081in}}%
\pgfpathlineto{\pgfqpoint{4.348593in}{2.919815in}}%
\pgfpathlineto{\pgfqpoint{4.354770in}{2.783861in}}%
\pgfpathlineto{\pgfqpoint{4.360946in}{2.842127in}}%
\pgfpathlineto{\pgfqpoint{4.363005in}{2.842127in}}%
\pgfpathlineto{\pgfqpoint{4.365064in}{2.832416in}}%
\pgfpathlineto{\pgfqpoint{4.367123in}{2.812994in}}%
\pgfpathlineto{\pgfqpoint{4.369182in}{2.812994in}}%
\pgfpathlineto{\pgfqpoint{4.375359in}{2.832416in}}%
\pgfpathlineto{\pgfqpoint{4.377418in}{2.832416in}}%
\pgfpathlineto{\pgfqpoint{4.381536in}{2.677040in}}%
\pgfpathlineto{\pgfqpoint{4.383595in}{2.686751in}}%
\pgfpathlineto{\pgfqpoint{4.389772in}{2.735306in}}%
\pgfpathlineto{\pgfqpoint{4.391831in}{2.735306in}}%
\pgfpathlineto{\pgfqpoint{4.393890in}{2.754728in}}%
\pgfpathlineto{\pgfqpoint{4.395949in}{2.754728in}}%
\pgfpathlineto{\pgfqpoint{4.398007in}{2.725595in}}%
\pgfpathlineto{\pgfqpoint{4.404184in}{2.735306in}}%
\pgfpathlineto{\pgfqpoint{4.408302in}{2.715884in}}%
\pgfpathlineto{\pgfqpoint{4.412420in}{2.774150in}}%
\pgfpathlineto{\pgfqpoint{4.418597in}{2.774150in}}%
\pgfpathlineto{\pgfqpoint{4.420656in}{2.783861in}}%
\pgfpathlineto{\pgfqpoint{4.422715in}{2.706173in}}%
\pgfpathlineto{\pgfqpoint{4.424774in}{2.715884in}}%
\pgfpathlineto{\pgfqpoint{4.426833in}{2.706173in}}%
\pgfpathlineto{\pgfqpoint{4.433010in}{2.696462in}}%
\pgfpathlineto{\pgfqpoint{4.435068in}{2.696462in}}%
\pgfpathlineto{\pgfqpoint{4.439186in}{2.647908in}}%
\pgfpathlineto{\pgfqpoint{4.447422in}{2.715884in}}%
\pgfpathlineto{\pgfqpoint{4.449481in}{2.725595in}}%
\pgfpathlineto{\pgfqpoint{4.451540in}{2.715884in}}%
\pgfpathlineto{\pgfqpoint{4.453599in}{2.715884in}}%
\pgfpathlineto{\pgfqpoint{4.455658in}{2.735306in}}%
\pgfpathlineto{\pgfqpoint{4.461835in}{2.735306in}}%
\pgfpathlineto{\pgfqpoint{4.463894in}{2.706173in}}%
\pgfpathlineto{\pgfqpoint{4.465953in}{2.735306in}}%
\pgfpathlineto{\pgfqpoint{4.468012in}{2.677040in}}%
\pgfpathlineto{\pgfqpoint{4.470071in}{2.657618in}}%
\pgfpathlineto{\pgfqpoint{4.476247in}{2.628486in}}%
\pgfpathlineto{\pgfqpoint{4.478306in}{2.638197in}}%
\pgfpathlineto{\pgfqpoint{4.480365in}{2.589642in}}%
\pgfpathlineto{\pgfqpoint{4.482424in}{2.599353in}}%
\pgfpathlineto{\pgfqpoint{4.484483in}{2.589642in}}%
\pgfpathlineto{\pgfqpoint{4.490660in}{2.579931in}}%
\pgfpathlineto{\pgfqpoint{4.492719in}{2.599353in}}%
\pgfpathlineto{\pgfqpoint{4.498896in}{2.492532in}}%
\pgfpathlineto{\pgfqpoint{4.505073in}{2.541087in}}%
\pgfpathlineto{\pgfqpoint{4.507132in}{2.531376in}}%
\pgfpathlineto{\pgfqpoint{4.511249in}{2.550798in}}%
\pgfpathlineto{\pgfqpoint{4.513308in}{2.511954in}}%
\pgfpathlineto{\pgfqpoint{4.519485in}{2.541087in}}%
\pgfpathlineto{\pgfqpoint{4.521544in}{2.579931in}}%
\pgfpathlineto{\pgfqpoint{4.523603in}{2.531376in}}%
\pgfpathlineto{\pgfqpoint{4.527721in}{2.531376in}}%
\pgfpathlineto{\pgfqpoint{4.535957in}{2.521665in}}%
\pgfpathlineto{\pgfqpoint{4.538016in}{2.511954in}}%
\pgfpathlineto{\pgfqpoint{4.540075in}{2.521665in}}%
\pgfpathlineto{\pgfqpoint{4.542134in}{2.521665in}}%
\pgfpathlineto{\pgfqpoint{4.548311in}{2.511954in}}%
\pgfpathlineto{\pgfqpoint{4.550369in}{2.531376in}}%
\pgfpathlineto{\pgfqpoint{4.552428in}{2.521665in}}%
\pgfpathlineto{\pgfqpoint{4.556546in}{2.560509in}}%
\pgfpathlineto{\pgfqpoint{4.562723in}{2.570220in}}%
\pgfpathlineto{\pgfqpoint{4.564782in}{2.570220in}}%
\pgfpathlineto{\pgfqpoint{4.566841in}{2.550798in}}%
\pgfpathlineto{\pgfqpoint{4.568900in}{2.560509in}}%
\pgfpathlineto{\pgfqpoint{4.570959in}{2.550798in}}%
\pgfpathlineto{\pgfqpoint{4.577136in}{2.570220in}}%
\pgfpathlineto{\pgfqpoint{4.585372in}{2.492532in}}%
\pgfpathlineto{\pgfqpoint{4.591548in}{2.473110in}}%
\pgfpathlineto{\pgfqpoint{4.593607in}{2.453688in}}%
\pgfpathlineto{\pgfqpoint{4.599784in}{2.298312in}}%
\pgfpathlineto{\pgfqpoint{4.605961in}{2.376000in}}%
\pgfpathlineto{\pgfqpoint{4.608020in}{2.337156in}}%
\pgfpathlineto{\pgfqpoint{4.610079in}{2.337156in}}%
\pgfpathlineto{\pgfqpoint{4.612138in}{2.327445in}}%
\pgfpathlineto{\pgfqpoint{4.614197in}{2.327445in}}%
\pgfpathlineto{\pgfqpoint{4.622433in}{2.317734in}}%
\pgfpathlineto{\pgfqpoint{4.624491in}{2.288601in}}%
\pgfpathlineto{\pgfqpoint{4.626550in}{2.278890in}}%
\pgfpathlineto{\pgfqpoint{4.628609in}{2.278890in}}%
\pgfpathlineto{\pgfqpoint{4.634786in}{2.298312in}}%
\pgfpathlineto{\pgfqpoint{4.636845in}{2.288601in}}%
\pgfpathlineto{\pgfqpoint{4.638904in}{2.288601in}}%
\pgfpathlineto{\pgfqpoint{4.643022in}{2.308023in}}%
\pgfpathlineto{\pgfqpoint{4.649199in}{2.298312in}}%
\pgfpathlineto{\pgfqpoint{4.651258in}{2.288601in}}%
\pgfpathlineto{\pgfqpoint{4.653317in}{2.269179in}}%
\pgfpathlineto{\pgfqpoint{4.655376in}{2.220624in}}%
\pgfpathlineto{\pgfqpoint{4.657435in}{2.201202in}}%
\pgfpathlineto{\pgfqpoint{4.665670in}{2.230335in}}%
\pgfpathlineto{\pgfqpoint{4.667729in}{2.220624in}}%
\pgfpathlineto{\pgfqpoint{4.669788in}{2.230335in}}%
\pgfpathlineto{\pgfqpoint{4.671847in}{2.230335in}}%
\pgfpathlineto{\pgfqpoint{4.680083in}{2.240046in}}%
\pgfpathlineto{\pgfqpoint{4.682142in}{2.240046in}}%
\pgfpathlineto{\pgfqpoint{4.684201in}{2.230335in}}%
\pgfpathlineto{\pgfqpoint{4.686260in}{2.240046in}}%
\pgfpathlineto{\pgfqpoint{4.692437in}{2.230335in}}%
\pgfpathlineto{\pgfqpoint{4.696555in}{2.230335in}}%
\pgfpathlineto{\pgfqpoint{4.698614in}{2.240046in}}%
\pgfpathlineto{\pgfqpoint{4.700672in}{2.240046in}}%
\pgfpathlineto{\pgfqpoint{4.708908in}{2.269179in}}%
\pgfpathlineto{\pgfqpoint{4.710967in}{2.278890in}}%
\pgfpathlineto{\pgfqpoint{4.715085in}{2.278890in}}%
\pgfpathlineto{\pgfqpoint{4.721262in}{2.269179in}}%
\pgfpathlineto{\pgfqpoint{4.723321in}{2.220624in}}%
\pgfpathlineto{\pgfqpoint{4.727439in}{2.201202in}}%
\pgfpathlineto{\pgfqpoint{4.729498in}{2.210913in}}%
\pgfpathlineto{\pgfqpoint{4.735675in}{2.230335in}}%
\pgfpathlineto{\pgfqpoint{4.737733in}{2.220624in}}%
\pgfpathlineto{\pgfqpoint{4.739792in}{2.230335in}}%
\pgfpathlineto{\pgfqpoint{4.743910in}{2.210913in}}%
\pgfpathlineto{\pgfqpoint{4.750087in}{2.230335in}}%
\pgfpathlineto{\pgfqpoint{4.754205in}{2.230335in}}%
\pgfpathlineto{\pgfqpoint{4.756264in}{2.220624in}}%
\pgfpathlineto{\pgfqpoint{4.758323in}{2.230335in}}%
\pgfpathlineto{\pgfqpoint{4.766559in}{2.259468in}}%
\pgfpathlineto{\pgfqpoint{4.770677in}{2.259468in}}%
\pgfpathlineto{\pgfqpoint{4.772736in}{2.240046in}}%
\pgfpathlineto{\pgfqpoint{4.778912in}{2.249757in}}%
\pgfpathlineto{\pgfqpoint{4.780971in}{2.249757in}}%
\pgfpathlineto{\pgfqpoint{4.787148in}{2.201202in}}%
\pgfpathlineto{\pgfqpoint{4.793325in}{2.210913in}}%
\pgfpathlineto{\pgfqpoint{4.799502in}{2.210913in}}%
\pgfpathlineto{\pgfqpoint{4.801561in}{2.201202in}}%
\pgfpathlineto{\pgfqpoint{4.807738in}{2.220624in}}%
\pgfpathlineto{\pgfqpoint{4.809797in}{2.220624in}}%
\pgfpathlineto{\pgfqpoint{4.811856in}{2.230335in}}%
\pgfpathlineto{\pgfqpoint{4.813914in}{2.210913in}}%
\pgfpathlineto{\pgfqpoint{4.815973in}{2.220624in}}%
\pgfpathlineto{\pgfqpoint{4.824209in}{2.220624in}}%
\pgfpathlineto{\pgfqpoint{4.826268in}{2.210913in}}%
\pgfpathlineto{\pgfqpoint{4.828327in}{2.210913in}}%
\pgfpathlineto{\pgfqpoint{4.830386in}{2.201202in}}%
\pgfpathlineto{\pgfqpoint{4.838622in}{2.230335in}}%
\pgfpathlineto{\pgfqpoint{4.840681in}{2.220624in}}%
\pgfpathlineto{\pgfqpoint{4.842740in}{2.220624in}}%
\pgfpathlineto{\pgfqpoint{4.844799in}{2.191491in}}%
\pgfpathlineto{\pgfqpoint{4.857152in}{2.230335in}}%
\pgfpathlineto{\pgfqpoint{4.859211in}{2.220624in}}%
\pgfpathlineto{\pgfqpoint{4.865388in}{2.230335in}}%
\pgfpathlineto{\pgfqpoint{4.867447in}{2.201202in}}%
\pgfpathlineto{\pgfqpoint{4.869506in}{2.210913in}}%
\pgfpathlineto{\pgfqpoint{4.883919in}{2.210913in}}%
\pgfpathlineto{\pgfqpoint{4.885978in}{2.201202in}}%
\pgfpathlineto{\pgfqpoint{4.888037in}{2.181780in}}%
\pgfpathlineto{\pgfqpoint{4.896272in}{2.123514in}}%
\pgfpathlineto{\pgfqpoint{4.898331in}{2.074960in}}%
\pgfpathlineto{\pgfqpoint{4.900390in}{1.987561in}}%
\pgfpathlineto{\pgfqpoint{4.902449in}{1.773919in}}%
\pgfpathlineto{\pgfqpoint{4.908626in}{1.618543in}}%
\pgfpathlineto{\pgfqpoint{4.916862in}{1.094150in}}%
\pgfpathlineto{\pgfqpoint{4.923039in}{0.958197in}}%
\pgfpathlineto{\pgfqpoint{4.925098in}{1.113572in}}%
\pgfpathlineto{\pgfqpoint{4.929215in}{1.055306in}}%
\pgfpathlineto{\pgfqpoint{4.931274in}{1.065017in}}%
\pgfpathlineto{\pgfqpoint{4.937451in}{0.977618in}}%
\pgfpathlineto{\pgfqpoint{4.939510in}{0.929064in}}%
\pgfpathlineto{\pgfqpoint{4.941569in}{0.773688in}}%
\pgfpathlineto{\pgfqpoint{4.945687in}{0.744555in}}%
\pgfpathlineto{\pgfqpoint{4.953923in}{0.783399in}}%
\pgfpathlineto{\pgfqpoint{4.960100in}{0.715422in}}%
\pgfpathlineto{\pgfqpoint{4.968335in}{0.841665in}}%
\pgfpathlineto{\pgfqpoint{4.970394in}{0.831954in}}%
\pgfpathlineto{\pgfqpoint{4.972453in}{0.841665in}}%
\pgfpathlineto{\pgfqpoint{4.974512in}{0.841665in}}%
\pgfpathlineto{\pgfqpoint{4.980689in}{0.861087in}}%
\pgfpathlineto{\pgfqpoint{4.984807in}{0.929064in}}%
\pgfpathlineto{\pgfqpoint{4.986866in}{0.929064in}}%
\pgfpathlineto{\pgfqpoint{4.995102in}{0.958197in}}%
\pgfpathlineto{\pgfqpoint{4.997161in}{0.929064in}}%
\pgfpathlineto{\pgfqpoint{4.999220in}{0.880509in}}%
\pgfpathlineto{\pgfqpoint{5.001279in}{0.870798in}}%
\pgfpathlineto{\pgfqpoint{5.003337in}{0.851376in}}%
\pgfpathlineto{\pgfqpoint{5.009514in}{0.841665in}}%
\pgfpathlineto{\pgfqpoint{5.011573in}{0.831954in}}%
\pgfpathlineto{\pgfqpoint{5.017750in}{0.831954in}}%
\pgfpathlineto{\pgfqpoint{5.023927in}{0.841665in}}%
\pgfpathlineto{\pgfqpoint{5.028045in}{0.812532in}}%
\pgfpathlineto{\pgfqpoint{5.030104in}{0.802821in}}%
\pgfpathlineto{\pgfqpoint{5.032163in}{0.812532in}}%
\pgfpathlineto{\pgfqpoint{5.040399in}{0.841665in}}%
\pgfpathlineto{\pgfqpoint{5.042457in}{0.851376in}}%
\pgfpathlineto{\pgfqpoint{5.044516in}{0.831954in}}%
\pgfpathlineto{\pgfqpoint{5.046575in}{0.841665in}}%
\pgfpathlineto{\pgfqpoint{5.052752in}{0.851376in}}%
\pgfpathlineto{\pgfqpoint{5.054811in}{0.851376in}}%
\pgfpathlineto{\pgfqpoint{5.056870in}{0.841665in}}%
\pgfpathlineto{\pgfqpoint{5.069224in}{0.841665in}}%
\pgfpathlineto{\pgfqpoint{5.071283in}{0.831954in}}%
\pgfpathlineto{\pgfqpoint{5.073342in}{0.831954in}}%
\pgfpathlineto{\pgfqpoint{5.075401in}{0.851376in}}%
\pgfpathlineto{\pgfqpoint{5.083636in}{0.861087in}}%
\pgfpathlineto{\pgfqpoint{5.085695in}{0.861087in}}%
\pgfpathlineto{\pgfqpoint{5.087754in}{0.870798in}}%
\pgfpathlineto{\pgfqpoint{5.098049in}{0.870798in}}%
\pgfpathlineto{\pgfqpoint{5.100108in}{0.880509in}}%
\pgfpathlineto{\pgfqpoint{5.102167in}{0.870798in}}%
\pgfpathlineto{\pgfqpoint{5.104226in}{0.870798in}}%
\pgfpathlineto{\pgfqpoint{5.110403in}{0.880509in}}%
\pgfpathlineto{\pgfqpoint{5.114521in}{0.880509in}}%
\pgfpathlineto{\pgfqpoint{5.116579in}{0.870798in}}%
\pgfpathlineto{\pgfqpoint{5.118638in}{0.870798in}}%
\pgfpathlineto{\pgfqpoint{5.124815in}{0.880509in}}%
\pgfpathlineto{\pgfqpoint{5.126874in}{0.880509in}}%
\pgfpathlineto{\pgfqpoint{5.130992in}{0.861087in}}%
\pgfpathlineto{\pgfqpoint{5.133051in}{0.861087in}}%
\pgfpathlineto{\pgfqpoint{5.139228in}{0.870798in}}%
\pgfpathlineto{\pgfqpoint{5.141287in}{0.861087in}}%
\pgfpathlineto{\pgfqpoint{5.147464in}{0.861087in}}%
\pgfpathlineto{\pgfqpoint{5.153641in}{0.870798in}}%
\pgfpathlineto{\pgfqpoint{5.155699in}{0.870798in}}%
\pgfpathlineto{\pgfqpoint{5.159817in}{0.851376in}}%
\pgfpathlineto{\pgfqpoint{5.168053in}{0.851376in}}%
\pgfpathlineto{\pgfqpoint{5.170112in}{0.861087in}}%
\pgfpathlineto{\pgfqpoint{5.172171in}{0.861087in}}%
\pgfpathlineto{\pgfqpoint{5.176289in}{0.841665in}}%
\pgfpathlineto{\pgfqpoint{5.182466in}{0.841665in}}%
\pgfpathlineto{\pgfqpoint{5.184525in}{0.831954in}}%
\pgfpathlineto{\pgfqpoint{5.186584in}{0.841665in}}%
\pgfpathlineto{\pgfqpoint{5.188643in}{0.822243in}}%
\pgfpathlineto{\pgfqpoint{5.190702in}{0.822243in}}%
\pgfpathlineto{\pgfqpoint{5.196878in}{0.831954in}}%
\pgfpathlineto{\pgfqpoint{5.198937in}{0.822243in}}%
\pgfpathlineto{\pgfqpoint{5.200996in}{0.831954in}}%
\pgfpathlineto{\pgfqpoint{5.203055in}{0.822243in}}%
\pgfpathlineto{\pgfqpoint{5.205114in}{0.831954in}}%
\pgfpathlineto{\pgfqpoint{5.211291in}{0.831954in}}%
\pgfpathlineto{\pgfqpoint{5.213350in}{0.812532in}}%
\pgfpathlineto{\pgfqpoint{5.215409in}{0.812532in}}%
\pgfpathlineto{\pgfqpoint{5.219527in}{0.793110in}}%
\pgfpathlineto{\pgfqpoint{5.225704in}{0.802821in}}%
\pgfpathlineto{\pgfqpoint{5.227763in}{0.802821in}}%
\pgfpathlineto{\pgfqpoint{5.229822in}{0.812532in}}%
\pgfpathlineto{\pgfqpoint{5.231880in}{0.802821in}}%
\pgfpathlineto{\pgfqpoint{5.233939in}{0.812532in}}%
\pgfpathlineto{\pgfqpoint{5.240116in}{0.822243in}}%
\pgfpathlineto{\pgfqpoint{5.242175in}{0.812532in}}%
\pgfpathlineto{\pgfqpoint{5.254529in}{0.812532in}}%
\pgfpathlineto{\pgfqpoint{5.256588in}{0.822243in}}%
\pgfpathlineto{\pgfqpoint{5.258647in}{0.812532in}}%
\pgfpathlineto{\pgfqpoint{5.260706in}{0.822243in}}%
\pgfpathlineto{\pgfqpoint{5.262765in}{0.812532in}}%
\pgfpathlineto{\pgfqpoint{5.273059in}{0.812532in}}%
\pgfpathlineto{\pgfqpoint{5.275118in}{0.802821in}}%
\pgfpathlineto{\pgfqpoint{5.277177in}{0.802821in}}%
\pgfpathlineto{\pgfqpoint{5.283354in}{0.822243in}}%
\pgfpathlineto{\pgfqpoint{5.285413in}{0.822243in}}%
\pgfpathlineto{\pgfqpoint{5.287472in}{0.812532in}}%
\pgfpathlineto{\pgfqpoint{5.291590in}{0.812532in}}%
\pgfpathlineto{\pgfqpoint{5.299826in}{0.831954in}}%
\pgfpathlineto{\pgfqpoint{5.301885in}{0.831954in}}%
\pgfpathlineto{\pgfqpoint{5.303944in}{0.812532in}}%
\pgfpathlineto{\pgfqpoint{5.306002in}{0.812532in}}%
\pgfpathlineto{\pgfqpoint{5.312179in}{0.822243in}}%
\pgfpathlineto{\pgfqpoint{5.314238in}{0.812532in}}%
\pgfpathlineto{\pgfqpoint{5.316297in}{0.812532in}}%
\pgfpathlineto{\pgfqpoint{5.318356in}{0.802821in}}%
\pgfpathlineto{\pgfqpoint{5.320415in}{0.812532in}}%
\pgfpathlineto{\pgfqpoint{5.326592in}{0.802821in}}%
\pgfpathlineto{\pgfqpoint{5.345122in}{0.802821in}}%
\pgfpathlineto{\pgfqpoint{5.347181in}{0.793110in}}%
\pgfpathlineto{\pgfqpoint{5.349240in}{0.802821in}}%
\pgfpathlineto{\pgfqpoint{5.357476in}{0.802821in}}%
\pgfpathlineto{\pgfqpoint{5.359535in}{0.812532in}}%
\pgfpathlineto{\pgfqpoint{5.378066in}{0.812532in}}%
\pgfpathlineto{\pgfqpoint{5.384242in}{0.802821in}}%
\pgfpathlineto{\pgfqpoint{5.386301in}{0.812532in}}%
\pgfpathlineto{\pgfqpoint{5.388360in}{0.812532in}}%
\pgfpathlineto{\pgfqpoint{5.390419in}{0.802821in}}%
\pgfpathlineto{\pgfqpoint{5.392478in}{0.802821in}}%
\pgfpathlineto{\pgfqpoint{5.398655in}{0.812532in}}%
\pgfpathlineto{\pgfqpoint{5.400714in}{0.802821in}}%
\pgfpathlineto{\pgfqpoint{5.402773in}{0.802821in}}%
\pgfpathlineto{\pgfqpoint{5.404832in}{0.793110in}}%
\pgfpathlineto{\pgfqpoint{5.406891in}{0.802821in}}%
\pgfpathlineto{\pgfqpoint{5.413068in}{0.802821in}}%
\pgfpathlineto{\pgfqpoint{5.415127in}{0.812532in}}%
\pgfpathlineto{\pgfqpoint{5.417186in}{0.793110in}}%
\pgfpathlineto{\pgfqpoint{5.419244in}{0.793110in}}%
\pgfpathlineto{\pgfqpoint{5.421303in}{0.802821in}}%
\pgfpathlineto{\pgfqpoint{5.429539in}{0.802821in}}%
\pgfpathlineto{\pgfqpoint{5.433657in}{0.793110in}}%
\pgfpathlineto{\pgfqpoint{5.435716in}{0.793110in}}%
\pgfpathlineto{\pgfqpoint{5.441893in}{0.812532in}}%
\pgfpathlineto{\pgfqpoint{5.443952in}{0.793110in}}%
\pgfpathlineto{\pgfqpoint{5.450129in}{0.793110in}}%
\pgfpathlineto{\pgfqpoint{5.456306in}{0.773688in}}%
\pgfpathlineto{\pgfqpoint{5.458364in}{0.793110in}}%
\pgfpathlineto{\pgfqpoint{5.460423in}{0.783399in}}%
\pgfpathlineto{\pgfqpoint{5.464541in}{0.793110in}}%
\pgfpathlineto{\pgfqpoint{5.470718in}{0.783399in}}%
\pgfpathlineto{\pgfqpoint{5.472777in}{0.793110in}}%
\pgfpathlineto{\pgfqpoint{5.474836in}{0.793110in}}%
\pgfpathlineto{\pgfqpoint{5.476895in}{0.783399in}}%
\pgfpathlineto{\pgfqpoint{5.478954in}{0.793110in}}%
\pgfpathlineto{\pgfqpoint{5.485131in}{0.793110in}}%
\pgfpathlineto{\pgfqpoint{5.487190in}{0.783399in}}%
\pgfpathlineto{\pgfqpoint{5.491308in}{0.783399in}}%
\pgfpathlineto{\pgfqpoint{5.493367in}{0.773688in}}%
\pgfpathlineto{\pgfqpoint{5.501602in}{0.773688in}}%
\pgfpathlineto{\pgfqpoint{5.503661in}{0.783399in}}%
\pgfpathlineto{\pgfqpoint{5.520133in}{0.783399in}}%
\pgfpathlineto{\pgfqpoint{5.528369in}{0.802821in}}%
\pgfpathlineto{\pgfqpoint{5.530428in}{0.812532in}}%
\pgfpathlineto{\pgfqpoint{5.532487in}{0.783399in}}%
\pgfpathlineto{\pgfqpoint{5.534545in}{0.783399in}}%
\pgfpathlineto{\pgfqpoint{5.534545in}{0.783399in}}%
\pgfusepath{stroke}%
\end{pgfscope}%
\begin{pgfscope}%
\pgfpathrectangle{\pgfqpoint{0.800000in}{0.528000in}}{\pgfqpoint{4.960000in}{3.696000in}}%
\pgfusepath{clip}%
\pgfsetrectcap%
\pgfsetroundjoin%
\pgfsetlinewidth{1.003750pt}%
\definecolor{currentstroke}{rgb}{0.501961,0.501961,0.501961}%
\pgfsetstrokecolor{currentstroke}%
\pgfsetstrokeopacity{0.900000}%
\pgfsetdash{}{0pt}%
\pgfpathmoveto{\pgfqpoint{1.025455in}{0.938775in}}%
\pgfpathlineto{\pgfqpoint{1.031631in}{0.948486in}}%
\pgfpathlineto{\pgfqpoint{1.033690in}{0.938775in}}%
\pgfpathlineto{\pgfqpoint{1.035749in}{0.938775in}}%
\pgfpathlineto{\pgfqpoint{1.039867in}{0.909642in}}%
\pgfpathlineto{\pgfqpoint{1.046044in}{0.880509in}}%
\pgfpathlineto{\pgfqpoint{1.048103in}{0.890220in}}%
\pgfpathlineto{\pgfqpoint{1.052221in}{0.851376in}}%
\pgfpathlineto{\pgfqpoint{1.054280in}{0.861087in}}%
\pgfpathlineto{\pgfqpoint{1.068692in}{0.861087in}}%
\pgfpathlineto{\pgfqpoint{1.074869in}{0.870798in}}%
\pgfpathlineto{\pgfqpoint{1.076928in}{0.861087in}}%
\pgfpathlineto{\pgfqpoint{1.081046in}{0.861087in}}%
\pgfpathlineto{\pgfqpoint{1.083105in}{0.870798in}}%
\pgfpathlineto{\pgfqpoint{1.089282in}{0.861087in}}%
\pgfpathlineto{\pgfqpoint{1.091341in}{0.899931in}}%
\pgfpathlineto{\pgfqpoint{1.093400in}{0.890220in}}%
\pgfpathlineto{\pgfqpoint{1.095459in}{0.890220in}}%
\pgfpathlineto{\pgfqpoint{1.097518in}{0.948486in}}%
\pgfpathlineto{\pgfqpoint{1.103694in}{0.929064in}}%
\pgfpathlineto{\pgfqpoint{1.105753in}{0.938775in}}%
\pgfpathlineto{\pgfqpoint{1.109871in}{0.919353in}}%
\pgfpathlineto{\pgfqpoint{1.111930in}{0.919353in}}%
\pgfpathlineto{\pgfqpoint{1.120166in}{0.938775in}}%
\pgfpathlineto{\pgfqpoint{1.122225in}{0.919353in}}%
\pgfpathlineto{\pgfqpoint{1.126343in}{0.919353in}}%
\pgfpathlineto{\pgfqpoint{1.132520in}{0.909642in}}%
\pgfpathlineto{\pgfqpoint{1.134579in}{0.909642in}}%
\pgfpathlineto{\pgfqpoint{1.136638in}{0.899931in}}%
\pgfpathlineto{\pgfqpoint{1.138697in}{0.909642in}}%
\pgfpathlineto{\pgfqpoint{1.146932in}{0.909642in}}%
\pgfpathlineto{\pgfqpoint{1.148991in}{0.948486in}}%
\pgfpathlineto{\pgfqpoint{1.151050in}{0.948486in}}%
\pgfpathlineto{\pgfqpoint{1.153109in}{0.938775in}}%
\pgfpathlineto{\pgfqpoint{1.155168in}{0.958197in}}%
\pgfpathlineto{\pgfqpoint{1.161345in}{0.958197in}}%
\pgfpathlineto{\pgfqpoint{1.163404in}{0.938775in}}%
\pgfpathlineto{\pgfqpoint{1.165463in}{0.938775in}}%
\pgfpathlineto{\pgfqpoint{1.167522in}{0.929064in}}%
\pgfpathlineto{\pgfqpoint{1.169581in}{0.929064in}}%
\pgfpathlineto{\pgfqpoint{1.177817in}{0.958197in}}%
\pgfpathlineto{\pgfqpoint{1.179875in}{0.919353in}}%
\pgfpathlineto{\pgfqpoint{1.181934in}{0.948486in}}%
\pgfpathlineto{\pgfqpoint{1.183993in}{0.929064in}}%
\pgfpathlineto{\pgfqpoint{1.192229in}{0.929064in}}%
\pgfpathlineto{\pgfqpoint{1.194288in}{0.938775in}}%
\pgfpathlineto{\pgfqpoint{1.196347in}{0.967908in}}%
\pgfpathlineto{\pgfqpoint{1.198406in}{0.958197in}}%
\pgfpathlineto{\pgfqpoint{1.204583in}{0.958197in}}%
\pgfpathlineto{\pgfqpoint{1.206642in}{0.948486in}}%
\pgfpathlineto{\pgfqpoint{1.208701in}{0.958197in}}%
\pgfpathlineto{\pgfqpoint{1.210760in}{0.938775in}}%
\pgfpathlineto{\pgfqpoint{1.212819in}{0.899931in}}%
\pgfpathlineto{\pgfqpoint{1.218995in}{0.899931in}}%
\pgfpathlineto{\pgfqpoint{1.221054in}{0.909642in}}%
\pgfpathlineto{\pgfqpoint{1.225172in}{0.909642in}}%
\pgfpathlineto{\pgfqpoint{1.227231in}{0.929064in}}%
\pgfpathlineto{\pgfqpoint{1.233408in}{0.919353in}}%
\pgfpathlineto{\pgfqpoint{1.237526in}{0.919353in}}%
\pgfpathlineto{\pgfqpoint{1.239585in}{0.909642in}}%
\pgfpathlineto{\pgfqpoint{1.241644in}{0.919353in}}%
\pgfpathlineto{\pgfqpoint{1.247821in}{0.929064in}}%
\pgfpathlineto{\pgfqpoint{1.249880in}{0.919353in}}%
\pgfpathlineto{\pgfqpoint{1.251939in}{0.919353in}}%
\pgfpathlineto{\pgfqpoint{1.253998in}{0.929064in}}%
\pgfpathlineto{\pgfqpoint{1.256056in}{0.929064in}}%
\pgfpathlineto{\pgfqpoint{1.262233in}{0.938775in}}%
\pgfpathlineto{\pgfqpoint{1.264292in}{0.929064in}}%
\pgfpathlineto{\pgfqpoint{1.266351in}{0.938775in}}%
\pgfpathlineto{\pgfqpoint{1.268410in}{0.929064in}}%
\pgfpathlineto{\pgfqpoint{1.270469in}{0.938775in}}%
\pgfpathlineto{\pgfqpoint{1.276646in}{0.938775in}}%
\pgfpathlineto{\pgfqpoint{1.278705in}{0.929064in}}%
\pgfpathlineto{\pgfqpoint{1.280764in}{0.938775in}}%
\pgfpathlineto{\pgfqpoint{1.284882in}{0.919353in}}%
\pgfpathlineto{\pgfqpoint{1.291059in}{0.938775in}}%
\pgfpathlineto{\pgfqpoint{1.293117in}{0.938775in}}%
\pgfpathlineto{\pgfqpoint{1.297235in}{0.919353in}}%
\pgfpathlineto{\pgfqpoint{1.299294in}{0.919353in}}%
\pgfpathlineto{\pgfqpoint{1.305471in}{0.909642in}}%
\pgfpathlineto{\pgfqpoint{1.307530in}{0.919353in}}%
\pgfpathlineto{\pgfqpoint{1.309589in}{0.919353in}}%
\pgfpathlineto{\pgfqpoint{1.311648in}{0.909642in}}%
\pgfpathlineto{\pgfqpoint{1.313707in}{0.919353in}}%
\pgfpathlineto{\pgfqpoint{1.321943in}{0.929064in}}%
\pgfpathlineto{\pgfqpoint{1.324002in}{0.958197in}}%
\pgfpathlineto{\pgfqpoint{1.326061in}{0.948486in}}%
\pgfpathlineto{\pgfqpoint{1.338414in}{0.948486in}}%
\pgfpathlineto{\pgfqpoint{1.340473in}{0.958197in}}%
\pgfpathlineto{\pgfqpoint{1.342532in}{0.977618in}}%
\pgfpathlineto{\pgfqpoint{1.348709in}{0.958197in}}%
\pgfpathlineto{\pgfqpoint{1.350768in}{0.958197in}}%
\pgfpathlineto{\pgfqpoint{1.352827in}{0.967908in}}%
\pgfpathlineto{\pgfqpoint{1.365181in}{0.967908in}}%
\pgfpathlineto{\pgfqpoint{1.371357in}{0.938775in}}%
\pgfpathlineto{\pgfqpoint{1.377534in}{0.958197in}}%
\pgfpathlineto{\pgfqpoint{1.379593in}{0.987329in}}%
\pgfpathlineto{\pgfqpoint{1.381652in}{0.987329in}}%
\pgfpathlineto{\pgfqpoint{1.383711in}{0.977618in}}%
\pgfpathlineto{\pgfqpoint{1.385770in}{0.977618in}}%
\pgfpathlineto{\pgfqpoint{1.391947in}{0.958197in}}%
\pgfpathlineto{\pgfqpoint{1.394006in}{0.967908in}}%
\pgfpathlineto{\pgfqpoint{1.396065in}{0.967908in}}%
\pgfpathlineto{\pgfqpoint{1.398124in}{0.948486in}}%
\pgfpathlineto{\pgfqpoint{1.406359in}{0.948486in}}%
\pgfpathlineto{\pgfqpoint{1.410477in}{0.929064in}}%
\pgfpathlineto{\pgfqpoint{1.412536in}{0.938775in}}%
\pgfpathlineto{\pgfqpoint{1.414595in}{0.967908in}}%
\pgfpathlineto{\pgfqpoint{1.420772in}{0.967908in}}%
\pgfpathlineto{\pgfqpoint{1.422831in}{0.958197in}}%
\pgfpathlineto{\pgfqpoint{1.426949in}{0.977618in}}%
\pgfpathlineto{\pgfqpoint{1.429008in}{0.977618in}}%
\pgfpathlineto{\pgfqpoint{1.435185in}{0.997040in}}%
\pgfpathlineto{\pgfqpoint{1.437244in}{1.026173in}}%
\pgfpathlineto{\pgfqpoint{1.439303in}{1.026173in}}%
\pgfpathlineto{\pgfqpoint{1.443421in}{1.006751in}}%
\pgfpathlineto{\pgfqpoint{1.451656in}{1.006751in}}%
\pgfpathlineto{\pgfqpoint{1.453715in}{1.016462in}}%
\pgfpathlineto{\pgfqpoint{1.455774in}{1.045595in}}%
\pgfpathlineto{\pgfqpoint{1.457833in}{1.016462in}}%
\pgfpathlineto{\pgfqpoint{1.464010in}{1.016462in}}%
\pgfpathlineto{\pgfqpoint{1.466069in}{1.055306in}}%
\pgfpathlineto{\pgfqpoint{1.468128in}{1.065017in}}%
\pgfpathlineto{\pgfqpoint{1.470187in}{1.035884in}}%
\pgfpathlineto{\pgfqpoint{1.472246in}{1.065017in}}%
\pgfpathlineto{\pgfqpoint{1.478423in}{1.084439in}}%
\pgfpathlineto{\pgfqpoint{1.480482in}{1.055306in}}%
\pgfpathlineto{\pgfqpoint{1.482540in}{1.055306in}}%
\pgfpathlineto{\pgfqpoint{1.484599in}{1.084439in}}%
\pgfpathlineto{\pgfqpoint{1.486658in}{1.094150in}}%
\pgfpathlineto{\pgfqpoint{1.492835in}{1.084439in}}%
\pgfpathlineto{\pgfqpoint{1.494894in}{1.103861in}}%
\pgfpathlineto{\pgfqpoint{1.496953in}{1.074728in}}%
\pgfpathlineto{\pgfqpoint{1.499012in}{1.074728in}}%
\pgfpathlineto{\pgfqpoint{1.501071in}{1.045595in}}%
\pgfpathlineto{\pgfqpoint{1.507248in}{1.016462in}}%
\pgfpathlineto{\pgfqpoint{1.509307in}{1.045595in}}%
\pgfpathlineto{\pgfqpoint{1.511366in}{1.035884in}}%
\pgfpathlineto{\pgfqpoint{1.513425in}{1.045595in}}%
\pgfpathlineto{\pgfqpoint{1.515484in}{1.065017in}}%
\pgfpathlineto{\pgfqpoint{1.521660in}{1.074728in}}%
\pgfpathlineto{\pgfqpoint{1.523719in}{1.074728in}}%
\pgfpathlineto{\pgfqpoint{1.527837in}{1.045595in}}%
\pgfpathlineto{\pgfqpoint{1.529896in}{1.045595in}}%
\pgfpathlineto{\pgfqpoint{1.538132in}{1.074728in}}%
\pgfpathlineto{\pgfqpoint{1.542250in}{1.074728in}}%
\pgfpathlineto{\pgfqpoint{1.544309in}{1.084439in}}%
\pgfpathlineto{\pgfqpoint{1.550486in}{1.084439in}}%
\pgfpathlineto{\pgfqpoint{1.552545in}{1.152416in}}%
\pgfpathlineto{\pgfqpoint{1.554604in}{1.142705in}}%
\pgfpathlineto{\pgfqpoint{1.558721in}{1.035884in}}%
\pgfpathlineto{\pgfqpoint{1.564898in}{1.045595in}}%
\pgfpathlineto{\pgfqpoint{1.569016in}{1.026173in}}%
\pgfpathlineto{\pgfqpoint{1.571075in}{1.006751in}}%
\pgfpathlineto{\pgfqpoint{1.573134in}{1.035884in}}%
\pgfpathlineto{\pgfqpoint{1.579311in}{1.026173in}}%
\pgfpathlineto{\pgfqpoint{1.581370in}{1.016462in}}%
\pgfpathlineto{\pgfqpoint{1.583429in}{1.016462in}}%
\pgfpathlineto{\pgfqpoint{1.585488in}{0.997040in}}%
\pgfpathlineto{\pgfqpoint{1.587547in}{0.938775in}}%
\pgfpathlineto{\pgfqpoint{1.593724in}{0.948486in}}%
\pgfpathlineto{\pgfqpoint{1.595782in}{0.948486in}}%
\pgfpathlineto{\pgfqpoint{1.597841in}{0.958197in}}%
\pgfpathlineto{\pgfqpoint{1.599900in}{0.958197in}}%
\pgfpathlineto{\pgfqpoint{1.601959in}{0.967908in}}%
\pgfpathlineto{\pgfqpoint{1.610195in}{0.958197in}}%
\pgfpathlineto{\pgfqpoint{1.612254in}{0.899931in}}%
\pgfpathlineto{\pgfqpoint{1.616372in}{0.919353in}}%
\pgfpathlineto{\pgfqpoint{1.628726in}{0.919353in}}%
\pgfpathlineto{\pgfqpoint{1.630785in}{0.929064in}}%
\pgfpathlineto{\pgfqpoint{1.636961in}{0.938775in}}%
\pgfpathlineto{\pgfqpoint{1.641079in}{1.016462in}}%
\pgfpathlineto{\pgfqpoint{1.643138in}{1.016462in}}%
\pgfpathlineto{\pgfqpoint{1.651374in}{1.055306in}}%
\pgfpathlineto{\pgfqpoint{1.657551in}{1.103861in}}%
\pgfpathlineto{\pgfqpoint{1.659610in}{1.152416in}}%
\pgfpathlineto{\pgfqpoint{1.665787in}{1.152416in}}%
\pgfpathlineto{\pgfqpoint{1.667846in}{1.191260in}}%
\pgfpathlineto{\pgfqpoint{1.671963in}{1.191260in}}%
\pgfpathlineto{\pgfqpoint{1.674022in}{1.181549in}}%
\pgfpathlineto{\pgfqpoint{1.682258in}{1.181549in}}%
\pgfpathlineto{\pgfqpoint{1.684317in}{1.171838in}}%
\pgfpathlineto{\pgfqpoint{1.688435in}{1.171838in}}%
\pgfpathlineto{\pgfqpoint{1.694612in}{1.181549in}}%
\pgfpathlineto{\pgfqpoint{1.696671in}{1.200971in}}%
\pgfpathlineto{\pgfqpoint{1.698730in}{1.200971in}}%
\pgfpathlineto{\pgfqpoint{1.702848in}{1.181549in}}%
\pgfpathlineto{\pgfqpoint{1.709024in}{1.191260in}}%
\pgfpathlineto{\pgfqpoint{1.711083in}{1.191260in}}%
\pgfpathlineto{\pgfqpoint{1.713142in}{1.200971in}}%
\pgfpathlineto{\pgfqpoint{1.717260in}{1.278659in}}%
\pgfpathlineto{\pgfqpoint{1.723437in}{1.346636in}}%
\pgfpathlineto{\pgfqpoint{1.725496in}{1.434035in}}%
\pgfpathlineto{\pgfqpoint{1.727555in}{1.395191in}}%
\pgfpathlineto{\pgfqpoint{1.729614in}{1.385480in}}%
\pgfpathlineto{\pgfqpoint{1.731673in}{1.356347in}}%
\pgfpathlineto{\pgfqpoint{1.737850in}{1.356347in}}%
\pgfpathlineto{\pgfqpoint{1.741968in}{1.375769in}}%
\pgfpathlineto{\pgfqpoint{1.744027in}{1.366058in}}%
\pgfpathlineto{\pgfqpoint{1.746086in}{1.346636in}}%
\pgfpathlineto{\pgfqpoint{1.752262in}{1.317503in}}%
\pgfpathlineto{\pgfqpoint{1.754321in}{1.336925in}}%
\pgfpathlineto{\pgfqpoint{1.758439in}{1.317503in}}%
\pgfpathlineto{\pgfqpoint{1.766675in}{1.336925in}}%
\pgfpathlineto{\pgfqpoint{1.768734in}{1.346636in}}%
\pgfpathlineto{\pgfqpoint{1.770793in}{1.317503in}}%
\pgfpathlineto{\pgfqpoint{1.772852in}{1.327214in}}%
\pgfpathlineto{\pgfqpoint{1.781088in}{1.288370in}}%
\pgfpathlineto{\pgfqpoint{1.783147in}{1.356347in}}%
\pgfpathlineto{\pgfqpoint{1.787264in}{1.336925in}}%
\pgfpathlineto{\pgfqpoint{1.789323in}{1.317503in}}%
\pgfpathlineto{\pgfqpoint{1.795500in}{1.307792in}}%
\pgfpathlineto{\pgfqpoint{1.797559in}{1.298081in}}%
\pgfpathlineto{\pgfqpoint{1.799618in}{1.278659in}}%
\pgfpathlineto{\pgfqpoint{1.803736in}{1.171838in}}%
\pgfpathlineto{\pgfqpoint{1.811972in}{1.162127in}}%
\pgfpathlineto{\pgfqpoint{1.814031in}{1.113572in}}%
\pgfpathlineto{\pgfqpoint{1.816090in}{1.123283in}}%
\pgfpathlineto{\pgfqpoint{1.818149in}{1.152416in}}%
\pgfpathlineto{\pgfqpoint{1.838738in}{1.152416in}}%
\pgfpathlineto{\pgfqpoint{1.840797in}{1.220393in}}%
\pgfpathlineto{\pgfqpoint{1.842856in}{1.220393in}}%
\pgfpathlineto{\pgfqpoint{1.844915in}{1.200971in}}%
\pgfpathlineto{\pgfqpoint{1.846974in}{1.230104in}}%
\pgfpathlineto{\pgfqpoint{1.853151in}{1.191260in}}%
\pgfpathlineto{\pgfqpoint{1.855210in}{1.200971in}}%
\pgfpathlineto{\pgfqpoint{1.857269in}{1.200971in}}%
\pgfpathlineto{\pgfqpoint{1.859328in}{1.152416in}}%
\pgfpathlineto{\pgfqpoint{1.861386in}{1.191260in}}%
\pgfpathlineto{\pgfqpoint{1.869622in}{1.191260in}}%
\pgfpathlineto{\pgfqpoint{1.871681in}{1.210682in}}%
\pgfpathlineto{\pgfqpoint{1.875799in}{1.210682in}}%
\pgfpathlineto{\pgfqpoint{1.881976in}{1.230104in}}%
\pgfpathlineto{\pgfqpoint{1.886094in}{1.230104in}}%
\pgfpathlineto{\pgfqpoint{1.888153in}{1.239815in}}%
\pgfpathlineto{\pgfqpoint{1.890212in}{1.278659in}}%
\pgfpathlineto{\pgfqpoint{1.896389in}{1.298081in}}%
\pgfpathlineto{\pgfqpoint{1.898447in}{1.356347in}}%
\pgfpathlineto{\pgfqpoint{1.900506in}{1.346636in}}%
\pgfpathlineto{\pgfqpoint{1.902565in}{1.327214in}}%
\pgfpathlineto{\pgfqpoint{1.904624in}{1.346636in}}%
\pgfpathlineto{\pgfqpoint{1.910801in}{1.346636in}}%
\pgfpathlineto{\pgfqpoint{1.912860in}{1.356347in}}%
\pgfpathlineto{\pgfqpoint{1.914919in}{1.356347in}}%
\pgfpathlineto{\pgfqpoint{1.919037in}{1.375769in}}%
\pgfpathlineto{\pgfqpoint{1.925214in}{1.375769in}}%
\pgfpathlineto{\pgfqpoint{1.927273in}{1.385480in}}%
\pgfpathlineto{\pgfqpoint{1.929332in}{1.336925in}}%
\pgfpathlineto{\pgfqpoint{1.933450in}{1.298081in}}%
\pgfpathlineto{\pgfqpoint{1.939626in}{1.307792in}}%
\pgfpathlineto{\pgfqpoint{1.941685in}{1.317503in}}%
\pgfpathlineto{\pgfqpoint{1.943744in}{1.317503in}}%
\pgfpathlineto{\pgfqpoint{1.945803in}{1.307792in}}%
\pgfpathlineto{\pgfqpoint{1.954039in}{1.327214in}}%
\pgfpathlineto{\pgfqpoint{1.960216in}{1.268948in}}%
\pgfpathlineto{\pgfqpoint{1.962275in}{1.298081in}}%
\pgfpathlineto{\pgfqpoint{1.968452in}{1.268948in}}%
\pgfpathlineto{\pgfqpoint{1.970511in}{1.239815in}}%
\pgfpathlineto{\pgfqpoint{1.972570in}{1.230104in}}%
\pgfpathlineto{\pgfqpoint{1.974628in}{1.200971in}}%
\pgfpathlineto{\pgfqpoint{1.976687in}{1.220393in}}%
\pgfpathlineto{\pgfqpoint{1.982864in}{1.210682in}}%
\pgfpathlineto{\pgfqpoint{1.986982in}{1.230104in}}%
\pgfpathlineto{\pgfqpoint{1.989041in}{1.230104in}}%
\pgfpathlineto{\pgfqpoint{1.991100in}{1.210682in}}%
\pgfpathlineto{\pgfqpoint{1.997277in}{1.200971in}}%
\pgfpathlineto{\pgfqpoint{2.001395in}{1.220393in}}%
\pgfpathlineto{\pgfqpoint{2.003454in}{1.239815in}}%
\pgfpathlineto{\pgfqpoint{2.005513in}{1.239815in}}%
\pgfpathlineto{\pgfqpoint{2.011689in}{1.249526in}}%
\pgfpathlineto{\pgfqpoint{2.013748in}{1.288370in}}%
\pgfpathlineto{\pgfqpoint{2.017866in}{1.239815in}}%
\pgfpathlineto{\pgfqpoint{2.019925in}{1.239815in}}%
\pgfpathlineto{\pgfqpoint{2.026102in}{1.230104in}}%
\pgfpathlineto{\pgfqpoint{2.030220in}{1.200971in}}%
\pgfpathlineto{\pgfqpoint{2.032279in}{1.191260in}}%
\pgfpathlineto{\pgfqpoint{2.040515in}{1.191260in}}%
\pgfpathlineto{\pgfqpoint{2.048751in}{1.230104in}}%
\pgfpathlineto{\pgfqpoint{2.056986in}{1.259237in}}%
\pgfpathlineto{\pgfqpoint{2.059045in}{1.307792in}}%
\pgfpathlineto{\pgfqpoint{2.061104in}{1.317503in}}%
\pgfpathlineto{\pgfqpoint{2.063163in}{1.346636in}}%
\pgfpathlineto{\pgfqpoint{2.069340in}{1.366058in}}%
\pgfpathlineto{\pgfqpoint{2.071399in}{1.366058in}}%
\pgfpathlineto{\pgfqpoint{2.075517in}{1.327214in}}%
\pgfpathlineto{\pgfqpoint{2.077576in}{1.356347in}}%
\pgfpathlineto{\pgfqpoint{2.085812in}{1.356347in}}%
\pgfpathlineto{\pgfqpoint{2.087870in}{1.375769in}}%
\pgfpathlineto{\pgfqpoint{2.089929in}{1.356347in}}%
\pgfpathlineto{\pgfqpoint{2.091988in}{1.278659in}}%
\pgfpathlineto{\pgfqpoint{2.098165in}{1.278659in}}%
\pgfpathlineto{\pgfqpoint{2.100224in}{1.268948in}}%
\pgfpathlineto{\pgfqpoint{2.102283in}{1.278659in}}%
\pgfpathlineto{\pgfqpoint{2.104342in}{1.268948in}}%
\pgfpathlineto{\pgfqpoint{2.106401in}{1.249526in}}%
\pgfpathlineto{\pgfqpoint{2.112578in}{1.230104in}}%
\pgfpathlineto{\pgfqpoint{2.114637in}{1.230104in}}%
\pgfpathlineto{\pgfqpoint{2.116696in}{1.200971in}}%
\pgfpathlineto{\pgfqpoint{2.118755in}{1.210682in}}%
\pgfpathlineto{\pgfqpoint{2.120814in}{1.191260in}}%
\pgfpathlineto{\pgfqpoint{2.129049in}{1.249526in}}%
\pgfpathlineto{\pgfqpoint{2.131108in}{1.239815in}}%
\pgfpathlineto{\pgfqpoint{2.133167in}{1.259237in}}%
\pgfpathlineto{\pgfqpoint{2.135226in}{1.162127in}}%
\pgfpathlineto{\pgfqpoint{2.141403in}{1.132994in}}%
\pgfpathlineto{\pgfqpoint{2.143462in}{1.132994in}}%
\pgfpathlineto{\pgfqpoint{2.145521in}{1.142705in}}%
\pgfpathlineto{\pgfqpoint{2.147580in}{1.132994in}}%
\pgfpathlineto{\pgfqpoint{2.149639in}{1.132994in}}%
\pgfpathlineto{\pgfqpoint{2.157875in}{1.123283in}}%
\pgfpathlineto{\pgfqpoint{2.161993in}{1.152416in}}%
\pgfpathlineto{\pgfqpoint{2.164051in}{1.162127in}}%
\pgfpathlineto{\pgfqpoint{2.170228in}{1.181549in}}%
\pgfpathlineto{\pgfqpoint{2.172287in}{1.200971in}}%
\pgfpathlineto{\pgfqpoint{2.174346in}{1.191260in}}%
\pgfpathlineto{\pgfqpoint{2.176405in}{1.210682in}}%
\pgfpathlineto{\pgfqpoint{2.178464in}{1.200971in}}%
\pgfpathlineto{\pgfqpoint{2.184641in}{1.200971in}}%
\pgfpathlineto{\pgfqpoint{2.186700in}{1.239815in}}%
\pgfpathlineto{\pgfqpoint{2.188759in}{1.239815in}}%
\pgfpathlineto{\pgfqpoint{2.190818in}{1.220393in}}%
\pgfpathlineto{\pgfqpoint{2.192877in}{1.230104in}}%
\pgfpathlineto{\pgfqpoint{2.201112in}{1.230104in}}%
\pgfpathlineto{\pgfqpoint{2.203171in}{1.210682in}}%
\pgfpathlineto{\pgfqpoint{2.205230in}{1.210682in}}%
\pgfpathlineto{\pgfqpoint{2.207289in}{1.181549in}}%
\pgfpathlineto{\pgfqpoint{2.215525in}{1.181549in}}%
\pgfpathlineto{\pgfqpoint{2.217584in}{1.210682in}}%
\pgfpathlineto{\pgfqpoint{2.219643in}{1.191260in}}%
\pgfpathlineto{\pgfqpoint{2.221702in}{1.239815in}}%
\pgfpathlineto{\pgfqpoint{2.227879in}{1.249526in}}%
\pgfpathlineto{\pgfqpoint{2.229938in}{1.230104in}}%
\pgfpathlineto{\pgfqpoint{2.234056in}{1.230104in}}%
\pgfpathlineto{\pgfqpoint{2.236115in}{1.239815in}}%
\pgfpathlineto{\pgfqpoint{2.242291in}{1.239815in}}%
\pgfpathlineto{\pgfqpoint{2.246409in}{1.259237in}}%
\pgfpathlineto{\pgfqpoint{2.248468in}{1.259237in}}%
\pgfpathlineto{\pgfqpoint{2.250527in}{1.268948in}}%
\pgfpathlineto{\pgfqpoint{2.256704in}{1.259237in}}%
\pgfpathlineto{\pgfqpoint{2.258763in}{1.259237in}}%
\pgfpathlineto{\pgfqpoint{2.262881in}{1.278659in}}%
\pgfpathlineto{\pgfqpoint{2.264940in}{1.298081in}}%
\pgfpathlineto{\pgfqpoint{2.271117in}{1.298081in}}%
\pgfpathlineto{\pgfqpoint{2.273176in}{1.288370in}}%
\pgfpathlineto{\pgfqpoint{2.275235in}{1.288370in}}%
\pgfpathlineto{\pgfqpoint{2.279352in}{1.268948in}}%
\pgfpathlineto{\pgfqpoint{2.287588in}{1.239815in}}%
\pgfpathlineto{\pgfqpoint{2.289647in}{1.249526in}}%
\pgfpathlineto{\pgfqpoint{2.291706in}{1.249526in}}%
\pgfpathlineto{\pgfqpoint{2.293765in}{1.259237in}}%
\pgfpathlineto{\pgfqpoint{2.299942in}{1.249526in}}%
\pgfpathlineto{\pgfqpoint{2.302001in}{1.307792in}}%
\pgfpathlineto{\pgfqpoint{2.304060in}{1.298081in}}%
\pgfpathlineto{\pgfqpoint{2.306119in}{1.278659in}}%
\pgfpathlineto{\pgfqpoint{2.308178in}{1.288370in}}%
\pgfpathlineto{\pgfqpoint{2.314355in}{1.278659in}}%
\pgfpathlineto{\pgfqpoint{2.316413in}{1.288370in}}%
\pgfpathlineto{\pgfqpoint{2.318472in}{1.288370in}}%
\pgfpathlineto{\pgfqpoint{2.320531in}{1.278659in}}%
\pgfpathlineto{\pgfqpoint{2.322590in}{1.278659in}}%
\pgfpathlineto{\pgfqpoint{2.328767in}{1.259237in}}%
\pgfpathlineto{\pgfqpoint{2.330826in}{1.259237in}}%
\pgfpathlineto{\pgfqpoint{2.332885in}{1.278659in}}%
\pgfpathlineto{\pgfqpoint{2.334944in}{1.268948in}}%
\pgfpathlineto{\pgfqpoint{2.337003in}{1.268948in}}%
\pgfpathlineto{\pgfqpoint{2.347298in}{1.327214in}}%
\pgfpathlineto{\pgfqpoint{2.349357in}{1.327214in}}%
\pgfpathlineto{\pgfqpoint{2.351416in}{1.336925in}}%
\pgfpathlineto{\pgfqpoint{2.359651in}{1.366058in}}%
\pgfpathlineto{\pgfqpoint{2.361710in}{1.356347in}}%
\pgfpathlineto{\pgfqpoint{2.363769in}{1.336925in}}%
\pgfpathlineto{\pgfqpoint{2.365828in}{1.336925in}}%
\pgfpathlineto{\pgfqpoint{2.372005in}{1.327214in}}%
\pgfpathlineto{\pgfqpoint{2.374064in}{1.336925in}}%
\pgfpathlineto{\pgfqpoint{2.376123in}{1.327214in}}%
\pgfpathlineto{\pgfqpoint{2.378182in}{1.336925in}}%
\pgfpathlineto{\pgfqpoint{2.388477in}{1.336925in}}%
\pgfpathlineto{\pgfqpoint{2.392594in}{1.356347in}}%
\pgfpathlineto{\pgfqpoint{2.394653in}{1.336925in}}%
\pgfpathlineto{\pgfqpoint{2.400830in}{1.336925in}}%
\pgfpathlineto{\pgfqpoint{2.404948in}{1.317503in}}%
\pgfpathlineto{\pgfqpoint{2.407007in}{1.317503in}}%
\pgfpathlineto{\pgfqpoint{2.409066in}{1.298081in}}%
\pgfpathlineto{\pgfqpoint{2.415243in}{1.307792in}}%
\pgfpathlineto{\pgfqpoint{2.417302in}{1.385480in}}%
\pgfpathlineto{\pgfqpoint{2.419361in}{1.395191in}}%
\pgfpathlineto{\pgfqpoint{2.421420in}{1.395191in}}%
\pgfpathlineto{\pgfqpoint{2.431714in}{1.453457in}}%
\pgfpathlineto{\pgfqpoint{2.433773in}{1.434035in}}%
\pgfpathlineto{\pgfqpoint{2.435832in}{1.443746in}}%
\pgfpathlineto{\pgfqpoint{2.437891in}{1.443746in}}%
\pgfpathlineto{\pgfqpoint{2.444068in}{1.453457in}}%
\pgfpathlineto{\pgfqpoint{2.446127in}{1.453457in}}%
\pgfpathlineto{\pgfqpoint{2.448186in}{1.472879in}}%
\pgfpathlineto{\pgfqpoint{2.452304in}{1.482590in}}%
\pgfpathlineto{\pgfqpoint{2.460540in}{1.453457in}}%
\pgfpathlineto{\pgfqpoint{2.464658in}{1.492301in}}%
\pgfpathlineto{\pgfqpoint{2.466716in}{1.472879in}}%
\pgfpathlineto{\pgfqpoint{2.474952in}{1.502012in}}%
\pgfpathlineto{\pgfqpoint{2.477011in}{1.521434in}}%
\pgfpathlineto{\pgfqpoint{2.479070in}{1.511723in}}%
\pgfpathlineto{\pgfqpoint{2.481129in}{1.521434in}}%
\pgfpathlineto{\pgfqpoint{2.487306in}{1.521434in}}%
\pgfpathlineto{\pgfqpoint{2.491424in}{1.589410in}}%
\pgfpathlineto{\pgfqpoint{2.493483in}{1.579699in}}%
\pgfpathlineto{\pgfqpoint{2.495542in}{1.579699in}}%
\pgfpathlineto{\pgfqpoint{2.501719in}{1.569988in}}%
\pgfpathlineto{\pgfqpoint{2.503778in}{1.569988in}}%
\pgfpathlineto{\pgfqpoint{2.507895in}{1.540855in}}%
\pgfpathlineto{\pgfqpoint{2.509954in}{1.540855in}}%
\pgfpathlineto{\pgfqpoint{2.518190in}{1.560277in}}%
\pgfpathlineto{\pgfqpoint{2.520249in}{1.569988in}}%
\pgfpathlineto{\pgfqpoint{2.522308in}{1.521434in}}%
\pgfpathlineto{\pgfqpoint{2.524367in}{1.521434in}}%
\pgfpathlineto{\pgfqpoint{2.532603in}{1.560277in}}%
\pgfpathlineto{\pgfqpoint{2.534662in}{1.540855in}}%
\pgfpathlineto{\pgfqpoint{2.536721in}{1.502012in}}%
\pgfpathlineto{\pgfqpoint{2.538780in}{1.521434in}}%
\pgfpathlineto{\pgfqpoint{2.544956in}{1.492301in}}%
\pgfpathlineto{\pgfqpoint{2.549074in}{1.492301in}}%
\pgfpathlineto{\pgfqpoint{2.551133in}{1.482590in}}%
\pgfpathlineto{\pgfqpoint{2.553192in}{1.492301in}}%
\pgfpathlineto{\pgfqpoint{2.561428in}{1.472879in}}%
\pgfpathlineto{\pgfqpoint{2.565546in}{1.502012in}}%
\pgfpathlineto{\pgfqpoint{2.573782in}{1.463168in}}%
\pgfpathlineto{\pgfqpoint{2.577900in}{1.492301in}}%
\pgfpathlineto{\pgfqpoint{2.582017in}{1.492301in}}%
\pgfpathlineto{\pgfqpoint{2.588194in}{1.482590in}}%
\pgfpathlineto{\pgfqpoint{2.590253in}{1.511723in}}%
\pgfpathlineto{\pgfqpoint{2.592312in}{1.502012in}}%
\pgfpathlineto{\pgfqpoint{2.594371in}{1.511723in}}%
\pgfpathlineto{\pgfqpoint{2.596430in}{1.492301in}}%
\pgfpathlineto{\pgfqpoint{2.602607in}{1.463168in}}%
\pgfpathlineto{\pgfqpoint{2.604666in}{1.472879in}}%
\pgfpathlineto{\pgfqpoint{2.606725in}{1.463168in}}%
\pgfpathlineto{\pgfqpoint{2.610843in}{1.482590in}}%
\pgfpathlineto{\pgfqpoint{2.617020in}{1.492301in}}%
\pgfpathlineto{\pgfqpoint{2.621137in}{1.531145in}}%
\pgfpathlineto{\pgfqpoint{2.623196in}{1.492301in}}%
\pgfpathlineto{\pgfqpoint{2.625255in}{1.492301in}}%
\pgfpathlineto{\pgfqpoint{2.633491in}{1.502012in}}%
\pgfpathlineto{\pgfqpoint{2.639668in}{1.472879in}}%
\pgfpathlineto{\pgfqpoint{2.645845in}{1.482590in}}%
\pgfpathlineto{\pgfqpoint{2.649963in}{1.589410in}}%
\pgfpathlineto{\pgfqpoint{2.652022in}{1.647676in}}%
\pgfpathlineto{\pgfqpoint{2.654081in}{1.647676in}}%
\pgfpathlineto{\pgfqpoint{2.660257in}{1.637965in}}%
\pgfpathlineto{\pgfqpoint{2.662316in}{1.686520in}}%
\pgfpathlineto{\pgfqpoint{2.666434in}{1.705942in}}%
\pgfpathlineto{\pgfqpoint{2.668493in}{1.696231in}}%
\pgfpathlineto{\pgfqpoint{2.674670in}{1.725364in}}%
\pgfpathlineto{\pgfqpoint{2.676729in}{1.725364in}}%
\pgfpathlineto{\pgfqpoint{2.678788in}{1.686520in}}%
\pgfpathlineto{\pgfqpoint{2.682906in}{1.667098in}}%
\pgfpathlineto{\pgfqpoint{2.689083in}{1.676809in}}%
\pgfpathlineto{\pgfqpoint{2.693200in}{1.657387in}}%
\pgfpathlineto{\pgfqpoint{2.695259in}{1.657387in}}%
\pgfpathlineto{\pgfqpoint{2.697318in}{1.667098in}}%
\pgfpathlineto{\pgfqpoint{2.703495in}{1.667098in}}%
\pgfpathlineto{\pgfqpoint{2.705554in}{1.696231in}}%
\pgfpathlineto{\pgfqpoint{2.707613in}{1.705942in}}%
\pgfpathlineto{\pgfqpoint{2.709672in}{1.696231in}}%
\pgfpathlineto{\pgfqpoint{2.711731in}{1.696231in}}%
\pgfpathlineto{\pgfqpoint{2.717908in}{1.686520in}}%
\pgfpathlineto{\pgfqpoint{2.719967in}{1.696231in}}%
\pgfpathlineto{\pgfqpoint{2.722026in}{1.696231in}}%
\pgfpathlineto{\pgfqpoint{2.726144in}{1.744786in}}%
\pgfpathlineto{\pgfqpoint{2.732320in}{1.735075in}}%
\pgfpathlineto{\pgfqpoint{2.736438in}{1.705942in}}%
\pgfpathlineto{\pgfqpoint{2.738497in}{1.696231in}}%
\pgfpathlineto{\pgfqpoint{2.746733in}{1.705942in}}%
\pgfpathlineto{\pgfqpoint{2.748792in}{1.686520in}}%
\pgfpathlineto{\pgfqpoint{2.750851in}{1.686520in}}%
\pgfpathlineto{\pgfqpoint{2.752910in}{1.676809in}}%
\pgfpathlineto{\pgfqpoint{2.754969in}{1.657387in}}%
\pgfpathlineto{\pgfqpoint{2.761146in}{1.696231in}}%
\pgfpathlineto{\pgfqpoint{2.763205in}{1.754497in}}%
\pgfpathlineto{\pgfqpoint{2.767323in}{1.725364in}}%
\pgfpathlineto{\pgfqpoint{2.769381in}{1.735075in}}%
\pgfpathlineto{\pgfqpoint{2.775558in}{1.754497in}}%
\pgfpathlineto{\pgfqpoint{2.777617in}{1.744786in}}%
\pgfpathlineto{\pgfqpoint{2.781735in}{1.773919in}}%
\pgfpathlineto{\pgfqpoint{2.783794in}{1.764208in}}%
\pgfpathlineto{\pgfqpoint{2.789971in}{1.783630in}}%
\pgfpathlineto{\pgfqpoint{2.792030in}{1.803052in}}%
\pgfpathlineto{\pgfqpoint{2.794089in}{1.793341in}}%
\pgfpathlineto{\pgfqpoint{2.796148in}{1.793341in}}%
\pgfpathlineto{\pgfqpoint{2.798207in}{1.773919in}}%
\pgfpathlineto{\pgfqpoint{2.806443in}{1.773919in}}%
\pgfpathlineto{\pgfqpoint{2.808501in}{1.744786in}}%
\pgfpathlineto{\pgfqpoint{2.812619in}{1.764208in}}%
\pgfpathlineto{\pgfqpoint{2.818796in}{1.783630in}}%
\pgfpathlineto{\pgfqpoint{2.820855in}{1.803052in}}%
\pgfpathlineto{\pgfqpoint{2.822914in}{1.841896in}}%
\pgfpathlineto{\pgfqpoint{2.824973in}{1.822474in}}%
\pgfpathlineto{\pgfqpoint{2.827032in}{1.832185in}}%
\pgfpathlineto{\pgfqpoint{2.835268in}{1.822474in}}%
\pgfpathlineto{\pgfqpoint{2.837327in}{1.832185in}}%
\pgfpathlineto{\pgfqpoint{2.839386in}{1.822474in}}%
\pgfpathlineto{\pgfqpoint{2.849680in}{1.822474in}}%
\pgfpathlineto{\pgfqpoint{2.851739in}{1.832185in}}%
\pgfpathlineto{\pgfqpoint{2.855857in}{1.861318in}}%
\pgfpathlineto{\pgfqpoint{2.862034in}{1.851607in}}%
\pgfpathlineto{\pgfqpoint{2.864093in}{1.880740in}}%
\pgfpathlineto{\pgfqpoint{2.866152in}{1.861318in}}%
\pgfpathlineto{\pgfqpoint{2.868211in}{1.871029in}}%
\pgfpathlineto{\pgfqpoint{2.870270in}{1.871029in}}%
\pgfpathlineto{\pgfqpoint{2.876447in}{1.880740in}}%
\pgfpathlineto{\pgfqpoint{2.882623in}{1.880740in}}%
\pgfpathlineto{\pgfqpoint{2.884682in}{1.871029in}}%
\pgfpathlineto{\pgfqpoint{2.890859in}{1.861318in}}%
\pgfpathlineto{\pgfqpoint{2.892918in}{1.880740in}}%
\pgfpathlineto{\pgfqpoint{2.894977in}{1.871029in}}%
\pgfpathlineto{\pgfqpoint{2.899095in}{1.900162in}}%
\pgfpathlineto{\pgfqpoint{2.909390in}{1.900162in}}%
\pgfpathlineto{\pgfqpoint{2.913508in}{1.880740in}}%
\pgfpathlineto{\pgfqpoint{2.919685in}{1.890451in}}%
\pgfpathlineto{\pgfqpoint{2.921743in}{1.861318in}}%
\pgfpathlineto{\pgfqpoint{2.923802in}{1.871029in}}%
\pgfpathlineto{\pgfqpoint{2.925861in}{1.890451in}}%
\pgfpathlineto{\pgfqpoint{2.927920in}{1.880740in}}%
\pgfpathlineto{\pgfqpoint{2.934097in}{1.880740in}}%
\pgfpathlineto{\pgfqpoint{2.936156in}{1.851607in}}%
\pgfpathlineto{\pgfqpoint{2.938215in}{1.890451in}}%
\pgfpathlineto{\pgfqpoint{2.940274in}{1.880740in}}%
\pgfpathlineto{\pgfqpoint{2.942333in}{1.880740in}}%
\pgfpathlineto{\pgfqpoint{2.948510in}{1.890451in}}%
\pgfpathlineto{\pgfqpoint{2.950569in}{1.900162in}}%
\pgfpathlineto{\pgfqpoint{2.954687in}{1.880740in}}%
\pgfpathlineto{\pgfqpoint{2.956746in}{1.880740in}}%
\pgfpathlineto{\pgfqpoint{2.962922in}{1.890451in}}%
\pgfpathlineto{\pgfqpoint{2.964981in}{1.880740in}}%
\pgfpathlineto{\pgfqpoint{2.967040in}{1.900162in}}%
\pgfpathlineto{\pgfqpoint{2.969099in}{1.880740in}}%
\pgfpathlineto{\pgfqpoint{2.971158in}{1.890451in}}%
\pgfpathlineto{\pgfqpoint{2.977335in}{1.880740in}}%
\pgfpathlineto{\pgfqpoint{2.979394in}{1.900162in}}%
\pgfpathlineto{\pgfqpoint{2.981453in}{1.871029in}}%
\pgfpathlineto{\pgfqpoint{2.983512in}{1.880740in}}%
\pgfpathlineto{\pgfqpoint{2.985571in}{1.871029in}}%
\pgfpathlineto{\pgfqpoint{2.991748in}{1.890451in}}%
\pgfpathlineto{\pgfqpoint{2.993807in}{1.890451in}}%
\pgfpathlineto{\pgfqpoint{2.995866in}{1.900162in}}%
\pgfpathlineto{\pgfqpoint{2.999983in}{1.900162in}}%
\pgfpathlineto{\pgfqpoint{3.006160in}{1.890451in}}%
\pgfpathlineto{\pgfqpoint{3.008219in}{1.900162in}}%
\pgfpathlineto{\pgfqpoint{3.010278in}{1.880740in}}%
\pgfpathlineto{\pgfqpoint{3.012337in}{1.890451in}}%
\pgfpathlineto{\pgfqpoint{3.014396in}{1.890451in}}%
\pgfpathlineto{\pgfqpoint{3.020573in}{1.900162in}}%
\pgfpathlineto{\pgfqpoint{3.022632in}{1.890451in}}%
\pgfpathlineto{\pgfqpoint{3.026750in}{1.890451in}}%
\pgfpathlineto{\pgfqpoint{3.028809in}{1.900162in}}%
\pgfpathlineto{\pgfqpoint{3.037044in}{1.890451in}}%
\pgfpathlineto{\pgfqpoint{3.039103in}{1.900162in}}%
\pgfpathlineto{\pgfqpoint{3.041162in}{1.871029in}}%
\pgfpathlineto{\pgfqpoint{3.043221in}{1.880740in}}%
\pgfpathlineto{\pgfqpoint{3.049398in}{1.900162in}}%
\pgfpathlineto{\pgfqpoint{3.051457in}{1.929295in}}%
\pgfpathlineto{\pgfqpoint{3.053516in}{1.929295in}}%
\pgfpathlineto{\pgfqpoint{3.055575in}{1.939006in}}%
\pgfpathlineto{\pgfqpoint{3.057634in}{1.958428in}}%
\pgfpathlineto{\pgfqpoint{3.063811in}{1.958428in}}%
\pgfpathlineto{\pgfqpoint{3.067929in}{1.977850in}}%
\pgfpathlineto{\pgfqpoint{3.072046in}{1.958428in}}%
\pgfpathlineto{\pgfqpoint{3.078223in}{1.958428in}}%
\pgfpathlineto{\pgfqpoint{3.080282in}{1.968139in}}%
\pgfpathlineto{\pgfqpoint{3.082341in}{1.987561in}}%
\pgfpathlineto{\pgfqpoint{3.084400in}{1.968139in}}%
\pgfpathlineto{\pgfqpoint{3.092636in}{1.968139in}}%
\pgfpathlineto{\pgfqpoint{3.096754in}{1.987561in}}%
\pgfpathlineto{\pgfqpoint{3.098813in}{2.006983in}}%
\pgfpathlineto{\pgfqpoint{3.100872in}{2.006983in}}%
\pgfpathlineto{\pgfqpoint{3.109108in}{2.074960in}}%
\pgfpathlineto{\pgfqpoint{3.111166in}{2.055538in}}%
\pgfpathlineto{\pgfqpoint{3.113225in}{2.065249in}}%
\pgfpathlineto{\pgfqpoint{3.115284in}{2.045827in}}%
\pgfpathlineto{\pgfqpoint{3.121461in}{2.074960in}}%
\pgfpathlineto{\pgfqpoint{3.123520in}{2.065249in}}%
\pgfpathlineto{\pgfqpoint{3.125579in}{2.074960in}}%
\pgfpathlineto{\pgfqpoint{3.127638in}{2.065249in}}%
\pgfpathlineto{\pgfqpoint{3.129697in}{2.084671in}}%
\pgfpathlineto{\pgfqpoint{3.135874in}{2.074960in}}%
\pgfpathlineto{\pgfqpoint{3.137933in}{2.084671in}}%
\pgfpathlineto{\pgfqpoint{3.142051in}{2.084671in}}%
\pgfpathlineto{\pgfqpoint{3.144110in}{2.074960in}}%
\pgfpathlineto{\pgfqpoint{3.150286in}{2.074960in}}%
\pgfpathlineto{\pgfqpoint{3.152345in}{2.084671in}}%
\pgfpathlineto{\pgfqpoint{3.154404in}{2.113803in}}%
\pgfpathlineto{\pgfqpoint{3.156463in}{2.113803in}}%
\pgfpathlineto{\pgfqpoint{3.158522in}{2.142936in}}%
\pgfpathlineto{\pgfqpoint{3.164699in}{2.152647in}}%
\pgfpathlineto{\pgfqpoint{3.166758in}{2.142936in}}%
\pgfpathlineto{\pgfqpoint{3.168817in}{2.181780in}}%
\pgfpathlineto{\pgfqpoint{3.170876in}{2.181780in}}%
\pgfpathlineto{\pgfqpoint{3.172935in}{2.191491in}}%
\pgfpathlineto{\pgfqpoint{3.179112in}{2.201202in}}%
\pgfpathlineto{\pgfqpoint{3.183230in}{2.201202in}}%
\pgfpathlineto{\pgfqpoint{3.185289in}{2.240046in}}%
\pgfpathlineto{\pgfqpoint{3.187347in}{2.249757in}}%
\pgfpathlineto{\pgfqpoint{3.193524in}{2.269179in}}%
\pgfpathlineto{\pgfqpoint{3.195583in}{2.269179in}}%
\pgfpathlineto{\pgfqpoint{3.197642in}{2.259468in}}%
\pgfpathlineto{\pgfqpoint{3.201760in}{2.259468in}}%
\pgfpathlineto{\pgfqpoint{3.207937in}{2.269179in}}%
\pgfpathlineto{\pgfqpoint{3.209996in}{2.259468in}}%
\pgfpathlineto{\pgfqpoint{3.212055in}{2.259468in}}%
\pgfpathlineto{\pgfqpoint{3.214114in}{2.269179in}}%
\pgfpathlineto{\pgfqpoint{3.216173in}{2.269179in}}%
\pgfpathlineto{\pgfqpoint{3.222350in}{2.308023in}}%
\pgfpathlineto{\pgfqpoint{3.224408in}{2.288601in}}%
\pgfpathlineto{\pgfqpoint{3.226467in}{2.327445in}}%
\pgfpathlineto{\pgfqpoint{3.228526in}{2.317734in}}%
\pgfpathlineto{\pgfqpoint{3.230585in}{2.298312in}}%
\pgfpathlineto{\pgfqpoint{3.238821in}{2.346867in}}%
\pgfpathlineto{\pgfqpoint{3.240880in}{2.327445in}}%
\pgfpathlineto{\pgfqpoint{3.244998in}{2.356578in}}%
\pgfpathlineto{\pgfqpoint{3.251175in}{2.346867in}}%
\pgfpathlineto{\pgfqpoint{3.257352in}{2.376000in}}%
\pgfpathlineto{\pgfqpoint{3.259411in}{2.376000in}}%
\pgfpathlineto{\pgfqpoint{3.267646in}{2.395422in}}%
\pgfpathlineto{\pgfqpoint{3.269705in}{2.395422in}}%
\pgfpathlineto{\pgfqpoint{3.271764in}{2.405133in}}%
\pgfpathlineto{\pgfqpoint{3.273823in}{2.405133in}}%
\pgfpathlineto{\pgfqpoint{3.282059in}{2.473110in}}%
\pgfpathlineto{\pgfqpoint{3.284118in}{2.453688in}}%
\pgfpathlineto{\pgfqpoint{3.286177in}{2.463399in}}%
\pgfpathlineto{\pgfqpoint{3.288236in}{2.443977in}}%
\pgfpathlineto{\pgfqpoint{3.294413in}{2.434266in}}%
\pgfpathlineto{\pgfqpoint{3.296472in}{2.424555in}}%
\pgfpathlineto{\pgfqpoint{3.298531in}{2.424555in}}%
\pgfpathlineto{\pgfqpoint{3.300589in}{2.414844in}}%
\pgfpathlineto{\pgfqpoint{3.302648in}{2.424555in}}%
\pgfpathlineto{\pgfqpoint{3.310884in}{2.434266in}}%
\pgfpathlineto{\pgfqpoint{3.323238in}{2.434266in}}%
\pgfpathlineto{\pgfqpoint{3.325297in}{2.424555in}}%
\pgfpathlineto{\pgfqpoint{3.329415in}{2.443977in}}%
\pgfpathlineto{\pgfqpoint{3.337650in}{2.443977in}}%
\pgfpathlineto{\pgfqpoint{3.339709in}{2.521665in}}%
\pgfpathlineto{\pgfqpoint{3.341768in}{2.541087in}}%
\pgfpathlineto{\pgfqpoint{3.352063in}{2.492532in}}%
\pgfpathlineto{\pgfqpoint{3.354122in}{2.511954in}}%
\pgfpathlineto{\pgfqpoint{3.356181in}{2.550798in}}%
\pgfpathlineto{\pgfqpoint{3.358240in}{2.550798in}}%
\pgfpathlineto{\pgfqpoint{3.360299in}{2.531376in}}%
\pgfpathlineto{\pgfqpoint{3.368535in}{2.589642in}}%
\pgfpathlineto{\pgfqpoint{3.370594in}{2.618775in}}%
\pgfpathlineto{\pgfqpoint{3.374711in}{2.638197in}}%
\pgfpathlineto{\pgfqpoint{3.382947in}{2.647908in}}%
\pgfpathlineto{\pgfqpoint{3.385006in}{2.667329in}}%
\pgfpathlineto{\pgfqpoint{3.387065in}{2.657618in}}%
\pgfpathlineto{\pgfqpoint{3.389124in}{2.657618in}}%
\pgfpathlineto{\pgfqpoint{3.395301in}{2.667329in}}%
\pgfpathlineto{\pgfqpoint{3.397360in}{2.715884in}}%
\pgfpathlineto{\pgfqpoint{3.399419in}{2.706173in}}%
\pgfpathlineto{\pgfqpoint{3.401478in}{2.686751in}}%
\pgfpathlineto{\pgfqpoint{3.403537in}{2.696462in}}%
\pgfpathlineto{\pgfqpoint{3.411773in}{2.696462in}}%
\pgfpathlineto{\pgfqpoint{3.413831in}{2.686751in}}%
\pgfpathlineto{\pgfqpoint{3.415890in}{2.686751in}}%
\pgfpathlineto{\pgfqpoint{3.417949in}{2.667329in}}%
\pgfpathlineto{\pgfqpoint{3.424126in}{2.686751in}}%
\pgfpathlineto{\pgfqpoint{3.426185in}{2.667329in}}%
\pgfpathlineto{\pgfqpoint{3.432362in}{2.715884in}}%
\pgfpathlineto{\pgfqpoint{3.440598in}{2.715884in}}%
\pgfpathlineto{\pgfqpoint{3.444716in}{2.686751in}}%
\pgfpathlineto{\pgfqpoint{3.446775in}{2.677040in}}%
\pgfpathlineto{\pgfqpoint{3.452951in}{2.696462in}}%
\pgfpathlineto{\pgfqpoint{3.457069in}{2.754728in}}%
\pgfpathlineto{\pgfqpoint{3.459128in}{2.725595in}}%
\pgfpathlineto{\pgfqpoint{3.467364in}{2.715884in}}%
\pgfpathlineto{\pgfqpoint{3.469423in}{2.725595in}}%
\pgfpathlineto{\pgfqpoint{3.471482in}{2.706173in}}%
\pgfpathlineto{\pgfqpoint{3.473541in}{2.706173in}}%
\pgfpathlineto{\pgfqpoint{3.475600in}{2.696462in}}%
\pgfpathlineto{\pgfqpoint{3.485895in}{2.725595in}}%
\pgfpathlineto{\pgfqpoint{3.490012in}{2.754728in}}%
\pgfpathlineto{\pgfqpoint{3.496189in}{2.754728in}}%
\pgfpathlineto{\pgfqpoint{3.498248in}{2.793572in}}%
\pgfpathlineto{\pgfqpoint{3.500307in}{2.803283in}}%
\pgfpathlineto{\pgfqpoint{3.502366in}{2.842127in}}%
\pgfpathlineto{\pgfqpoint{3.510602in}{2.880971in}}%
\pgfpathlineto{\pgfqpoint{3.512661in}{2.880971in}}%
\pgfpathlineto{\pgfqpoint{3.514720in}{2.890682in}}%
\pgfpathlineto{\pgfqpoint{3.518838in}{2.871260in}}%
\pgfpathlineto{\pgfqpoint{3.525015in}{2.871260in}}%
\pgfpathlineto{\pgfqpoint{3.527073in}{2.890682in}}%
\pgfpathlineto{\pgfqpoint{3.529132in}{2.871260in}}%
\pgfpathlineto{\pgfqpoint{3.533250in}{2.871260in}}%
\pgfpathlineto{\pgfqpoint{3.539427in}{2.880971in}}%
\pgfpathlineto{\pgfqpoint{3.543545in}{2.900393in}}%
\pgfpathlineto{\pgfqpoint{3.545604in}{2.900393in}}%
\pgfpathlineto{\pgfqpoint{3.547663in}{2.910104in}}%
\pgfpathlineto{\pgfqpoint{3.553840in}{2.910104in}}%
\pgfpathlineto{\pgfqpoint{3.555899in}{2.939237in}}%
\pgfpathlineto{\pgfqpoint{3.557958in}{2.948948in}}%
\pgfpathlineto{\pgfqpoint{3.562076in}{2.948948in}}%
\pgfpathlineto{\pgfqpoint{3.568252in}{2.978081in}}%
\pgfpathlineto{\pgfqpoint{3.570311in}{2.968370in}}%
\pgfpathlineto{\pgfqpoint{3.572370in}{2.919815in}}%
\pgfpathlineto{\pgfqpoint{3.576488in}{2.900393in}}%
\pgfpathlineto{\pgfqpoint{3.584724in}{2.803283in}}%
\pgfpathlineto{\pgfqpoint{3.586783in}{2.861549in}}%
\pgfpathlineto{\pgfqpoint{3.588842in}{2.861549in}}%
\pgfpathlineto{\pgfqpoint{3.590901in}{2.910104in}}%
\pgfpathlineto{\pgfqpoint{3.597078in}{2.929526in}}%
\pgfpathlineto{\pgfqpoint{3.599137in}{2.948948in}}%
\pgfpathlineto{\pgfqpoint{3.601196in}{2.948948in}}%
\pgfpathlineto{\pgfqpoint{3.605313in}{2.929526in}}%
\pgfpathlineto{\pgfqpoint{3.611490in}{2.948948in}}%
\pgfpathlineto{\pgfqpoint{3.613549in}{2.939237in}}%
\pgfpathlineto{\pgfqpoint{3.615608in}{2.978081in}}%
\pgfpathlineto{\pgfqpoint{3.625903in}{2.978081in}}%
\pgfpathlineto{\pgfqpoint{3.627962in}{2.968370in}}%
\pgfpathlineto{\pgfqpoint{3.630021in}{2.987792in}}%
\pgfpathlineto{\pgfqpoint{3.634139in}{2.958659in}}%
\pgfpathlineto{\pgfqpoint{3.640315in}{2.968370in}}%
\pgfpathlineto{\pgfqpoint{3.642374in}{2.958659in}}%
\pgfpathlineto{\pgfqpoint{3.648551in}{2.958659in}}%
\pgfpathlineto{\pgfqpoint{3.654728in}{2.968370in}}%
\pgfpathlineto{\pgfqpoint{3.656787in}{2.958659in}}%
\pgfpathlineto{\pgfqpoint{3.660905in}{2.948948in}}%
\pgfpathlineto{\pgfqpoint{3.662964in}{2.968370in}}%
\pgfpathlineto{\pgfqpoint{3.669141in}{2.968370in}}%
\pgfpathlineto{\pgfqpoint{3.671200in}{2.987792in}}%
\pgfpathlineto{\pgfqpoint{3.673259in}{2.987792in}}%
\pgfpathlineto{\pgfqpoint{3.675318in}{3.016925in}}%
\pgfpathlineto{\pgfqpoint{3.677377in}{2.997503in}}%
\pgfpathlineto{\pgfqpoint{3.683553in}{3.016925in}}%
\pgfpathlineto{\pgfqpoint{3.685612in}{3.016925in}}%
\pgfpathlineto{\pgfqpoint{3.687671in}{3.055769in}}%
\pgfpathlineto{\pgfqpoint{3.689730in}{3.026636in}}%
\pgfpathlineto{\pgfqpoint{3.691789in}{3.036347in}}%
\pgfpathlineto{\pgfqpoint{3.697966in}{3.046058in}}%
\pgfpathlineto{\pgfqpoint{3.702084in}{3.046058in}}%
\pgfpathlineto{\pgfqpoint{3.704143in}{3.036347in}}%
\pgfpathlineto{\pgfqpoint{3.706202in}{3.055769in}}%
\pgfpathlineto{\pgfqpoint{3.712379in}{3.055769in}}%
\pgfpathlineto{\pgfqpoint{3.716496in}{3.075191in}}%
\pgfpathlineto{\pgfqpoint{3.718555in}{3.075191in}}%
\pgfpathlineto{\pgfqpoint{3.720614in}{3.055769in}}%
\pgfpathlineto{\pgfqpoint{3.726791in}{3.065480in}}%
\pgfpathlineto{\pgfqpoint{3.728850in}{3.075191in}}%
\pgfpathlineto{\pgfqpoint{3.730909in}{3.065480in}}%
\pgfpathlineto{\pgfqpoint{3.732968in}{3.065480in}}%
\pgfpathlineto{\pgfqpoint{3.735027in}{3.046058in}}%
\pgfpathlineto{\pgfqpoint{3.741204in}{3.046058in}}%
\pgfpathlineto{\pgfqpoint{3.745322in}{3.075191in}}%
\pgfpathlineto{\pgfqpoint{3.747381in}{3.075191in}}%
\pgfpathlineto{\pgfqpoint{3.749440in}{3.065480in}}%
\pgfpathlineto{\pgfqpoint{3.755616in}{3.055769in}}%
\pgfpathlineto{\pgfqpoint{3.757675in}{3.075191in}}%
\pgfpathlineto{\pgfqpoint{3.759734in}{3.055769in}}%
\pgfpathlineto{\pgfqpoint{3.761793in}{3.055769in}}%
\pgfpathlineto{\pgfqpoint{3.770029in}{3.094613in}}%
\pgfpathlineto{\pgfqpoint{3.772088in}{3.094613in}}%
\pgfpathlineto{\pgfqpoint{3.774147in}{3.104324in}}%
\pgfpathlineto{\pgfqpoint{3.778265in}{3.084902in}}%
\pgfpathlineto{\pgfqpoint{3.786501in}{3.114035in}}%
\pgfpathlineto{\pgfqpoint{3.788560in}{3.114035in}}%
\pgfpathlineto{\pgfqpoint{3.790619in}{3.123746in}}%
\pgfpathlineto{\pgfqpoint{3.792677in}{3.152879in}}%
\pgfpathlineto{\pgfqpoint{3.798854in}{3.162590in}}%
\pgfpathlineto{\pgfqpoint{3.802972in}{3.182012in}}%
\pgfpathlineto{\pgfqpoint{3.805031in}{3.172301in}}%
\pgfpathlineto{\pgfqpoint{3.807090in}{3.182012in}}%
\pgfpathlineto{\pgfqpoint{3.813267in}{3.191723in}}%
\pgfpathlineto{\pgfqpoint{3.815326in}{3.201434in}}%
\pgfpathlineto{\pgfqpoint{3.821503in}{3.201434in}}%
\pgfpathlineto{\pgfqpoint{3.827680in}{3.220855in}}%
\pgfpathlineto{\pgfqpoint{3.831797in}{3.201434in}}%
\pgfpathlineto{\pgfqpoint{3.833856in}{3.201434in}}%
\pgfpathlineto{\pgfqpoint{3.835915in}{3.211145in}}%
\pgfpathlineto{\pgfqpoint{3.842092in}{3.220855in}}%
\pgfpathlineto{\pgfqpoint{3.850328in}{3.259699in}}%
\pgfpathlineto{\pgfqpoint{3.858564in}{3.269410in}}%
\pgfpathlineto{\pgfqpoint{3.860623in}{3.288832in}}%
\pgfpathlineto{\pgfqpoint{3.862682in}{3.279121in}}%
\pgfpathlineto{\pgfqpoint{3.864741in}{3.279121in}}%
\pgfpathlineto{\pgfqpoint{3.870917in}{3.288832in}}%
\pgfpathlineto{\pgfqpoint{3.872976in}{3.279121in}}%
\pgfpathlineto{\pgfqpoint{3.875035in}{3.279121in}}%
\pgfpathlineto{\pgfqpoint{3.877094in}{3.288832in}}%
\pgfpathlineto{\pgfqpoint{3.879153in}{3.288832in}}%
\pgfpathlineto{\pgfqpoint{3.885330in}{3.298543in}}%
\pgfpathlineto{\pgfqpoint{3.887389in}{3.288832in}}%
\pgfpathlineto{\pgfqpoint{3.889448in}{3.259699in}}%
\pgfpathlineto{\pgfqpoint{3.891507in}{3.279121in}}%
\pgfpathlineto{\pgfqpoint{3.893566in}{3.249988in}}%
\pgfpathlineto{\pgfqpoint{3.899743in}{3.259699in}}%
\pgfpathlineto{\pgfqpoint{3.903861in}{3.308254in}}%
\pgfpathlineto{\pgfqpoint{3.905919in}{3.288832in}}%
\pgfpathlineto{\pgfqpoint{3.907978in}{3.317965in}}%
\pgfpathlineto{\pgfqpoint{3.914155in}{3.327676in}}%
\pgfpathlineto{\pgfqpoint{3.916214in}{3.337387in}}%
\pgfpathlineto{\pgfqpoint{3.918273in}{3.356809in}}%
\pgfpathlineto{\pgfqpoint{3.920332in}{3.356809in}}%
\pgfpathlineto{\pgfqpoint{3.922391in}{3.347098in}}%
\pgfpathlineto{\pgfqpoint{3.930627in}{3.337387in}}%
\pgfpathlineto{\pgfqpoint{3.934745in}{3.317965in}}%
\pgfpathlineto{\pgfqpoint{3.936804in}{3.298543in}}%
\pgfpathlineto{\pgfqpoint{3.942980in}{3.279121in}}%
\pgfpathlineto{\pgfqpoint{3.945039in}{3.288832in}}%
\pgfpathlineto{\pgfqpoint{3.951216in}{3.288832in}}%
\pgfpathlineto{\pgfqpoint{3.957393in}{3.317965in}}%
\pgfpathlineto{\pgfqpoint{3.959452in}{3.317965in}}%
\pgfpathlineto{\pgfqpoint{3.961511in}{3.308254in}}%
\pgfpathlineto{\pgfqpoint{3.963570in}{3.308254in}}%
\pgfpathlineto{\pgfqpoint{3.965629in}{3.317965in}}%
\pgfpathlineto{\pgfqpoint{3.971806in}{3.337387in}}%
\pgfpathlineto{\pgfqpoint{3.973865in}{3.327676in}}%
\pgfpathlineto{\pgfqpoint{3.977983in}{3.317965in}}%
\pgfpathlineto{\pgfqpoint{3.980042in}{3.298543in}}%
\pgfpathlineto{\pgfqpoint{3.986218in}{3.308254in}}%
\pgfpathlineto{\pgfqpoint{3.988277in}{3.317965in}}%
\pgfpathlineto{\pgfqpoint{3.990336in}{3.317965in}}%
\pgfpathlineto{\pgfqpoint{3.994454in}{3.298543in}}%
\pgfpathlineto{\pgfqpoint{4.000631in}{3.279121in}}%
\pgfpathlineto{\pgfqpoint{4.004749in}{3.240277in}}%
\pgfpathlineto{\pgfqpoint{4.006808in}{3.259699in}}%
\pgfpathlineto{\pgfqpoint{4.008867in}{3.240277in}}%
\pgfpathlineto{\pgfqpoint{4.015044in}{3.230566in}}%
\pgfpathlineto{\pgfqpoint{4.019161in}{3.230566in}}%
\pgfpathlineto{\pgfqpoint{4.021220in}{3.201434in}}%
\pgfpathlineto{\pgfqpoint{4.023279in}{3.191723in}}%
\pgfpathlineto{\pgfqpoint{4.029456in}{3.249988in}}%
\pgfpathlineto{\pgfqpoint{4.033574in}{3.220855in}}%
\pgfpathlineto{\pgfqpoint{4.035633in}{3.123746in}}%
\pgfpathlineto{\pgfqpoint{4.037692in}{3.191723in}}%
\pgfpathlineto{\pgfqpoint{4.043869in}{3.201434in}}%
\pgfpathlineto{\pgfqpoint{4.045928in}{3.220855in}}%
\pgfpathlineto{\pgfqpoint{4.047987in}{3.211145in}}%
\pgfpathlineto{\pgfqpoint{4.050046in}{3.211145in}}%
\pgfpathlineto{\pgfqpoint{4.052105in}{3.201434in}}%
\pgfpathlineto{\pgfqpoint{4.058281in}{3.191723in}}%
\pgfpathlineto{\pgfqpoint{4.064458in}{3.191723in}}%
\pgfpathlineto{\pgfqpoint{4.066517in}{3.220855in}}%
\pgfpathlineto{\pgfqpoint{4.074753in}{3.211145in}}%
\pgfpathlineto{\pgfqpoint{4.076812in}{3.211145in}}%
\pgfpathlineto{\pgfqpoint{4.078871in}{3.201434in}}%
\pgfpathlineto{\pgfqpoint{4.080930in}{3.220855in}}%
\pgfpathlineto{\pgfqpoint{4.089166in}{3.220855in}}%
\pgfpathlineto{\pgfqpoint{4.093284in}{3.172301in}}%
\pgfpathlineto{\pgfqpoint{4.095342in}{3.182012in}}%
\pgfpathlineto{\pgfqpoint{4.101519in}{3.191723in}}%
\pgfpathlineto{\pgfqpoint{4.103578in}{3.182012in}}%
\pgfpathlineto{\pgfqpoint{4.105637in}{3.182012in}}%
\pgfpathlineto{\pgfqpoint{4.109755in}{3.162590in}}%
\pgfpathlineto{\pgfqpoint{4.115932in}{3.172301in}}%
\pgfpathlineto{\pgfqpoint{4.120050in}{3.172301in}}%
\pgfpathlineto{\pgfqpoint{4.122109in}{3.152879in}}%
\pgfpathlineto{\pgfqpoint{4.124168in}{3.172301in}}%
\pgfpathlineto{\pgfqpoint{4.132403in}{3.162590in}}%
\pgfpathlineto{\pgfqpoint{4.134462in}{3.162590in}}%
\pgfpathlineto{\pgfqpoint{4.136521in}{3.172301in}}%
\pgfpathlineto{\pgfqpoint{4.138580in}{3.172301in}}%
\pgfpathlineto{\pgfqpoint{4.144757in}{3.182012in}}%
\pgfpathlineto{\pgfqpoint{4.148875in}{3.162590in}}%
\pgfpathlineto{\pgfqpoint{4.150934in}{3.162590in}}%
\pgfpathlineto{\pgfqpoint{4.152993in}{3.172301in}}%
\pgfpathlineto{\pgfqpoint{4.159170in}{3.162590in}}%
\pgfpathlineto{\pgfqpoint{4.163288in}{3.162590in}}%
\pgfpathlineto{\pgfqpoint{4.165347in}{3.143168in}}%
\pgfpathlineto{\pgfqpoint{4.167406in}{3.152879in}}%
\pgfpathlineto{\pgfqpoint{4.173582in}{3.152879in}}%
\pgfpathlineto{\pgfqpoint{4.175641in}{3.143168in}}%
\pgfpathlineto{\pgfqpoint{4.177700in}{3.152879in}}%
\pgfpathlineto{\pgfqpoint{4.179759in}{3.143168in}}%
\pgfpathlineto{\pgfqpoint{4.187995in}{3.143168in}}%
\pgfpathlineto{\pgfqpoint{4.192113in}{3.094613in}}%
\pgfpathlineto{\pgfqpoint{4.194172in}{3.104324in}}%
\pgfpathlineto{\pgfqpoint{4.196231in}{3.075191in}}%
\pgfpathlineto{\pgfqpoint{4.202408in}{3.036347in}}%
\pgfpathlineto{\pgfqpoint{4.204467in}{3.065480in}}%
\pgfpathlineto{\pgfqpoint{4.206526in}{3.026636in}}%
\pgfpathlineto{\pgfqpoint{4.210643in}{3.026636in}}%
\pgfpathlineto{\pgfqpoint{4.216820in}{3.036347in}}%
\pgfpathlineto{\pgfqpoint{4.222997in}{3.036347in}}%
\pgfpathlineto{\pgfqpoint{4.225056in}{3.055769in}}%
\pgfpathlineto{\pgfqpoint{4.231233in}{3.055769in}}%
\pgfpathlineto{\pgfqpoint{4.235351in}{3.036347in}}%
\pgfpathlineto{\pgfqpoint{4.237410in}{3.065480in}}%
\pgfpathlineto{\pgfqpoint{4.239469in}{3.065480in}}%
\pgfpathlineto{\pgfqpoint{4.245645in}{3.055769in}}%
\pgfpathlineto{\pgfqpoint{4.247704in}{3.075191in}}%
\pgfpathlineto{\pgfqpoint{4.249763in}{3.065480in}}%
\pgfpathlineto{\pgfqpoint{4.251822in}{3.065480in}}%
\pgfpathlineto{\pgfqpoint{4.260058in}{3.084902in}}%
\pgfpathlineto{\pgfqpoint{4.262117in}{3.055769in}}%
\pgfpathlineto{\pgfqpoint{4.264176in}{3.046058in}}%
\pgfpathlineto{\pgfqpoint{4.266235in}{3.046058in}}%
\pgfpathlineto{\pgfqpoint{4.268294in}{3.036347in}}%
\pgfpathlineto{\pgfqpoint{4.274471in}{3.046058in}}%
\pgfpathlineto{\pgfqpoint{4.276530in}{3.016925in}}%
\pgfpathlineto{\pgfqpoint{4.278589in}{3.016925in}}%
\pgfpathlineto{\pgfqpoint{4.280648in}{3.036347in}}%
\pgfpathlineto{\pgfqpoint{4.282707in}{3.036347in}}%
\pgfpathlineto{\pgfqpoint{4.288883in}{3.016925in}}%
\pgfpathlineto{\pgfqpoint{4.290942in}{2.997503in}}%
\pgfpathlineto{\pgfqpoint{4.293001in}{2.997503in}}%
\pgfpathlineto{\pgfqpoint{4.295060in}{2.987792in}}%
\pgfpathlineto{\pgfqpoint{4.297119in}{2.987792in}}%
\pgfpathlineto{\pgfqpoint{4.303296in}{2.948948in}}%
\pgfpathlineto{\pgfqpoint{4.305355in}{2.948948in}}%
\pgfpathlineto{\pgfqpoint{4.307414in}{2.929526in}}%
\pgfpathlineto{\pgfqpoint{4.309473in}{2.958659in}}%
\pgfpathlineto{\pgfqpoint{4.311532in}{2.958659in}}%
\pgfpathlineto{\pgfqpoint{4.317709in}{2.968370in}}%
\pgfpathlineto{\pgfqpoint{4.321826in}{2.997503in}}%
\pgfpathlineto{\pgfqpoint{4.323885in}{2.948948in}}%
\pgfpathlineto{\pgfqpoint{4.325944in}{2.958659in}}%
\pgfpathlineto{\pgfqpoint{4.334180in}{2.939237in}}%
\pgfpathlineto{\pgfqpoint{4.338298in}{2.919815in}}%
\pgfpathlineto{\pgfqpoint{4.340357in}{2.842127in}}%
\pgfpathlineto{\pgfqpoint{4.346534in}{2.745017in}}%
\pgfpathlineto{\pgfqpoint{4.348593in}{2.745017in}}%
\pgfpathlineto{\pgfqpoint{4.350652in}{2.677040in}}%
\pgfpathlineto{\pgfqpoint{4.352711in}{2.657618in}}%
\pgfpathlineto{\pgfqpoint{4.354770in}{2.609064in}}%
\pgfpathlineto{\pgfqpoint{4.363005in}{2.686751in}}%
\pgfpathlineto{\pgfqpoint{4.367123in}{2.638197in}}%
\pgfpathlineto{\pgfqpoint{4.369182in}{2.638197in}}%
\pgfpathlineto{\pgfqpoint{4.377418in}{2.677040in}}%
\pgfpathlineto{\pgfqpoint{4.381536in}{2.550798in}}%
\pgfpathlineto{\pgfqpoint{4.383595in}{2.589642in}}%
\pgfpathlineto{\pgfqpoint{4.389772in}{2.560509in}}%
\pgfpathlineto{\pgfqpoint{4.391831in}{2.570220in}}%
\pgfpathlineto{\pgfqpoint{4.393890in}{2.599353in}}%
\pgfpathlineto{\pgfqpoint{4.395949in}{2.570220in}}%
\pgfpathlineto{\pgfqpoint{4.398007in}{2.560509in}}%
\pgfpathlineto{\pgfqpoint{4.404184in}{2.579931in}}%
\pgfpathlineto{\pgfqpoint{4.406243in}{2.550798in}}%
\pgfpathlineto{\pgfqpoint{4.408302in}{2.550798in}}%
\pgfpathlineto{\pgfqpoint{4.412420in}{2.618775in}}%
\pgfpathlineto{\pgfqpoint{4.418597in}{2.628486in}}%
\pgfpathlineto{\pgfqpoint{4.420656in}{2.638197in}}%
\pgfpathlineto{\pgfqpoint{4.422715in}{2.570220in}}%
\pgfpathlineto{\pgfqpoint{4.424774in}{2.609064in}}%
\pgfpathlineto{\pgfqpoint{4.426833in}{2.599353in}}%
\pgfpathlineto{\pgfqpoint{4.433010in}{2.589642in}}%
\pgfpathlineto{\pgfqpoint{4.435068in}{2.638197in}}%
\pgfpathlineto{\pgfqpoint{4.439186in}{2.541087in}}%
\pgfpathlineto{\pgfqpoint{4.441245in}{2.579931in}}%
\pgfpathlineto{\pgfqpoint{4.447422in}{2.589642in}}%
\pgfpathlineto{\pgfqpoint{4.451540in}{2.618775in}}%
\pgfpathlineto{\pgfqpoint{4.455658in}{2.638197in}}%
\pgfpathlineto{\pgfqpoint{4.463894in}{2.609064in}}%
\pgfpathlineto{\pgfqpoint{4.465953in}{2.638197in}}%
\pgfpathlineto{\pgfqpoint{4.468012in}{2.521665in}}%
\pgfpathlineto{\pgfqpoint{4.470071in}{2.492532in}}%
\pgfpathlineto{\pgfqpoint{4.476247in}{2.424555in}}%
\pgfpathlineto{\pgfqpoint{4.478306in}{2.443977in}}%
\pgfpathlineto{\pgfqpoint{4.480365in}{2.395422in}}%
\pgfpathlineto{\pgfqpoint{4.482424in}{2.434266in}}%
\pgfpathlineto{\pgfqpoint{4.490660in}{2.395422in}}%
\pgfpathlineto{\pgfqpoint{4.492719in}{2.502243in}}%
\pgfpathlineto{\pgfqpoint{4.496837in}{2.366289in}}%
\pgfpathlineto{\pgfqpoint{4.498896in}{2.356578in}}%
\pgfpathlineto{\pgfqpoint{4.505073in}{2.395422in}}%
\pgfpathlineto{\pgfqpoint{4.507132in}{2.366289in}}%
\pgfpathlineto{\pgfqpoint{4.509191in}{2.414844in}}%
\pgfpathlineto{\pgfqpoint{4.511249in}{2.434266in}}%
\pgfpathlineto{\pgfqpoint{4.513308in}{2.376000in}}%
\pgfpathlineto{\pgfqpoint{4.519485in}{2.395422in}}%
\pgfpathlineto{\pgfqpoint{4.521544in}{2.414844in}}%
\pgfpathlineto{\pgfqpoint{4.523603in}{2.385711in}}%
\pgfpathlineto{\pgfqpoint{4.527721in}{2.405133in}}%
\pgfpathlineto{\pgfqpoint{4.535957in}{2.366289in}}%
\pgfpathlineto{\pgfqpoint{4.538016in}{2.337156in}}%
\pgfpathlineto{\pgfqpoint{4.540075in}{2.376000in}}%
\pgfpathlineto{\pgfqpoint{4.542134in}{2.376000in}}%
\pgfpathlineto{\pgfqpoint{4.548311in}{2.385711in}}%
\pgfpathlineto{\pgfqpoint{4.550369in}{2.453688in}}%
\pgfpathlineto{\pgfqpoint{4.552428in}{2.434266in}}%
\pgfpathlineto{\pgfqpoint{4.554487in}{2.463399in}}%
\pgfpathlineto{\pgfqpoint{4.556546in}{2.521665in}}%
\pgfpathlineto{\pgfqpoint{4.562723in}{2.502243in}}%
\pgfpathlineto{\pgfqpoint{4.564782in}{2.511954in}}%
\pgfpathlineto{\pgfqpoint{4.566841in}{2.511954in}}%
\pgfpathlineto{\pgfqpoint{4.568900in}{2.521665in}}%
\pgfpathlineto{\pgfqpoint{4.570959in}{2.482821in}}%
\pgfpathlineto{\pgfqpoint{4.577136in}{2.453688in}}%
\pgfpathlineto{\pgfqpoint{4.579195in}{2.424555in}}%
\pgfpathlineto{\pgfqpoint{4.581254in}{2.463399in}}%
\pgfpathlineto{\pgfqpoint{4.583313in}{2.434266in}}%
\pgfpathlineto{\pgfqpoint{4.585372in}{2.385711in}}%
\pgfpathlineto{\pgfqpoint{4.591548in}{2.395422in}}%
\pgfpathlineto{\pgfqpoint{4.593607in}{2.376000in}}%
\pgfpathlineto{\pgfqpoint{4.599784in}{2.230335in}}%
\pgfpathlineto{\pgfqpoint{4.605961in}{2.288601in}}%
\pgfpathlineto{\pgfqpoint{4.610079in}{2.240046in}}%
\pgfpathlineto{\pgfqpoint{4.614197in}{2.317734in}}%
\pgfpathlineto{\pgfqpoint{4.622433in}{2.298312in}}%
\pgfpathlineto{\pgfqpoint{4.624491in}{2.240046in}}%
\pgfpathlineto{\pgfqpoint{4.626550in}{2.240046in}}%
\pgfpathlineto{\pgfqpoint{4.628609in}{2.230335in}}%
\pgfpathlineto{\pgfqpoint{4.634786in}{2.240046in}}%
\pgfpathlineto{\pgfqpoint{4.636845in}{2.240046in}}%
\pgfpathlineto{\pgfqpoint{4.638904in}{2.230335in}}%
\pgfpathlineto{\pgfqpoint{4.643022in}{2.249757in}}%
\pgfpathlineto{\pgfqpoint{4.649199in}{2.249757in}}%
\pgfpathlineto{\pgfqpoint{4.651258in}{2.240046in}}%
\pgfpathlineto{\pgfqpoint{4.653317in}{2.240046in}}%
\pgfpathlineto{\pgfqpoint{4.655376in}{2.181780in}}%
\pgfpathlineto{\pgfqpoint{4.657435in}{2.181780in}}%
\pgfpathlineto{\pgfqpoint{4.663611in}{2.210913in}}%
\pgfpathlineto{\pgfqpoint{4.665670in}{2.269179in}}%
\pgfpathlineto{\pgfqpoint{4.667729in}{2.230335in}}%
\pgfpathlineto{\pgfqpoint{4.680083in}{2.230335in}}%
\pgfpathlineto{\pgfqpoint{4.682142in}{2.220624in}}%
\pgfpathlineto{\pgfqpoint{4.686260in}{2.191491in}}%
\pgfpathlineto{\pgfqpoint{4.696555in}{2.191491in}}%
\pgfpathlineto{\pgfqpoint{4.700672in}{2.210913in}}%
\pgfpathlineto{\pgfqpoint{4.708908in}{2.240046in}}%
\pgfpathlineto{\pgfqpoint{4.710967in}{2.249757in}}%
\pgfpathlineto{\pgfqpoint{4.721262in}{2.249757in}}%
\pgfpathlineto{\pgfqpoint{4.723321in}{2.220624in}}%
\pgfpathlineto{\pgfqpoint{4.725380in}{2.210913in}}%
\pgfpathlineto{\pgfqpoint{4.727439in}{2.210913in}}%
\pgfpathlineto{\pgfqpoint{4.729498in}{2.220624in}}%
\pgfpathlineto{\pgfqpoint{4.735675in}{2.210913in}}%
\pgfpathlineto{\pgfqpoint{4.737733in}{2.210913in}}%
\pgfpathlineto{\pgfqpoint{4.739792in}{2.201202in}}%
\pgfpathlineto{\pgfqpoint{4.741851in}{2.201202in}}%
\pgfpathlineto{\pgfqpoint{4.743910in}{2.191491in}}%
\pgfpathlineto{\pgfqpoint{4.750087in}{2.191491in}}%
\pgfpathlineto{\pgfqpoint{4.752146in}{2.181780in}}%
\pgfpathlineto{\pgfqpoint{4.754205in}{2.191491in}}%
\pgfpathlineto{\pgfqpoint{4.756264in}{2.172069in}}%
\pgfpathlineto{\pgfqpoint{4.758323in}{2.172069in}}%
\pgfpathlineto{\pgfqpoint{4.764500in}{2.181780in}}%
\pgfpathlineto{\pgfqpoint{4.770677in}{2.181780in}}%
\pgfpathlineto{\pgfqpoint{4.772736in}{2.162358in}}%
\pgfpathlineto{\pgfqpoint{4.780971in}{2.240046in}}%
\pgfpathlineto{\pgfqpoint{4.787148in}{2.201202in}}%
\pgfpathlineto{\pgfqpoint{4.793325in}{2.191491in}}%
\pgfpathlineto{\pgfqpoint{4.795384in}{2.181780in}}%
\pgfpathlineto{\pgfqpoint{4.797443in}{2.201202in}}%
\pgfpathlineto{\pgfqpoint{4.801561in}{2.181780in}}%
\pgfpathlineto{\pgfqpoint{4.809797in}{2.181780in}}%
\pgfpathlineto{\pgfqpoint{4.811856in}{2.191491in}}%
\pgfpathlineto{\pgfqpoint{4.813914in}{2.191491in}}%
\pgfpathlineto{\pgfqpoint{4.815973in}{2.210913in}}%
\pgfpathlineto{\pgfqpoint{4.824209in}{2.191491in}}%
\pgfpathlineto{\pgfqpoint{4.826268in}{2.201202in}}%
\pgfpathlineto{\pgfqpoint{4.830386in}{2.201202in}}%
\pgfpathlineto{\pgfqpoint{4.836563in}{2.181780in}}%
\pgfpathlineto{\pgfqpoint{4.838622in}{2.181780in}}%
\pgfpathlineto{\pgfqpoint{4.842740in}{2.133225in}}%
\pgfpathlineto{\pgfqpoint{4.844799in}{2.104092in}}%
\pgfpathlineto{\pgfqpoint{4.850976in}{2.113803in}}%
\pgfpathlineto{\pgfqpoint{4.857152in}{2.162358in}}%
\pgfpathlineto{\pgfqpoint{4.859211in}{2.142936in}}%
\pgfpathlineto{\pgfqpoint{4.865388in}{2.104092in}}%
\pgfpathlineto{\pgfqpoint{4.867447in}{2.133225in}}%
\pgfpathlineto{\pgfqpoint{4.869506in}{2.142936in}}%
\pgfpathlineto{\pgfqpoint{4.871565in}{2.133225in}}%
\pgfpathlineto{\pgfqpoint{4.873624in}{2.142936in}}%
\pgfpathlineto{\pgfqpoint{4.881860in}{2.123514in}}%
\pgfpathlineto{\pgfqpoint{4.883919in}{2.123514in}}%
\pgfpathlineto{\pgfqpoint{4.885978in}{2.113803in}}%
\pgfpathlineto{\pgfqpoint{4.896272in}{1.958428in}}%
\pgfpathlineto{\pgfqpoint{4.898331in}{1.919584in}}%
\pgfpathlineto{\pgfqpoint{4.900390in}{1.841896in}}%
\pgfpathlineto{\pgfqpoint{4.902449in}{1.637965in}}%
\pgfpathlineto{\pgfqpoint{4.908626in}{1.560277in}}%
\pgfpathlineto{\pgfqpoint{4.914803in}{1.162127in}}%
\pgfpathlineto{\pgfqpoint{4.916862in}{1.074728in}}%
\pgfpathlineto{\pgfqpoint{4.923039in}{0.997040in}}%
\pgfpathlineto{\pgfqpoint{4.925098in}{1.113572in}}%
\pgfpathlineto{\pgfqpoint{4.927156in}{1.084439in}}%
\pgfpathlineto{\pgfqpoint{4.931274in}{1.065017in}}%
\pgfpathlineto{\pgfqpoint{4.937451in}{0.977618in}}%
\pgfpathlineto{\pgfqpoint{4.939510in}{0.987329in}}%
\pgfpathlineto{\pgfqpoint{4.941569in}{0.899931in}}%
\pgfpathlineto{\pgfqpoint{4.943628in}{0.890220in}}%
\pgfpathlineto{\pgfqpoint{4.945687in}{0.841665in}}%
\pgfpathlineto{\pgfqpoint{4.951864in}{0.861087in}}%
\pgfpathlineto{\pgfqpoint{4.953923in}{0.938775in}}%
\pgfpathlineto{\pgfqpoint{4.958041in}{0.822243in}}%
\pgfpathlineto{\pgfqpoint{4.960100in}{0.802821in}}%
\pgfpathlineto{\pgfqpoint{4.966276in}{0.831954in}}%
\pgfpathlineto{\pgfqpoint{4.968335in}{0.861087in}}%
\pgfpathlineto{\pgfqpoint{4.970394in}{0.851376in}}%
\pgfpathlineto{\pgfqpoint{4.972453in}{0.831954in}}%
\pgfpathlineto{\pgfqpoint{4.974512in}{0.841665in}}%
\pgfpathlineto{\pgfqpoint{4.980689in}{0.890220in}}%
\pgfpathlineto{\pgfqpoint{4.982748in}{0.890220in}}%
\pgfpathlineto{\pgfqpoint{4.986866in}{0.938775in}}%
\pgfpathlineto{\pgfqpoint{4.995102in}{0.958197in}}%
\pgfpathlineto{\pgfqpoint{4.997161in}{0.938775in}}%
\pgfpathlineto{\pgfqpoint{4.999220in}{0.880509in}}%
\pgfpathlineto{\pgfqpoint{5.003337in}{0.851376in}}%
\pgfpathlineto{\pgfqpoint{5.009514in}{0.841665in}}%
\pgfpathlineto{\pgfqpoint{5.011573in}{0.861087in}}%
\pgfpathlineto{\pgfqpoint{5.013632in}{0.851376in}}%
\pgfpathlineto{\pgfqpoint{5.017750in}{0.870798in}}%
\pgfpathlineto{\pgfqpoint{5.023927in}{0.861087in}}%
\pgfpathlineto{\pgfqpoint{5.025986in}{0.851376in}}%
\pgfpathlineto{\pgfqpoint{5.028045in}{0.870798in}}%
\pgfpathlineto{\pgfqpoint{5.030104in}{0.851376in}}%
\pgfpathlineto{\pgfqpoint{5.032163in}{0.861087in}}%
\pgfpathlineto{\pgfqpoint{5.038340in}{0.851376in}}%
\pgfpathlineto{\pgfqpoint{5.042457in}{0.851376in}}%
\pgfpathlineto{\pgfqpoint{5.044516in}{0.841665in}}%
\pgfpathlineto{\pgfqpoint{5.046575in}{0.841665in}}%
\pgfpathlineto{\pgfqpoint{5.052752in}{0.851376in}}%
\pgfpathlineto{\pgfqpoint{5.054811in}{0.851376in}}%
\pgfpathlineto{\pgfqpoint{5.056870in}{0.841665in}}%
\pgfpathlineto{\pgfqpoint{5.060988in}{0.841665in}}%
\pgfpathlineto{\pgfqpoint{5.067165in}{0.861087in}}%
\pgfpathlineto{\pgfqpoint{5.069224in}{0.851376in}}%
\pgfpathlineto{\pgfqpoint{5.073342in}{0.851376in}}%
\pgfpathlineto{\pgfqpoint{5.075401in}{0.861087in}}%
\pgfpathlineto{\pgfqpoint{5.083636in}{0.861087in}}%
\pgfpathlineto{\pgfqpoint{5.085695in}{0.870798in}}%
\pgfpathlineto{\pgfqpoint{5.087754in}{0.861087in}}%
\pgfpathlineto{\pgfqpoint{5.102167in}{0.861087in}}%
\pgfpathlineto{\pgfqpoint{5.104226in}{0.870798in}}%
\pgfpathlineto{\pgfqpoint{5.110403in}{0.880509in}}%
\pgfpathlineto{\pgfqpoint{5.112462in}{0.880509in}}%
\pgfpathlineto{\pgfqpoint{5.114521in}{0.870798in}}%
\pgfpathlineto{\pgfqpoint{5.116579in}{0.880509in}}%
\pgfpathlineto{\pgfqpoint{5.118638in}{0.870798in}}%
\pgfpathlineto{\pgfqpoint{5.124815in}{0.861087in}}%
\pgfpathlineto{\pgfqpoint{5.128933in}{0.880509in}}%
\pgfpathlineto{\pgfqpoint{5.130992in}{0.880509in}}%
\pgfpathlineto{\pgfqpoint{5.133051in}{0.870798in}}%
\pgfpathlineto{\pgfqpoint{5.139228in}{0.861087in}}%
\pgfpathlineto{\pgfqpoint{5.141287in}{0.870798in}}%
\pgfpathlineto{\pgfqpoint{5.143346in}{0.861087in}}%
\pgfpathlineto{\pgfqpoint{5.147464in}{0.861087in}}%
\pgfpathlineto{\pgfqpoint{5.153641in}{0.851376in}}%
\pgfpathlineto{\pgfqpoint{5.168053in}{0.851376in}}%
\pgfpathlineto{\pgfqpoint{5.170112in}{0.841665in}}%
\pgfpathlineto{\pgfqpoint{5.176289in}{0.841665in}}%
\pgfpathlineto{\pgfqpoint{5.182466in}{0.851376in}}%
\pgfpathlineto{\pgfqpoint{5.184525in}{0.861087in}}%
\pgfpathlineto{\pgfqpoint{5.188643in}{0.831954in}}%
\pgfpathlineto{\pgfqpoint{5.196878in}{0.831954in}}%
\pgfpathlineto{\pgfqpoint{5.198937in}{0.841665in}}%
\pgfpathlineto{\pgfqpoint{5.200996in}{0.831954in}}%
\pgfpathlineto{\pgfqpoint{5.203055in}{0.831954in}}%
\pgfpathlineto{\pgfqpoint{5.205114in}{0.851376in}}%
\pgfpathlineto{\pgfqpoint{5.211291in}{0.831954in}}%
\pgfpathlineto{\pgfqpoint{5.213350in}{0.831954in}}%
\pgfpathlineto{\pgfqpoint{5.215409in}{0.822243in}}%
\pgfpathlineto{\pgfqpoint{5.217468in}{0.802821in}}%
\pgfpathlineto{\pgfqpoint{5.219527in}{0.802821in}}%
\pgfpathlineto{\pgfqpoint{5.225704in}{0.812532in}}%
\pgfpathlineto{\pgfqpoint{5.227763in}{0.831954in}}%
\pgfpathlineto{\pgfqpoint{5.229822in}{0.812532in}}%
\pgfpathlineto{\pgfqpoint{5.231880in}{0.831954in}}%
\pgfpathlineto{\pgfqpoint{5.233939in}{0.831954in}}%
\pgfpathlineto{\pgfqpoint{5.240116in}{0.822243in}}%
\pgfpathlineto{\pgfqpoint{5.242175in}{0.841665in}}%
\pgfpathlineto{\pgfqpoint{5.244234in}{0.822243in}}%
\pgfpathlineto{\pgfqpoint{5.246293in}{0.831954in}}%
\pgfpathlineto{\pgfqpoint{5.248352in}{0.822243in}}%
\pgfpathlineto{\pgfqpoint{5.258647in}{0.822243in}}%
\pgfpathlineto{\pgfqpoint{5.260706in}{0.812532in}}%
\pgfpathlineto{\pgfqpoint{5.262765in}{0.822243in}}%
\pgfpathlineto{\pgfqpoint{5.268941in}{0.831954in}}%
\pgfpathlineto{\pgfqpoint{5.273059in}{0.812532in}}%
\pgfpathlineto{\pgfqpoint{5.275118in}{0.822243in}}%
\pgfpathlineto{\pgfqpoint{5.277177in}{0.812532in}}%
\pgfpathlineto{\pgfqpoint{5.285413in}{0.812532in}}%
\pgfpathlineto{\pgfqpoint{5.287472in}{0.822243in}}%
\pgfpathlineto{\pgfqpoint{5.289531in}{0.812532in}}%
\pgfpathlineto{\pgfqpoint{5.291590in}{0.822243in}}%
\pgfpathlineto{\pgfqpoint{5.299826in}{0.841665in}}%
\pgfpathlineto{\pgfqpoint{5.301885in}{0.831954in}}%
\pgfpathlineto{\pgfqpoint{5.303944in}{0.841665in}}%
\pgfpathlineto{\pgfqpoint{5.306002in}{0.822243in}}%
\pgfpathlineto{\pgfqpoint{5.312179in}{0.831954in}}%
\pgfpathlineto{\pgfqpoint{5.316297in}{0.812532in}}%
\pgfpathlineto{\pgfqpoint{5.318356in}{0.812532in}}%
\pgfpathlineto{\pgfqpoint{5.320415in}{0.822243in}}%
\pgfpathlineto{\pgfqpoint{5.326592in}{0.812532in}}%
\pgfpathlineto{\pgfqpoint{5.328651in}{0.812532in}}%
\pgfpathlineto{\pgfqpoint{5.330710in}{0.822243in}}%
\pgfpathlineto{\pgfqpoint{5.332769in}{0.812532in}}%
\pgfpathlineto{\pgfqpoint{5.355417in}{0.812532in}}%
\pgfpathlineto{\pgfqpoint{5.357476in}{0.831954in}}%
\pgfpathlineto{\pgfqpoint{5.359535in}{0.822243in}}%
\pgfpathlineto{\pgfqpoint{5.361594in}{0.822243in}}%
\pgfpathlineto{\pgfqpoint{5.363653in}{0.841665in}}%
\pgfpathlineto{\pgfqpoint{5.371889in}{0.822243in}}%
\pgfpathlineto{\pgfqpoint{5.373948in}{0.822243in}}%
\pgfpathlineto{\pgfqpoint{5.376007in}{0.812532in}}%
\pgfpathlineto{\pgfqpoint{5.378066in}{0.812532in}}%
\pgfpathlineto{\pgfqpoint{5.384242in}{0.822243in}}%
\pgfpathlineto{\pgfqpoint{5.390419in}{0.822243in}}%
\pgfpathlineto{\pgfqpoint{5.392478in}{0.812532in}}%
\pgfpathlineto{\pgfqpoint{5.404832in}{0.812532in}}%
\pgfpathlineto{\pgfqpoint{5.406891in}{0.822243in}}%
\pgfpathlineto{\pgfqpoint{5.413068in}{0.822243in}}%
\pgfpathlineto{\pgfqpoint{5.415127in}{0.831954in}}%
\pgfpathlineto{\pgfqpoint{5.417186in}{0.812532in}}%
\pgfpathlineto{\pgfqpoint{5.429539in}{0.812532in}}%
\pgfpathlineto{\pgfqpoint{5.433657in}{0.822243in}}%
\pgfpathlineto{\pgfqpoint{5.435716in}{0.812532in}}%
\pgfpathlineto{\pgfqpoint{5.443952in}{0.812532in}}%
\pgfpathlineto{\pgfqpoint{5.446011in}{0.802821in}}%
\pgfpathlineto{\pgfqpoint{5.450129in}{0.802821in}}%
\pgfpathlineto{\pgfqpoint{5.456306in}{0.793110in}}%
\pgfpathlineto{\pgfqpoint{5.458364in}{0.802821in}}%
\pgfpathlineto{\pgfqpoint{5.470718in}{0.802821in}}%
\pgfpathlineto{\pgfqpoint{5.472777in}{0.812532in}}%
\pgfpathlineto{\pgfqpoint{5.476895in}{0.793110in}}%
\pgfpathlineto{\pgfqpoint{5.478954in}{0.802821in}}%
\pgfpathlineto{\pgfqpoint{5.485131in}{0.793110in}}%
\pgfpathlineto{\pgfqpoint{5.499543in}{0.793110in}}%
\pgfpathlineto{\pgfqpoint{5.501602in}{0.783399in}}%
\pgfpathlineto{\pgfqpoint{5.518074in}{0.783399in}}%
\pgfpathlineto{\pgfqpoint{5.520133in}{0.793110in}}%
\pgfpathlineto{\pgfqpoint{5.528369in}{0.802821in}}%
\pgfpathlineto{\pgfqpoint{5.530428in}{0.802821in}}%
\pgfpathlineto{\pgfqpoint{5.532487in}{0.812532in}}%
\pgfpathlineto{\pgfqpoint{5.534545in}{0.793110in}}%
\pgfpathlineto{\pgfqpoint{5.534545in}{0.793110in}}%
\pgfusepath{stroke}%
\end{pgfscope}%
\begin{pgfscope}%
\pgfpathrectangle{\pgfqpoint{0.800000in}{0.528000in}}{\pgfqpoint{4.960000in}{3.696000in}}%
\pgfusepath{clip}%
\pgfsetrectcap%
\pgfsetroundjoin%
\pgfsetlinewidth{1.003750pt}%
\definecolor{currentstroke}{rgb}{0.501961,0.501961,0.501961}%
\pgfsetstrokecolor{currentstroke}%
\pgfsetstrokeopacity{0.900000}%
\pgfsetdash{}{0pt}%
\pgfpathmoveto{\pgfqpoint{1.025455in}{1.336925in}}%
\pgfpathlineto{\pgfqpoint{1.031631in}{1.356347in}}%
\pgfpathlineto{\pgfqpoint{1.035749in}{1.298081in}}%
\pgfpathlineto{\pgfqpoint{1.037808in}{1.298081in}}%
\pgfpathlineto{\pgfqpoint{1.039867in}{1.268948in}}%
\pgfpathlineto{\pgfqpoint{1.046044in}{1.239815in}}%
\pgfpathlineto{\pgfqpoint{1.050162in}{1.191260in}}%
\pgfpathlineto{\pgfqpoint{1.052221in}{1.123283in}}%
\pgfpathlineto{\pgfqpoint{1.054280in}{1.171838in}}%
\pgfpathlineto{\pgfqpoint{1.062516in}{1.210682in}}%
\pgfpathlineto{\pgfqpoint{1.066633in}{1.210682in}}%
\pgfpathlineto{\pgfqpoint{1.068692in}{1.200971in}}%
\pgfpathlineto{\pgfqpoint{1.074869in}{1.220393in}}%
\pgfpathlineto{\pgfqpoint{1.076928in}{1.220393in}}%
\pgfpathlineto{\pgfqpoint{1.078987in}{1.181549in}}%
\pgfpathlineto{\pgfqpoint{1.081046in}{1.191260in}}%
\pgfpathlineto{\pgfqpoint{1.083105in}{1.152416in}}%
\pgfpathlineto{\pgfqpoint{1.089282in}{1.171838in}}%
\pgfpathlineto{\pgfqpoint{1.091341in}{1.200971in}}%
\pgfpathlineto{\pgfqpoint{1.095459in}{1.200971in}}%
\pgfpathlineto{\pgfqpoint{1.097518in}{1.327214in}}%
\pgfpathlineto{\pgfqpoint{1.103694in}{1.327214in}}%
\pgfpathlineto{\pgfqpoint{1.105753in}{1.346636in}}%
\pgfpathlineto{\pgfqpoint{1.107812in}{1.346636in}}%
\pgfpathlineto{\pgfqpoint{1.109871in}{1.288370in}}%
\pgfpathlineto{\pgfqpoint{1.111930in}{1.336925in}}%
\pgfpathlineto{\pgfqpoint{1.120166in}{1.375769in}}%
\pgfpathlineto{\pgfqpoint{1.122225in}{1.298081in}}%
\pgfpathlineto{\pgfqpoint{1.124284in}{1.346636in}}%
\pgfpathlineto{\pgfqpoint{1.126343in}{1.346636in}}%
\pgfpathlineto{\pgfqpoint{1.132520in}{1.317503in}}%
\pgfpathlineto{\pgfqpoint{1.134579in}{1.278659in}}%
\pgfpathlineto{\pgfqpoint{1.136638in}{1.288370in}}%
\pgfpathlineto{\pgfqpoint{1.138697in}{1.336925in}}%
\pgfpathlineto{\pgfqpoint{1.140756in}{1.307792in}}%
\pgfpathlineto{\pgfqpoint{1.146932in}{1.336925in}}%
\pgfpathlineto{\pgfqpoint{1.148991in}{1.356347in}}%
\pgfpathlineto{\pgfqpoint{1.153109in}{1.327214in}}%
\pgfpathlineto{\pgfqpoint{1.155168in}{1.404902in}}%
\pgfpathlineto{\pgfqpoint{1.161345in}{1.375769in}}%
\pgfpathlineto{\pgfqpoint{1.165463in}{1.375769in}}%
\pgfpathlineto{\pgfqpoint{1.167522in}{1.346636in}}%
\pgfpathlineto{\pgfqpoint{1.169581in}{1.356347in}}%
\pgfpathlineto{\pgfqpoint{1.175758in}{1.336925in}}%
\pgfpathlineto{\pgfqpoint{1.177817in}{1.375769in}}%
\pgfpathlineto{\pgfqpoint{1.179875in}{1.249526in}}%
\pgfpathlineto{\pgfqpoint{1.181934in}{1.307792in}}%
\pgfpathlineto{\pgfqpoint{1.183993in}{1.278659in}}%
\pgfpathlineto{\pgfqpoint{1.190170in}{1.278659in}}%
\pgfpathlineto{\pgfqpoint{1.192229in}{1.259237in}}%
\pgfpathlineto{\pgfqpoint{1.194288in}{1.268948in}}%
\pgfpathlineto{\pgfqpoint{1.196347in}{1.288370in}}%
\pgfpathlineto{\pgfqpoint{1.198406in}{1.259237in}}%
\pgfpathlineto{\pgfqpoint{1.204583in}{1.259237in}}%
\pgfpathlineto{\pgfqpoint{1.208701in}{1.230104in}}%
\pgfpathlineto{\pgfqpoint{1.210760in}{1.230104in}}%
\pgfpathlineto{\pgfqpoint{1.212819in}{1.171838in}}%
\pgfpathlineto{\pgfqpoint{1.221054in}{1.200971in}}%
\pgfpathlineto{\pgfqpoint{1.227231in}{1.249526in}}%
\pgfpathlineto{\pgfqpoint{1.235467in}{1.210682in}}%
\pgfpathlineto{\pgfqpoint{1.239585in}{1.181549in}}%
\pgfpathlineto{\pgfqpoint{1.241644in}{1.191260in}}%
\pgfpathlineto{\pgfqpoint{1.247821in}{1.230104in}}%
\pgfpathlineto{\pgfqpoint{1.249880in}{1.230104in}}%
\pgfpathlineto{\pgfqpoint{1.251939in}{1.249526in}}%
\pgfpathlineto{\pgfqpoint{1.256056in}{1.220393in}}%
\pgfpathlineto{\pgfqpoint{1.262233in}{1.220393in}}%
\pgfpathlineto{\pgfqpoint{1.264292in}{1.239815in}}%
\pgfpathlineto{\pgfqpoint{1.266351in}{1.239815in}}%
\pgfpathlineto{\pgfqpoint{1.270469in}{1.278659in}}%
\pgfpathlineto{\pgfqpoint{1.276646in}{1.278659in}}%
\pgfpathlineto{\pgfqpoint{1.280764in}{1.327214in}}%
\pgfpathlineto{\pgfqpoint{1.282823in}{1.307792in}}%
\pgfpathlineto{\pgfqpoint{1.284882in}{1.268948in}}%
\pgfpathlineto{\pgfqpoint{1.291059in}{1.298081in}}%
\pgfpathlineto{\pgfqpoint{1.293117in}{1.288370in}}%
\pgfpathlineto{\pgfqpoint{1.299294in}{1.230104in}}%
\pgfpathlineto{\pgfqpoint{1.305471in}{1.259237in}}%
\pgfpathlineto{\pgfqpoint{1.307530in}{1.307792in}}%
\pgfpathlineto{\pgfqpoint{1.309589in}{1.278659in}}%
\pgfpathlineto{\pgfqpoint{1.311648in}{1.278659in}}%
\pgfpathlineto{\pgfqpoint{1.313707in}{1.317503in}}%
\pgfpathlineto{\pgfqpoint{1.324002in}{1.317503in}}%
\pgfpathlineto{\pgfqpoint{1.328120in}{1.288370in}}%
\pgfpathlineto{\pgfqpoint{1.334296in}{1.317503in}}%
\pgfpathlineto{\pgfqpoint{1.336355in}{1.317503in}}%
\pgfpathlineto{\pgfqpoint{1.338414in}{1.366058in}}%
\pgfpathlineto{\pgfqpoint{1.340473in}{1.336925in}}%
\pgfpathlineto{\pgfqpoint{1.342532in}{1.404902in}}%
\pgfpathlineto{\pgfqpoint{1.348709in}{1.375769in}}%
\pgfpathlineto{\pgfqpoint{1.352827in}{1.424324in}}%
\pgfpathlineto{\pgfqpoint{1.354886in}{1.404902in}}%
\pgfpathlineto{\pgfqpoint{1.356945in}{1.414613in}}%
\pgfpathlineto{\pgfqpoint{1.365181in}{1.385480in}}%
\pgfpathlineto{\pgfqpoint{1.367240in}{1.346636in}}%
\pgfpathlineto{\pgfqpoint{1.371357in}{1.327214in}}%
\pgfpathlineto{\pgfqpoint{1.377534in}{1.356347in}}%
\pgfpathlineto{\pgfqpoint{1.379593in}{1.375769in}}%
\pgfpathlineto{\pgfqpoint{1.381652in}{1.356347in}}%
\pgfpathlineto{\pgfqpoint{1.383711in}{1.356347in}}%
\pgfpathlineto{\pgfqpoint{1.385770in}{1.395191in}}%
\pgfpathlineto{\pgfqpoint{1.391947in}{1.317503in}}%
\pgfpathlineto{\pgfqpoint{1.394006in}{1.317503in}}%
\pgfpathlineto{\pgfqpoint{1.396065in}{1.366058in}}%
\pgfpathlineto{\pgfqpoint{1.398124in}{1.317503in}}%
\pgfpathlineto{\pgfqpoint{1.406359in}{1.278659in}}%
\pgfpathlineto{\pgfqpoint{1.410477in}{1.230104in}}%
\pgfpathlineto{\pgfqpoint{1.414595in}{1.327214in}}%
\pgfpathlineto{\pgfqpoint{1.420772in}{1.366058in}}%
\pgfpathlineto{\pgfqpoint{1.424890in}{1.317503in}}%
\pgfpathlineto{\pgfqpoint{1.426949in}{1.346636in}}%
\pgfpathlineto{\pgfqpoint{1.435185in}{1.385480in}}%
\pgfpathlineto{\pgfqpoint{1.437244in}{1.385480in}}%
\pgfpathlineto{\pgfqpoint{1.439303in}{1.424324in}}%
\pgfpathlineto{\pgfqpoint{1.441362in}{1.385480in}}%
\pgfpathlineto{\pgfqpoint{1.443421in}{1.375769in}}%
\pgfpathlineto{\pgfqpoint{1.449597in}{1.356347in}}%
\pgfpathlineto{\pgfqpoint{1.453715in}{1.375769in}}%
\pgfpathlineto{\pgfqpoint{1.455774in}{1.395191in}}%
\pgfpathlineto{\pgfqpoint{1.457833in}{1.346636in}}%
\pgfpathlineto{\pgfqpoint{1.464010in}{1.356347in}}%
\pgfpathlineto{\pgfqpoint{1.466069in}{1.414613in}}%
\pgfpathlineto{\pgfqpoint{1.468128in}{1.404902in}}%
\pgfpathlineto{\pgfqpoint{1.470187in}{1.385480in}}%
\pgfpathlineto{\pgfqpoint{1.472246in}{1.404902in}}%
\pgfpathlineto{\pgfqpoint{1.478423in}{1.404902in}}%
\pgfpathlineto{\pgfqpoint{1.480482in}{1.356347in}}%
\pgfpathlineto{\pgfqpoint{1.482540in}{1.346636in}}%
\pgfpathlineto{\pgfqpoint{1.484599in}{1.395191in}}%
\pgfpathlineto{\pgfqpoint{1.486658in}{1.404902in}}%
\pgfpathlineto{\pgfqpoint{1.492835in}{1.395191in}}%
\pgfpathlineto{\pgfqpoint{1.494894in}{1.414613in}}%
\pgfpathlineto{\pgfqpoint{1.496953in}{1.346636in}}%
\pgfpathlineto{\pgfqpoint{1.499012in}{1.366058in}}%
\pgfpathlineto{\pgfqpoint{1.501071in}{1.317503in}}%
\pgfpathlineto{\pgfqpoint{1.507248in}{1.268948in}}%
\pgfpathlineto{\pgfqpoint{1.509307in}{1.346636in}}%
\pgfpathlineto{\pgfqpoint{1.511366in}{1.346636in}}%
\pgfpathlineto{\pgfqpoint{1.513425in}{1.356347in}}%
\pgfpathlineto{\pgfqpoint{1.515484in}{1.395191in}}%
\pgfpathlineto{\pgfqpoint{1.521660in}{1.414613in}}%
\pgfpathlineto{\pgfqpoint{1.523719in}{1.375769in}}%
\pgfpathlineto{\pgfqpoint{1.525778in}{1.395191in}}%
\pgfpathlineto{\pgfqpoint{1.527837in}{1.385480in}}%
\pgfpathlineto{\pgfqpoint{1.529896in}{1.385480in}}%
\pgfpathlineto{\pgfqpoint{1.540191in}{1.424324in}}%
\pgfpathlineto{\pgfqpoint{1.542250in}{1.424324in}}%
\pgfpathlineto{\pgfqpoint{1.544309in}{1.385480in}}%
\pgfpathlineto{\pgfqpoint{1.550486in}{1.404902in}}%
\pgfpathlineto{\pgfqpoint{1.552545in}{1.492301in}}%
\pgfpathlineto{\pgfqpoint{1.554604in}{1.492301in}}%
\pgfpathlineto{\pgfqpoint{1.556663in}{1.375769in}}%
\pgfpathlineto{\pgfqpoint{1.558721in}{1.366058in}}%
\pgfpathlineto{\pgfqpoint{1.564898in}{1.395191in}}%
\pgfpathlineto{\pgfqpoint{1.566957in}{1.366058in}}%
\pgfpathlineto{\pgfqpoint{1.569016in}{1.375769in}}%
\pgfpathlineto{\pgfqpoint{1.571075in}{1.346636in}}%
\pgfpathlineto{\pgfqpoint{1.573134in}{1.375769in}}%
\pgfpathlineto{\pgfqpoint{1.579311in}{1.346636in}}%
\pgfpathlineto{\pgfqpoint{1.581370in}{1.317503in}}%
\pgfpathlineto{\pgfqpoint{1.585488in}{1.317503in}}%
\pgfpathlineto{\pgfqpoint{1.587547in}{1.259237in}}%
\pgfpathlineto{\pgfqpoint{1.593724in}{1.288370in}}%
\pgfpathlineto{\pgfqpoint{1.595782in}{1.288370in}}%
\pgfpathlineto{\pgfqpoint{1.597841in}{1.327214in}}%
\pgfpathlineto{\pgfqpoint{1.601959in}{1.327214in}}%
\pgfpathlineto{\pgfqpoint{1.610195in}{1.317503in}}%
\pgfpathlineto{\pgfqpoint{1.612254in}{1.249526in}}%
\pgfpathlineto{\pgfqpoint{1.614313in}{1.288370in}}%
\pgfpathlineto{\pgfqpoint{1.622549in}{1.288370in}}%
\pgfpathlineto{\pgfqpoint{1.624608in}{1.336925in}}%
\pgfpathlineto{\pgfqpoint{1.628726in}{1.288370in}}%
\pgfpathlineto{\pgfqpoint{1.630785in}{1.336925in}}%
\pgfpathlineto{\pgfqpoint{1.636961in}{1.336925in}}%
\pgfpathlineto{\pgfqpoint{1.639020in}{1.327214in}}%
\pgfpathlineto{\pgfqpoint{1.641079in}{1.404902in}}%
\pgfpathlineto{\pgfqpoint{1.643138in}{1.424324in}}%
\pgfpathlineto{\pgfqpoint{1.645197in}{1.424324in}}%
\pgfpathlineto{\pgfqpoint{1.651374in}{1.443746in}}%
\pgfpathlineto{\pgfqpoint{1.659610in}{1.569988in}}%
\pgfpathlineto{\pgfqpoint{1.665787in}{1.560277in}}%
\pgfpathlineto{\pgfqpoint{1.667846in}{1.540855in}}%
\pgfpathlineto{\pgfqpoint{1.671963in}{1.560277in}}%
\pgfpathlineto{\pgfqpoint{1.674022in}{1.531145in}}%
\pgfpathlineto{\pgfqpoint{1.680199in}{1.550566in}}%
\pgfpathlineto{\pgfqpoint{1.682258in}{1.531145in}}%
\pgfpathlineto{\pgfqpoint{1.684317in}{1.569988in}}%
\pgfpathlineto{\pgfqpoint{1.686376in}{1.579699in}}%
\pgfpathlineto{\pgfqpoint{1.688435in}{1.599121in}}%
\pgfpathlineto{\pgfqpoint{1.694612in}{1.608832in}}%
\pgfpathlineto{\pgfqpoint{1.696671in}{1.599121in}}%
\pgfpathlineto{\pgfqpoint{1.698730in}{1.599121in}}%
\pgfpathlineto{\pgfqpoint{1.702848in}{1.589410in}}%
\pgfpathlineto{\pgfqpoint{1.709024in}{1.608832in}}%
\pgfpathlineto{\pgfqpoint{1.711083in}{1.579699in}}%
\pgfpathlineto{\pgfqpoint{1.715201in}{1.628254in}}%
\pgfpathlineto{\pgfqpoint{1.717260in}{1.628254in}}%
\pgfpathlineto{\pgfqpoint{1.723437in}{1.608832in}}%
\pgfpathlineto{\pgfqpoint{1.725496in}{1.608832in}}%
\pgfpathlineto{\pgfqpoint{1.727555in}{1.599121in}}%
\pgfpathlineto{\pgfqpoint{1.729614in}{1.618543in}}%
\pgfpathlineto{\pgfqpoint{1.731673in}{1.550566in}}%
\pgfpathlineto{\pgfqpoint{1.737850in}{1.637965in}}%
\pgfpathlineto{\pgfqpoint{1.739909in}{1.647676in}}%
\pgfpathlineto{\pgfqpoint{1.741968in}{1.686520in}}%
\pgfpathlineto{\pgfqpoint{1.746086in}{1.637965in}}%
\pgfpathlineto{\pgfqpoint{1.752262in}{1.628254in}}%
\pgfpathlineto{\pgfqpoint{1.756380in}{1.676809in}}%
\pgfpathlineto{\pgfqpoint{1.758439in}{1.696231in}}%
\pgfpathlineto{\pgfqpoint{1.766675in}{1.715653in}}%
\pgfpathlineto{\pgfqpoint{1.768734in}{1.754497in}}%
\pgfpathlineto{\pgfqpoint{1.770793in}{1.744786in}}%
\pgfpathlineto{\pgfqpoint{1.772852in}{1.725364in}}%
\pgfpathlineto{\pgfqpoint{1.781088in}{1.686520in}}%
\pgfpathlineto{\pgfqpoint{1.783147in}{1.705942in}}%
\pgfpathlineto{\pgfqpoint{1.787264in}{1.628254in}}%
\pgfpathlineto{\pgfqpoint{1.789323in}{1.608832in}}%
\pgfpathlineto{\pgfqpoint{1.795500in}{1.608832in}}%
\pgfpathlineto{\pgfqpoint{1.797559in}{1.599121in}}%
\pgfpathlineto{\pgfqpoint{1.801677in}{1.569988in}}%
\pgfpathlineto{\pgfqpoint{1.803736in}{1.521434in}}%
\pgfpathlineto{\pgfqpoint{1.811972in}{1.550566in}}%
\pgfpathlineto{\pgfqpoint{1.814031in}{1.521434in}}%
\pgfpathlineto{\pgfqpoint{1.816090in}{1.511723in}}%
\pgfpathlineto{\pgfqpoint{1.818149in}{1.550566in}}%
\pgfpathlineto{\pgfqpoint{1.824325in}{1.550566in}}%
\pgfpathlineto{\pgfqpoint{1.826384in}{1.521434in}}%
\pgfpathlineto{\pgfqpoint{1.830502in}{1.502012in}}%
\pgfpathlineto{\pgfqpoint{1.832561in}{1.434035in}}%
\pgfpathlineto{\pgfqpoint{1.838738in}{1.482590in}}%
\pgfpathlineto{\pgfqpoint{1.842856in}{1.395191in}}%
\pgfpathlineto{\pgfqpoint{1.844915in}{1.375769in}}%
\pgfpathlineto{\pgfqpoint{1.846974in}{1.414613in}}%
\pgfpathlineto{\pgfqpoint{1.853151in}{1.336925in}}%
\pgfpathlineto{\pgfqpoint{1.857269in}{1.385480in}}%
\pgfpathlineto{\pgfqpoint{1.859328in}{1.317503in}}%
\pgfpathlineto{\pgfqpoint{1.861386in}{1.385480in}}%
\pgfpathlineto{\pgfqpoint{1.869622in}{1.414613in}}%
\pgfpathlineto{\pgfqpoint{1.871681in}{1.414613in}}%
\pgfpathlineto{\pgfqpoint{1.873740in}{1.385480in}}%
\pgfpathlineto{\pgfqpoint{1.875799in}{1.434035in}}%
\pgfpathlineto{\pgfqpoint{1.881976in}{1.453457in}}%
\pgfpathlineto{\pgfqpoint{1.888153in}{1.395191in}}%
\pgfpathlineto{\pgfqpoint{1.890212in}{1.472879in}}%
\pgfpathlineto{\pgfqpoint{1.896389in}{1.453457in}}%
\pgfpathlineto{\pgfqpoint{1.898447in}{1.521434in}}%
\pgfpathlineto{\pgfqpoint{1.902565in}{1.521434in}}%
\pgfpathlineto{\pgfqpoint{1.904624in}{1.550566in}}%
\pgfpathlineto{\pgfqpoint{1.910801in}{1.579699in}}%
\pgfpathlineto{\pgfqpoint{1.912860in}{1.550566in}}%
\pgfpathlineto{\pgfqpoint{1.916978in}{1.599121in}}%
\pgfpathlineto{\pgfqpoint{1.919037in}{1.637965in}}%
\pgfpathlineto{\pgfqpoint{1.925214in}{1.637965in}}%
\pgfpathlineto{\pgfqpoint{1.927273in}{1.647676in}}%
\pgfpathlineto{\pgfqpoint{1.929332in}{1.540855in}}%
\pgfpathlineto{\pgfqpoint{1.931391in}{1.540855in}}%
\pgfpathlineto{\pgfqpoint{1.933450in}{1.511723in}}%
\pgfpathlineto{\pgfqpoint{1.939626in}{1.540855in}}%
\pgfpathlineto{\pgfqpoint{1.941685in}{1.579699in}}%
\pgfpathlineto{\pgfqpoint{1.943744in}{1.540855in}}%
\pgfpathlineto{\pgfqpoint{1.945803in}{1.560277in}}%
\pgfpathlineto{\pgfqpoint{1.954039in}{1.560277in}}%
\pgfpathlineto{\pgfqpoint{1.956098in}{1.453457in}}%
\pgfpathlineto{\pgfqpoint{1.960216in}{1.404902in}}%
\pgfpathlineto{\pgfqpoint{1.962275in}{1.434035in}}%
\pgfpathlineto{\pgfqpoint{1.968452in}{1.424324in}}%
\pgfpathlineto{\pgfqpoint{1.970511in}{1.395191in}}%
\pgfpathlineto{\pgfqpoint{1.972570in}{1.404902in}}%
\pgfpathlineto{\pgfqpoint{1.974628in}{1.375769in}}%
\pgfpathlineto{\pgfqpoint{1.982864in}{1.375769in}}%
\pgfpathlineto{\pgfqpoint{1.984923in}{1.414613in}}%
\pgfpathlineto{\pgfqpoint{1.986982in}{1.424324in}}%
\pgfpathlineto{\pgfqpoint{1.989041in}{1.443746in}}%
\pgfpathlineto{\pgfqpoint{1.991100in}{1.414613in}}%
\pgfpathlineto{\pgfqpoint{1.997277in}{1.424324in}}%
\pgfpathlineto{\pgfqpoint{2.005513in}{1.511723in}}%
\pgfpathlineto{\pgfqpoint{2.011689in}{1.521434in}}%
\pgfpathlineto{\pgfqpoint{2.013748in}{1.531145in}}%
\pgfpathlineto{\pgfqpoint{2.015807in}{1.502012in}}%
\pgfpathlineto{\pgfqpoint{2.017866in}{1.453457in}}%
\pgfpathlineto{\pgfqpoint{2.019925in}{1.443746in}}%
\pgfpathlineto{\pgfqpoint{2.026102in}{1.472879in}}%
\pgfpathlineto{\pgfqpoint{2.028161in}{1.424324in}}%
\pgfpathlineto{\pgfqpoint{2.030220in}{1.424324in}}%
\pgfpathlineto{\pgfqpoint{2.032279in}{1.395191in}}%
\pgfpathlineto{\pgfqpoint{2.034338in}{1.414613in}}%
\pgfpathlineto{\pgfqpoint{2.040515in}{1.395191in}}%
\pgfpathlineto{\pgfqpoint{2.042574in}{1.395191in}}%
\pgfpathlineto{\pgfqpoint{2.046692in}{1.434035in}}%
\pgfpathlineto{\pgfqpoint{2.048751in}{1.434035in}}%
\pgfpathlineto{\pgfqpoint{2.054927in}{1.463168in}}%
\pgfpathlineto{\pgfqpoint{2.056986in}{1.492301in}}%
\pgfpathlineto{\pgfqpoint{2.059045in}{1.569988in}}%
\pgfpathlineto{\pgfqpoint{2.061104in}{1.560277in}}%
\pgfpathlineto{\pgfqpoint{2.063163in}{1.560277in}}%
\pgfpathlineto{\pgfqpoint{2.073458in}{1.589410in}}%
\pgfpathlineto{\pgfqpoint{2.075517in}{1.540855in}}%
\pgfpathlineto{\pgfqpoint{2.077576in}{1.569988in}}%
\pgfpathlineto{\pgfqpoint{2.085812in}{1.540855in}}%
\pgfpathlineto{\pgfqpoint{2.087870in}{1.579699in}}%
\pgfpathlineto{\pgfqpoint{2.089929in}{1.560277in}}%
\pgfpathlineto{\pgfqpoint{2.091988in}{1.453457in}}%
\pgfpathlineto{\pgfqpoint{2.098165in}{1.472879in}}%
\pgfpathlineto{\pgfqpoint{2.100224in}{1.453457in}}%
\pgfpathlineto{\pgfqpoint{2.102283in}{1.453457in}}%
\pgfpathlineto{\pgfqpoint{2.104342in}{1.443746in}}%
\pgfpathlineto{\pgfqpoint{2.106401in}{1.404902in}}%
\pgfpathlineto{\pgfqpoint{2.112578in}{1.404902in}}%
\pgfpathlineto{\pgfqpoint{2.114637in}{1.414613in}}%
\pgfpathlineto{\pgfqpoint{2.116696in}{1.366058in}}%
\pgfpathlineto{\pgfqpoint{2.118755in}{1.375769in}}%
\pgfpathlineto{\pgfqpoint{2.120814in}{1.375769in}}%
\pgfpathlineto{\pgfqpoint{2.129049in}{1.434035in}}%
\pgfpathlineto{\pgfqpoint{2.131108in}{1.424324in}}%
\pgfpathlineto{\pgfqpoint{2.133167in}{1.453457in}}%
\pgfpathlineto{\pgfqpoint{2.135226in}{1.317503in}}%
\pgfpathlineto{\pgfqpoint{2.141403in}{1.288370in}}%
\pgfpathlineto{\pgfqpoint{2.143462in}{1.288370in}}%
\pgfpathlineto{\pgfqpoint{2.145521in}{1.298081in}}%
\pgfpathlineto{\pgfqpoint{2.147580in}{1.259237in}}%
\pgfpathlineto{\pgfqpoint{2.149639in}{1.268948in}}%
\pgfpathlineto{\pgfqpoint{2.157875in}{1.239815in}}%
\pgfpathlineto{\pgfqpoint{2.159934in}{1.259237in}}%
\pgfpathlineto{\pgfqpoint{2.161993in}{1.259237in}}%
\pgfpathlineto{\pgfqpoint{2.164051in}{1.288370in}}%
\pgfpathlineto{\pgfqpoint{2.170228in}{1.336925in}}%
\pgfpathlineto{\pgfqpoint{2.172287in}{1.366058in}}%
\pgfpathlineto{\pgfqpoint{2.174346in}{1.356347in}}%
\pgfpathlineto{\pgfqpoint{2.176405in}{1.356347in}}%
\pgfpathlineto{\pgfqpoint{2.178464in}{1.385480in}}%
\pgfpathlineto{\pgfqpoint{2.184641in}{1.356347in}}%
\pgfpathlineto{\pgfqpoint{2.188759in}{1.404902in}}%
\pgfpathlineto{\pgfqpoint{2.190818in}{1.375769in}}%
\pgfpathlineto{\pgfqpoint{2.192877in}{1.385480in}}%
\pgfpathlineto{\pgfqpoint{2.199054in}{1.395191in}}%
\pgfpathlineto{\pgfqpoint{2.201112in}{1.424324in}}%
\pgfpathlineto{\pgfqpoint{2.205230in}{1.395191in}}%
\pgfpathlineto{\pgfqpoint{2.207289in}{1.346636in}}%
\pgfpathlineto{\pgfqpoint{2.217584in}{1.346636in}}%
\pgfpathlineto{\pgfqpoint{2.219643in}{1.317503in}}%
\pgfpathlineto{\pgfqpoint{2.221702in}{1.395191in}}%
\pgfpathlineto{\pgfqpoint{2.227879in}{1.414613in}}%
\pgfpathlineto{\pgfqpoint{2.231997in}{1.366058in}}%
\pgfpathlineto{\pgfqpoint{2.234056in}{1.434035in}}%
\pgfpathlineto{\pgfqpoint{2.236115in}{1.385480in}}%
\pgfpathlineto{\pgfqpoint{2.242291in}{1.395191in}}%
\pgfpathlineto{\pgfqpoint{2.244350in}{1.434035in}}%
\pgfpathlineto{\pgfqpoint{2.248468in}{1.385480in}}%
\pgfpathlineto{\pgfqpoint{2.250527in}{1.434035in}}%
\pgfpathlineto{\pgfqpoint{2.256704in}{1.434035in}}%
\pgfpathlineto{\pgfqpoint{2.258763in}{1.414613in}}%
\pgfpathlineto{\pgfqpoint{2.262881in}{1.453457in}}%
\pgfpathlineto{\pgfqpoint{2.264940in}{1.511723in}}%
\pgfpathlineto{\pgfqpoint{2.273176in}{1.472879in}}%
\pgfpathlineto{\pgfqpoint{2.275235in}{1.472879in}}%
\pgfpathlineto{\pgfqpoint{2.277293in}{1.453457in}}%
\pgfpathlineto{\pgfqpoint{2.279352in}{1.472879in}}%
\pgfpathlineto{\pgfqpoint{2.287588in}{1.414613in}}%
\pgfpathlineto{\pgfqpoint{2.289647in}{1.414613in}}%
\pgfpathlineto{\pgfqpoint{2.291706in}{1.453457in}}%
\pgfpathlineto{\pgfqpoint{2.293765in}{1.463168in}}%
\pgfpathlineto{\pgfqpoint{2.299942in}{1.463168in}}%
\pgfpathlineto{\pgfqpoint{2.302001in}{1.472879in}}%
\pgfpathlineto{\pgfqpoint{2.306119in}{1.414613in}}%
\pgfpathlineto{\pgfqpoint{2.308178in}{1.443746in}}%
\pgfpathlineto{\pgfqpoint{2.314355in}{1.463168in}}%
\pgfpathlineto{\pgfqpoint{2.320531in}{1.463168in}}%
\pgfpathlineto{\pgfqpoint{2.322590in}{1.443746in}}%
\pgfpathlineto{\pgfqpoint{2.328767in}{1.434035in}}%
\pgfpathlineto{\pgfqpoint{2.330826in}{1.424324in}}%
\pgfpathlineto{\pgfqpoint{2.332885in}{1.424324in}}%
\pgfpathlineto{\pgfqpoint{2.334944in}{1.404902in}}%
\pgfpathlineto{\pgfqpoint{2.337003in}{1.443746in}}%
\pgfpathlineto{\pgfqpoint{2.343180in}{1.472879in}}%
\pgfpathlineto{\pgfqpoint{2.349357in}{1.531145in}}%
\pgfpathlineto{\pgfqpoint{2.351416in}{1.502012in}}%
\pgfpathlineto{\pgfqpoint{2.359651in}{1.540855in}}%
\pgfpathlineto{\pgfqpoint{2.361710in}{1.540855in}}%
\pgfpathlineto{\pgfqpoint{2.365828in}{1.511723in}}%
\pgfpathlineto{\pgfqpoint{2.372005in}{1.482590in}}%
\pgfpathlineto{\pgfqpoint{2.374064in}{1.492301in}}%
\pgfpathlineto{\pgfqpoint{2.376123in}{1.482590in}}%
\pgfpathlineto{\pgfqpoint{2.378182in}{1.511723in}}%
\pgfpathlineto{\pgfqpoint{2.386418in}{1.511723in}}%
\pgfpathlineto{\pgfqpoint{2.388477in}{1.531145in}}%
\pgfpathlineto{\pgfqpoint{2.390535in}{1.531145in}}%
\pgfpathlineto{\pgfqpoint{2.392594in}{1.540855in}}%
\pgfpathlineto{\pgfqpoint{2.394653in}{1.531145in}}%
\pgfpathlineto{\pgfqpoint{2.400830in}{1.531145in}}%
\pgfpathlineto{\pgfqpoint{2.404948in}{1.482590in}}%
\pgfpathlineto{\pgfqpoint{2.407007in}{1.482590in}}%
\pgfpathlineto{\pgfqpoint{2.409066in}{1.472879in}}%
\pgfpathlineto{\pgfqpoint{2.415243in}{1.492301in}}%
\pgfpathlineto{\pgfqpoint{2.419361in}{1.569988in}}%
\pgfpathlineto{\pgfqpoint{2.431714in}{1.686520in}}%
\pgfpathlineto{\pgfqpoint{2.433773in}{1.667098in}}%
\pgfpathlineto{\pgfqpoint{2.437891in}{1.735075in}}%
\pgfpathlineto{\pgfqpoint{2.444068in}{1.744786in}}%
\pgfpathlineto{\pgfqpoint{2.446127in}{1.735075in}}%
\pgfpathlineto{\pgfqpoint{2.448186in}{1.783630in}}%
\pgfpathlineto{\pgfqpoint{2.452304in}{1.783630in}}%
\pgfpathlineto{\pgfqpoint{2.458481in}{1.773919in}}%
\pgfpathlineto{\pgfqpoint{2.460540in}{1.754497in}}%
\pgfpathlineto{\pgfqpoint{2.464658in}{1.803052in}}%
\pgfpathlineto{\pgfqpoint{2.466716in}{1.773919in}}%
\pgfpathlineto{\pgfqpoint{2.472893in}{1.793341in}}%
\pgfpathlineto{\pgfqpoint{2.474952in}{1.783630in}}%
\pgfpathlineto{\pgfqpoint{2.477011in}{1.764208in}}%
\pgfpathlineto{\pgfqpoint{2.481129in}{1.812763in}}%
\pgfpathlineto{\pgfqpoint{2.487306in}{1.812763in}}%
\pgfpathlineto{\pgfqpoint{2.489365in}{1.832185in}}%
\pgfpathlineto{\pgfqpoint{2.491424in}{1.929295in}}%
\pgfpathlineto{\pgfqpoint{2.493483in}{1.948717in}}%
\pgfpathlineto{\pgfqpoint{2.495542in}{1.939006in}}%
\pgfpathlineto{\pgfqpoint{2.501719in}{1.900162in}}%
\pgfpathlineto{\pgfqpoint{2.503778in}{1.909873in}}%
\pgfpathlineto{\pgfqpoint{2.505836in}{1.871029in}}%
\pgfpathlineto{\pgfqpoint{2.507895in}{1.880740in}}%
\pgfpathlineto{\pgfqpoint{2.509954in}{1.880740in}}%
\pgfpathlineto{\pgfqpoint{2.518190in}{1.939006in}}%
\pgfpathlineto{\pgfqpoint{2.520249in}{1.919584in}}%
\pgfpathlineto{\pgfqpoint{2.524367in}{1.861318in}}%
\pgfpathlineto{\pgfqpoint{2.532603in}{1.880740in}}%
\pgfpathlineto{\pgfqpoint{2.534662in}{1.900162in}}%
\pgfpathlineto{\pgfqpoint{2.536721in}{1.832185in}}%
\pgfpathlineto{\pgfqpoint{2.538780in}{1.880740in}}%
\pgfpathlineto{\pgfqpoint{2.544956in}{1.871029in}}%
\pgfpathlineto{\pgfqpoint{2.547015in}{1.851607in}}%
\pgfpathlineto{\pgfqpoint{2.549074in}{1.861318in}}%
\pgfpathlineto{\pgfqpoint{2.551133in}{1.841896in}}%
\pgfpathlineto{\pgfqpoint{2.553192in}{1.871029in}}%
\pgfpathlineto{\pgfqpoint{2.561428in}{1.832185in}}%
\pgfpathlineto{\pgfqpoint{2.563487in}{1.890451in}}%
\pgfpathlineto{\pgfqpoint{2.565546in}{1.909873in}}%
\pgfpathlineto{\pgfqpoint{2.567605in}{1.861318in}}%
\pgfpathlineto{\pgfqpoint{2.573782in}{1.822474in}}%
\pgfpathlineto{\pgfqpoint{2.575841in}{1.871029in}}%
\pgfpathlineto{\pgfqpoint{2.577900in}{1.890451in}}%
\pgfpathlineto{\pgfqpoint{2.579958in}{1.871029in}}%
\pgfpathlineto{\pgfqpoint{2.582017in}{1.880740in}}%
\pgfpathlineto{\pgfqpoint{2.588194in}{1.880740in}}%
\pgfpathlineto{\pgfqpoint{2.590253in}{1.851607in}}%
\pgfpathlineto{\pgfqpoint{2.592312in}{1.880740in}}%
\pgfpathlineto{\pgfqpoint{2.594371in}{1.871029in}}%
\pgfpathlineto{\pgfqpoint{2.596430in}{1.871029in}}%
\pgfpathlineto{\pgfqpoint{2.602607in}{1.822474in}}%
\pgfpathlineto{\pgfqpoint{2.604666in}{1.822474in}}%
\pgfpathlineto{\pgfqpoint{2.606725in}{1.812763in}}%
\pgfpathlineto{\pgfqpoint{2.608784in}{1.861318in}}%
\pgfpathlineto{\pgfqpoint{2.617020in}{1.861318in}}%
\pgfpathlineto{\pgfqpoint{2.619078in}{1.909873in}}%
\pgfpathlineto{\pgfqpoint{2.621137in}{1.929295in}}%
\pgfpathlineto{\pgfqpoint{2.623196in}{1.880740in}}%
\pgfpathlineto{\pgfqpoint{2.625255in}{1.871029in}}%
\pgfpathlineto{\pgfqpoint{2.633491in}{1.880740in}}%
\pgfpathlineto{\pgfqpoint{2.635550in}{1.880740in}}%
\pgfpathlineto{\pgfqpoint{2.639668in}{1.783630in}}%
\pgfpathlineto{\pgfqpoint{2.647904in}{1.880740in}}%
\pgfpathlineto{\pgfqpoint{2.649963in}{1.948717in}}%
\pgfpathlineto{\pgfqpoint{2.652022in}{1.977850in}}%
\pgfpathlineto{\pgfqpoint{2.654081in}{1.977850in}}%
\pgfpathlineto{\pgfqpoint{2.660257in}{1.968139in}}%
\pgfpathlineto{\pgfqpoint{2.662316in}{1.977850in}}%
\pgfpathlineto{\pgfqpoint{2.664375in}{2.016694in}}%
\pgfpathlineto{\pgfqpoint{2.666434in}{2.026405in}}%
\pgfpathlineto{\pgfqpoint{2.668493in}{2.016694in}}%
\pgfpathlineto{\pgfqpoint{2.674670in}{2.055538in}}%
\pgfpathlineto{\pgfqpoint{2.676729in}{2.055538in}}%
\pgfpathlineto{\pgfqpoint{2.678788in}{1.987561in}}%
\pgfpathlineto{\pgfqpoint{2.680847in}{2.006983in}}%
\pgfpathlineto{\pgfqpoint{2.682906in}{1.987561in}}%
\pgfpathlineto{\pgfqpoint{2.689083in}{1.958428in}}%
\pgfpathlineto{\pgfqpoint{2.691142in}{1.929295in}}%
\pgfpathlineto{\pgfqpoint{2.693200in}{1.929295in}}%
\pgfpathlineto{\pgfqpoint{2.695259in}{1.919584in}}%
\pgfpathlineto{\pgfqpoint{2.697318in}{1.919584in}}%
\pgfpathlineto{\pgfqpoint{2.703495in}{1.929295in}}%
\pgfpathlineto{\pgfqpoint{2.705554in}{1.958428in}}%
\pgfpathlineto{\pgfqpoint{2.707613in}{1.919584in}}%
\pgfpathlineto{\pgfqpoint{2.709672in}{1.939006in}}%
\pgfpathlineto{\pgfqpoint{2.717908in}{1.900162in}}%
\pgfpathlineto{\pgfqpoint{2.719967in}{1.909873in}}%
\pgfpathlineto{\pgfqpoint{2.722026in}{1.900162in}}%
\pgfpathlineto{\pgfqpoint{2.724085in}{1.900162in}}%
\pgfpathlineto{\pgfqpoint{2.726144in}{1.948717in}}%
\pgfpathlineto{\pgfqpoint{2.732320in}{1.948717in}}%
\pgfpathlineto{\pgfqpoint{2.734379in}{1.900162in}}%
\pgfpathlineto{\pgfqpoint{2.736438in}{1.900162in}}%
\pgfpathlineto{\pgfqpoint{2.738497in}{1.871029in}}%
\pgfpathlineto{\pgfqpoint{2.746733in}{1.871029in}}%
\pgfpathlineto{\pgfqpoint{2.748792in}{1.841896in}}%
\pgfpathlineto{\pgfqpoint{2.750851in}{1.851607in}}%
\pgfpathlineto{\pgfqpoint{2.752910in}{1.871029in}}%
\pgfpathlineto{\pgfqpoint{2.754969in}{1.861318in}}%
\pgfpathlineto{\pgfqpoint{2.761146in}{1.909873in}}%
\pgfpathlineto{\pgfqpoint{2.763205in}{1.948717in}}%
\pgfpathlineto{\pgfqpoint{2.765264in}{1.939006in}}%
\pgfpathlineto{\pgfqpoint{2.767323in}{1.909873in}}%
\pgfpathlineto{\pgfqpoint{2.769381in}{1.939006in}}%
\pgfpathlineto{\pgfqpoint{2.775558in}{1.939006in}}%
\pgfpathlineto{\pgfqpoint{2.777617in}{1.929295in}}%
\pgfpathlineto{\pgfqpoint{2.781735in}{1.977850in}}%
\pgfpathlineto{\pgfqpoint{2.783794in}{1.977850in}}%
\pgfpathlineto{\pgfqpoint{2.789971in}{1.987561in}}%
\pgfpathlineto{\pgfqpoint{2.792030in}{2.026405in}}%
\pgfpathlineto{\pgfqpoint{2.794089in}{2.006983in}}%
\pgfpathlineto{\pgfqpoint{2.796148in}{2.006983in}}%
\pgfpathlineto{\pgfqpoint{2.798207in}{1.948717in}}%
\pgfpathlineto{\pgfqpoint{2.804384in}{1.968139in}}%
\pgfpathlineto{\pgfqpoint{2.808501in}{1.919584in}}%
\pgfpathlineto{\pgfqpoint{2.812619in}{1.939006in}}%
\pgfpathlineto{\pgfqpoint{2.818796in}{1.948717in}}%
\pgfpathlineto{\pgfqpoint{2.820855in}{1.968139in}}%
\pgfpathlineto{\pgfqpoint{2.822914in}{1.948717in}}%
\pgfpathlineto{\pgfqpoint{2.824973in}{1.958428in}}%
\pgfpathlineto{\pgfqpoint{2.827032in}{1.958428in}}%
\pgfpathlineto{\pgfqpoint{2.835268in}{1.939006in}}%
\pgfpathlineto{\pgfqpoint{2.841445in}{1.939006in}}%
\pgfpathlineto{\pgfqpoint{2.847621in}{1.977850in}}%
\pgfpathlineto{\pgfqpoint{2.849680in}{1.958428in}}%
\pgfpathlineto{\pgfqpoint{2.855857in}{2.006983in}}%
\pgfpathlineto{\pgfqpoint{2.862034in}{2.006983in}}%
\pgfpathlineto{\pgfqpoint{2.864093in}{2.036116in}}%
\pgfpathlineto{\pgfqpoint{2.866152in}{2.006983in}}%
\pgfpathlineto{\pgfqpoint{2.868211in}{2.006983in}}%
\pgfpathlineto{\pgfqpoint{2.870270in}{1.977850in}}%
\pgfpathlineto{\pgfqpoint{2.876447in}{2.016694in}}%
\pgfpathlineto{\pgfqpoint{2.880565in}{2.016694in}}%
\pgfpathlineto{\pgfqpoint{2.882623in}{1.997272in}}%
\pgfpathlineto{\pgfqpoint{2.884682in}{1.997272in}}%
\pgfpathlineto{\pgfqpoint{2.890859in}{2.016694in}}%
\pgfpathlineto{\pgfqpoint{2.892918in}{2.036116in}}%
\pgfpathlineto{\pgfqpoint{2.894977in}{1.997272in}}%
\pgfpathlineto{\pgfqpoint{2.897036in}{2.036116in}}%
\pgfpathlineto{\pgfqpoint{2.899095in}{2.036116in}}%
\pgfpathlineto{\pgfqpoint{2.905272in}{2.065249in}}%
\pgfpathlineto{\pgfqpoint{2.909390in}{2.065249in}}%
\pgfpathlineto{\pgfqpoint{2.911449in}{2.055538in}}%
\pgfpathlineto{\pgfqpoint{2.919685in}{2.055538in}}%
\pgfpathlineto{\pgfqpoint{2.923802in}{2.006983in}}%
\pgfpathlineto{\pgfqpoint{2.925861in}{2.026405in}}%
\pgfpathlineto{\pgfqpoint{2.927920in}{2.006983in}}%
\pgfpathlineto{\pgfqpoint{2.934097in}{2.016694in}}%
\pgfpathlineto{\pgfqpoint{2.936156in}{2.016694in}}%
\pgfpathlineto{\pgfqpoint{2.938215in}{2.026405in}}%
\pgfpathlineto{\pgfqpoint{2.940274in}{2.026405in}}%
\pgfpathlineto{\pgfqpoint{2.942333in}{2.016694in}}%
\pgfpathlineto{\pgfqpoint{2.948510in}{2.026405in}}%
\pgfpathlineto{\pgfqpoint{2.950569in}{2.055538in}}%
\pgfpathlineto{\pgfqpoint{2.952628in}{2.016694in}}%
\pgfpathlineto{\pgfqpoint{2.954687in}{2.016694in}}%
\pgfpathlineto{\pgfqpoint{2.956746in}{1.997272in}}%
\pgfpathlineto{\pgfqpoint{2.964981in}{1.997272in}}%
\pgfpathlineto{\pgfqpoint{2.967040in}{2.016694in}}%
\pgfpathlineto{\pgfqpoint{2.969099in}{1.997272in}}%
\pgfpathlineto{\pgfqpoint{2.971158in}{2.016694in}}%
\pgfpathlineto{\pgfqpoint{2.979394in}{2.016694in}}%
\pgfpathlineto{\pgfqpoint{2.981453in}{1.987561in}}%
\pgfpathlineto{\pgfqpoint{2.983512in}{1.987561in}}%
\pgfpathlineto{\pgfqpoint{2.985571in}{1.958428in}}%
\pgfpathlineto{\pgfqpoint{2.991748in}{1.987561in}}%
\pgfpathlineto{\pgfqpoint{2.993807in}{2.006983in}}%
\pgfpathlineto{\pgfqpoint{2.997924in}{1.977850in}}%
\pgfpathlineto{\pgfqpoint{2.999983in}{1.987561in}}%
\pgfpathlineto{\pgfqpoint{3.006160in}{1.977850in}}%
\pgfpathlineto{\pgfqpoint{3.008219in}{1.987561in}}%
\pgfpathlineto{\pgfqpoint{3.010278in}{1.977850in}}%
\pgfpathlineto{\pgfqpoint{3.012337in}{1.987561in}}%
\pgfpathlineto{\pgfqpoint{3.014396in}{2.006983in}}%
\pgfpathlineto{\pgfqpoint{3.020573in}{1.987561in}}%
\pgfpathlineto{\pgfqpoint{3.026750in}{1.987561in}}%
\pgfpathlineto{\pgfqpoint{3.028809in}{2.006983in}}%
\pgfpathlineto{\pgfqpoint{3.037044in}{1.958428in}}%
\pgfpathlineto{\pgfqpoint{3.039103in}{1.958428in}}%
\pgfpathlineto{\pgfqpoint{3.041162in}{1.929295in}}%
\pgfpathlineto{\pgfqpoint{3.043221in}{1.929295in}}%
\pgfpathlineto{\pgfqpoint{3.057634in}{2.045827in}}%
\pgfpathlineto{\pgfqpoint{3.063811in}{2.055538in}}%
\pgfpathlineto{\pgfqpoint{3.065870in}{2.055538in}}%
\pgfpathlineto{\pgfqpoint{3.067929in}{2.104092in}}%
\pgfpathlineto{\pgfqpoint{3.069988in}{2.104092in}}%
\pgfpathlineto{\pgfqpoint{3.072046in}{2.113803in}}%
\pgfpathlineto{\pgfqpoint{3.078223in}{2.094382in}}%
\pgfpathlineto{\pgfqpoint{3.080282in}{2.104092in}}%
\pgfpathlineto{\pgfqpoint{3.082341in}{2.123514in}}%
\pgfpathlineto{\pgfqpoint{3.084400in}{2.104092in}}%
\pgfpathlineto{\pgfqpoint{3.086459in}{2.123514in}}%
\pgfpathlineto{\pgfqpoint{3.092636in}{2.142936in}}%
\pgfpathlineto{\pgfqpoint{3.094695in}{2.123514in}}%
\pgfpathlineto{\pgfqpoint{3.096754in}{2.123514in}}%
\pgfpathlineto{\pgfqpoint{3.098813in}{2.142936in}}%
\pgfpathlineto{\pgfqpoint{3.100872in}{2.191491in}}%
\pgfpathlineto{\pgfqpoint{3.109108in}{2.162358in}}%
\pgfpathlineto{\pgfqpoint{3.115284in}{2.162358in}}%
\pgfpathlineto{\pgfqpoint{3.121461in}{2.191491in}}%
\pgfpathlineto{\pgfqpoint{3.123520in}{2.191491in}}%
\pgfpathlineto{\pgfqpoint{3.125579in}{2.240046in}}%
\pgfpathlineto{\pgfqpoint{3.127638in}{2.230335in}}%
\pgfpathlineto{\pgfqpoint{3.129697in}{2.249757in}}%
\pgfpathlineto{\pgfqpoint{3.135874in}{2.230335in}}%
\pgfpathlineto{\pgfqpoint{3.142051in}{2.278890in}}%
\pgfpathlineto{\pgfqpoint{3.144110in}{2.240046in}}%
\pgfpathlineto{\pgfqpoint{3.150286in}{2.230335in}}%
\pgfpathlineto{\pgfqpoint{3.154404in}{2.259468in}}%
\pgfpathlineto{\pgfqpoint{3.156463in}{2.259468in}}%
\pgfpathlineto{\pgfqpoint{3.158522in}{2.278890in}}%
\pgfpathlineto{\pgfqpoint{3.164699in}{2.259468in}}%
\pgfpathlineto{\pgfqpoint{3.168817in}{2.298312in}}%
\pgfpathlineto{\pgfqpoint{3.170876in}{2.278890in}}%
\pgfpathlineto{\pgfqpoint{3.172935in}{2.317734in}}%
\pgfpathlineto{\pgfqpoint{3.179112in}{2.346867in}}%
\pgfpathlineto{\pgfqpoint{3.181171in}{2.327445in}}%
\pgfpathlineto{\pgfqpoint{3.183230in}{2.327445in}}%
\pgfpathlineto{\pgfqpoint{3.185289in}{2.366289in}}%
\pgfpathlineto{\pgfqpoint{3.187347in}{2.376000in}}%
\pgfpathlineto{\pgfqpoint{3.193524in}{2.414844in}}%
\pgfpathlineto{\pgfqpoint{3.195583in}{2.414844in}}%
\pgfpathlineto{\pgfqpoint{3.197642in}{2.385711in}}%
\pgfpathlineto{\pgfqpoint{3.201760in}{2.395422in}}%
\pgfpathlineto{\pgfqpoint{3.207937in}{2.385711in}}%
\pgfpathlineto{\pgfqpoint{3.209996in}{2.395422in}}%
\pgfpathlineto{\pgfqpoint{3.212055in}{2.424555in}}%
\pgfpathlineto{\pgfqpoint{3.216173in}{2.424555in}}%
\pgfpathlineto{\pgfqpoint{3.222350in}{2.443977in}}%
\pgfpathlineto{\pgfqpoint{3.224408in}{2.473110in}}%
\pgfpathlineto{\pgfqpoint{3.226467in}{2.424555in}}%
\pgfpathlineto{\pgfqpoint{3.228526in}{2.443977in}}%
\pgfpathlineto{\pgfqpoint{3.230585in}{2.443977in}}%
\pgfpathlineto{\pgfqpoint{3.238821in}{2.473110in}}%
\pgfpathlineto{\pgfqpoint{3.240880in}{2.434266in}}%
\pgfpathlineto{\pgfqpoint{3.244998in}{2.482821in}}%
\pgfpathlineto{\pgfqpoint{3.251175in}{2.482821in}}%
\pgfpathlineto{\pgfqpoint{3.253234in}{2.511954in}}%
\pgfpathlineto{\pgfqpoint{3.255293in}{2.511954in}}%
\pgfpathlineto{\pgfqpoint{3.259411in}{2.550798in}}%
\pgfpathlineto{\pgfqpoint{3.267646in}{2.560509in}}%
\pgfpathlineto{\pgfqpoint{3.269705in}{2.531376in}}%
\pgfpathlineto{\pgfqpoint{3.271764in}{2.550798in}}%
\pgfpathlineto{\pgfqpoint{3.273823in}{2.531376in}}%
\pgfpathlineto{\pgfqpoint{3.282059in}{2.560509in}}%
\pgfpathlineto{\pgfqpoint{3.286177in}{2.599353in}}%
\pgfpathlineto{\pgfqpoint{3.294413in}{2.599353in}}%
\pgfpathlineto{\pgfqpoint{3.296472in}{2.618775in}}%
\pgfpathlineto{\pgfqpoint{3.300589in}{2.618775in}}%
\pgfpathlineto{\pgfqpoint{3.310884in}{2.667329in}}%
\pgfpathlineto{\pgfqpoint{3.312943in}{2.686751in}}%
\pgfpathlineto{\pgfqpoint{3.315002in}{2.686751in}}%
\pgfpathlineto{\pgfqpoint{3.317061in}{2.696462in}}%
\pgfpathlineto{\pgfqpoint{3.323238in}{2.715884in}}%
\pgfpathlineto{\pgfqpoint{3.325297in}{2.696462in}}%
\pgfpathlineto{\pgfqpoint{3.327356in}{2.715884in}}%
\pgfpathlineto{\pgfqpoint{3.329415in}{2.715884in}}%
\pgfpathlineto{\pgfqpoint{3.331474in}{2.764439in}}%
\pgfpathlineto{\pgfqpoint{3.337650in}{2.745017in}}%
\pgfpathlineto{\pgfqpoint{3.343827in}{2.793572in}}%
\pgfpathlineto{\pgfqpoint{3.345886in}{2.783861in}}%
\pgfpathlineto{\pgfqpoint{3.352063in}{2.715884in}}%
\pgfpathlineto{\pgfqpoint{3.354122in}{2.735306in}}%
\pgfpathlineto{\pgfqpoint{3.356181in}{2.783861in}}%
\pgfpathlineto{\pgfqpoint{3.358240in}{2.764439in}}%
\pgfpathlineto{\pgfqpoint{3.360299in}{2.686751in}}%
\pgfpathlineto{\pgfqpoint{3.368535in}{2.735306in}}%
\pgfpathlineto{\pgfqpoint{3.370594in}{2.803283in}}%
\pgfpathlineto{\pgfqpoint{3.374711in}{2.842127in}}%
\pgfpathlineto{\pgfqpoint{3.385006in}{2.890682in}}%
\pgfpathlineto{\pgfqpoint{3.387065in}{2.880971in}}%
\pgfpathlineto{\pgfqpoint{3.389124in}{2.880971in}}%
\pgfpathlineto{\pgfqpoint{3.395301in}{2.851838in}}%
\pgfpathlineto{\pgfqpoint{3.397360in}{2.900393in}}%
\pgfpathlineto{\pgfqpoint{3.401478in}{2.851838in}}%
\pgfpathlineto{\pgfqpoint{3.403537in}{2.880971in}}%
\pgfpathlineto{\pgfqpoint{3.409714in}{2.871260in}}%
\pgfpathlineto{\pgfqpoint{3.411773in}{2.880971in}}%
\pgfpathlineto{\pgfqpoint{3.415890in}{2.880971in}}%
\pgfpathlineto{\pgfqpoint{3.417949in}{2.900393in}}%
\pgfpathlineto{\pgfqpoint{3.424126in}{2.900393in}}%
\pgfpathlineto{\pgfqpoint{3.426185in}{2.890682in}}%
\pgfpathlineto{\pgfqpoint{3.428244in}{2.890682in}}%
\pgfpathlineto{\pgfqpoint{3.432362in}{2.939237in}}%
\pgfpathlineto{\pgfqpoint{3.438539in}{2.939237in}}%
\pgfpathlineto{\pgfqpoint{3.440598in}{2.968370in}}%
\pgfpathlineto{\pgfqpoint{3.444716in}{2.919815in}}%
\pgfpathlineto{\pgfqpoint{3.446775in}{2.910104in}}%
\pgfpathlineto{\pgfqpoint{3.452951in}{2.958659in}}%
\pgfpathlineto{\pgfqpoint{3.455010in}{2.890682in}}%
\pgfpathlineto{\pgfqpoint{3.457069in}{2.910104in}}%
\pgfpathlineto{\pgfqpoint{3.459128in}{2.900393in}}%
\pgfpathlineto{\pgfqpoint{3.467364in}{2.880971in}}%
\pgfpathlineto{\pgfqpoint{3.469423in}{2.910104in}}%
\pgfpathlineto{\pgfqpoint{3.471482in}{2.910104in}}%
\pgfpathlineto{\pgfqpoint{3.473541in}{2.929526in}}%
\pgfpathlineto{\pgfqpoint{3.475600in}{2.900393in}}%
\pgfpathlineto{\pgfqpoint{3.481777in}{2.919815in}}%
\pgfpathlineto{\pgfqpoint{3.483836in}{2.948948in}}%
\pgfpathlineto{\pgfqpoint{3.485895in}{2.948948in}}%
\pgfpathlineto{\pgfqpoint{3.490012in}{2.997503in}}%
\pgfpathlineto{\pgfqpoint{3.496189in}{3.016925in}}%
\pgfpathlineto{\pgfqpoint{3.504425in}{3.084902in}}%
\pgfpathlineto{\pgfqpoint{3.510602in}{3.114035in}}%
\pgfpathlineto{\pgfqpoint{3.512661in}{3.104324in}}%
\pgfpathlineto{\pgfqpoint{3.514720in}{3.114035in}}%
\pgfpathlineto{\pgfqpoint{3.525015in}{3.114035in}}%
\pgfpathlineto{\pgfqpoint{3.527073in}{3.123746in}}%
\pgfpathlineto{\pgfqpoint{3.529132in}{3.114035in}}%
\pgfpathlineto{\pgfqpoint{3.531191in}{3.114035in}}%
\pgfpathlineto{\pgfqpoint{3.533250in}{3.133457in}}%
\pgfpathlineto{\pgfqpoint{3.539427in}{3.114035in}}%
\pgfpathlineto{\pgfqpoint{3.545604in}{3.162590in}}%
\pgfpathlineto{\pgfqpoint{3.547663in}{3.162590in}}%
\pgfpathlineto{\pgfqpoint{3.553840in}{3.172301in}}%
\pgfpathlineto{\pgfqpoint{3.555899in}{3.201434in}}%
\pgfpathlineto{\pgfqpoint{3.557958in}{3.201434in}}%
\pgfpathlineto{\pgfqpoint{3.560017in}{3.191723in}}%
\pgfpathlineto{\pgfqpoint{3.562076in}{3.172301in}}%
\pgfpathlineto{\pgfqpoint{3.570311in}{3.211145in}}%
\pgfpathlineto{\pgfqpoint{3.574429in}{3.123746in}}%
\pgfpathlineto{\pgfqpoint{3.576488in}{3.104324in}}%
\pgfpathlineto{\pgfqpoint{3.584724in}{2.948948in}}%
\pgfpathlineto{\pgfqpoint{3.586783in}{3.046058in}}%
\pgfpathlineto{\pgfqpoint{3.588842in}{3.026636in}}%
\pgfpathlineto{\pgfqpoint{3.590901in}{3.094613in}}%
\pgfpathlineto{\pgfqpoint{3.597078in}{3.143168in}}%
\pgfpathlineto{\pgfqpoint{3.599137in}{3.114035in}}%
\pgfpathlineto{\pgfqpoint{3.601196in}{3.143168in}}%
\pgfpathlineto{\pgfqpoint{3.603254in}{3.123746in}}%
\pgfpathlineto{\pgfqpoint{3.605313in}{3.123746in}}%
\pgfpathlineto{\pgfqpoint{3.611490in}{3.143168in}}%
\pgfpathlineto{\pgfqpoint{3.613549in}{3.162590in}}%
\pgfpathlineto{\pgfqpoint{3.615608in}{3.211145in}}%
\pgfpathlineto{\pgfqpoint{3.617667in}{3.211145in}}%
\pgfpathlineto{\pgfqpoint{3.619726in}{3.172301in}}%
\pgfpathlineto{\pgfqpoint{3.625903in}{3.182012in}}%
\pgfpathlineto{\pgfqpoint{3.627962in}{3.162590in}}%
\pgfpathlineto{\pgfqpoint{3.630021in}{3.182012in}}%
\pgfpathlineto{\pgfqpoint{3.634139in}{3.182012in}}%
\pgfpathlineto{\pgfqpoint{3.642374in}{3.152879in}}%
\pgfpathlineto{\pgfqpoint{3.644433in}{3.143168in}}%
\pgfpathlineto{\pgfqpoint{3.648551in}{3.143168in}}%
\pgfpathlineto{\pgfqpoint{3.654728in}{3.191723in}}%
\pgfpathlineto{\pgfqpoint{3.656787in}{3.152879in}}%
\pgfpathlineto{\pgfqpoint{3.660905in}{3.172301in}}%
\pgfpathlineto{\pgfqpoint{3.662964in}{3.152879in}}%
\pgfpathlineto{\pgfqpoint{3.671200in}{3.211145in}}%
\pgfpathlineto{\pgfqpoint{3.673259in}{3.201434in}}%
\pgfpathlineto{\pgfqpoint{3.675318in}{3.220855in}}%
\pgfpathlineto{\pgfqpoint{3.677377in}{3.211145in}}%
\pgfpathlineto{\pgfqpoint{3.683553in}{3.211145in}}%
\pgfpathlineto{\pgfqpoint{3.685612in}{3.240277in}}%
\pgfpathlineto{\pgfqpoint{3.687671in}{3.220855in}}%
\pgfpathlineto{\pgfqpoint{3.691789in}{3.220855in}}%
\pgfpathlineto{\pgfqpoint{3.697966in}{3.259699in}}%
\pgfpathlineto{\pgfqpoint{3.700025in}{3.249988in}}%
\pgfpathlineto{\pgfqpoint{3.704143in}{3.308254in}}%
\pgfpathlineto{\pgfqpoint{3.706202in}{3.288832in}}%
\pgfpathlineto{\pgfqpoint{3.712379in}{3.279121in}}%
\pgfpathlineto{\pgfqpoint{3.714438in}{3.288832in}}%
\pgfpathlineto{\pgfqpoint{3.716496in}{3.288832in}}%
\pgfpathlineto{\pgfqpoint{3.718555in}{3.279121in}}%
\pgfpathlineto{\pgfqpoint{3.720614in}{3.249988in}}%
\pgfpathlineto{\pgfqpoint{3.726791in}{3.259699in}}%
\pgfpathlineto{\pgfqpoint{3.728850in}{3.298543in}}%
\pgfpathlineto{\pgfqpoint{3.730909in}{3.298543in}}%
\pgfpathlineto{\pgfqpoint{3.735027in}{3.230566in}}%
\pgfpathlineto{\pgfqpoint{3.741204in}{3.230566in}}%
\pgfpathlineto{\pgfqpoint{3.743263in}{3.249988in}}%
\pgfpathlineto{\pgfqpoint{3.745322in}{3.230566in}}%
\pgfpathlineto{\pgfqpoint{3.747381in}{3.249988in}}%
\pgfpathlineto{\pgfqpoint{3.749440in}{3.230566in}}%
\pgfpathlineto{\pgfqpoint{3.755616in}{3.220855in}}%
\pgfpathlineto{\pgfqpoint{3.757675in}{3.230566in}}%
\pgfpathlineto{\pgfqpoint{3.759734in}{3.220855in}}%
\pgfpathlineto{\pgfqpoint{3.761793in}{3.230566in}}%
\pgfpathlineto{\pgfqpoint{3.763852in}{3.249988in}}%
\pgfpathlineto{\pgfqpoint{3.770029in}{3.288832in}}%
\pgfpathlineto{\pgfqpoint{3.774147in}{3.288832in}}%
\pgfpathlineto{\pgfqpoint{3.778265in}{3.240277in}}%
\pgfpathlineto{\pgfqpoint{3.786501in}{3.279121in}}%
\pgfpathlineto{\pgfqpoint{3.788560in}{3.279121in}}%
\pgfpathlineto{\pgfqpoint{3.790619in}{3.259699in}}%
\pgfpathlineto{\pgfqpoint{3.792677in}{3.327676in}}%
\pgfpathlineto{\pgfqpoint{3.798854in}{3.347098in}}%
\pgfpathlineto{\pgfqpoint{3.800913in}{3.376231in}}%
\pgfpathlineto{\pgfqpoint{3.802972in}{3.356809in}}%
\pgfpathlineto{\pgfqpoint{3.807090in}{3.395653in}}%
\pgfpathlineto{\pgfqpoint{3.813267in}{3.395653in}}%
\pgfpathlineto{\pgfqpoint{3.815326in}{3.424786in}}%
\pgfpathlineto{\pgfqpoint{3.821503in}{3.424786in}}%
\pgfpathlineto{\pgfqpoint{3.827680in}{3.444208in}}%
\pgfpathlineto{\pgfqpoint{3.833856in}{3.444208in}}%
\pgfpathlineto{\pgfqpoint{3.835915in}{3.424786in}}%
\pgfpathlineto{\pgfqpoint{3.842092in}{3.434497in}}%
\pgfpathlineto{\pgfqpoint{3.844151in}{3.434497in}}%
\pgfpathlineto{\pgfqpoint{3.848269in}{3.483052in}}%
\pgfpathlineto{\pgfqpoint{3.850328in}{3.492763in}}%
\pgfpathlineto{\pgfqpoint{3.860623in}{3.492763in}}%
\pgfpathlineto{\pgfqpoint{3.862682in}{3.463630in}}%
\pgfpathlineto{\pgfqpoint{3.870917in}{3.463630in}}%
\pgfpathlineto{\pgfqpoint{3.875035in}{3.502474in}}%
\pgfpathlineto{\pgfqpoint{3.877094in}{3.483052in}}%
\pgfpathlineto{\pgfqpoint{3.879153in}{3.531607in}}%
\pgfpathlineto{\pgfqpoint{3.885330in}{3.531607in}}%
\pgfpathlineto{\pgfqpoint{3.887389in}{3.502474in}}%
\pgfpathlineto{\pgfqpoint{3.889448in}{3.453919in}}%
\pgfpathlineto{\pgfqpoint{3.891507in}{3.473341in}}%
\pgfpathlineto{\pgfqpoint{3.893566in}{3.424786in}}%
\pgfpathlineto{\pgfqpoint{3.899743in}{3.424786in}}%
\pgfpathlineto{\pgfqpoint{3.903861in}{3.483052in}}%
\pgfpathlineto{\pgfqpoint{3.905919in}{3.453919in}}%
\pgfpathlineto{\pgfqpoint{3.907978in}{3.521896in}}%
\pgfpathlineto{\pgfqpoint{3.914155in}{3.521896in}}%
\pgfpathlineto{\pgfqpoint{3.920332in}{3.589873in}}%
\pgfpathlineto{\pgfqpoint{3.922391in}{3.551029in}}%
\pgfpathlineto{\pgfqpoint{3.930627in}{3.502474in}}%
\pgfpathlineto{\pgfqpoint{3.932686in}{3.473341in}}%
\pgfpathlineto{\pgfqpoint{3.934745in}{3.473341in}}%
\pgfpathlineto{\pgfqpoint{3.936804in}{3.424786in}}%
\pgfpathlineto{\pgfqpoint{3.942980in}{3.405364in}}%
\pgfpathlineto{\pgfqpoint{3.945039in}{3.405364in}}%
\pgfpathlineto{\pgfqpoint{3.947098in}{3.424786in}}%
\pgfpathlineto{\pgfqpoint{3.951216in}{3.424786in}}%
\pgfpathlineto{\pgfqpoint{3.957393in}{3.453919in}}%
\pgfpathlineto{\pgfqpoint{3.959452in}{3.444208in}}%
\pgfpathlineto{\pgfqpoint{3.961511in}{3.424786in}}%
\pgfpathlineto{\pgfqpoint{3.963570in}{3.424786in}}%
\pgfpathlineto{\pgfqpoint{3.965629in}{3.415075in}}%
\pgfpathlineto{\pgfqpoint{3.971806in}{3.444208in}}%
\pgfpathlineto{\pgfqpoint{3.980042in}{3.337387in}}%
\pgfpathlineto{\pgfqpoint{3.986218in}{3.337387in}}%
\pgfpathlineto{\pgfqpoint{3.988277in}{3.395653in}}%
\pgfpathlineto{\pgfqpoint{3.990336in}{3.385942in}}%
\pgfpathlineto{\pgfqpoint{3.994454in}{3.347098in}}%
\pgfpathlineto{\pgfqpoint{4.000631in}{3.317965in}}%
\pgfpathlineto{\pgfqpoint{4.002690in}{3.269410in}}%
\pgfpathlineto{\pgfqpoint{4.004749in}{3.249988in}}%
\pgfpathlineto{\pgfqpoint{4.006808in}{3.288832in}}%
\pgfpathlineto{\pgfqpoint{4.008867in}{3.249988in}}%
\pgfpathlineto{\pgfqpoint{4.015044in}{3.172301in}}%
\pgfpathlineto{\pgfqpoint{4.019161in}{3.230566in}}%
\pgfpathlineto{\pgfqpoint{4.023279in}{3.143168in}}%
\pgfpathlineto{\pgfqpoint{4.029456in}{3.104324in}}%
\pgfpathlineto{\pgfqpoint{4.033574in}{3.123746in}}%
\pgfpathlineto{\pgfqpoint{4.035633in}{3.016925in}}%
\pgfpathlineto{\pgfqpoint{4.037692in}{3.123746in}}%
\pgfpathlineto{\pgfqpoint{4.043869in}{3.152879in}}%
\pgfpathlineto{\pgfqpoint{4.045928in}{3.201434in}}%
\pgfpathlineto{\pgfqpoint{4.047987in}{3.182012in}}%
\pgfpathlineto{\pgfqpoint{4.050046in}{3.182012in}}%
\pgfpathlineto{\pgfqpoint{4.052105in}{3.172301in}}%
\pgfpathlineto{\pgfqpoint{4.058281in}{3.152879in}}%
\pgfpathlineto{\pgfqpoint{4.060340in}{3.152879in}}%
\pgfpathlineto{\pgfqpoint{4.064458in}{3.182012in}}%
\pgfpathlineto{\pgfqpoint{4.066517in}{3.240277in}}%
\pgfpathlineto{\pgfqpoint{4.074753in}{3.201434in}}%
\pgfpathlineto{\pgfqpoint{4.076812in}{3.201434in}}%
\pgfpathlineto{\pgfqpoint{4.078871in}{3.182012in}}%
\pgfpathlineto{\pgfqpoint{4.080930in}{3.220855in}}%
\pgfpathlineto{\pgfqpoint{4.087107in}{3.220855in}}%
\pgfpathlineto{\pgfqpoint{4.091225in}{3.143168in}}%
\pgfpathlineto{\pgfqpoint{4.093284in}{3.075191in}}%
\pgfpathlineto{\pgfqpoint{4.095342in}{3.143168in}}%
\pgfpathlineto{\pgfqpoint{4.101519in}{3.152879in}}%
\pgfpathlineto{\pgfqpoint{4.103578in}{3.152879in}}%
\pgfpathlineto{\pgfqpoint{4.105637in}{3.143168in}}%
\pgfpathlineto{\pgfqpoint{4.109755in}{3.075191in}}%
\pgfpathlineto{\pgfqpoint{4.115932in}{3.104324in}}%
\pgfpathlineto{\pgfqpoint{4.120050in}{3.152879in}}%
\pgfpathlineto{\pgfqpoint{4.122109in}{3.123746in}}%
\pgfpathlineto{\pgfqpoint{4.124168in}{3.143168in}}%
\pgfpathlineto{\pgfqpoint{4.132403in}{3.123746in}}%
\pgfpathlineto{\pgfqpoint{4.134462in}{3.123746in}}%
\pgfpathlineto{\pgfqpoint{4.136521in}{3.152879in}}%
\pgfpathlineto{\pgfqpoint{4.138580in}{3.104324in}}%
\pgfpathlineto{\pgfqpoint{4.144757in}{3.133457in}}%
\pgfpathlineto{\pgfqpoint{4.146816in}{3.104324in}}%
\pgfpathlineto{\pgfqpoint{4.152993in}{3.172301in}}%
\pgfpathlineto{\pgfqpoint{4.161229in}{3.172301in}}%
\pgfpathlineto{\pgfqpoint{4.163288in}{3.143168in}}%
\pgfpathlineto{\pgfqpoint{4.165347in}{3.094613in}}%
\pgfpathlineto{\pgfqpoint{4.167406in}{3.075191in}}%
\pgfpathlineto{\pgfqpoint{4.173582in}{3.094613in}}%
\pgfpathlineto{\pgfqpoint{4.175641in}{3.075191in}}%
\pgfpathlineto{\pgfqpoint{4.177700in}{3.075191in}}%
\pgfpathlineto{\pgfqpoint{4.179759in}{3.084902in}}%
\pgfpathlineto{\pgfqpoint{4.181818in}{3.055769in}}%
\pgfpathlineto{\pgfqpoint{4.190054in}{3.084902in}}%
\pgfpathlineto{\pgfqpoint{4.192113in}{3.026636in}}%
\pgfpathlineto{\pgfqpoint{4.194172in}{3.036347in}}%
\pgfpathlineto{\pgfqpoint{4.196231in}{2.939237in}}%
\pgfpathlineto{\pgfqpoint{4.206526in}{2.851838in}}%
\pgfpathlineto{\pgfqpoint{4.208584in}{2.861549in}}%
\pgfpathlineto{\pgfqpoint{4.210643in}{2.900393in}}%
\pgfpathlineto{\pgfqpoint{4.216820in}{2.958659in}}%
\pgfpathlineto{\pgfqpoint{4.218879in}{2.929526in}}%
\pgfpathlineto{\pgfqpoint{4.220938in}{2.958659in}}%
\pgfpathlineto{\pgfqpoint{4.222997in}{2.958659in}}%
\pgfpathlineto{\pgfqpoint{4.225056in}{2.978081in}}%
\pgfpathlineto{\pgfqpoint{4.231233in}{2.987792in}}%
\pgfpathlineto{\pgfqpoint{4.233292in}{2.978081in}}%
\pgfpathlineto{\pgfqpoint{4.235351in}{2.939237in}}%
\pgfpathlineto{\pgfqpoint{4.239469in}{3.026636in}}%
\pgfpathlineto{\pgfqpoint{4.245645in}{3.026636in}}%
\pgfpathlineto{\pgfqpoint{4.247704in}{3.036347in}}%
\pgfpathlineto{\pgfqpoint{4.249763in}{3.026636in}}%
\pgfpathlineto{\pgfqpoint{4.251822in}{3.007214in}}%
\pgfpathlineto{\pgfqpoint{4.260058in}{3.007214in}}%
\pgfpathlineto{\pgfqpoint{4.262117in}{2.987792in}}%
\pgfpathlineto{\pgfqpoint{4.264176in}{2.948948in}}%
\pgfpathlineto{\pgfqpoint{4.266235in}{2.958659in}}%
\pgfpathlineto{\pgfqpoint{4.268294in}{2.910104in}}%
\pgfpathlineto{\pgfqpoint{4.274471in}{2.929526in}}%
\pgfpathlineto{\pgfqpoint{4.276530in}{2.900393in}}%
\pgfpathlineto{\pgfqpoint{4.280648in}{2.978081in}}%
\pgfpathlineto{\pgfqpoint{4.282707in}{2.958659in}}%
\pgfpathlineto{\pgfqpoint{4.288883in}{2.939237in}}%
\pgfpathlineto{\pgfqpoint{4.290942in}{2.910104in}}%
\pgfpathlineto{\pgfqpoint{4.293001in}{2.929526in}}%
\pgfpathlineto{\pgfqpoint{4.295060in}{2.890682in}}%
\pgfpathlineto{\pgfqpoint{4.297119in}{2.890682in}}%
\pgfpathlineto{\pgfqpoint{4.303296in}{2.812994in}}%
\pgfpathlineto{\pgfqpoint{4.305355in}{2.832416in}}%
\pgfpathlineto{\pgfqpoint{4.307414in}{2.793572in}}%
\pgfpathlineto{\pgfqpoint{4.309473in}{2.832416in}}%
\pgfpathlineto{\pgfqpoint{4.311532in}{2.832416in}}%
\pgfpathlineto{\pgfqpoint{4.317709in}{2.842127in}}%
\pgfpathlineto{\pgfqpoint{4.319768in}{2.890682in}}%
\pgfpathlineto{\pgfqpoint{4.321826in}{2.861549in}}%
\pgfpathlineto{\pgfqpoint{4.323885in}{2.754728in}}%
\pgfpathlineto{\pgfqpoint{4.325944in}{2.793572in}}%
\pgfpathlineto{\pgfqpoint{4.334180in}{2.754728in}}%
\pgfpathlineto{\pgfqpoint{4.338298in}{2.696462in}}%
\pgfpathlineto{\pgfqpoint{4.340357in}{2.589642in}}%
\pgfpathlineto{\pgfqpoint{4.346534in}{2.463399in}}%
\pgfpathlineto{\pgfqpoint{4.348593in}{2.521665in}}%
\pgfpathlineto{\pgfqpoint{4.350652in}{2.473110in}}%
\pgfpathlineto{\pgfqpoint{4.352711in}{2.521665in}}%
\pgfpathlineto{\pgfqpoint{4.354770in}{2.492532in}}%
\pgfpathlineto{\pgfqpoint{4.360946in}{2.541087in}}%
\pgfpathlineto{\pgfqpoint{4.363005in}{2.570220in}}%
\pgfpathlineto{\pgfqpoint{4.367123in}{2.473110in}}%
\pgfpathlineto{\pgfqpoint{4.369182in}{2.482821in}}%
\pgfpathlineto{\pgfqpoint{4.375359in}{2.502243in}}%
\pgfpathlineto{\pgfqpoint{4.377418in}{2.502243in}}%
\pgfpathlineto{\pgfqpoint{4.379477in}{2.385711in}}%
\pgfpathlineto{\pgfqpoint{4.381536in}{2.366289in}}%
\pgfpathlineto{\pgfqpoint{4.383595in}{2.414844in}}%
\pgfpathlineto{\pgfqpoint{4.391831in}{2.356578in}}%
\pgfpathlineto{\pgfqpoint{4.393890in}{2.414844in}}%
\pgfpathlineto{\pgfqpoint{4.395949in}{2.385711in}}%
\pgfpathlineto{\pgfqpoint{4.404184in}{2.424555in}}%
\pgfpathlineto{\pgfqpoint{4.406243in}{2.414844in}}%
\pgfpathlineto{\pgfqpoint{4.408302in}{2.414844in}}%
\pgfpathlineto{\pgfqpoint{4.412420in}{2.511954in}}%
\pgfpathlineto{\pgfqpoint{4.418597in}{2.521665in}}%
\pgfpathlineto{\pgfqpoint{4.420656in}{2.560509in}}%
\pgfpathlineto{\pgfqpoint{4.422715in}{2.463399in}}%
\pgfpathlineto{\pgfqpoint{4.424774in}{2.492532in}}%
\pgfpathlineto{\pgfqpoint{4.426833in}{2.482821in}}%
\pgfpathlineto{\pgfqpoint{4.433010in}{2.473110in}}%
\pgfpathlineto{\pgfqpoint{4.435068in}{2.511954in}}%
\pgfpathlineto{\pgfqpoint{4.439186in}{2.414844in}}%
\pgfpathlineto{\pgfqpoint{4.441245in}{2.443977in}}%
\pgfpathlineto{\pgfqpoint{4.447422in}{2.443977in}}%
\pgfpathlineto{\pgfqpoint{4.449481in}{2.473110in}}%
\pgfpathlineto{\pgfqpoint{4.451540in}{2.473110in}}%
\pgfpathlineto{\pgfqpoint{4.453599in}{2.502243in}}%
\pgfpathlineto{\pgfqpoint{4.455658in}{2.502243in}}%
\pgfpathlineto{\pgfqpoint{4.461835in}{2.492532in}}%
\pgfpathlineto{\pgfqpoint{4.463894in}{2.492532in}}%
\pgfpathlineto{\pgfqpoint{4.465953in}{2.531376in}}%
\pgfpathlineto{\pgfqpoint{4.468012in}{2.376000in}}%
\pgfpathlineto{\pgfqpoint{4.470071in}{2.366289in}}%
\pgfpathlineto{\pgfqpoint{4.476247in}{2.240046in}}%
\pgfpathlineto{\pgfqpoint{4.478306in}{2.249757in}}%
\pgfpathlineto{\pgfqpoint{4.480365in}{2.240046in}}%
\pgfpathlineto{\pgfqpoint{4.482424in}{2.269179in}}%
\pgfpathlineto{\pgfqpoint{4.484483in}{2.278890in}}%
\pgfpathlineto{\pgfqpoint{4.490660in}{2.230335in}}%
\pgfpathlineto{\pgfqpoint{4.492719in}{2.308023in}}%
\pgfpathlineto{\pgfqpoint{4.496837in}{2.133225in}}%
\pgfpathlineto{\pgfqpoint{4.498896in}{2.133225in}}%
\pgfpathlineto{\pgfqpoint{4.505073in}{2.181780in}}%
\pgfpathlineto{\pgfqpoint{4.507132in}{2.152647in}}%
\pgfpathlineto{\pgfqpoint{4.511249in}{2.259468in}}%
\pgfpathlineto{\pgfqpoint{4.513308in}{2.162358in}}%
\pgfpathlineto{\pgfqpoint{4.519485in}{2.191491in}}%
\pgfpathlineto{\pgfqpoint{4.521544in}{2.181780in}}%
\pgfpathlineto{\pgfqpoint{4.523603in}{2.152647in}}%
\pgfpathlineto{\pgfqpoint{4.525662in}{2.181780in}}%
\pgfpathlineto{\pgfqpoint{4.527721in}{2.152647in}}%
\pgfpathlineto{\pgfqpoint{4.535957in}{2.123514in}}%
\pgfpathlineto{\pgfqpoint{4.538016in}{2.084671in}}%
\pgfpathlineto{\pgfqpoint{4.540075in}{2.201202in}}%
\pgfpathlineto{\pgfqpoint{4.542134in}{2.181780in}}%
\pgfpathlineto{\pgfqpoint{4.548311in}{2.230335in}}%
\pgfpathlineto{\pgfqpoint{4.550369in}{2.317734in}}%
\pgfpathlineto{\pgfqpoint{4.552428in}{2.327445in}}%
\pgfpathlineto{\pgfqpoint{4.554487in}{2.366289in}}%
\pgfpathlineto{\pgfqpoint{4.556546in}{2.434266in}}%
\pgfpathlineto{\pgfqpoint{4.564782in}{2.366289in}}%
\pgfpathlineto{\pgfqpoint{4.566841in}{2.414844in}}%
\pgfpathlineto{\pgfqpoint{4.568900in}{2.385711in}}%
\pgfpathlineto{\pgfqpoint{4.570959in}{2.337156in}}%
\pgfpathlineto{\pgfqpoint{4.577136in}{2.327445in}}%
\pgfpathlineto{\pgfqpoint{4.579195in}{2.249757in}}%
\pgfpathlineto{\pgfqpoint{4.581254in}{2.327445in}}%
\pgfpathlineto{\pgfqpoint{4.585372in}{2.278890in}}%
\pgfpathlineto{\pgfqpoint{4.591548in}{2.278890in}}%
\pgfpathlineto{\pgfqpoint{4.597725in}{2.045827in}}%
\pgfpathlineto{\pgfqpoint{4.599784in}{2.055538in}}%
\pgfpathlineto{\pgfqpoint{4.605961in}{2.113803in}}%
\pgfpathlineto{\pgfqpoint{4.608020in}{2.074960in}}%
\pgfpathlineto{\pgfqpoint{4.612138in}{2.181780in}}%
\pgfpathlineto{\pgfqpoint{4.614197in}{2.278890in}}%
\pgfpathlineto{\pgfqpoint{4.622433in}{2.259468in}}%
\pgfpathlineto{\pgfqpoint{4.624491in}{2.230335in}}%
\pgfpathlineto{\pgfqpoint{4.626550in}{2.249757in}}%
\pgfpathlineto{\pgfqpoint{4.628609in}{2.230335in}}%
\pgfpathlineto{\pgfqpoint{4.634786in}{2.269179in}}%
\pgfpathlineto{\pgfqpoint{4.638904in}{2.230335in}}%
\pgfpathlineto{\pgfqpoint{4.640963in}{2.230335in}}%
\pgfpathlineto{\pgfqpoint{4.643022in}{2.278890in}}%
\pgfpathlineto{\pgfqpoint{4.649199in}{2.288601in}}%
\pgfpathlineto{\pgfqpoint{4.651258in}{2.288601in}}%
\pgfpathlineto{\pgfqpoint{4.653317in}{2.259468in}}%
\pgfpathlineto{\pgfqpoint{4.655376in}{2.172069in}}%
\pgfpathlineto{\pgfqpoint{4.657435in}{2.210913in}}%
\pgfpathlineto{\pgfqpoint{4.663611in}{2.249757in}}%
\pgfpathlineto{\pgfqpoint{4.665670in}{2.278890in}}%
\pgfpathlineto{\pgfqpoint{4.667729in}{2.259468in}}%
\pgfpathlineto{\pgfqpoint{4.669788in}{2.327445in}}%
\pgfpathlineto{\pgfqpoint{4.671847in}{2.327445in}}%
\pgfpathlineto{\pgfqpoint{4.680083in}{2.308023in}}%
\pgfpathlineto{\pgfqpoint{4.682142in}{2.278890in}}%
\pgfpathlineto{\pgfqpoint{4.684201in}{2.230335in}}%
\pgfpathlineto{\pgfqpoint{4.686260in}{2.259468in}}%
\pgfpathlineto{\pgfqpoint{4.692437in}{2.249757in}}%
\pgfpathlineto{\pgfqpoint{4.694496in}{2.249757in}}%
\pgfpathlineto{\pgfqpoint{4.696555in}{2.210913in}}%
\pgfpathlineto{\pgfqpoint{4.698614in}{2.249757in}}%
\pgfpathlineto{\pgfqpoint{4.700672in}{2.259468in}}%
\pgfpathlineto{\pgfqpoint{4.706849in}{2.259468in}}%
\pgfpathlineto{\pgfqpoint{4.708908in}{2.230335in}}%
\pgfpathlineto{\pgfqpoint{4.710967in}{2.278890in}}%
\pgfpathlineto{\pgfqpoint{4.715085in}{2.259468in}}%
\pgfpathlineto{\pgfqpoint{4.721262in}{2.259468in}}%
\pgfpathlineto{\pgfqpoint{4.723321in}{2.181780in}}%
\pgfpathlineto{\pgfqpoint{4.725380in}{2.230335in}}%
\pgfpathlineto{\pgfqpoint{4.727439in}{2.230335in}}%
\pgfpathlineto{\pgfqpoint{4.729498in}{2.259468in}}%
\pgfpathlineto{\pgfqpoint{4.735675in}{2.278890in}}%
\pgfpathlineto{\pgfqpoint{4.737733in}{2.298312in}}%
\pgfpathlineto{\pgfqpoint{4.739792in}{2.259468in}}%
\pgfpathlineto{\pgfqpoint{4.741851in}{2.308023in}}%
\pgfpathlineto{\pgfqpoint{4.743910in}{2.259468in}}%
\pgfpathlineto{\pgfqpoint{4.750087in}{2.298312in}}%
\pgfpathlineto{\pgfqpoint{4.752146in}{2.278890in}}%
\pgfpathlineto{\pgfqpoint{4.754205in}{2.278890in}}%
\pgfpathlineto{\pgfqpoint{4.756264in}{2.269179in}}%
\pgfpathlineto{\pgfqpoint{4.758323in}{2.278890in}}%
\pgfpathlineto{\pgfqpoint{4.764500in}{2.288601in}}%
\pgfpathlineto{\pgfqpoint{4.766559in}{2.269179in}}%
\pgfpathlineto{\pgfqpoint{4.770677in}{2.288601in}}%
\pgfpathlineto{\pgfqpoint{4.772736in}{2.240046in}}%
\pgfpathlineto{\pgfqpoint{4.778912in}{2.230335in}}%
\pgfpathlineto{\pgfqpoint{4.785089in}{2.230335in}}%
\pgfpathlineto{\pgfqpoint{4.787148in}{2.181780in}}%
\pgfpathlineto{\pgfqpoint{4.793325in}{2.191491in}}%
\pgfpathlineto{\pgfqpoint{4.795384in}{2.191491in}}%
\pgfpathlineto{\pgfqpoint{4.797443in}{2.230335in}}%
\pgfpathlineto{\pgfqpoint{4.799502in}{2.230335in}}%
\pgfpathlineto{\pgfqpoint{4.801561in}{2.210913in}}%
\pgfpathlineto{\pgfqpoint{4.807738in}{2.230335in}}%
\pgfpathlineto{\pgfqpoint{4.809797in}{2.230335in}}%
\pgfpathlineto{\pgfqpoint{4.811856in}{2.210913in}}%
\pgfpathlineto{\pgfqpoint{4.813914in}{2.230335in}}%
\pgfpathlineto{\pgfqpoint{4.815973in}{2.230335in}}%
\pgfpathlineto{\pgfqpoint{4.824209in}{2.181780in}}%
\pgfpathlineto{\pgfqpoint{4.826268in}{2.181780in}}%
\pgfpathlineto{\pgfqpoint{4.836563in}{2.094382in}}%
\pgfpathlineto{\pgfqpoint{4.838622in}{2.104092in}}%
\pgfpathlineto{\pgfqpoint{4.840681in}{2.074960in}}%
\pgfpathlineto{\pgfqpoint{4.842740in}{2.065249in}}%
\pgfpathlineto{\pgfqpoint{4.844799in}{1.987561in}}%
\pgfpathlineto{\pgfqpoint{4.850976in}{2.016694in}}%
\pgfpathlineto{\pgfqpoint{4.855093in}{2.094382in}}%
\pgfpathlineto{\pgfqpoint{4.857152in}{2.094382in}}%
\pgfpathlineto{\pgfqpoint{4.859211in}{2.065249in}}%
\pgfpathlineto{\pgfqpoint{4.865388in}{2.026405in}}%
\pgfpathlineto{\pgfqpoint{4.869506in}{2.094382in}}%
\pgfpathlineto{\pgfqpoint{4.871565in}{2.094382in}}%
\pgfpathlineto{\pgfqpoint{4.873624in}{2.074960in}}%
\pgfpathlineto{\pgfqpoint{4.881860in}{2.065249in}}%
\pgfpathlineto{\pgfqpoint{4.883919in}{2.074960in}}%
\pgfpathlineto{\pgfqpoint{4.885978in}{2.045827in}}%
\pgfpathlineto{\pgfqpoint{4.888037in}{1.997272in}}%
\pgfpathlineto{\pgfqpoint{4.894213in}{1.919584in}}%
\pgfpathlineto{\pgfqpoint{4.900390in}{1.773919in}}%
\pgfpathlineto{\pgfqpoint{4.902449in}{1.531145in}}%
\pgfpathlineto{\pgfqpoint{4.908626in}{1.511723in}}%
\pgfpathlineto{\pgfqpoint{4.910685in}{1.385480in}}%
\pgfpathlineto{\pgfqpoint{4.912744in}{1.346636in}}%
\pgfpathlineto{\pgfqpoint{4.916862in}{1.171838in}}%
\pgfpathlineto{\pgfqpoint{4.923039in}{1.065017in}}%
\pgfpathlineto{\pgfqpoint{4.925098in}{1.181549in}}%
\pgfpathlineto{\pgfqpoint{4.929215in}{1.181549in}}%
\pgfpathlineto{\pgfqpoint{4.931274in}{1.171838in}}%
\pgfpathlineto{\pgfqpoint{4.937451in}{1.045595in}}%
\pgfpathlineto{\pgfqpoint{4.941569in}{1.220393in}}%
\pgfpathlineto{\pgfqpoint{4.945687in}{1.055306in}}%
\pgfpathlineto{\pgfqpoint{4.951864in}{0.967908in}}%
\pgfpathlineto{\pgfqpoint{4.953923in}{1.065017in}}%
\pgfpathlineto{\pgfqpoint{4.960100in}{0.938775in}}%
\pgfpathlineto{\pgfqpoint{4.966276in}{0.919353in}}%
\pgfpathlineto{\pgfqpoint{4.974512in}{0.919353in}}%
\pgfpathlineto{\pgfqpoint{4.982748in}{0.967908in}}%
\pgfpathlineto{\pgfqpoint{4.984807in}{0.958197in}}%
\pgfpathlineto{\pgfqpoint{4.986866in}{0.919353in}}%
\pgfpathlineto{\pgfqpoint{4.995102in}{0.938775in}}%
\pgfpathlineto{\pgfqpoint{5.001279in}{0.890220in}}%
\pgfpathlineto{\pgfqpoint{5.011573in}{0.890220in}}%
\pgfpathlineto{\pgfqpoint{5.013632in}{0.909642in}}%
\pgfpathlineto{\pgfqpoint{5.017750in}{0.909642in}}%
\pgfpathlineto{\pgfqpoint{5.023927in}{0.929064in}}%
\pgfpathlineto{\pgfqpoint{5.025986in}{0.890220in}}%
\pgfpathlineto{\pgfqpoint{5.032163in}{0.890220in}}%
\pgfpathlineto{\pgfqpoint{5.038340in}{0.880509in}}%
\pgfpathlineto{\pgfqpoint{5.040399in}{0.880509in}}%
\pgfpathlineto{\pgfqpoint{5.042457in}{0.861087in}}%
\pgfpathlineto{\pgfqpoint{5.044516in}{0.822243in}}%
\pgfpathlineto{\pgfqpoint{5.046575in}{0.851376in}}%
\pgfpathlineto{\pgfqpoint{5.052752in}{0.861087in}}%
\pgfpathlineto{\pgfqpoint{5.054811in}{0.861087in}}%
\pgfpathlineto{\pgfqpoint{5.056870in}{0.851376in}}%
\pgfpathlineto{\pgfqpoint{5.060988in}{0.851376in}}%
\pgfpathlineto{\pgfqpoint{5.067165in}{0.870798in}}%
\pgfpathlineto{\pgfqpoint{5.071283in}{0.851376in}}%
\pgfpathlineto{\pgfqpoint{5.073342in}{0.861087in}}%
\pgfpathlineto{\pgfqpoint{5.075401in}{0.861087in}}%
\pgfpathlineto{\pgfqpoint{5.083636in}{0.870798in}}%
\pgfpathlineto{\pgfqpoint{5.085695in}{0.880509in}}%
\pgfpathlineto{\pgfqpoint{5.089813in}{0.851376in}}%
\pgfpathlineto{\pgfqpoint{5.095990in}{0.831954in}}%
\pgfpathlineto{\pgfqpoint{5.100108in}{0.880509in}}%
\pgfpathlineto{\pgfqpoint{5.102167in}{0.880509in}}%
\pgfpathlineto{\pgfqpoint{5.104226in}{0.909642in}}%
\pgfpathlineto{\pgfqpoint{5.110403in}{0.909642in}}%
\pgfpathlineto{\pgfqpoint{5.114521in}{0.861087in}}%
\pgfpathlineto{\pgfqpoint{5.116579in}{0.880509in}}%
\pgfpathlineto{\pgfqpoint{5.124815in}{0.880509in}}%
\pgfpathlineto{\pgfqpoint{5.126874in}{0.899931in}}%
\pgfpathlineto{\pgfqpoint{5.128933in}{0.880509in}}%
\pgfpathlineto{\pgfqpoint{5.139228in}{0.880509in}}%
\pgfpathlineto{\pgfqpoint{5.141287in}{0.870798in}}%
\pgfpathlineto{\pgfqpoint{5.143346in}{0.880509in}}%
\pgfpathlineto{\pgfqpoint{5.145405in}{0.861087in}}%
\pgfpathlineto{\pgfqpoint{5.147464in}{0.861087in}}%
\pgfpathlineto{\pgfqpoint{5.153641in}{0.851376in}}%
\pgfpathlineto{\pgfqpoint{5.155699in}{0.851376in}}%
\pgfpathlineto{\pgfqpoint{5.157758in}{0.861087in}}%
\pgfpathlineto{\pgfqpoint{5.159817in}{0.851376in}}%
\pgfpathlineto{\pgfqpoint{5.182466in}{0.851376in}}%
\pgfpathlineto{\pgfqpoint{5.184525in}{0.831954in}}%
\pgfpathlineto{\pgfqpoint{5.186584in}{0.851376in}}%
\pgfpathlineto{\pgfqpoint{5.188643in}{0.851376in}}%
\pgfpathlineto{\pgfqpoint{5.190702in}{0.831954in}}%
\pgfpathlineto{\pgfqpoint{5.196878in}{0.851376in}}%
\pgfpathlineto{\pgfqpoint{5.198937in}{0.831954in}}%
\pgfpathlineto{\pgfqpoint{5.200996in}{0.831954in}}%
\pgfpathlineto{\pgfqpoint{5.203055in}{0.851376in}}%
\pgfpathlineto{\pgfqpoint{5.205114in}{0.831954in}}%
\pgfpathlineto{\pgfqpoint{5.211291in}{0.841665in}}%
\pgfpathlineto{\pgfqpoint{5.213350in}{0.831954in}}%
\pgfpathlineto{\pgfqpoint{5.217468in}{0.802821in}}%
\pgfpathlineto{\pgfqpoint{5.231880in}{0.802821in}}%
\pgfpathlineto{\pgfqpoint{5.233939in}{0.822243in}}%
\pgfpathlineto{\pgfqpoint{5.240116in}{0.831954in}}%
\pgfpathlineto{\pgfqpoint{5.242175in}{0.851376in}}%
\pgfpathlineto{\pgfqpoint{5.246293in}{0.851376in}}%
\pgfpathlineto{\pgfqpoint{5.248352in}{0.831954in}}%
\pgfpathlineto{\pgfqpoint{5.258647in}{0.831954in}}%
\pgfpathlineto{\pgfqpoint{5.260706in}{0.822243in}}%
\pgfpathlineto{\pgfqpoint{5.262765in}{0.851376in}}%
\pgfpathlineto{\pgfqpoint{5.268941in}{0.831954in}}%
\pgfpathlineto{\pgfqpoint{5.273059in}{0.851376in}}%
\pgfpathlineto{\pgfqpoint{5.275118in}{0.851376in}}%
\pgfpathlineto{\pgfqpoint{5.277177in}{0.831954in}}%
\pgfpathlineto{\pgfqpoint{5.283354in}{0.831954in}}%
\pgfpathlineto{\pgfqpoint{5.285413in}{0.822243in}}%
\pgfpathlineto{\pgfqpoint{5.287472in}{0.831954in}}%
\pgfpathlineto{\pgfqpoint{5.289531in}{0.822243in}}%
\pgfpathlineto{\pgfqpoint{5.291590in}{0.831954in}}%
\pgfpathlineto{\pgfqpoint{5.303944in}{0.831954in}}%
\pgfpathlineto{\pgfqpoint{5.306002in}{0.822243in}}%
\pgfpathlineto{\pgfqpoint{5.312179in}{0.831954in}}%
\pgfpathlineto{\pgfqpoint{5.316297in}{0.831954in}}%
\pgfpathlineto{\pgfqpoint{5.318356in}{0.822243in}}%
\pgfpathlineto{\pgfqpoint{5.320415in}{0.831954in}}%
\pgfpathlineto{\pgfqpoint{5.326592in}{0.831954in}}%
\pgfpathlineto{\pgfqpoint{5.328651in}{0.822243in}}%
\pgfpathlineto{\pgfqpoint{5.330710in}{0.831954in}}%
\pgfpathlineto{\pgfqpoint{5.332769in}{0.831954in}}%
\pgfpathlineto{\pgfqpoint{5.334828in}{0.812532in}}%
\pgfpathlineto{\pgfqpoint{5.341005in}{0.831954in}}%
\pgfpathlineto{\pgfqpoint{5.343064in}{0.802821in}}%
\pgfpathlineto{\pgfqpoint{5.347181in}{0.831954in}}%
\pgfpathlineto{\pgfqpoint{5.349240in}{0.822243in}}%
\pgfpathlineto{\pgfqpoint{5.355417in}{0.831954in}}%
\pgfpathlineto{\pgfqpoint{5.357476in}{0.831954in}}%
\pgfpathlineto{\pgfqpoint{5.359535in}{0.851376in}}%
\pgfpathlineto{\pgfqpoint{5.361594in}{0.822243in}}%
\pgfpathlineto{\pgfqpoint{5.363653in}{0.851376in}}%
\pgfpathlineto{\pgfqpoint{5.371889in}{0.851376in}}%
\pgfpathlineto{\pgfqpoint{5.373948in}{0.831954in}}%
\pgfpathlineto{\pgfqpoint{5.378066in}{0.831954in}}%
\pgfpathlineto{\pgfqpoint{5.384242in}{0.851376in}}%
\pgfpathlineto{\pgfqpoint{5.386301in}{0.831954in}}%
\pgfpathlineto{\pgfqpoint{5.388360in}{0.831954in}}%
\pgfpathlineto{\pgfqpoint{5.392478in}{0.870798in}}%
\pgfpathlineto{\pgfqpoint{5.398655in}{0.851376in}}%
\pgfpathlineto{\pgfqpoint{5.404832in}{0.851376in}}%
\pgfpathlineto{\pgfqpoint{5.406891in}{0.831954in}}%
\pgfpathlineto{\pgfqpoint{5.415127in}{0.861087in}}%
\pgfpathlineto{\pgfqpoint{5.417186in}{0.831954in}}%
\pgfpathlineto{\pgfqpoint{5.419244in}{0.831954in}}%
\pgfpathlineto{\pgfqpoint{5.421303in}{0.851376in}}%
\pgfpathlineto{\pgfqpoint{5.427480in}{0.861087in}}%
\pgfpathlineto{\pgfqpoint{5.429539in}{0.880509in}}%
\pgfpathlineto{\pgfqpoint{5.433657in}{0.861087in}}%
\pgfpathlineto{\pgfqpoint{5.435716in}{0.861087in}}%
\pgfpathlineto{\pgfqpoint{5.441893in}{0.880509in}}%
\pgfpathlineto{\pgfqpoint{5.443952in}{0.870798in}}%
\pgfpathlineto{\pgfqpoint{5.446011in}{0.851376in}}%
\pgfpathlineto{\pgfqpoint{5.448070in}{0.870798in}}%
\pgfpathlineto{\pgfqpoint{5.450129in}{0.851376in}}%
\pgfpathlineto{\pgfqpoint{5.458364in}{0.851376in}}%
\pgfpathlineto{\pgfqpoint{5.460423in}{0.861087in}}%
\pgfpathlineto{\pgfqpoint{5.464541in}{0.851376in}}%
\pgfpathlineto{\pgfqpoint{5.470718in}{0.851376in}}%
\pgfpathlineto{\pgfqpoint{5.472777in}{0.861087in}}%
\pgfpathlineto{\pgfqpoint{5.474836in}{0.851376in}}%
\pgfpathlineto{\pgfqpoint{5.478954in}{0.851376in}}%
\pgfpathlineto{\pgfqpoint{5.485131in}{0.831954in}}%
\pgfpathlineto{\pgfqpoint{5.487190in}{0.831954in}}%
\pgfpathlineto{\pgfqpoint{5.489249in}{0.851376in}}%
\pgfpathlineto{\pgfqpoint{5.493367in}{0.802821in}}%
\pgfpathlineto{\pgfqpoint{5.499543in}{0.822243in}}%
\pgfpathlineto{\pgfqpoint{5.501602in}{0.802821in}}%
\pgfpathlineto{\pgfqpoint{5.503661in}{0.822243in}}%
\pgfpathlineto{\pgfqpoint{5.528369in}{0.822243in}}%
\pgfpathlineto{\pgfqpoint{5.530428in}{0.812532in}}%
\pgfpathlineto{\pgfqpoint{5.532487in}{0.812532in}}%
\pgfpathlineto{\pgfqpoint{5.534545in}{0.822243in}}%
\pgfpathlineto{\pgfqpoint{5.534545in}{0.822243in}}%
\pgfusepath{stroke}%
\end{pgfscope}%
\begin{pgfscope}%
\pgfpathrectangle{\pgfqpoint{0.800000in}{0.528000in}}{\pgfqpoint{4.960000in}{3.696000in}}%
\pgfusepath{clip}%
\pgfsetrectcap%
\pgfsetroundjoin%
\pgfsetlinewidth{1.003750pt}%
\definecolor{currentstroke}{rgb}{0.501961,0.501961,0.501961}%
\pgfsetstrokecolor{currentstroke}%
\pgfsetstrokeopacity{0.900000}%
\pgfsetdash{}{0pt}%
\pgfpathmoveto{\pgfqpoint{1.025455in}{1.735075in}}%
\pgfpathlineto{\pgfqpoint{1.031631in}{1.725364in}}%
\pgfpathlineto{\pgfqpoint{1.035749in}{1.667098in}}%
\pgfpathlineto{\pgfqpoint{1.037808in}{1.667098in}}%
\pgfpathlineto{\pgfqpoint{1.039867in}{1.628254in}}%
\pgfpathlineto{\pgfqpoint{1.046044in}{1.579699in}}%
\pgfpathlineto{\pgfqpoint{1.048103in}{1.550566in}}%
\pgfpathlineto{\pgfqpoint{1.050162in}{1.502012in}}%
\pgfpathlineto{\pgfqpoint{1.052221in}{1.424324in}}%
\pgfpathlineto{\pgfqpoint{1.054280in}{1.472879in}}%
\pgfpathlineto{\pgfqpoint{1.062516in}{1.521434in}}%
\pgfpathlineto{\pgfqpoint{1.066633in}{1.569988in}}%
\pgfpathlineto{\pgfqpoint{1.068692in}{1.531145in}}%
\pgfpathlineto{\pgfqpoint{1.074869in}{1.560277in}}%
\pgfpathlineto{\pgfqpoint{1.076928in}{1.540855in}}%
\pgfpathlineto{\pgfqpoint{1.078987in}{1.482590in}}%
\pgfpathlineto{\pgfqpoint{1.081046in}{1.511723in}}%
\pgfpathlineto{\pgfqpoint{1.083105in}{1.443746in}}%
\pgfpathlineto{\pgfqpoint{1.089282in}{1.453457in}}%
\pgfpathlineto{\pgfqpoint{1.091341in}{1.521434in}}%
\pgfpathlineto{\pgfqpoint{1.095459in}{1.540855in}}%
\pgfpathlineto{\pgfqpoint{1.097518in}{1.696231in}}%
\pgfpathlineto{\pgfqpoint{1.103694in}{1.705942in}}%
\pgfpathlineto{\pgfqpoint{1.107812in}{1.725364in}}%
\pgfpathlineto{\pgfqpoint{1.109871in}{1.686520in}}%
\pgfpathlineto{\pgfqpoint{1.111930in}{1.696231in}}%
\pgfpathlineto{\pgfqpoint{1.120166in}{1.764208in}}%
\pgfpathlineto{\pgfqpoint{1.122225in}{1.667098in}}%
\pgfpathlineto{\pgfqpoint{1.124284in}{1.715653in}}%
\pgfpathlineto{\pgfqpoint{1.126343in}{1.735075in}}%
\pgfpathlineto{\pgfqpoint{1.132520in}{1.696231in}}%
\pgfpathlineto{\pgfqpoint{1.134579in}{1.637965in}}%
\pgfpathlineto{\pgfqpoint{1.136638in}{1.647676in}}%
\pgfpathlineto{\pgfqpoint{1.138697in}{1.705942in}}%
\pgfpathlineto{\pgfqpoint{1.140756in}{1.676809in}}%
\pgfpathlineto{\pgfqpoint{1.146932in}{1.725364in}}%
\pgfpathlineto{\pgfqpoint{1.148991in}{1.754497in}}%
\pgfpathlineto{\pgfqpoint{1.153109in}{1.715653in}}%
\pgfpathlineto{\pgfqpoint{1.155168in}{1.822474in}}%
\pgfpathlineto{\pgfqpoint{1.161345in}{1.793341in}}%
\pgfpathlineto{\pgfqpoint{1.163404in}{1.764208in}}%
\pgfpathlineto{\pgfqpoint{1.165463in}{1.754497in}}%
\pgfpathlineto{\pgfqpoint{1.167522in}{1.725364in}}%
\pgfpathlineto{\pgfqpoint{1.169581in}{1.735075in}}%
\pgfpathlineto{\pgfqpoint{1.175758in}{1.725364in}}%
\pgfpathlineto{\pgfqpoint{1.177817in}{1.735075in}}%
\pgfpathlineto{\pgfqpoint{1.179875in}{1.589410in}}%
\pgfpathlineto{\pgfqpoint{1.181934in}{1.657387in}}%
\pgfpathlineto{\pgfqpoint{1.183993in}{1.618543in}}%
\pgfpathlineto{\pgfqpoint{1.190170in}{1.599121in}}%
\pgfpathlineto{\pgfqpoint{1.192229in}{1.579699in}}%
\pgfpathlineto{\pgfqpoint{1.196347in}{1.647676in}}%
\pgfpathlineto{\pgfqpoint{1.198406in}{1.589410in}}%
\pgfpathlineto{\pgfqpoint{1.204583in}{1.599121in}}%
\pgfpathlineto{\pgfqpoint{1.208701in}{1.531145in}}%
\pgfpathlineto{\pgfqpoint{1.210760in}{1.540855in}}%
\pgfpathlineto{\pgfqpoint{1.212819in}{1.472879in}}%
\pgfpathlineto{\pgfqpoint{1.218995in}{1.502012in}}%
\pgfpathlineto{\pgfqpoint{1.227231in}{1.579699in}}%
\pgfpathlineto{\pgfqpoint{1.233408in}{1.560277in}}%
\pgfpathlineto{\pgfqpoint{1.235467in}{1.521434in}}%
\pgfpathlineto{\pgfqpoint{1.237526in}{1.521434in}}%
\pgfpathlineto{\pgfqpoint{1.239585in}{1.482590in}}%
\pgfpathlineto{\pgfqpoint{1.241644in}{1.511723in}}%
\pgfpathlineto{\pgfqpoint{1.247821in}{1.531145in}}%
\pgfpathlineto{\pgfqpoint{1.249880in}{1.531145in}}%
\pgfpathlineto{\pgfqpoint{1.251939in}{1.579699in}}%
\pgfpathlineto{\pgfqpoint{1.256056in}{1.511723in}}%
\pgfpathlineto{\pgfqpoint{1.262233in}{1.540855in}}%
\pgfpathlineto{\pgfqpoint{1.264292in}{1.569988in}}%
\pgfpathlineto{\pgfqpoint{1.266351in}{1.579699in}}%
\pgfpathlineto{\pgfqpoint{1.268410in}{1.579699in}}%
\pgfpathlineto{\pgfqpoint{1.270469in}{1.637965in}}%
\pgfpathlineto{\pgfqpoint{1.276646in}{1.628254in}}%
\pgfpathlineto{\pgfqpoint{1.280764in}{1.696231in}}%
\pgfpathlineto{\pgfqpoint{1.282823in}{1.686520in}}%
\pgfpathlineto{\pgfqpoint{1.284882in}{1.628254in}}%
\pgfpathlineto{\pgfqpoint{1.291059in}{1.686520in}}%
\pgfpathlineto{\pgfqpoint{1.295176in}{1.637965in}}%
\pgfpathlineto{\pgfqpoint{1.297235in}{1.579699in}}%
\pgfpathlineto{\pgfqpoint{1.299294in}{1.569988in}}%
\pgfpathlineto{\pgfqpoint{1.305471in}{1.618543in}}%
\pgfpathlineto{\pgfqpoint{1.307530in}{1.676809in}}%
\pgfpathlineto{\pgfqpoint{1.311648in}{1.618543in}}%
\pgfpathlineto{\pgfqpoint{1.313707in}{1.676809in}}%
\pgfpathlineto{\pgfqpoint{1.321943in}{1.657387in}}%
\pgfpathlineto{\pgfqpoint{1.324002in}{1.667098in}}%
\pgfpathlineto{\pgfqpoint{1.328120in}{1.608832in}}%
\pgfpathlineto{\pgfqpoint{1.334296in}{1.657387in}}%
\pgfpathlineto{\pgfqpoint{1.336355in}{1.686520in}}%
\pgfpathlineto{\pgfqpoint{1.338414in}{1.735075in}}%
\pgfpathlineto{\pgfqpoint{1.340473in}{1.705942in}}%
\pgfpathlineto{\pgfqpoint{1.342532in}{1.793341in}}%
\pgfpathlineto{\pgfqpoint{1.348709in}{1.754497in}}%
\pgfpathlineto{\pgfqpoint{1.352827in}{1.812763in}}%
\pgfpathlineto{\pgfqpoint{1.354886in}{1.783630in}}%
\pgfpathlineto{\pgfqpoint{1.356945in}{1.783630in}}%
\pgfpathlineto{\pgfqpoint{1.363122in}{1.764208in}}%
\pgfpathlineto{\pgfqpoint{1.365181in}{1.744786in}}%
\pgfpathlineto{\pgfqpoint{1.367240in}{1.696231in}}%
\pgfpathlineto{\pgfqpoint{1.369298in}{1.696231in}}%
\pgfpathlineto{\pgfqpoint{1.371357in}{1.657387in}}%
\pgfpathlineto{\pgfqpoint{1.377534in}{1.725364in}}%
\pgfpathlineto{\pgfqpoint{1.379593in}{1.735075in}}%
\pgfpathlineto{\pgfqpoint{1.381652in}{1.725364in}}%
\pgfpathlineto{\pgfqpoint{1.383711in}{1.725364in}}%
\pgfpathlineto{\pgfqpoint{1.385770in}{1.754497in}}%
\pgfpathlineto{\pgfqpoint{1.391947in}{1.667098in}}%
\pgfpathlineto{\pgfqpoint{1.394006in}{1.676809in}}%
\pgfpathlineto{\pgfqpoint{1.396065in}{1.744786in}}%
\pgfpathlineto{\pgfqpoint{1.398124in}{1.676809in}}%
\pgfpathlineto{\pgfqpoint{1.406359in}{1.618543in}}%
\pgfpathlineto{\pgfqpoint{1.408418in}{1.628254in}}%
\pgfpathlineto{\pgfqpoint{1.410477in}{1.579699in}}%
\pgfpathlineto{\pgfqpoint{1.412536in}{1.618543in}}%
\pgfpathlineto{\pgfqpoint{1.414595in}{1.705942in}}%
\pgfpathlineto{\pgfqpoint{1.420772in}{1.725364in}}%
\pgfpathlineto{\pgfqpoint{1.424890in}{1.667098in}}%
\pgfpathlineto{\pgfqpoint{1.426949in}{1.715653in}}%
\pgfpathlineto{\pgfqpoint{1.429008in}{1.715653in}}%
\pgfpathlineto{\pgfqpoint{1.435185in}{1.754497in}}%
\pgfpathlineto{\pgfqpoint{1.437244in}{1.735075in}}%
\pgfpathlineto{\pgfqpoint{1.439303in}{1.744786in}}%
\pgfpathlineto{\pgfqpoint{1.443421in}{1.705942in}}%
\pgfpathlineto{\pgfqpoint{1.449597in}{1.667098in}}%
\pgfpathlineto{\pgfqpoint{1.453715in}{1.715653in}}%
\pgfpathlineto{\pgfqpoint{1.455774in}{1.735075in}}%
\pgfpathlineto{\pgfqpoint{1.457833in}{1.667098in}}%
\pgfpathlineto{\pgfqpoint{1.464010in}{1.657387in}}%
\pgfpathlineto{\pgfqpoint{1.466069in}{1.744786in}}%
\pgfpathlineto{\pgfqpoint{1.468128in}{1.764208in}}%
\pgfpathlineto{\pgfqpoint{1.470187in}{1.744786in}}%
\pgfpathlineto{\pgfqpoint{1.472246in}{1.744786in}}%
\pgfpathlineto{\pgfqpoint{1.478423in}{1.754497in}}%
\pgfpathlineto{\pgfqpoint{1.480482in}{1.696231in}}%
\pgfpathlineto{\pgfqpoint{1.482540in}{1.676809in}}%
\pgfpathlineto{\pgfqpoint{1.484599in}{1.725364in}}%
\pgfpathlineto{\pgfqpoint{1.486658in}{1.744786in}}%
\pgfpathlineto{\pgfqpoint{1.492835in}{1.735075in}}%
\pgfpathlineto{\pgfqpoint{1.494894in}{1.735075in}}%
\pgfpathlineto{\pgfqpoint{1.496953in}{1.647676in}}%
\pgfpathlineto{\pgfqpoint{1.499012in}{1.667098in}}%
\pgfpathlineto{\pgfqpoint{1.501071in}{1.618543in}}%
\pgfpathlineto{\pgfqpoint{1.507248in}{1.569988in}}%
\pgfpathlineto{\pgfqpoint{1.509307in}{1.647676in}}%
\pgfpathlineto{\pgfqpoint{1.511366in}{1.647676in}}%
\pgfpathlineto{\pgfqpoint{1.513425in}{1.657387in}}%
\pgfpathlineto{\pgfqpoint{1.515484in}{1.705942in}}%
\pgfpathlineto{\pgfqpoint{1.521660in}{1.735075in}}%
\pgfpathlineto{\pgfqpoint{1.523719in}{1.696231in}}%
\pgfpathlineto{\pgfqpoint{1.525778in}{1.705942in}}%
\pgfpathlineto{\pgfqpoint{1.527837in}{1.667098in}}%
\pgfpathlineto{\pgfqpoint{1.529896in}{1.667098in}}%
\pgfpathlineto{\pgfqpoint{1.540191in}{1.725364in}}%
\pgfpathlineto{\pgfqpoint{1.542250in}{1.725364in}}%
\pgfpathlineto{\pgfqpoint{1.544309in}{1.696231in}}%
\pgfpathlineto{\pgfqpoint{1.550486in}{1.696231in}}%
\pgfpathlineto{\pgfqpoint{1.552545in}{1.783630in}}%
\pgfpathlineto{\pgfqpoint{1.554604in}{1.793341in}}%
\pgfpathlineto{\pgfqpoint{1.556663in}{1.667098in}}%
\pgfpathlineto{\pgfqpoint{1.558721in}{1.637965in}}%
\pgfpathlineto{\pgfqpoint{1.564898in}{1.676809in}}%
\pgfpathlineto{\pgfqpoint{1.566957in}{1.637965in}}%
\pgfpathlineto{\pgfqpoint{1.569016in}{1.667098in}}%
\pgfpathlineto{\pgfqpoint{1.571075in}{1.637965in}}%
\pgfpathlineto{\pgfqpoint{1.573134in}{1.667098in}}%
\pgfpathlineto{\pgfqpoint{1.579311in}{1.637965in}}%
\pgfpathlineto{\pgfqpoint{1.581370in}{1.589410in}}%
\pgfpathlineto{\pgfqpoint{1.585488in}{1.589410in}}%
\pgfpathlineto{\pgfqpoint{1.587547in}{1.521434in}}%
\pgfpathlineto{\pgfqpoint{1.599900in}{1.599121in}}%
\pgfpathlineto{\pgfqpoint{1.601959in}{1.628254in}}%
\pgfpathlineto{\pgfqpoint{1.610195in}{1.589410in}}%
\pgfpathlineto{\pgfqpoint{1.612254in}{1.511723in}}%
\pgfpathlineto{\pgfqpoint{1.614313in}{1.569988in}}%
\pgfpathlineto{\pgfqpoint{1.616372in}{1.579699in}}%
\pgfpathlineto{\pgfqpoint{1.622549in}{1.560277in}}%
\pgfpathlineto{\pgfqpoint{1.624608in}{1.599121in}}%
\pgfpathlineto{\pgfqpoint{1.628726in}{1.560277in}}%
\pgfpathlineto{\pgfqpoint{1.630785in}{1.608832in}}%
\pgfpathlineto{\pgfqpoint{1.636961in}{1.608832in}}%
\pgfpathlineto{\pgfqpoint{1.639020in}{1.589410in}}%
\pgfpathlineto{\pgfqpoint{1.643138in}{1.715653in}}%
\pgfpathlineto{\pgfqpoint{1.645197in}{1.715653in}}%
\pgfpathlineto{\pgfqpoint{1.651374in}{1.744786in}}%
\pgfpathlineto{\pgfqpoint{1.653433in}{1.764208in}}%
\pgfpathlineto{\pgfqpoint{1.655492in}{1.822474in}}%
\pgfpathlineto{\pgfqpoint{1.657551in}{1.822474in}}%
\pgfpathlineto{\pgfqpoint{1.659610in}{1.890451in}}%
\pgfpathlineto{\pgfqpoint{1.665787in}{1.929295in}}%
\pgfpathlineto{\pgfqpoint{1.667846in}{1.880740in}}%
\pgfpathlineto{\pgfqpoint{1.671963in}{1.900162in}}%
\pgfpathlineto{\pgfqpoint{1.674022in}{1.861318in}}%
\pgfpathlineto{\pgfqpoint{1.680199in}{1.841896in}}%
\pgfpathlineto{\pgfqpoint{1.682258in}{1.841896in}}%
\pgfpathlineto{\pgfqpoint{1.684317in}{1.871029in}}%
\pgfpathlineto{\pgfqpoint{1.686376in}{1.871029in}}%
\pgfpathlineto{\pgfqpoint{1.688435in}{1.890451in}}%
\pgfpathlineto{\pgfqpoint{1.694612in}{1.919584in}}%
\pgfpathlineto{\pgfqpoint{1.696671in}{1.890451in}}%
\pgfpathlineto{\pgfqpoint{1.698730in}{1.909873in}}%
\pgfpathlineto{\pgfqpoint{1.702848in}{1.890451in}}%
\pgfpathlineto{\pgfqpoint{1.709024in}{1.900162in}}%
\pgfpathlineto{\pgfqpoint{1.711083in}{1.851607in}}%
\pgfpathlineto{\pgfqpoint{1.715201in}{1.929295in}}%
\pgfpathlineto{\pgfqpoint{1.717260in}{1.909873in}}%
\pgfpathlineto{\pgfqpoint{1.723437in}{1.900162in}}%
\pgfpathlineto{\pgfqpoint{1.725496in}{1.909873in}}%
\pgfpathlineto{\pgfqpoint{1.727555in}{1.880740in}}%
\pgfpathlineto{\pgfqpoint{1.729614in}{1.909873in}}%
\pgfpathlineto{\pgfqpoint{1.731673in}{1.822474in}}%
\pgfpathlineto{\pgfqpoint{1.739909in}{1.948717in}}%
\pgfpathlineto{\pgfqpoint{1.741968in}{2.006983in}}%
\pgfpathlineto{\pgfqpoint{1.744027in}{1.987561in}}%
\pgfpathlineto{\pgfqpoint{1.746086in}{1.929295in}}%
\pgfpathlineto{\pgfqpoint{1.752262in}{1.939006in}}%
\pgfpathlineto{\pgfqpoint{1.754321in}{1.968139in}}%
\pgfpathlineto{\pgfqpoint{1.758439in}{1.987561in}}%
\pgfpathlineto{\pgfqpoint{1.766675in}{1.987561in}}%
\pgfpathlineto{\pgfqpoint{1.768734in}{2.036116in}}%
\pgfpathlineto{\pgfqpoint{1.770793in}{2.016694in}}%
\pgfpathlineto{\pgfqpoint{1.772852in}{1.968139in}}%
\pgfpathlineto{\pgfqpoint{1.781088in}{1.968139in}}%
\pgfpathlineto{\pgfqpoint{1.783147in}{1.977850in}}%
\pgfpathlineto{\pgfqpoint{1.787264in}{1.880740in}}%
\pgfpathlineto{\pgfqpoint{1.789323in}{1.861318in}}%
\pgfpathlineto{\pgfqpoint{1.795500in}{1.861318in}}%
\pgfpathlineto{\pgfqpoint{1.801677in}{1.803052in}}%
\pgfpathlineto{\pgfqpoint{1.803736in}{1.744786in}}%
\pgfpathlineto{\pgfqpoint{1.811972in}{1.773919in}}%
\pgfpathlineto{\pgfqpoint{1.814031in}{1.725364in}}%
\pgfpathlineto{\pgfqpoint{1.816090in}{1.725364in}}%
\pgfpathlineto{\pgfqpoint{1.818149in}{1.773919in}}%
\pgfpathlineto{\pgfqpoint{1.824325in}{1.764208in}}%
\pgfpathlineto{\pgfqpoint{1.826384in}{1.735075in}}%
\pgfpathlineto{\pgfqpoint{1.828443in}{1.735075in}}%
\pgfpathlineto{\pgfqpoint{1.830502in}{1.715653in}}%
\pgfpathlineto{\pgfqpoint{1.832561in}{1.637965in}}%
\pgfpathlineto{\pgfqpoint{1.838738in}{1.676809in}}%
\pgfpathlineto{\pgfqpoint{1.840797in}{1.599121in}}%
\pgfpathlineto{\pgfqpoint{1.844915in}{1.569988in}}%
\pgfpathlineto{\pgfqpoint{1.846974in}{1.579699in}}%
\pgfpathlineto{\pgfqpoint{1.853151in}{1.502012in}}%
\pgfpathlineto{\pgfqpoint{1.855210in}{1.521434in}}%
\pgfpathlineto{\pgfqpoint{1.857269in}{1.521434in}}%
\pgfpathlineto{\pgfqpoint{1.859328in}{1.482590in}}%
\pgfpathlineto{\pgfqpoint{1.861386in}{1.560277in}}%
\pgfpathlineto{\pgfqpoint{1.869622in}{1.579699in}}%
\pgfpathlineto{\pgfqpoint{1.871681in}{1.599121in}}%
\pgfpathlineto{\pgfqpoint{1.873740in}{1.550566in}}%
\pgfpathlineto{\pgfqpoint{1.875799in}{1.579699in}}%
\pgfpathlineto{\pgfqpoint{1.881976in}{1.589410in}}%
\pgfpathlineto{\pgfqpoint{1.884035in}{1.569988in}}%
\pgfpathlineto{\pgfqpoint{1.886094in}{1.569988in}}%
\pgfpathlineto{\pgfqpoint{1.888153in}{1.521434in}}%
\pgfpathlineto{\pgfqpoint{1.890212in}{1.599121in}}%
\pgfpathlineto{\pgfqpoint{1.896389in}{1.579699in}}%
\pgfpathlineto{\pgfqpoint{1.898447in}{1.647676in}}%
\pgfpathlineto{\pgfqpoint{1.900506in}{1.667098in}}%
\pgfpathlineto{\pgfqpoint{1.902565in}{1.657387in}}%
\pgfpathlineto{\pgfqpoint{1.904624in}{1.705942in}}%
\pgfpathlineto{\pgfqpoint{1.910801in}{1.744786in}}%
\pgfpathlineto{\pgfqpoint{1.912860in}{1.705942in}}%
\pgfpathlineto{\pgfqpoint{1.916978in}{1.773919in}}%
\pgfpathlineto{\pgfqpoint{1.919037in}{1.822474in}}%
\pgfpathlineto{\pgfqpoint{1.925214in}{1.812763in}}%
\pgfpathlineto{\pgfqpoint{1.927273in}{1.822474in}}%
\pgfpathlineto{\pgfqpoint{1.929332in}{1.715653in}}%
\pgfpathlineto{\pgfqpoint{1.931391in}{1.705942in}}%
\pgfpathlineto{\pgfqpoint{1.933450in}{1.667098in}}%
\pgfpathlineto{\pgfqpoint{1.939626in}{1.715653in}}%
\pgfpathlineto{\pgfqpoint{1.941685in}{1.744786in}}%
\pgfpathlineto{\pgfqpoint{1.943744in}{1.696231in}}%
\pgfpathlineto{\pgfqpoint{1.945803in}{1.715653in}}%
\pgfpathlineto{\pgfqpoint{1.954039in}{1.705942in}}%
\pgfpathlineto{\pgfqpoint{1.956098in}{1.608832in}}%
\pgfpathlineto{\pgfqpoint{1.960216in}{1.540855in}}%
\pgfpathlineto{\pgfqpoint{1.962275in}{1.569988in}}%
\pgfpathlineto{\pgfqpoint{1.968452in}{1.550566in}}%
\pgfpathlineto{\pgfqpoint{1.970511in}{1.521434in}}%
\pgfpathlineto{\pgfqpoint{1.972570in}{1.550566in}}%
\pgfpathlineto{\pgfqpoint{1.974628in}{1.502012in}}%
\pgfpathlineto{\pgfqpoint{1.976687in}{1.511723in}}%
\pgfpathlineto{\pgfqpoint{1.982864in}{1.521434in}}%
\pgfpathlineto{\pgfqpoint{1.984923in}{1.569988in}}%
\pgfpathlineto{\pgfqpoint{1.986982in}{1.569988in}}%
\pgfpathlineto{\pgfqpoint{1.989041in}{1.589410in}}%
\pgfpathlineto{\pgfqpoint{1.991100in}{1.540855in}}%
\pgfpathlineto{\pgfqpoint{1.997277in}{1.569988in}}%
\pgfpathlineto{\pgfqpoint{1.999336in}{1.589410in}}%
\pgfpathlineto{\pgfqpoint{2.001395in}{1.637965in}}%
\pgfpathlineto{\pgfqpoint{2.003454in}{1.647676in}}%
\pgfpathlineto{\pgfqpoint{2.005513in}{1.676809in}}%
\pgfpathlineto{\pgfqpoint{2.011689in}{1.676809in}}%
\pgfpathlineto{\pgfqpoint{2.013748in}{1.705942in}}%
\pgfpathlineto{\pgfqpoint{2.017866in}{1.599121in}}%
\pgfpathlineto{\pgfqpoint{2.019925in}{1.589410in}}%
\pgfpathlineto{\pgfqpoint{2.026102in}{1.628254in}}%
\pgfpathlineto{\pgfqpoint{2.030220in}{1.560277in}}%
\pgfpathlineto{\pgfqpoint{2.032279in}{1.540855in}}%
\pgfpathlineto{\pgfqpoint{2.034338in}{1.569988in}}%
\pgfpathlineto{\pgfqpoint{2.040515in}{1.531145in}}%
\pgfpathlineto{\pgfqpoint{2.042574in}{1.550566in}}%
\pgfpathlineto{\pgfqpoint{2.044633in}{1.540855in}}%
\pgfpathlineto{\pgfqpoint{2.046692in}{1.589410in}}%
\pgfpathlineto{\pgfqpoint{2.048751in}{1.579699in}}%
\pgfpathlineto{\pgfqpoint{2.054927in}{1.608832in}}%
\pgfpathlineto{\pgfqpoint{2.056986in}{1.637965in}}%
\pgfpathlineto{\pgfqpoint{2.059045in}{1.744786in}}%
\pgfpathlineto{\pgfqpoint{2.063163in}{1.715653in}}%
\pgfpathlineto{\pgfqpoint{2.069340in}{1.715653in}}%
\pgfpathlineto{\pgfqpoint{2.071399in}{1.744786in}}%
\pgfpathlineto{\pgfqpoint{2.073458in}{1.744786in}}%
\pgfpathlineto{\pgfqpoint{2.075517in}{1.696231in}}%
\pgfpathlineto{\pgfqpoint{2.077576in}{1.725364in}}%
\pgfpathlineto{\pgfqpoint{2.085812in}{1.696231in}}%
\pgfpathlineto{\pgfqpoint{2.087870in}{1.735075in}}%
\pgfpathlineto{\pgfqpoint{2.089929in}{1.696231in}}%
\pgfpathlineto{\pgfqpoint{2.091988in}{1.589410in}}%
\pgfpathlineto{\pgfqpoint{2.098165in}{1.608832in}}%
\pgfpathlineto{\pgfqpoint{2.100224in}{1.608832in}}%
\pgfpathlineto{\pgfqpoint{2.102283in}{1.599121in}}%
\pgfpathlineto{\pgfqpoint{2.104342in}{1.579699in}}%
\pgfpathlineto{\pgfqpoint{2.106401in}{1.540855in}}%
\pgfpathlineto{\pgfqpoint{2.112578in}{1.511723in}}%
\pgfpathlineto{\pgfqpoint{2.114637in}{1.521434in}}%
\pgfpathlineto{\pgfqpoint{2.116696in}{1.482590in}}%
\pgfpathlineto{\pgfqpoint{2.118755in}{1.482590in}}%
\pgfpathlineto{\pgfqpoint{2.120814in}{1.502012in}}%
\pgfpathlineto{\pgfqpoint{2.129049in}{1.560277in}}%
\pgfpathlineto{\pgfqpoint{2.131108in}{1.550566in}}%
\pgfpathlineto{\pgfqpoint{2.133167in}{1.589410in}}%
\pgfpathlineto{\pgfqpoint{2.135226in}{1.434035in}}%
\pgfpathlineto{\pgfqpoint{2.141403in}{1.375769in}}%
\pgfpathlineto{\pgfqpoint{2.143462in}{1.385480in}}%
\pgfpathlineto{\pgfqpoint{2.145521in}{1.414613in}}%
\pgfpathlineto{\pgfqpoint{2.147580in}{1.385480in}}%
\pgfpathlineto{\pgfqpoint{2.149639in}{1.385480in}}%
\pgfpathlineto{\pgfqpoint{2.157875in}{1.336925in}}%
\pgfpathlineto{\pgfqpoint{2.159934in}{1.366058in}}%
\pgfpathlineto{\pgfqpoint{2.161993in}{1.366058in}}%
\pgfpathlineto{\pgfqpoint{2.170228in}{1.443746in}}%
\pgfpathlineto{\pgfqpoint{2.172287in}{1.482590in}}%
\pgfpathlineto{\pgfqpoint{2.174346in}{1.472879in}}%
\pgfpathlineto{\pgfqpoint{2.176405in}{1.492301in}}%
\pgfpathlineto{\pgfqpoint{2.178464in}{1.540855in}}%
\pgfpathlineto{\pgfqpoint{2.186700in}{1.511723in}}%
\pgfpathlineto{\pgfqpoint{2.188759in}{1.531145in}}%
\pgfpathlineto{\pgfqpoint{2.190818in}{1.492301in}}%
\pgfpathlineto{\pgfqpoint{2.192877in}{1.511723in}}%
\pgfpathlineto{\pgfqpoint{2.199054in}{1.540855in}}%
\pgfpathlineto{\pgfqpoint{2.201112in}{1.540855in}}%
\pgfpathlineto{\pgfqpoint{2.203171in}{1.502012in}}%
\pgfpathlineto{\pgfqpoint{2.205230in}{1.492301in}}%
\pgfpathlineto{\pgfqpoint{2.207289in}{1.434035in}}%
\pgfpathlineto{\pgfqpoint{2.215525in}{1.463168in}}%
\pgfpathlineto{\pgfqpoint{2.217584in}{1.453457in}}%
\pgfpathlineto{\pgfqpoint{2.219643in}{1.434035in}}%
\pgfpathlineto{\pgfqpoint{2.221702in}{1.531145in}}%
\pgfpathlineto{\pgfqpoint{2.227879in}{1.531145in}}%
\pgfpathlineto{\pgfqpoint{2.229938in}{1.511723in}}%
\pgfpathlineto{\pgfqpoint{2.231997in}{1.472879in}}%
\pgfpathlineto{\pgfqpoint{2.234056in}{1.550566in}}%
\pgfpathlineto{\pgfqpoint{2.236115in}{1.492301in}}%
\pgfpathlineto{\pgfqpoint{2.242291in}{1.521434in}}%
\pgfpathlineto{\pgfqpoint{2.244350in}{1.540855in}}%
\pgfpathlineto{\pgfqpoint{2.246409in}{1.531145in}}%
\pgfpathlineto{\pgfqpoint{2.248468in}{1.482590in}}%
\pgfpathlineto{\pgfqpoint{2.250527in}{1.550566in}}%
\pgfpathlineto{\pgfqpoint{2.256704in}{1.531145in}}%
\pgfpathlineto{\pgfqpoint{2.258763in}{1.531145in}}%
\pgfpathlineto{\pgfqpoint{2.260822in}{1.540855in}}%
\pgfpathlineto{\pgfqpoint{2.262881in}{1.560277in}}%
\pgfpathlineto{\pgfqpoint{2.264940in}{1.628254in}}%
\pgfpathlineto{\pgfqpoint{2.271117in}{1.589410in}}%
\pgfpathlineto{\pgfqpoint{2.275235in}{1.589410in}}%
\pgfpathlineto{\pgfqpoint{2.277293in}{1.579699in}}%
\pgfpathlineto{\pgfqpoint{2.279352in}{1.589410in}}%
\pgfpathlineto{\pgfqpoint{2.287588in}{1.531145in}}%
\pgfpathlineto{\pgfqpoint{2.289647in}{1.531145in}}%
\pgfpathlineto{\pgfqpoint{2.291706in}{1.579699in}}%
\pgfpathlineto{\pgfqpoint{2.293765in}{1.599121in}}%
\pgfpathlineto{\pgfqpoint{2.299942in}{1.589410in}}%
\pgfpathlineto{\pgfqpoint{2.302001in}{1.618543in}}%
\pgfpathlineto{\pgfqpoint{2.306119in}{1.540855in}}%
\pgfpathlineto{\pgfqpoint{2.308178in}{1.579699in}}%
\pgfpathlineto{\pgfqpoint{2.314355in}{1.589410in}}%
\pgfpathlineto{\pgfqpoint{2.316413in}{1.599121in}}%
\pgfpathlineto{\pgfqpoint{2.330826in}{1.531145in}}%
\pgfpathlineto{\pgfqpoint{2.332885in}{1.540855in}}%
\pgfpathlineto{\pgfqpoint{2.334944in}{1.521434in}}%
\pgfpathlineto{\pgfqpoint{2.337003in}{1.550566in}}%
\pgfpathlineto{\pgfqpoint{2.343180in}{1.579699in}}%
\pgfpathlineto{\pgfqpoint{2.347298in}{1.647676in}}%
\pgfpathlineto{\pgfqpoint{2.349357in}{1.667098in}}%
\pgfpathlineto{\pgfqpoint{2.351416in}{1.657387in}}%
\pgfpathlineto{\pgfqpoint{2.359651in}{1.696231in}}%
\pgfpathlineto{\pgfqpoint{2.361710in}{1.657387in}}%
\pgfpathlineto{\pgfqpoint{2.363769in}{1.667098in}}%
\pgfpathlineto{\pgfqpoint{2.365828in}{1.667098in}}%
\pgfpathlineto{\pgfqpoint{2.372005in}{1.647676in}}%
\pgfpathlineto{\pgfqpoint{2.374064in}{1.628254in}}%
\pgfpathlineto{\pgfqpoint{2.376123in}{1.628254in}}%
\pgfpathlineto{\pgfqpoint{2.378182in}{1.647676in}}%
\pgfpathlineto{\pgfqpoint{2.380241in}{1.647676in}}%
\pgfpathlineto{\pgfqpoint{2.386418in}{1.667098in}}%
\pgfpathlineto{\pgfqpoint{2.388477in}{1.667098in}}%
\pgfpathlineto{\pgfqpoint{2.390535in}{1.676809in}}%
\pgfpathlineto{\pgfqpoint{2.392594in}{1.705942in}}%
\pgfpathlineto{\pgfqpoint{2.394653in}{1.686520in}}%
\pgfpathlineto{\pgfqpoint{2.402889in}{1.657387in}}%
\pgfpathlineto{\pgfqpoint{2.404948in}{1.647676in}}%
\pgfpathlineto{\pgfqpoint{2.407007in}{1.647676in}}%
\pgfpathlineto{\pgfqpoint{2.409066in}{1.618543in}}%
\pgfpathlineto{\pgfqpoint{2.415243in}{1.657387in}}%
\pgfpathlineto{\pgfqpoint{2.417302in}{1.705942in}}%
\pgfpathlineto{\pgfqpoint{2.421420in}{1.832185in}}%
\pgfpathlineto{\pgfqpoint{2.429655in}{1.929295in}}%
\pgfpathlineto{\pgfqpoint{2.431714in}{1.939006in}}%
\pgfpathlineto{\pgfqpoint{2.433773in}{1.939006in}}%
\pgfpathlineto{\pgfqpoint{2.435832in}{1.968139in}}%
\pgfpathlineto{\pgfqpoint{2.437891in}{2.016694in}}%
\pgfpathlineto{\pgfqpoint{2.444068in}{2.016694in}}%
\pgfpathlineto{\pgfqpoint{2.446127in}{2.006983in}}%
\pgfpathlineto{\pgfqpoint{2.448186in}{2.055538in}}%
\pgfpathlineto{\pgfqpoint{2.452304in}{2.065249in}}%
\pgfpathlineto{\pgfqpoint{2.460540in}{2.026405in}}%
\pgfpathlineto{\pgfqpoint{2.462599in}{2.055538in}}%
\pgfpathlineto{\pgfqpoint{2.464658in}{2.104092in}}%
\pgfpathlineto{\pgfqpoint{2.466716in}{2.055538in}}%
\pgfpathlineto{\pgfqpoint{2.472893in}{2.074960in}}%
\pgfpathlineto{\pgfqpoint{2.474952in}{2.065249in}}%
\pgfpathlineto{\pgfqpoint{2.477011in}{2.045827in}}%
\pgfpathlineto{\pgfqpoint{2.479070in}{2.055538in}}%
\pgfpathlineto{\pgfqpoint{2.481129in}{2.084671in}}%
\pgfpathlineto{\pgfqpoint{2.487306in}{2.094382in}}%
\pgfpathlineto{\pgfqpoint{2.489365in}{2.113803in}}%
\pgfpathlineto{\pgfqpoint{2.491424in}{2.220624in}}%
\pgfpathlineto{\pgfqpoint{2.493483in}{2.259468in}}%
\pgfpathlineto{\pgfqpoint{2.495542in}{2.240046in}}%
\pgfpathlineto{\pgfqpoint{2.501719in}{2.201202in}}%
\pgfpathlineto{\pgfqpoint{2.503778in}{2.210913in}}%
\pgfpathlineto{\pgfqpoint{2.505836in}{2.191491in}}%
\pgfpathlineto{\pgfqpoint{2.509954in}{2.191491in}}%
\pgfpathlineto{\pgfqpoint{2.518190in}{2.230335in}}%
\pgfpathlineto{\pgfqpoint{2.520249in}{2.201202in}}%
\pgfpathlineto{\pgfqpoint{2.522308in}{2.142936in}}%
\pgfpathlineto{\pgfqpoint{2.524367in}{2.123514in}}%
\pgfpathlineto{\pgfqpoint{2.532603in}{2.152647in}}%
\pgfpathlineto{\pgfqpoint{2.534662in}{2.152647in}}%
\pgfpathlineto{\pgfqpoint{2.536721in}{2.084671in}}%
\pgfpathlineto{\pgfqpoint{2.538780in}{2.152647in}}%
\pgfpathlineto{\pgfqpoint{2.544956in}{2.123514in}}%
\pgfpathlineto{\pgfqpoint{2.549074in}{2.123514in}}%
\pgfpathlineto{\pgfqpoint{2.551133in}{2.104092in}}%
\pgfpathlineto{\pgfqpoint{2.553192in}{2.133225in}}%
\pgfpathlineto{\pgfqpoint{2.561428in}{2.074960in}}%
\pgfpathlineto{\pgfqpoint{2.563487in}{2.162358in}}%
\pgfpathlineto{\pgfqpoint{2.565546in}{2.181780in}}%
\pgfpathlineto{\pgfqpoint{2.567605in}{2.152647in}}%
\pgfpathlineto{\pgfqpoint{2.573782in}{2.084671in}}%
\pgfpathlineto{\pgfqpoint{2.577900in}{2.172069in}}%
\pgfpathlineto{\pgfqpoint{2.579958in}{2.142936in}}%
\pgfpathlineto{\pgfqpoint{2.582017in}{2.133225in}}%
\pgfpathlineto{\pgfqpoint{2.588194in}{2.133225in}}%
\pgfpathlineto{\pgfqpoint{2.590253in}{2.113803in}}%
\pgfpathlineto{\pgfqpoint{2.592312in}{2.142936in}}%
\pgfpathlineto{\pgfqpoint{2.594371in}{2.133225in}}%
\pgfpathlineto{\pgfqpoint{2.596430in}{2.142936in}}%
\pgfpathlineto{\pgfqpoint{2.602607in}{2.084671in}}%
\pgfpathlineto{\pgfqpoint{2.604666in}{2.084671in}}%
\pgfpathlineto{\pgfqpoint{2.606725in}{2.055538in}}%
\pgfpathlineto{\pgfqpoint{2.608784in}{2.113803in}}%
\pgfpathlineto{\pgfqpoint{2.610843in}{2.123514in}}%
\pgfpathlineto{\pgfqpoint{2.617020in}{2.133225in}}%
\pgfpathlineto{\pgfqpoint{2.621137in}{2.220624in}}%
\pgfpathlineto{\pgfqpoint{2.623196in}{2.152647in}}%
\pgfpathlineto{\pgfqpoint{2.625255in}{2.133225in}}%
\pgfpathlineto{\pgfqpoint{2.633491in}{2.152647in}}%
\pgfpathlineto{\pgfqpoint{2.635550in}{2.142936in}}%
\pgfpathlineto{\pgfqpoint{2.639668in}{2.036116in}}%
\pgfpathlineto{\pgfqpoint{2.647904in}{2.142936in}}%
\pgfpathlineto{\pgfqpoint{2.649963in}{2.220624in}}%
\pgfpathlineto{\pgfqpoint{2.652022in}{2.249757in}}%
\pgfpathlineto{\pgfqpoint{2.654081in}{2.240046in}}%
\pgfpathlineto{\pgfqpoint{2.660257in}{2.249757in}}%
\pgfpathlineto{\pgfqpoint{2.666434in}{2.317734in}}%
\pgfpathlineto{\pgfqpoint{2.668493in}{2.308023in}}%
\pgfpathlineto{\pgfqpoint{2.674670in}{2.337156in}}%
\pgfpathlineto{\pgfqpoint{2.676729in}{2.327445in}}%
\pgfpathlineto{\pgfqpoint{2.678788in}{2.240046in}}%
\pgfpathlineto{\pgfqpoint{2.680847in}{2.278890in}}%
\pgfpathlineto{\pgfqpoint{2.682906in}{2.249757in}}%
\pgfpathlineto{\pgfqpoint{2.689083in}{2.220624in}}%
\pgfpathlineto{\pgfqpoint{2.693200in}{2.172069in}}%
\pgfpathlineto{\pgfqpoint{2.697318in}{2.172069in}}%
\pgfpathlineto{\pgfqpoint{2.703495in}{2.162358in}}%
\pgfpathlineto{\pgfqpoint{2.705554in}{2.210913in}}%
\pgfpathlineto{\pgfqpoint{2.707613in}{2.181780in}}%
\pgfpathlineto{\pgfqpoint{2.709672in}{2.201202in}}%
\pgfpathlineto{\pgfqpoint{2.711731in}{2.152647in}}%
\pgfpathlineto{\pgfqpoint{2.717908in}{2.123514in}}%
\pgfpathlineto{\pgfqpoint{2.719967in}{2.123514in}}%
\pgfpathlineto{\pgfqpoint{2.722026in}{2.094382in}}%
\pgfpathlineto{\pgfqpoint{2.724085in}{2.104092in}}%
\pgfpathlineto{\pgfqpoint{2.726144in}{2.172069in}}%
\pgfpathlineto{\pgfqpoint{2.732320in}{2.172069in}}%
\pgfpathlineto{\pgfqpoint{2.734379in}{2.104092in}}%
\pgfpathlineto{\pgfqpoint{2.736438in}{2.094382in}}%
\pgfpathlineto{\pgfqpoint{2.738497in}{2.055538in}}%
\pgfpathlineto{\pgfqpoint{2.746733in}{2.074960in}}%
\pgfpathlineto{\pgfqpoint{2.748792in}{2.006983in}}%
\pgfpathlineto{\pgfqpoint{2.752910in}{2.065249in}}%
\pgfpathlineto{\pgfqpoint{2.754969in}{2.055538in}}%
\pgfpathlineto{\pgfqpoint{2.761146in}{2.094382in}}%
\pgfpathlineto{\pgfqpoint{2.763205in}{2.142936in}}%
\pgfpathlineto{\pgfqpoint{2.767323in}{2.094382in}}%
\pgfpathlineto{\pgfqpoint{2.775558in}{2.133225in}}%
\pgfpathlineto{\pgfqpoint{2.777617in}{2.104092in}}%
\pgfpathlineto{\pgfqpoint{2.779676in}{2.152647in}}%
\pgfpathlineto{\pgfqpoint{2.783794in}{2.172069in}}%
\pgfpathlineto{\pgfqpoint{2.789971in}{2.181780in}}%
\pgfpathlineto{\pgfqpoint{2.792030in}{2.220624in}}%
\pgfpathlineto{\pgfqpoint{2.796148in}{2.201202in}}%
\pgfpathlineto{\pgfqpoint{2.798207in}{2.142936in}}%
\pgfpathlineto{\pgfqpoint{2.804384in}{2.142936in}}%
\pgfpathlineto{\pgfqpoint{2.806443in}{2.133225in}}%
\pgfpathlineto{\pgfqpoint{2.808501in}{2.074960in}}%
\pgfpathlineto{\pgfqpoint{2.812619in}{2.104092in}}%
\pgfpathlineto{\pgfqpoint{2.818796in}{2.104092in}}%
\pgfpathlineto{\pgfqpoint{2.820855in}{2.142936in}}%
\pgfpathlineto{\pgfqpoint{2.822914in}{2.113803in}}%
\pgfpathlineto{\pgfqpoint{2.827032in}{2.113803in}}%
\pgfpathlineto{\pgfqpoint{2.835268in}{2.094382in}}%
\pgfpathlineto{\pgfqpoint{2.837327in}{2.094382in}}%
\pgfpathlineto{\pgfqpoint{2.839386in}{2.104092in}}%
\pgfpathlineto{\pgfqpoint{2.841445in}{2.074960in}}%
\pgfpathlineto{\pgfqpoint{2.847621in}{2.104092in}}%
\pgfpathlineto{\pgfqpoint{2.849680in}{2.074960in}}%
\pgfpathlineto{\pgfqpoint{2.855857in}{2.152647in}}%
\pgfpathlineto{\pgfqpoint{2.862034in}{2.152647in}}%
\pgfpathlineto{\pgfqpoint{2.864093in}{2.162358in}}%
\pgfpathlineto{\pgfqpoint{2.866152in}{2.133225in}}%
\pgfpathlineto{\pgfqpoint{2.868211in}{2.142936in}}%
\pgfpathlineto{\pgfqpoint{2.870270in}{2.133225in}}%
\pgfpathlineto{\pgfqpoint{2.876447in}{2.172069in}}%
\pgfpathlineto{\pgfqpoint{2.878506in}{2.152647in}}%
\pgfpathlineto{\pgfqpoint{2.880565in}{2.152647in}}%
\pgfpathlineto{\pgfqpoint{2.882623in}{2.133225in}}%
\pgfpathlineto{\pgfqpoint{2.890859in}{2.133225in}}%
\pgfpathlineto{\pgfqpoint{2.892918in}{2.181780in}}%
\pgfpathlineto{\pgfqpoint{2.894977in}{2.162358in}}%
\pgfpathlineto{\pgfqpoint{2.905272in}{2.249757in}}%
\pgfpathlineto{\pgfqpoint{2.909390in}{2.240046in}}%
\pgfpathlineto{\pgfqpoint{2.911449in}{2.249757in}}%
\pgfpathlineto{\pgfqpoint{2.913508in}{2.249757in}}%
\pgfpathlineto{\pgfqpoint{2.919685in}{2.240046in}}%
\pgfpathlineto{\pgfqpoint{2.921743in}{2.220624in}}%
\pgfpathlineto{\pgfqpoint{2.923802in}{2.181780in}}%
\pgfpathlineto{\pgfqpoint{2.925861in}{2.201202in}}%
\pgfpathlineto{\pgfqpoint{2.927920in}{2.191491in}}%
\pgfpathlineto{\pgfqpoint{2.934097in}{2.181780in}}%
\pgfpathlineto{\pgfqpoint{2.936156in}{2.172069in}}%
\pgfpathlineto{\pgfqpoint{2.938215in}{2.172069in}}%
\pgfpathlineto{\pgfqpoint{2.942333in}{2.152647in}}%
\pgfpathlineto{\pgfqpoint{2.948510in}{2.181780in}}%
\pgfpathlineto{\pgfqpoint{2.950569in}{2.210913in}}%
\pgfpathlineto{\pgfqpoint{2.952628in}{2.152647in}}%
\pgfpathlineto{\pgfqpoint{2.954687in}{2.172069in}}%
\pgfpathlineto{\pgfqpoint{2.956746in}{2.162358in}}%
\pgfpathlineto{\pgfqpoint{2.962922in}{2.162358in}}%
\pgfpathlineto{\pgfqpoint{2.964981in}{2.152647in}}%
\pgfpathlineto{\pgfqpoint{2.967040in}{2.172069in}}%
\pgfpathlineto{\pgfqpoint{2.969099in}{2.142936in}}%
\pgfpathlineto{\pgfqpoint{2.971158in}{2.162358in}}%
\pgfpathlineto{\pgfqpoint{2.977335in}{2.172069in}}%
\pgfpathlineto{\pgfqpoint{2.979394in}{2.181780in}}%
\pgfpathlineto{\pgfqpoint{2.981453in}{2.152647in}}%
\pgfpathlineto{\pgfqpoint{2.983512in}{2.142936in}}%
\pgfpathlineto{\pgfqpoint{2.985571in}{2.084671in}}%
\pgfpathlineto{\pgfqpoint{2.991748in}{2.133225in}}%
\pgfpathlineto{\pgfqpoint{2.993807in}{2.162358in}}%
\pgfpathlineto{\pgfqpoint{2.997924in}{2.113803in}}%
\pgfpathlineto{\pgfqpoint{2.999983in}{2.123514in}}%
\pgfpathlineto{\pgfqpoint{3.006160in}{2.113803in}}%
\pgfpathlineto{\pgfqpoint{3.008219in}{2.133225in}}%
\pgfpathlineto{\pgfqpoint{3.010278in}{2.104092in}}%
\pgfpathlineto{\pgfqpoint{3.012337in}{2.123514in}}%
\pgfpathlineto{\pgfqpoint{3.014396in}{2.123514in}}%
\pgfpathlineto{\pgfqpoint{3.020573in}{2.113803in}}%
\pgfpathlineto{\pgfqpoint{3.022632in}{2.084671in}}%
\pgfpathlineto{\pgfqpoint{3.024691in}{2.094382in}}%
\pgfpathlineto{\pgfqpoint{3.026750in}{2.094382in}}%
\pgfpathlineto{\pgfqpoint{3.028809in}{2.113803in}}%
\pgfpathlineto{\pgfqpoint{3.037044in}{2.055538in}}%
\pgfpathlineto{\pgfqpoint{3.039103in}{2.074960in}}%
\pgfpathlineto{\pgfqpoint{3.041162in}{2.036116in}}%
\pgfpathlineto{\pgfqpoint{3.043221in}{2.045827in}}%
\pgfpathlineto{\pgfqpoint{3.053516in}{2.133225in}}%
\pgfpathlineto{\pgfqpoint{3.057634in}{2.181780in}}%
\pgfpathlineto{\pgfqpoint{3.063811in}{2.191491in}}%
\pgfpathlineto{\pgfqpoint{3.065870in}{2.201202in}}%
\pgfpathlineto{\pgfqpoint{3.067929in}{2.249757in}}%
\pgfpathlineto{\pgfqpoint{3.072046in}{2.230335in}}%
\pgfpathlineto{\pgfqpoint{3.078223in}{2.210913in}}%
\pgfpathlineto{\pgfqpoint{3.080282in}{2.220624in}}%
\pgfpathlineto{\pgfqpoint{3.082341in}{2.249757in}}%
\pgfpathlineto{\pgfqpoint{3.084400in}{2.240046in}}%
\pgfpathlineto{\pgfqpoint{3.086459in}{2.269179in}}%
\pgfpathlineto{\pgfqpoint{3.092636in}{2.278890in}}%
\pgfpathlineto{\pgfqpoint{3.094695in}{2.269179in}}%
\pgfpathlineto{\pgfqpoint{3.096754in}{2.269179in}}%
\pgfpathlineto{\pgfqpoint{3.098813in}{2.278890in}}%
\pgfpathlineto{\pgfqpoint{3.100872in}{2.308023in}}%
\pgfpathlineto{\pgfqpoint{3.109108in}{2.288601in}}%
\pgfpathlineto{\pgfqpoint{3.111166in}{2.308023in}}%
\pgfpathlineto{\pgfqpoint{3.113225in}{2.308023in}}%
\pgfpathlineto{\pgfqpoint{3.115284in}{2.288601in}}%
\pgfpathlineto{\pgfqpoint{3.125579in}{2.346867in}}%
\pgfpathlineto{\pgfqpoint{3.127638in}{2.337156in}}%
\pgfpathlineto{\pgfqpoint{3.129697in}{2.366289in}}%
\pgfpathlineto{\pgfqpoint{3.135874in}{2.346867in}}%
\pgfpathlineto{\pgfqpoint{3.137933in}{2.376000in}}%
\pgfpathlineto{\pgfqpoint{3.139992in}{2.385711in}}%
\pgfpathlineto{\pgfqpoint{3.142051in}{2.405133in}}%
\pgfpathlineto{\pgfqpoint{3.144110in}{2.376000in}}%
\pgfpathlineto{\pgfqpoint{3.150286in}{2.356578in}}%
\pgfpathlineto{\pgfqpoint{3.154404in}{2.385711in}}%
\pgfpathlineto{\pgfqpoint{3.156463in}{2.376000in}}%
\pgfpathlineto{\pgfqpoint{3.158522in}{2.385711in}}%
\pgfpathlineto{\pgfqpoint{3.164699in}{2.376000in}}%
\pgfpathlineto{\pgfqpoint{3.168817in}{2.414844in}}%
\pgfpathlineto{\pgfqpoint{3.170876in}{2.395422in}}%
\pgfpathlineto{\pgfqpoint{3.172935in}{2.434266in}}%
\pgfpathlineto{\pgfqpoint{3.179112in}{2.463399in}}%
\pgfpathlineto{\pgfqpoint{3.181171in}{2.453688in}}%
\pgfpathlineto{\pgfqpoint{3.183230in}{2.434266in}}%
\pgfpathlineto{\pgfqpoint{3.185289in}{2.473110in}}%
\pgfpathlineto{\pgfqpoint{3.187347in}{2.473110in}}%
\pgfpathlineto{\pgfqpoint{3.193524in}{2.502243in}}%
\pgfpathlineto{\pgfqpoint{3.195583in}{2.521665in}}%
\pgfpathlineto{\pgfqpoint{3.197642in}{2.482821in}}%
\pgfpathlineto{\pgfqpoint{3.201760in}{2.492532in}}%
\pgfpathlineto{\pgfqpoint{3.207937in}{2.482821in}}%
\pgfpathlineto{\pgfqpoint{3.212055in}{2.502243in}}%
\pgfpathlineto{\pgfqpoint{3.214114in}{2.541087in}}%
\pgfpathlineto{\pgfqpoint{3.216173in}{2.541087in}}%
\pgfpathlineto{\pgfqpoint{3.224408in}{2.579931in}}%
\pgfpathlineto{\pgfqpoint{3.226467in}{2.560509in}}%
\pgfpathlineto{\pgfqpoint{3.230585in}{2.560509in}}%
\pgfpathlineto{\pgfqpoint{3.236762in}{2.589642in}}%
\pgfpathlineto{\pgfqpoint{3.238821in}{2.589642in}}%
\pgfpathlineto{\pgfqpoint{3.240880in}{2.541087in}}%
\pgfpathlineto{\pgfqpoint{3.244998in}{2.589642in}}%
\pgfpathlineto{\pgfqpoint{3.251175in}{2.579931in}}%
\pgfpathlineto{\pgfqpoint{3.253234in}{2.609064in}}%
\pgfpathlineto{\pgfqpoint{3.255293in}{2.618775in}}%
\pgfpathlineto{\pgfqpoint{3.257352in}{2.647908in}}%
\pgfpathlineto{\pgfqpoint{3.259411in}{2.647908in}}%
\pgfpathlineto{\pgfqpoint{3.267646in}{2.657618in}}%
\pgfpathlineto{\pgfqpoint{3.269705in}{2.628486in}}%
\pgfpathlineto{\pgfqpoint{3.271764in}{2.638197in}}%
\pgfpathlineto{\pgfqpoint{3.273823in}{2.618775in}}%
\pgfpathlineto{\pgfqpoint{3.284118in}{2.657618in}}%
\pgfpathlineto{\pgfqpoint{3.286177in}{2.686751in}}%
\pgfpathlineto{\pgfqpoint{3.288236in}{2.696462in}}%
\pgfpathlineto{\pgfqpoint{3.294413in}{2.706173in}}%
\pgfpathlineto{\pgfqpoint{3.296472in}{2.725595in}}%
\pgfpathlineto{\pgfqpoint{3.298531in}{2.715884in}}%
\pgfpathlineto{\pgfqpoint{3.300589in}{2.725595in}}%
\pgfpathlineto{\pgfqpoint{3.302648in}{2.754728in}}%
\pgfpathlineto{\pgfqpoint{3.310884in}{2.754728in}}%
\pgfpathlineto{\pgfqpoint{3.317061in}{2.832416in}}%
\pgfpathlineto{\pgfqpoint{3.323238in}{2.842127in}}%
\pgfpathlineto{\pgfqpoint{3.325297in}{2.812994in}}%
\pgfpathlineto{\pgfqpoint{3.327356in}{2.832416in}}%
\pgfpathlineto{\pgfqpoint{3.329415in}{2.832416in}}%
\pgfpathlineto{\pgfqpoint{3.331474in}{2.871260in}}%
\pgfpathlineto{\pgfqpoint{3.339709in}{2.900393in}}%
\pgfpathlineto{\pgfqpoint{3.341768in}{2.919815in}}%
\pgfpathlineto{\pgfqpoint{3.343827in}{2.958659in}}%
\pgfpathlineto{\pgfqpoint{3.345886in}{2.958659in}}%
\pgfpathlineto{\pgfqpoint{3.352063in}{2.880971in}}%
\pgfpathlineto{\pgfqpoint{3.356181in}{2.958659in}}%
\pgfpathlineto{\pgfqpoint{3.358240in}{2.948948in}}%
\pgfpathlineto{\pgfqpoint{3.360299in}{2.890682in}}%
\pgfpathlineto{\pgfqpoint{3.366476in}{2.929526in}}%
\pgfpathlineto{\pgfqpoint{3.368535in}{2.929526in}}%
\pgfpathlineto{\pgfqpoint{3.370594in}{3.026636in}}%
\pgfpathlineto{\pgfqpoint{3.372653in}{3.026636in}}%
\pgfpathlineto{\pgfqpoint{3.374711in}{3.007214in}}%
\pgfpathlineto{\pgfqpoint{3.382947in}{3.026636in}}%
\pgfpathlineto{\pgfqpoint{3.385006in}{3.065480in}}%
\pgfpathlineto{\pgfqpoint{3.389124in}{3.016925in}}%
\pgfpathlineto{\pgfqpoint{3.395301in}{2.997503in}}%
\pgfpathlineto{\pgfqpoint{3.397360in}{3.055769in}}%
\pgfpathlineto{\pgfqpoint{3.399419in}{3.046058in}}%
\pgfpathlineto{\pgfqpoint{3.401478in}{2.987792in}}%
\pgfpathlineto{\pgfqpoint{3.403537in}{3.026636in}}%
\pgfpathlineto{\pgfqpoint{3.409714in}{3.036347in}}%
\pgfpathlineto{\pgfqpoint{3.411773in}{3.046058in}}%
\pgfpathlineto{\pgfqpoint{3.415890in}{3.046058in}}%
\pgfpathlineto{\pgfqpoint{3.417949in}{3.075191in}}%
\pgfpathlineto{\pgfqpoint{3.424126in}{3.055769in}}%
\pgfpathlineto{\pgfqpoint{3.426185in}{3.036347in}}%
\pgfpathlineto{\pgfqpoint{3.428244in}{3.036347in}}%
\pgfpathlineto{\pgfqpoint{3.430303in}{3.046058in}}%
\pgfpathlineto{\pgfqpoint{3.432362in}{3.065480in}}%
\pgfpathlineto{\pgfqpoint{3.438539in}{3.075191in}}%
\pgfpathlineto{\pgfqpoint{3.440598in}{3.114035in}}%
\pgfpathlineto{\pgfqpoint{3.446775in}{3.036347in}}%
\pgfpathlineto{\pgfqpoint{3.452951in}{3.065480in}}%
\pgfpathlineto{\pgfqpoint{3.455010in}{3.016925in}}%
\pgfpathlineto{\pgfqpoint{3.457069in}{3.036347in}}%
\pgfpathlineto{\pgfqpoint{3.459128in}{3.016925in}}%
\pgfpathlineto{\pgfqpoint{3.467364in}{2.997503in}}%
\pgfpathlineto{\pgfqpoint{3.469423in}{3.036347in}}%
\pgfpathlineto{\pgfqpoint{3.471482in}{3.046058in}}%
\pgfpathlineto{\pgfqpoint{3.473541in}{3.075191in}}%
\pgfpathlineto{\pgfqpoint{3.475600in}{3.026636in}}%
\pgfpathlineto{\pgfqpoint{3.481777in}{3.055769in}}%
\pgfpathlineto{\pgfqpoint{3.483836in}{3.075191in}}%
\pgfpathlineto{\pgfqpoint{3.485895in}{3.075191in}}%
\pgfpathlineto{\pgfqpoint{3.490012in}{3.133457in}}%
\pgfpathlineto{\pgfqpoint{3.496189in}{3.143168in}}%
\pgfpathlineto{\pgfqpoint{3.498248in}{3.143168in}}%
\pgfpathlineto{\pgfqpoint{3.500307in}{3.191723in}}%
\pgfpathlineto{\pgfqpoint{3.502366in}{3.201434in}}%
\pgfpathlineto{\pgfqpoint{3.504425in}{3.240277in}}%
\pgfpathlineto{\pgfqpoint{3.510602in}{3.259699in}}%
\pgfpathlineto{\pgfqpoint{3.512661in}{3.249988in}}%
\pgfpathlineto{\pgfqpoint{3.514720in}{3.259699in}}%
\pgfpathlineto{\pgfqpoint{3.518838in}{3.240277in}}%
\pgfpathlineto{\pgfqpoint{3.525015in}{3.240277in}}%
\pgfpathlineto{\pgfqpoint{3.527073in}{3.279121in}}%
\pgfpathlineto{\pgfqpoint{3.531191in}{3.240277in}}%
\pgfpathlineto{\pgfqpoint{3.533250in}{3.249988in}}%
\pgfpathlineto{\pgfqpoint{3.539427in}{3.259699in}}%
\pgfpathlineto{\pgfqpoint{3.545604in}{3.308254in}}%
\pgfpathlineto{\pgfqpoint{3.547663in}{3.308254in}}%
\pgfpathlineto{\pgfqpoint{3.553840in}{3.317965in}}%
\pgfpathlineto{\pgfqpoint{3.555899in}{3.366520in}}%
\pgfpathlineto{\pgfqpoint{3.557958in}{3.376231in}}%
\pgfpathlineto{\pgfqpoint{3.560017in}{3.366520in}}%
\pgfpathlineto{\pgfqpoint{3.562076in}{3.327676in}}%
\pgfpathlineto{\pgfqpoint{3.568252in}{3.347098in}}%
\pgfpathlineto{\pgfqpoint{3.570311in}{3.347098in}}%
\pgfpathlineto{\pgfqpoint{3.572370in}{3.288832in}}%
\pgfpathlineto{\pgfqpoint{3.574429in}{3.269410in}}%
\pgfpathlineto{\pgfqpoint{3.576488in}{3.220855in}}%
\pgfpathlineto{\pgfqpoint{3.584724in}{3.055769in}}%
\pgfpathlineto{\pgfqpoint{3.586783in}{3.152879in}}%
\pgfpathlineto{\pgfqpoint{3.588842in}{3.162590in}}%
\pgfpathlineto{\pgfqpoint{3.590901in}{3.230566in}}%
\pgfpathlineto{\pgfqpoint{3.597078in}{3.259699in}}%
\pgfpathlineto{\pgfqpoint{3.599137in}{3.240277in}}%
\pgfpathlineto{\pgfqpoint{3.601196in}{3.269410in}}%
\pgfpathlineto{\pgfqpoint{3.603254in}{3.249988in}}%
\pgfpathlineto{\pgfqpoint{3.605313in}{3.249988in}}%
\pgfpathlineto{\pgfqpoint{3.613549in}{3.288832in}}%
\pgfpathlineto{\pgfqpoint{3.615608in}{3.327676in}}%
\pgfpathlineto{\pgfqpoint{3.619726in}{3.298543in}}%
\pgfpathlineto{\pgfqpoint{3.625903in}{3.288832in}}%
\pgfpathlineto{\pgfqpoint{3.627962in}{3.259699in}}%
\pgfpathlineto{\pgfqpoint{3.630021in}{3.288832in}}%
\pgfpathlineto{\pgfqpoint{3.632080in}{3.269410in}}%
\pgfpathlineto{\pgfqpoint{3.634139in}{3.269410in}}%
\pgfpathlineto{\pgfqpoint{3.640315in}{3.249988in}}%
\pgfpathlineto{\pgfqpoint{3.642374in}{3.249988in}}%
\pgfpathlineto{\pgfqpoint{3.644433in}{3.211145in}}%
\pgfpathlineto{\pgfqpoint{3.646492in}{3.220855in}}%
\pgfpathlineto{\pgfqpoint{3.648551in}{3.249988in}}%
\pgfpathlineto{\pgfqpoint{3.654728in}{3.269410in}}%
\pgfpathlineto{\pgfqpoint{3.656787in}{3.249988in}}%
\pgfpathlineto{\pgfqpoint{3.660905in}{3.269410in}}%
\pgfpathlineto{\pgfqpoint{3.662964in}{3.259699in}}%
\pgfpathlineto{\pgfqpoint{3.669141in}{3.279121in}}%
\pgfpathlineto{\pgfqpoint{3.671200in}{3.308254in}}%
\pgfpathlineto{\pgfqpoint{3.673259in}{3.288832in}}%
\pgfpathlineto{\pgfqpoint{3.675318in}{3.298543in}}%
\pgfpathlineto{\pgfqpoint{3.677377in}{3.279121in}}%
\pgfpathlineto{\pgfqpoint{3.683553in}{3.288832in}}%
\pgfpathlineto{\pgfqpoint{3.685612in}{3.308254in}}%
\pgfpathlineto{\pgfqpoint{3.687671in}{3.308254in}}%
\pgfpathlineto{\pgfqpoint{3.689730in}{3.288832in}}%
\pgfpathlineto{\pgfqpoint{3.691789in}{3.298543in}}%
\pgfpathlineto{\pgfqpoint{3.702084in}{3.356809in}}%
\pgfpathlineto{\pgfqpoint{3.704143in}{3.395653in}}%
\pgfpathlineto{\pgfqpoint{3.706202in}{3.376231in}}%
\pgfpathlineto{\pgfqpoint{3.712379in}{3.385942in}}%
\pgfpathlineto{\pgfqpoint{3.714438in}{3.385942in}}%
\pgfpathlineto{\pgfqpoint{3.716496in}{3.395653in}}%
\pgfpathlineto{\pgfqpoint{3.720614in}{3.356809in}}%
\pgfpathlineto{\pgfqpoint{3.726791in}{3.347098in}}%
\pgfpathlineto{\pgfqpoint{3.728850in}{3.376231in}}%
\pgfpathlineto{\pgfqpoint{3.730909in}{3.385942in}}%
\pgfpathlineto{\pgfqpoint{3.732968in}{3.356809in}}%
\pgfpathlineto{\pgfqpoint{3.735027in}{3.298543in}}%
\pgfpathlineto{\pgfqpoint{3.741204in}{3.298543in}}%
\pgfpathlineto{\pgfqpoint{3.743263in}{3.327676in}}%
\pgfpathlineto{\pgfqpoint{3.745322in}{3.298543in}}%
\pgfpathlineto{\pgfqpoint{3.747381in}{3.317965in}}%
\pgfpathlineto{\pgfqpoint{3.749440in}{3.298543in}}%
\pgfpathlineto{\pgfqpoint{3.755616in}{3.269410in}}%
\pgfpathlineto{\pgfqpoint{3.757675in}{3.288832in}}%
\pgfpathlineto{\pgfqpoint{3.759734in}{3.269410in}}%
\pgfpathlineto{\pgfqpoint{3.761793in}{3.279121in}}%
\pgfpathlineto{\pgfqpoint{3.763852in}{3.298543in}}%
\pgfpathlineto{\pgfqpoint{3.770029in}{3.317965in}}%
\pgfpathlineto{\pgfqpoint{3.774147in}{3.366520in}}%
\pgfpathlineto{\pgfqpoint{3.778265in}{3.317965in}}%
\pgfpathlineto{\pgfqpoint{3.786501in}{3.347098in}}%
\pgfpathlineto{\pgfqpoint{3.790619in}{3.327676in}}%
\pgfpathlineto{\pgfqpoint{3.792677in}{3.395653in}}%
\pgfpathlineto{\pgfqpoint{3.798854in}{3.395653in}}%
\pgfpathlineto{\pgfqpoint{3.800913in}{3.444208in}}%
\pgfpathlineto{\pgfqpoint{3.802972in}{3.434497in}}%
\pgfpathlineto{\pgfqpoint{3.805031in}{3.444208in}}%
\pgfpathlineto{\pgfqpoint{3.807090in}{3.463630in}}%
\pgfpathlineto{\pgfqpoint{3.813267in}{3.463630in}}%
\pgfpathlineto{\pgfqpoint{3.815326in}{3.492763in}}%
\pgfpathlineto{\pgfqpoint{3.817385in}{3.502474in}}%
\pgfpathlineto{\pgfqpoint{3.827680in}{3.502474in}}%
\pgfpathlineto{\pgfqpoint{3.829738in}{3.521896in}}%
\pgfpathlineto{\pgfqpoint{3.831797in}{3.502474in}}%
\pgfpathlineto{\pgfqpoint{3.833856in}{3.502474in}}%
\pgfpathlineto{\pgfqpoint{3.835915in}{3.492763in}}%
\pgfpathlineto{\pgfqpoint{3.842092in}{3.512185in}}%
\pgfpathlineto{\pgfqpoint{3.844151in}{3.492763in}}%
\pgfpathlineto{\pgfqpoint{3.848269in}{3.580162in}}%
\pgfpathlineto{\pgfqpoint{3.850328in}{3.599584in}}%
\pgfpathlineto{\pgfqpoint{3.858564in}{3.589873in}}%
\pgfpathlineto{\pgfqpoint{3.860623in}{3.580162in}}%
\pgfpathlineto{\pgfqpoint{3.862682in}{3.551029in}}%
\pgfpathlineto{\pgfqpoint{3.864741in}{3.541318in}}%
\pgfpathlineto{\pgfqpoint{3.870917in}{3.551029in}}%
\pgfpathlineto{\pgfqpoint{3.872976in}{3.560740in}}%
\pgfpathlineto{\pgfqpoint{3.875035in}{3.580162in}}%
\pgfpathlineto{\pgfqpoint{3.877094in}{3.560740in}}%
\pgfpathlineto{\pgfqpoint{3.879153in}{3.599584in}}%
\pgfpathlineto{\pgfqpoint{3.885330in}{3.599584in}}%
\pgfpathlineto{\pgfqpoint{3.889448in}{3.502474in}}%
\pgfpathlineto{\pgfqpoint{3.891507in}{3.531607in}}%
\pgfpathlineto{\pgfqpoint{3.893566in}{3.463630in}}%
\pgfpathlineto{\pgfqpoint{3.899743in}{3.473341in}}%
\pgfpathlineto{\pgfqpoint{3.903861in}{3.541318in}}%
\pgfpathlineto{\pgfqpoint{3.905919in}{3.521896in}}%
\pgfpathlineto{\pgfqpoint{3.907978in}{3.589873in}}%
\pgfpathlineto{\pgfqpoint{3.914155in}{3.599584in}}%
\pgfpathlineto{\pgfqpoint{3.920332in}{3.657850in}}%
\pgfpathlineto{\pgfqpoint{3.922391in}{3.619006in}}%
\pgfpathlineto{\pgfqpoint{3.930627in}{3.560740in}}%
\pgfpathlineto{\pgfqpoint{3.932686in}{3.531607in}}%
\pgfpathlineto{\pgfqpoint{3.934745in}{3.521896in}}%
\pgfpathlineto{\pgfqpoint{3.936804in}{3.463630in}}%
\pgfpathlineto{\pgfqpoint{3.942980in}{3.434497in}}%
\pgfpathlineto{\pgfqpoint{3.947098in}{3.453919in}}%
\pgfpathlineto{\pgfqpoint{3.951216in}{3.444208in}}%
\pgfpathlineto{\pgfqpoint{3.957393in}{3.473341in}}%
\pgfpathlineto{\pgfqpoint{3.959452in}{3.473341in}}%
\pgfpathlineto{\pgfqpoint{3.963570in}{3.444208in}}%
\pgfpathlineto{\pgfqpoint{3.965629in}{3.444208in}}%
\pgfpathlineto{\pgfqpoint{3.971806in}{3.453919in}}%
\pgfpathlineto{\pgfqpoint{3.980042in}{3.337387in}}%
\pgfpathlineto{\pgfqpoint{3.986218in}{3.347098in}}%
\pgfpathlineto{\pgfqpoint{3.988277in}{3.395653in}}%
\pgfpathlineto{\pgfqpoint{3.990336in}{3.395653in}}%
\pgfpathlineto{\pgfqpoint{3.992395in}{3.376231in}}%
\pgfpathlineto{\pgfqpoint{3.994454in}{3.337387in}}%
\pgfpathlineto{\pgfqpoint{4.000631in}{3.298543in}}%
\pgfpathlineto{\pgfqpoint{4.004749in}{3.230566in}}%
\pgfpathlineto{\pgfqpoint{4.006808in}{3.269410in}}%
\pgfpathlineto{\pgfqpoint{4.008867in}{3.230566in}}%
\pgfpathlineto{\pgfqpoint{4.015044in}{3.182012in}}%
\pgfpathlineto{\pgfqpoint{4.019161in}{3.220855in}}%
\pgfpathlineto{\pgfqpoint{4.023279in}{3.123746in}}%
\pgfpathlineto{\pgfqpoint{4.029456in}{3.084902in}}%
\pgfpathlineto{\pgfqpoint{4.033574in}{3.094613in}}%
\pgfpathlineto{\pgfqpoint{4.035633in}{2.978081in}}%
\pgfpathlineto{\pgfqpoint{4.037692in}{3.094613in}}%
\pgfpathlineto{\pgfqpoint{4.043869in}{3.133457in}}%
\pgfpathlineto{\pgfqpoint{4.045928in}{3.191723in}}%
\pgfpathlineto{\pgfqpoint{4.047987in}{3.162590in}}%
\pgfpathlineto{\pgfqpoint{4.050046in}{3.162590in}}%
\pgfpathlineto{\pgfqpoint{4.052105in}{3.133457in}}%
\pgfpathlineto{\pgfqpoint{4.060340in}{3.133457in}}%
\pgfpathlineto{\pgfqpoint{4.064458in}{3.172301in}}%
\pgfpathlineto{\pgfqpoint{4.066517in}{3.220855in}}%
\pgfpathlineto{\pgfqpoint{4.074753in}{3.172301in}}%
\pgfpathlineto{\pgfqpoint{4.076812in}{3.191723in}}%
\pgfpathlineto{\pgfqpoint{4.078871in}{3.162590in}}%
\pgfpathlineto{\pgfqpoint{4.080930in}{3.201434in}}%
\pgfpathlineto{\pgfqpoint{4.087107in}{3.201434in}}%
\pgfpathlineto{\pgfqpoint{4.091225in}{3.114035in}}%
\pgfpathlineto{\pgfqpoint{4.093284in}{3.055769in}}%
\pgfpathlineto{\pgfqpoint{4.095342in}{3.123746in}}%
\pgfpathlineto{\pgfqpoint{4.101519in}{3.143168in}}%
\pgfpathlineto{\pgfqpoint{4.103578in}{3.123746in}}%
\pgfpathlineto{\pgfqpoint{4.105637in}{3.123746in}}%
\pgfpathlineto{\pgfqpoint{4.109755in}{3.055769in}}%
\pgfpathlineto{\pgfqpoint{4.117991in}{3.104324in}}%
\pgfpathlineto{\pgfqpoint{4.120050in}{3.143168in}}%
\pgfpathlineto{\pgfqpoint{4.122109in}{3.104324in}}%
\pgfpathlineto{\pgfqpoint{4.124168in}{3.123746in}}%
\pgfpathlineto{\pgfqpoint{4.132403in}{3.094613in}}%
\pgfpathlineto{\pgfqpoint{4.134462in}{3.094613in}}%
\pgfpathlineto{\pgfqpoint{4.136521in}{3.114035in}}%
\pgfpathlineto{\pgfqpoint{4.138580in}{3.084902in}}%
\pgfpathlineto{\pgfqpoint{4.144757in}{3.104324in}}%
\pgfpathlineto{\pgfqpoint{4.146816in}{3.065480in}}%
\pgfpathlineto{\pgfqpoint{4.152993in}{3.162590in}}%
\pgfpathlineto{\pgfqpoint{4.159170in}{3.143168in}}%
\pgfpathlineto{\pgfqpoint{4.161229in}{3.143168in}}%
\pgfpathlineto{\pgfqpoint{4.163288in}{3.114035in}}%
\pgfpathlineto{\pgfqpoint{4.165347in}{3.065480in}}%
\pgfpathlineto{\pgfqpoint{4.167406in}{3.055769in}}%
\pgfpathlineto{\pgfqpoint{4.173582in}{3.075191in}}%
\pgfpathlineto{\pgfqpoint{4.175641in}{3.036347in}}%
\pgfpathlineto{\pgfqpoint{4.177700in}{3.036347in}}%
\pgfpathlineto{\pgfqpoint{4.179759in}{3.046058in}}%
\pgfpathlineto{\pgfqpoint{4.181818in}{3.016925in}}%
\pgfpathlineto{\pgfqpoint{4.190054in}{3.046058in}}%
\pgfpathlineto{\pgfqpoint{4.192113in}{2.968370in}}%
\pgfpathlineto{\pgfqpoint{4.194172in}{2.968370in}}%
\pgfpathlineto{\pgfqpoint{4.196231in}{2.871260in}}%
\pgfpathlineto{\pgfqpoint{4.206526in}{2.793572in}}%
\pgfpathlineto{\pgfqpoint{4.210643in}{2.842127in}}%
\pgfpathlineto{\pgfqpoint{4.216820in}{2.919815in}}%
\pgfpathlineto{\pgfqpoint{4.218879in}{2.890682in}}%
\pgfpathlineto{\pgfqpoint{4.220938in}{2.919815in}}%
\pgfpathlineto{\pgfqpoint{4.222997in}{2.919815in}}%
\pgfpathlineto{\pgfqpoint{4.225056in}{2.929526in}}%
\pgfpathlineto{\pgfqpoint{4.231233in}{2.939237in}}%
\pgfpathlineto{\pgfqpoint{4.233292in}{2.939237in}}%
\pgfpathlineto{\pgfqpoint{4.235351in}{2.900393in}}%
\pgfpathlineto{\pgfqpoint{4.237410in}{2.929526in}}%
\pgfpathlineto{\pgfqpoint{4.239469in}{2.987792in}}%
\pgfpathlineto{\pgfqpoint{4.245645in}{2.987792in}}%
\pgfpathlineto{\pgfqpoint{4.247704in}{3.007214in}}%
\pgfpathlineto{\pgfqpoint{4.251822in}{2.987792in}}%
\pgfpathlineto{\pgfqpoint{4.260058in}{2.987792in}}%
\pgfpathlineto{\pgfqpoint{4.262117in}{2.968370in}}%
\pgfpathlineto{\pgfqpoint{4.264176in}{2.910104in}}%
\pgfpathlineto{\pgfqpoint{4.266235in}{2.919815in}}%
\pgfpathlineto{\pgfqpoint{4.268294in}{2.880971in}}%
\pgfpathlineto{\pgfqpoint{4.274471in}{2.900393in}}%
\pgfpathlineto{\pgfqpoint{4.276530in}{2.871260in}}%
\pgfpathlineto{\pgfqpoint{4.280648in}{2.948948in}}%
\pgfpathlineto{\pgfqpoint{4.282707in}{2.929526in}}%
\pgfpathlineto{\pgfqpoint{4.288883in}{2.900393in}}%
\pgfpathlineto{\pgfqpoint{4.290942in}{2.871260in}}%
\pgfpathlineto{\pgfqpoint{4.293001in}{2.890682in}}%
\pgfpathlineto{\pgfqpoint{4.295060in}{2.851838in}}%
\pgfpathlineto{\pgfqpoint{4.297119in}{2.861549in}}%
\pgfpathlineto{\pgfqpoint{4.303296in}{2.783861in}}%
\pgfpathlineto{\pgfqpoint{4.305355in}{2.803283in}}%
\pgfpathlineto{\pgfqpoint{4.307414in}{2.754728in}}%
\pgfpathlineto{\pgfqpoint{4.309473in}{2.783861in}}%
\pgfpathlineto{\pgfqpoint{4.311532in}{2.783861in}}%
\pgfpathlineto{\pgfqpoint{4.317709in}{2.803283in}}%
\pgfpathlineto{\pgfqpoint{4.319768in}{2.832416in}}%
\pgfpathlineto{\pgfqpoint{4.321826in}{2.803283in}}%
\pgfpathlineto{\pgfqpoint{4.323885in}{2.715884in}}%
\pgfpathlineto{\pgfqpoint{4.325944in}{2.735306in}}%
\pgfpathlineto{\pgfqpoint{4.334180in}{2.696462in}}%
\pgfpathlineto{\pgfqpoint{4.336239in}{2.677040in}}%
\pgfpathlineto{\pgfqpoint{4.338298in}{2.638197in}}%
\pgfpathlineto{\pgfqpoint{4.340357in}{2.541087in}}%
\pgfpathlineto{\pgfqpoint{4.346534in}{2.434266in}}%
\pgfpathlineto{\pgfqpoint{4.348593in}{2.482821in}}%
\pgfpathlineto{\pgfqpoint{4.350652in}{2.453688in}}%
\pgfpathlineto{\pgfqpoint{4.352711in}{2.492532in}}%
\pgfpathlineto{\pgfqpoint{4.354770in}{2.463399in}}%
\pgfpathlineto{\pgfqpoint{4.360946in}{2.511954in}}%
\pgfpathlineto{\pgfqpoint{4.363005in}{2.511954in}}%
\pgfpathlineto{\pgfqpoint{4.367123in}{2.424555in}}%
\pgfpathlineto{\pgfqpoint{4.369182in}{2.434266in}}%
\pgfpathlineto{\pgfqpoint{4.375359in}{2.443977in}}%
\pgfpathlineto{\pgfqpoint{4.377418in}{2.443977in}}%
\pgfpathlineto{\pgfqpoint{4.379477in}{2.346867in}}%
\pgfpathlineto{\pgfqpoint{4.381536in}{2.337156in}}%
\pgfpathlineto{\pgfqpoint{4.383595in}{2.385711in}}%
\pgfpathlineto{\pgfqpoint{4.391831in}{2.317734in}}%
\pgfpathlineto{\pgfqpoint{4.393890in}{2.385711in}}%
\pgfpathlineto{\pgfqpoint{4.395949in}{2.356578in}}%
\pgfpathlineto{\pgfqpoint{4.398007in}{2.356578in}}%
\pgfpathlineto{\pgfqpoint{4.404184in}{2.385711in}}%
\pgfpathlineto{\pgfqpoint{4.406243in}{2.356578in}}%
\pgfpathlineto{\pgfqpoint{4.408302in}{2.356578in}}%
\pgfpathlineto{\pgfqpoint{4.412420in}{2.463399in}}%
\pgfpathlineto{\pgfqpoint{4.418597in}{2.482821in}}%
\pgfpathlineto{\pgfqpoint{4.420656in}{2.521665in}}%
\pgfpathlineto{\pgfqpoint{4.422715in}{2.434266in}}%
\pgfpathlineto{\pgfqpoint{4.424774in}{2.482821in}}%
\pgfpathlineto{\pgfqpoint{4.426833in}{2.453688in}}%
\pgfpathlineto{\pgfqpoint{4.433010in}{2.443977in}}%
\pgfpathlineto{\pgfqpoint{4.435068in}{2.482821in}}%
\pgfpathlineto{\pgfqpoint{4.439186in}{2.385711in}}%
\pgfpathlineto{\pgfqpoint{4.441245in}{2.414844in}}%
\pgfpathlineto{\pgfqpoint{4.447422in}{2.414844in}}%
\pgfpathlineto{\pgfqpoint{4.449481in}{2.434266in}}%
\pgfpathlineto{\pgfqpoint{4.451540in}{2.434266in}}%
\pgfpathlineto{\pgfqpoint{4.453599in}{2.473110in}}%
\pgfpathlineto{\pgfqpoint{4.455658in}{2.473110in}}%
\pgfpathlineto{\pgfqpoint{4.461835in}{2.453688in}}%
\pgfpathlineto{\pgfqpoint{4.463894in}{2.463399in}}%
\pgfpathlineto{\pgfqpoint{4.465953in}{2.482821in}}%
\pgfpathlineto{\pgfqpoint{4.468012in}{2.317734in}}%
\pgfpathlineto{\pgfqpoint{4.470071in}{2.317734in}}%
\pgfpathlineto{\pgfqpoint{4.476247in}{2.201202in}}%
\pgfpathlineto{\pgfqpoint{4.478306in}{2.191491in}}%
\pgfpathlineto{\pgfqpoint{4.480365in}{2.162358in}}%
\pgfpathlineto{\pgfqpoint{4.484483in}{2.230335in}}%
\pgfpathlineto{\pgfqpoint{4.490660in}{2.162358in}}%
\pgfpathlineto{\pgfqpoint{4.492719in}{2.249757in}}%
\pgfpathlineto{\pgfqpoint{4.496837in}{2.094382in}}%
\pgfpathlineto{\pgfqpoint{4.498896in}{2.094382in}}%
\pgfpathlineto{\pgfqpoint{4.505073in}{2.142936in}}%
\pgfpathlineto{\pgfqpoint{4.507132in}{2.094382in}}%
\pgfpathlineto{\pgfqpoint{4.511249in}{2.181780in}}%
\pgfpathlineto{\pgfqpoint{4.513308in}{2.084671in}}%
\pgfpathlineto{\pgfqpoint{4.519485in}{2.123514in}}%
\pgfpathlineto{\pgfqpoint{4.521544in}{2.084671in}}%
\pgfpathlineto{\pgfqpoint{4.523603in}{2.074960in}}%
\pgfpathlineto{\pgfqpoint{4.525662in}{2.094382in}}%
\pgfpathlineto{\pgfqpoint{4.527721in}{2.074960in}}%
\pgfpathlineto{\pgfqpoint{4.535957in}{2.036116in}}%
\pgfpathlineto{\pgfqpoint{4.538016in}{2.016694in}}%
\pgfpathlineto{\pgfqpoint{4.540075in}{2.123514in}}%
\pgfpathlineto{\pgfqpoint{4.542134in}{2.113803in}}%
\pgfpathlineto{\pgfqpoint{4.548311in}{2.172069in}}%
\pgfpathlineto{\pgfqpoint{4.550369in}{2.259468in}}%
\pgfpathlineto{\pgfqpoint{4.552428in}{2.269179in}}%
\pgfpathlineto{\pgfqpoint{4.554487in}{2.317734in}}%
\pgfpathlineto{\pgfqpoint{4.556546in}{2.405133in}}%
\pgfpathlineto{\pgfqpoint{4.562723in}{2.356578in}}%
\pgfpathlineto{\pgfqpoint{4.564782in}{2.327445in}}%
\pgfpathlineto{\pgfqpoint{4.566841in}{2.366289in}}%
\pgfpathlineto{\pgfqpoint{4.570959in}{2.278890in}}%
\pgfpathlineto{\pgfqpoint{4.577136in}{2.259468in}}%
\pgfpathlineto{\pgfqpoint{4.579195in}{2.181780in}}%
\pgfpathlineto{\pgfqpoint{4.581254in}{2.259468in}}%
\pgfpathlineto{\pgfqpoint{4.583313in}{2.259468in}}%
\pgfpathlineto{\pgfqpoint{4.585372in}{2.230335in}}%
\pgfpathlineto{\pgfqpoint{4.591548in}{2.210913in}}%
\pgfpathlineto{\pgfqpoint{4.593607in}{2.162358in}}%
\pgfpathlineto{\pgfqpoint{4.597725in}{1.997272in}}%
\pgfpathlineto{\pgfqpoint{4.599784in}{2.006983in}}%
\pgfpathlineto{\pgfqpoint{4.605961in}{2.065249in}}%
\pgfpathlineto{\pgfqpoint{4.608020in}{2.036116in}}%
\pgfpathlineto{\pgfqpoint{4.612138in}{2.142936in}}%
\pgfpathlineto{\pgfqpoint{4.614197in}{2.249757in}}%
\pgfpathlineto{\pgfqpoint{4.622433in}{2.249757in}}%
\pgfpathlineto{\pgfqpoint{4.624491in}{2.220624in}}%
\pgfpathlineto{\pgfqpoint{4.626550in}{2.220624in}}%
\pgfpathlineto{\pgfqpoint{4.628609in}{2.210913in}}%
\pgfpathlineto{\pgfqpoint{4.634786in}{2.240046in}}%
\pgfpathlineto{\pgfqpoint{4.636845in}{2.240046in}}%
\pgfpathlineto{\pgfqpoint{4.638904in}{2.230335in}}%
\pgfpathlineto{\pgfqpoint{4.640963in}{2.230335in}}%
\pgfpathlineto{\pgfqpoint{4.643022in}{2.269179in}}%
\pgfpathlineto{\pgfqpoint{4.649199in}{2.298312in}}%
\pgfpathlineto{\pgfqpoint{4.651258in}{2.298312in}}%
\pgfpathlineto{\pgfqpoint{4.653317in}{2.249757in}}%
\pgfpathlineto{\pgfqpoint{4.655376in}{2.172069in}}%
\pgfpathlineto{\pgfqpoint{4.657435in}{2.201202in}}%
\pgfpathlineto{\pgfqpoint{4.663611in}{2.240046in}}%
\pgfpathlineto{\pgfqpoint{4.665670in}{2.278890in}}%
\pgfpathlineto{\pgfqpoint{4.667729in}{2.249757in}}%
\pgfpathlineto{\pgfqpoint{4.669788in}{2.346867in}}%
\pgfpathlineto{\pgfqpoint{4.671847in}{2.346867in}}%
\pgfpathlineto{\pgfqpoint{4.680083in}{2.337156in}}%
\pgfpathlineto{\pgfqpoint{4.684201in}{2.240046in}}%
\pgfpathlineto{\pgfqpoint{4.686260in}{2.259468in}}%
\pgfpathlineto{\pgfqpoint{4.692437in}{2.240046in}}%
\pgfpathlineto{\pgfqpoint{4.694496in}{2.240046in}}%
\pgfpathlineto{\pgfqpoint{4.696555in}{2.201202in}}%
\pgfpathlineto{\pgfqpoint{4.698614in}{2.240046in}}%
\pgfpathlineto{\pgfqpoint{4.700672in}{2.249757in}}%
\pgfpathlineto{\pgfqpoint{4.706849in}{2.249757in}}%
\pgfpathlineto{\pgfqpoint{4.708908in}{2.230335in}}%
\pgfpathlineto{\pgfqpoint{4.710967in}{2.259468in}}%
\pgfpathlineto{\pgfqpoint{4.715085in}{2.259468in}}%
\pgfpathlineto{\pgfqpoint{4.721262in}{2.278890in}}%
\pgfpathlineto{\pgfqpoint{4.723321in}{2.191491in}}%
\pgfpathlineto{\pgfqpoint{4.729498in}{2.288601in}}%
\pgfpathlineto{\pgfqpoint{4.735675in}{2.278890in}}%
\pgfpathlineto{\pgfqpoint{4.737733in}{2.317734in}}%
\pgfpathlineto{\pgfqpoint{4.739792in}{2.269179in}}%
\pgfpathlineto{\pgfqpoint{4.741851in}{2.337156in}}%
\pgfpathlineto{\pgfqpoint{4.743910in}{2.269179in}}%
\pgfpathlineto{\pgfqpoint{4.750087in}{2.317734in}}%
\pgfpathlineto{\pgfqpoint{4.752146in}{2.308023in}}%
\pgfpathlineto{\pgfqpoint{4.754205in}{2.317734in}}%
\pgfpathlineto{\pgfqpoint{4.756264in}{2.298312in}}%
\pgfpathlineto{\pgfqpoint{4.758323in}{2.317734in}}%
\pgfpathlineto{\pgfqpoint{4.764500in}{2.317734in}}%
\pgfpathlineto{\pgfqpoint{4.766559in}{2.288601in}}%
\pgfpathlineto{\pgfqpoint{4.770677in}{2.298312in}}%
\pgfpathlineto{\pgfqpoint{4.772736in}{2.249757in}}%
\pgfpathlineto{\pgfqpoint{4.778912in}{2.240046in}}%
\pgfpathlineto{\pgfqpoint{4.780971in}{2.269179in}}%
\pgfpathlineto{\pgfqpoint{4.785089in}{2.240046in}}%
\pgfpathlineto{\pgfqpoint{4.787148in}{2.191491in}}%
\pgfpathlineto{\pgfqpoint{4.793325in}{2.210913in}}%
\pgfpathlineto{\pgfqpoint{4.795384in}{2.201202in}}%
\pgfpathlineto{\pgfqpoint{4.797443in}{2.259468in}}%
\pgfpathlineto{\pgfqpoint{4.799502in}{2.240046in}}%
\pgfpathlineto{\pgfqpoint{4.801561in}{2.240046in}}%
\pgfpathlineto{\pgfqpoint{4.807738in}{2.249757in}}%
\pgfpathlineto{\pgfqpoint{4.809797in}{2.240046in}}%
\pgfpathlineto{\pgfqpoint{4.811856in}{2.210913in}}%
\pgfpathlineto{\pgfqpoint{4.813914in}{2.230335in}}%
\pgfpathlineto{\pgfqpoint{4.815973in}{2.210913in}}%
\pgfpathlineto{\pgfqpoint{4.826268in}{2.172069in}}%
\pgfpathlineto{\pgfqpoint{4.828327in}{2.162358in}}%
\pgfpathlineto{\pgfqpoint{4.830386in}{2.133225in}}%
\pgfpathlineto{\pgfqpoint{4.836563in}{2.065249in}}%
\pgfpathlineto{\pgfqpoint{4.838622in}{2.104092in}}%
\pgfpathlineto{\pgfqpoint{4.840681in}{2.045827in}}%
\pgfpathlineto{\pgfqpoint{4.842740in}{2.026405in}}%
\pgfpathlineto{\pgfqpoint{4.844799in}{1.958428in}}%
\pgfpathlineto{\pgfqpoint{4.850976in}{1.997272in}}%
\pgfpathlineto{\pgfqpoint{4.855093in}{2.084671in}}%
\pgfpathlineto{\pgfqpoint{4.857152in}{2.084671in}}%
\pgfpathlineto{\pgfqpoint{4.859211in}{2.045827in}}%
\pgfpathlineto{\pgfqpoint{4.865388in}{2.016694in}}%
\pgfpathlineto{\pgfqpoint{4.869506in}{2.074960in}}%
\pgfpathlineto{\pgfqpoint{4.871565in}{2.074960in}}%
\pgfpathlineto{\pgfqpoint{4.873624in}{2.055538in}}%
\pgfpathlineto{\pgfqpoint{4.881860in}{2.026405in}}%
\pgfpathlineto{\pgfqpoint{4.883919in}{2.045827in}}%
\pgfpathlineto{\pgfqpoint{4.896272in}{1.822474in}}%
\pgfpathlineto{\pgfqpoint{4.898331in}{1.803052in}}%
\pgfpathlineto{\pgfqpoint{4.900390in}{1.744786in}}%
\pgfpathlineto{\pgfqpoint{4.902449in}{1.521434in}}%
\pgfpathlineto{\pgfqpoint{4.908626in}{1.521434in}}%
\pgfpathlineto{\pgfqpoint{4.910685in}{1.395191in}}%
\pgfpathlineto{\pgfqpoint{4.912744in}{1.356347in}}%
\pgfpathlineto{\pgfqpoint{4.916862in}{1.210682in}}%
\pgfpathlineto{\pgfqpoint{4.923039in}{1.084439in}}%
\pgfpathlineto{\pgfqpoint{4.925098in}{1.259237in}}%
\pgfpathlineto{\pgfqpoint{4.931274in}{1.259237in}}%
\pgfpathlineto{\pgfqpoint{4.937451in}{1.113572in}}%
\pgfpathlineto{\pgfqpoint{4.941569in}{1.336925in}}%
\pgfpathlineto{\pgfqpoint{4.945687in}{1.094150in}}%
\pgfpathlineto{\pgfqpoint{4.951864in}{0.997040in}}%
\pgfpathlineto{\pgfqpoint{4.953923in}{1.123283in}}%
\pgfpathlineto{\pgfqpoint{4.955982in}{1.094150in}}%
\pgfpathlineto{\pgfqpoint{4.960100in}{0.987329in}}%
\pgfpathlineto{\pgfqpoint{4.966276in}{0.977618in}}%
\pgfpathlineto{\pgfqpoint{4.968335in}{0.977618in}}%
\pgfpathlineto{\pgfqpoint{4.970394in}{0.967908in}}%
\pgfpathlineto{\pgfqpoint{4.974512in}{0.987329in}}%
\pgfpathlineto{\pgfqpoint{4.982748in}{1.045595in}}%
\pgfpathlineto{\pgfqpoint{4.984807in}{1.026173in}}%
\pgfpathlineto{\pgfqpoint{4.986866in}{0.977618in}}%
\pgfpathlineto{\pgfqpoint{4.995102in}{0.997040in}}%
\pgfpathlineto{\pgfqpoint{4.997161in}{0.977618in}}%
\pgfpathlineto{\pgfqpoint{4.999220in}{0.929064in}}%
\pgfpathlineto{\pgfqpoint{5.003337in}{0.948486in}}%
\pgfpathlineto{\pgfqpoint{5.009514in}{0.948486in}}%
\pgfpathlineto{\pgfqpoint{5.011573in}{0.929064in}}%
\pgfpathlineto{\pgfqpoint{5.013632in}{0.948486in}}%
\pgfpathlineto{\pgfqpoint{5.017750in}{0.948486in}}%
\pgfpathlineto{\pgfqpoint{5.023927in}{0.977618in}}%
\pgfpathlineto{\pgfqpoint{5.028045in}{0.929064in}}%
\pgfpathlineto{\pgfqpoint{5.030104in}{0.929064in}}%
\pgfpathlineto{\pgfqpoint{5.032163in}{0.938775in}}%
\pgfpathlineto{\pgfqpoint{5.038340in}{0.929064in}}%
\pgfpathlineto{\pgfqpoint{5.042457in}{0.929064in}}%
\pgfpathlineto{\pgfqpoint{5.044516in}{0.880509in}}%
\pgfpathlineto{\pgfqpoint{5.046575in}{0.899931in}}%
\pgfpathlineto{\pgfqpoint{5.052752in}{0.929064in}}%
\pgfpathlineto{\pgfqpoint{5.056870in}{0.890220in}}%
\pgfpathlineto{\pgfqpoint{5.058929in}{0.890220in}}%
\pgfpathlineto{\pgfqpoint{5.060988in}{0.880509in}}%
\pgfpathlineto{\pgfqpoint{5.067165in}{0.929064in}}%
\pgfpathlineto{\pgfqpoint{5.071283in}{0.899931in}}%
\pgfpathlineto{\pgfqpoint{5.075401in}{0.899931in}}%
\pgfpathlineto{\pgfqpoint{5.083636in}{0.909642in}}%
\pgfpathlineto{\pgfqpoint{5.087754in}{0.909642in}}%
\pgfpathlineto{\pgfqpoint{5.089813in}{0.880509in}}%
\pgfpathlineto{\pgfqpoint{5.095990in}{0.890220in}}%
\pgfpathlineto{\pgfqpoint{5.098049in}{0.890220in}}%
\pgfpathlineto{\pgfqpoint{5.100108in}{0.948486in}}%
\pgfpathlineto{\pgfqpoint{5.102167in}{0.948486in}}%
\pgfpathlineto{\pgfqpoint{5.104226in}{0.977618in}}%
\pgfpathlineto{\pgfqpoint{5.110403in}{0.977618in}}%
\pgfpathlineto{\pgfqpoint{5.114521in}{0.909642in}}%
\pgfpathlineto{\pgfqpoint{5.124815in}{0.909642in}}%
\pgfpathlineto{\pgfqpoint{5.126874in}{0.919353in}}%
\pgfpathlineto{\pgfqpoint{5.128933in}{0.919353in}}%
\pgfpathlineto{\pgfqpoint{5.130992in}{0.909642in}}%
\pgfpathlineto{\pgfqpoint{5.141287in}{0.909642in}}%
\pgfpathlineto{\pgfqpoint{5.143346in}{0.899931in}}%
\pgfpathlineto{\pgfqpoint{5.145405in}{0.899931in}}%
\pgfpathlineto{\pgfqpoint{5.147464in}{0.880509in}}%
\pgfpathlineto{\pgfqpoint{5.153641in}{0.870798in}}%
\pgfpathlineto{\pgfqpoint{5.155699in}{0.870798in}}%
\pgfpathlineto{\pgfqpoint{5.157758in}{0.880509in}}%
\pgfpathlineto{\pgfqpoint{5.172171in}{0.880509in}}%
\pgfpathlineto{\pgfqpoint{5.174230in}{0.870798in}}%
\pgfpathlineto{\pgfqpoint{5.176289in}{0.880509in}}%
\pgfpathlineto{\pgfqpoint{5.184525in}{0.880509in}}%
\pgfpathlineto{\pgfqpoint{5.188643in}{0.861087in}}%
\pgfpathlineto{\pgfqpoint{5.190702in}{0.870798in}}%
\pgfpathlineto{\pgfqpoint{5.196878in}{0.870798in}}%
\pgfpathlineto{\pgfqpoint{5.198937in}{0.861087in}}%
\pgfpathlineto{\pgfqpoint{5.200996in}{0.861087in}}%
\pgfpathlineto{\pgfqpoint{5.203055in}{0.851376in}}%
\pgfpathlineto{\pgfqpoint{5.205114in}{0.861087in}}%
\pgfpathlineto{\pgfqpoint{5.211291in}{0.870798in}}%
\pgfpathlineto{\pgfqpoint{5.215409in}{0.841665in}}%
\pgfpathlineto{\pgfqpoint{5.217468in}{0.831954in}}%
\pgfpathlineto{\pgfqpoint{5.219527in}{0.802821in}}%
\pgfpathlineto{\pgfqpoint{5.225704in}{0.822243in}}%
\pgfpathlineto{\pgfqpoint{5.227763in}{0.793110in}}%
\pgfpathlineto{\pgfqpoint{5.229822in}{0.822243in}}%
\pgfpathlineto{\pgfqpoint{5.231880in}{0.822243in}}%
\pgfpathlineto{\pgfqpoint{5.233939in}{0.831954in}}%
\pgfpathlineto{\pgfqpoint{5.240116in}{0.841665in}}%
\pgfpathlineto{\pgfqpoint{5.242175in}{0.870798in}}%
\pgfpathlineto{\pgfqpoint{5.244234in}{0.880509in}}%
\pgfpathlineto{\pgfqpoint{5.246293in}{0.880509in}}%
\pgfpathlineto{\pgfqpoint{5.248352in}{0.870798in}}%
\pgfpathlineto{\pgfqpoint{5.254529in}{0.870798in}}%
\pgfpathlineto{\pgfqpoint{5.256588in}{0.861087in}}%
\pgfpathlineto{\pgfqpoint{5.258647in}{0.861087in}}%
\pgfpathlineto{\pgfqpoint{5.260706in}{0.841665in}}%
\pgfpathlineto{\pgfqpoint{5.262765in}{0.851376in}}%
\pgfpathlineto{\pgfqpoint{5.268941in}{0.870798in}}%
\pgfpathlineto{\pgfqpoint{5.273059in}{0.870798in}}%
\pgfpathlineto{\pgfqpoint{5.275118in}{0.880509in}}%
\pgfpathlineto{\pgfqpoint{5.277177in}{0.851376in}}%
\pgfpathlineto{\pgfqpoint{5.283354in}{0.841665in}}%
\pgfpathlineto{\pgfqpoint{5.285413in}{0.831954in}}%
\pgfpathlineto{\pgfqpoint{5.287472in}{0.851376in}}%
\pgfpathlineto{\pgfqpoint{5.289531in}{0.841665in}}%
\pgfpathlineto{\pgfqpoint{5.291590in}{0.870798in}}%
\pgfpathlineto{\pgfqpoint{5.299826in}{0.861087in}}%
\pgfpathlineto{\pgfqpoint{5.303944in}{0.861087in}}%
\pgfpathlineto{\pgfqpoint{5.306002in}{0.851376in}}%
\pgfpathlineto{\pgfqpoint{5.326592in}{0.851376in}}%
\pgfpathlineto{\pgfqpoint{5.328651in}{0.841665in}}%
\pgfpathlineto{\pgfqpoint{5.330710in}{0.841665in}}%
\pgfpathlineto{\pgfqpoint{5.332769in}{0.851376in}}%
\pgfpathlineto{\pgfqpoint{5.334828in}{0.841665in}}%
\pgfpathlineto{\pgfqpoint{5.341005in}{0.851376in}}%
\pgfpathlineto{\pgfqpoint{5.343064in}{0.822243in}}%
\pgfpathlineto{\pgfqpoint{5.345122in}{0.851376in}}%
\pgfpathlineto{\pgfqpoint{5.349240in}{0.851376in}}%
\pgfpathlineto{\pgfqpoint{5.355417in}{0.880509in}}%
\pgfpathlineto{\pgfqpoint{5.357476in}{0.861087in}}%
\pgfpathlineto{\pgfqpoint{5.359535in}{0.899931in}}%
\pgfpathlineto{\pgfqpoint{5.361594in}{0.870798in}}%
\pgfpathlineto{\pgfqpoint{5.363653in}{0.890220in}}%
\pgfpathlineto{\pgfqpoint{5.371889in}{0.870798in}}%
\pgfpathlineto{\pgfqpoint{5.378066in}{0.870798in}}%
\pgfpathlineto{\pgfqpoint{5.384242in}{0.880509in}}%
\pgfpathlineto{\pgfqpoint{5.386301in}{0.880509in}}%
\pgfpathlineto{\pgfqpoint{5.390419in}{0.899931in}}%
\pgfpathlineto{\pgfqpoint{5.392478in}{0.899931in}}%
\pgfpathlineto{\pgfqpoint{5.398655in}{0.870798in}}%
\pgfpathlineto{\pgfqpoint{5.400714in}{0.880509in}}%
\pgfpathlineto{\pgfqpoint{5.402773in}{0.880509in}}%
\pgfpathlineto{\pgfqpoint{5.404832in}{0.890220in}}%
\pgfpathlineto{\pgfqpoint{5.406891in}{0.880509in}}%
\pgfpathlineto{\pgfqpoint{5.413068in}{0.890220in}}%
\pgfpathlineto{\pgfqpoint{5.415127in}{0.899931in}}%
\pgfpathlineto{\pgfqpoint{5.417186in}{0.870798in}}%
\pgfpathlineto{\pgfqpoint{5.419244in}{0.870798in}}%
\pgfpathlineto{\pgfqpoint{5.421303in}{0.899931in}}%
\pgfpathlineto{\pgfqpoint{5.429539in}{0.948486in}}%
\pgfpathlineto{\pgfqpoint{5.433657in}{0.919353in}}%
\pgfpathlineto{\pgfqpoint{5.435716in}{0.919353in}}%
\pgfpathlineto{\pgfqpoint{5.441893in}{0.929064in}}%
\pgfpathlineto{\pgfqpoint{5.443952in}{0.909642in}}%
\pgfpathlineto{\pgfqpoint{5.448070in}{0.909642in}}%
\pgfpathlineto{\pgfqpoint{5.450129in}{0.899931in}}%
\pgfpathlineto{\pgfqpoint{5.456306in}{0.909642in}}%
\pgfpathlineto{\pgfqpoint{5.458364in}{0.909642in}}%
\pgfpathlineto{\pgfqpoint{5.460423in}{0.899931in}}%
\pgfpathlineto{\pgfqpoint{5.464541in}{0.890220in}}%
\pgfpathlineto{\pgfqpoint{5.470718in}{0.880509in}}%
\pgfpathlineto{\pgfqpoint{5.472777in}{0.909642in}}%
\pgfpathlineto{\pgfqpoint{5.474836in}{0.909642in}}%
\pgfpathlineto{\pgfqpoint{5.476895in}{0.899931in}}%
\pgfpathlineto{\pgfqpoint{5.478954in}{0.899931in}}%
\pgfpathlineto{\pgfqpoint{5.485131in}{0.890220in}}%
\pgfpathlineto{\pgfqpoint{5.487190in}{0.890220in}}%
\pgfpathlineto{\pgfqpoint{5.489249in}{0.899931in}}%
\pgfpathlineto{\pgfqpoint{5.491308in}{0.890220in}}%
\pgfpathlineto{\pgfqpoint{5.493367in}{0.870798in}}%
\pgfpathlineto{\pgfqpoint{5.499543in}{0.861087in}}%
\pgfpathlineto{\pgfqpoint{5.501602in}{0.870798in}}%
\pgfpathlineto{\pgfqpoint{5.503661in}{0.870798in}}%
\pgfpathlineto{\pgfqpoint{5.505720in}{0.880509in}}%
\pgfpathlineto{\pgfqpoint{5.507779in}{0.880509in}}%
\pgfpathlineto{\pgfqpoint{5.513956in}{0.870798in}}%
\pgfpathlineto{\pgfqpoint{5.516015in}{0.861087in}}%
\pgfpathlineto{\pgfqpoint{5.518074in}{0.870798in}}%
\pgfpathlineto{\pgfqpoint{5.520133in}{0.861087in}}%
\pgfpathlineto{\pgfqpoint{5.534545in}{0.861087in}}%
\pgfpathlineto{\pgfqpoint{5.534545in}{0.861087in}}%
\pgfusepath{stroke}%
\end{pgfscope}%
\begin{pgfscope}%
\pgfpathrectangle{\pgfqpoint{0.800000in}{0.528000in}}{\pgfqpoint{4.960000in}{3.696000in}}%
\pgfusepath{clip}%
\pgfsetrectcap%
\pgfsetroundjoin%
\pgfsetlinewidth{1.003750pt}%
\definecolor{currentstroke}{rgb}{0.501961,0.501961,0.501961}%
\pgfsetstrokecolor{currentstroke}%
\pgfsetstrokeopacity{0.900000}%
\pgfsetdash{}{0pt}%
\pgfpathmoveto{\pgfqpoint{1.025455in}{2.259468in}}%
\pgfpathlineto{\pgfqpoint{1.031631in}{2.220624in}}%
\pgfpathlineto{\pgfqpoint{1.033690in}{2.152647in}}%
\pgfpathlineto{\pgfqpoint{1.035749in}{2.123514in}}%
\pgfpathlineto{\pgfqpoint{1.037808in}{2.152647in}}%
\pgfpathlineto{\pgfqpoint{1.039867in}{2.104092in}}%
\pgfpathlineto{\pgfqpoint{1.048103in}{2.026405in}}%
\pgfpathlineto{\pgfqpoint{1.050162in}{1.987561in}}%
\pgfpathlineto{\pgfqpoint{1.052221in}{1.880740in}}%
\pgfpathlineto{\pgfqpoint{1.054280in}{1.948717in}}%
\pgfpathlineto{\pgfqpoint{1.062516in}{1.968139in}}%
\pgfpathlineto{\pgfqpoint{1.066633in}{2.045827in}}%
\pgfpathlineto{\pgfqpoint{1.068692in}{1.987561in}}%
\pgfpathlineto{\pgfqpoint{1.074869in}{2.016694in}}%
\pgfpathlineto{\pgfqpoint{1.076928in}{1.997272in}}%
\pgfpathlineto{\pgfqpoint{1.078987in}{1.909873in}}%
\pgfpathlineto{\pgfqpoint{1.081046in}{1.939006in}}%
\pgfpathlineto{\pgfqpoint{1.083105in}{1.841896in}}%
\pgfpathlineto{\pgfqpoint{1.089282in}{1.851607in}}%
\pgfpathlineto{\pgfqpoint{1.091341in}{1.939006in}}%
\pgfpathlineto{\pgfqpoint{1.095459in}{1.958428in}}%
\pgfpathlineto{\pgfqpoint{1.097518in}{2.133225in}}%
\pgfpathlineto{\pgfqpoint{1.103694in}{2.142936in}}%
\pgfpathlineto{\pgfqpoint{1.105753in}{2.172069in}}%
\pgfpathlineto{\pgfqpoint{1.107812in}{2.181780in}}%
\pgfpathlineto{\pgfqpoint{1.109871in}{2.152647in}}%
\pgfpathlineto{\pgfqpoint{1.111930in}{2.181780in}}%
\pgfpathlineto{\pgfqpoint{1.120166in}{2.269179in}}%
\pgfpathlineto{\pgfqpoint{1.122225in}{2.172069in}}%
\pgfpathlineto{\pgfqpoint{1.126343in}{2.259468in}}%
\pgfpathlineto{\pgfqpoint{1.132520in}{2.210913in}}%
\pgfpathlineto{\pgfqpoint{1.134579in}{2.123514in}}%
\pgfpathlineto{\pgfqpoint{1.136638in}{2.123514in}}%
\pgfpathlineto{\pgfqpoint{1.138697in}{2.191491in}}%
\pgfpathlineto{\pgfqpoint{1.140756in}{2.152647in}}%
\pgfpathlineto{\pgfqpoint{1.146932in}{2.220624in}}%
\pgfpathlineto{\pgfqpoint{1.148991in}{2.259468in}}%
\pgfpathlineto{\pgfqpoint{1.151050in}{2.249757in}}%
\pgfpathlineto{\pgfqpoint{1.153109in}{2.220624in}}%
\pgfpathlineto{\pgfqpoint{1.155168in}{2.346867in}}%
\pgfpathlineto{\pgfqpoint{1.161345in}{2.308023in}}%
\pgfpathlineto{\pgfqpoint{1.165463in}{2.249757in}}%
\pgfpathlineto{\pgfqpoint{1.167522in}{2.240046in}}%
\pgfpathlineto{\pgfqpoint{1.169581in}{2.249757in}}%
\pgfpathlineto{\pgfqpoint{1.177817in}{2.210913in}}%
\pgfpathlineto{\pgfqpoint{1.179875in}{2.065249in}}%
\pgfpathlineto{\pgfqpoint{1.181934in}{2.133225in}}%
\pgfpathlineto{\pgfqpoint{1.183993in}{2.074960in}}%
\pgfpathlineto{\pgfqpoint{1.190170in}{2.065249in}}%
\pgfpathlineto{\pgfqpoint{1.192229in}{2.026405in}}%
\pgfpathlineto{\pgfqpoint{1.196347in}{2.123514in}}%
\pgfpathlineto{\pgfqpoint{1.198406in}{2.074960in}}%
\pgfpathlineto{\pgfqpoint{1.204583in}{2.065249in}}%
\pgfpathlineto{\pgfqpoint{1.208701in}{1.977850in}}%
\pgfpathlineto{\pgfqpoint{1.210760in}{2.006983in}}%
\pgfpathlineto{\pgfqpoint{1.212819in}{1.919584in}}%
\pgfpathlineto{\pgfqpoint{1.223113in}{2.006983in}}%
\pgfpathlineto{\pgfqpoint{1.225172in}{2.055538in}}%
\pgfpathlineto{\pgfqpoint{1.227231in}{2.065249in}}%
\pgfpathlineto{\pgfqpoint{1.233408in}{2.036116in}}%
\pgfpathlineto{\pgfqpoint{1.235467in}{1.997272in}}%
\pgfpathlineto{\pgfqpoint{1.237526in}{1.987561in}}%
\pgfpathlineto{\pgfqpoint{1.239585in}{1.968139in}}%
\pgfpathlineto{\pgfqpoint{1.241644in}{1.968139in}}%
\pgfpathlineto{\pgfqpoint{1.247821in}{1.987561in}}%
\pgfpathlineto{\pgfqpoint{1.249880in}{2.006983in}}%
\pgfpathlineto{\pgfqpoint{1.251939in}{2.065249in}}%
\pgfpathlineto{\pgfqpoint{1.256056in}{1.997272in}}%
\pgfpathlineto{\pgfqpoint{1.262233in}{2.016694in}}%
\pgfpathlineto{\pgfqpoint{1.266351in}{2.084671in}}%
\pgfpathlineto{\pgfqpoint{1.268410in}{2.084671in}}%
\pgfpathlineto{\pgfqpoint{1.270469in}{2.152647in}}%
\pgfpathlineto{\pgfqpoint{1.276646in}{2.162358in}}%
\pgfpathlineto{\pgfqpoint{1.280764in}{2.230335in}}%
\pgfpathlineto{\pgfqpoint{1.282823in}{2.201202in}}%
\pgfpathlineto{\pgfqpoint{1.284882in}{2.152647in}}%
\pgfpathlineto{\pgfqpoint{1.291059in}{2.240046in}}%
\pgfpathlineto{\pgfqpoint{1.295176in}{2.220624in}}%
\pgfpathlineto{\pgfqpoint{1.299294in}{2.113803in}}%
\pgfpathlineto{\pgfqpoint{1.305471in}{2.191491in}}%
\pgfpathlineto{\pgfqpoint{1.307530in}{2.249757in}}%
\pgfpathlineto{\pgfqpoint{1.311648in}{2.181780in}}%
\pgfpathlineto{\pgfqpoint{1.313707in}{2.220624in}}%
\pgfpathlineto{\pgfqpoint{1.324002in}{2.181780in}}%
\pgfpathlineto{\pgfqpoint{1.328120in}{2.142936in}}%
\pgfpathlineto{\pgfqpoint{1.334296in}{2.201202in}}%
\pgfpathlineto{\pgfqpoint{1.338414in}{2.337156in}}%
\pgfpathlineto{\pgfqpoint{1.340473in}{2.298312in}}%
\pgfpathlineto{\pgfqpoint{1.342532in}{2.395422in}}%
\pgfpathlineto{\pgfqpoint{1.348709in}{2.366289in}}%
\pgfpathlineto{\pgfqpoint{1.350768in}{2.385711in}}%
\pgfpathlineto{\pgfqpoint{1.352827in}{2.443977in}}%
\pgfpathlineto{\pgfqpoint{1.354886in}{2.385711in}}%
\pgfpathlineto{\pgfqpoint{1.356945in}{2.395422in}}%
\pgfpathlineto{\pgfqpoint{1.363122in}{2.356578in}}%
\pgfpathlineto{\pgfqpoint{1.365181in}{2.327445in}}%
\pgfpathlineto{\pgfqpoint{1.367240in}{2.278890in}}%
\pgfpathlineto{\pgfqpoint{1.369298in}{2.298312in}}%
\pgfpathlineto{\pgfqpoint{1.371357in}{2.240046in}}%
\pgfpathlineto{\pgfqpoint{1.379593in}{2.356578in}}%
\pgfpathlineto{\pgfqpoint{1.381652in}{2.337156in}}%
\pgfpathlineto{\pgfqpoint{1.383711in}{2.346867in}}%
\pgfpathlineto{\pgfqpoint{1.385770in}{2.395422in}}%
\pgfpathlineto{\pgfqpoint{1.391947in}{2.269179in}}%
\pgfpathlineto{\pgfqpoint{1.394006in}{2.278890in}}%
\pgfpathlineto{\pgfqpoint{1.396065in}{2.346867in}}%
\pgfpathlineto{\pgfqpoint{1.398124in}{2.288601in}}%
\pgfpathlineto{\pgfqpoint{1.408418in}{2.201202in}}%
\pgfpathlineto{\pgfqpoint{1.410477in}{2.152647in}}%
\pgfpathlineto{\pgfqpoint{1.414595in}{2.327445in}}%
\pgfpathlineto{\pgfqpoint{1.420772in}{2.356578in}}%
\pgfpathlineto{\pgfqpoint{1.424890in}{2.278890in}}%
\pgfpathlineto{\pgfqpoint{1.426949in}{2.308023in}}%
\pgfpathlineto{\pgfqpoint{1.429008in}{2.317734in}}%
\pgfpathlineto{\pgfqpoint{1.435185in}{2.366289in}}%
\pgfpathlineto{\pgfqpoint{1.437244in}{2.337156in}}%
\pgfpathlineto{\pgfqpoint{1.439303in}{2.337156in}}%
\pgfpathlineto{\pgfqpoint{1.441362in}{2.298312in}}%
\pgfpathlineto{\pgfqpoint{1.443421in}{2.288601in}}%
\pgfpathlineto{\pgfqpoint{1.449597in}{2.230335in}}%
\pgfpathlineto{\pgfqpoint{1.451656in}{2.259468in}}%
\pgfpathlineto{\pgfqpoint{1.453715in}{2.269179in}}%
\pgfpathlineto{\pgfqpoint{1.455774in}{2.269179in}}%
\pgfpathlineto{\pgfqpoint{1.457833in}{2.191491in}}%
\pgfpathlineto{\pgfqpoint{1.464010in}{2.172069in}}%
\pgfpathlineto{\pgfqpoint{1.468128in}{2.298312in}}%
\pgfpathlineto{\pgfqpoint{1.472246in}{2.240046in}}%
\pgfpathlineto{\pgfqpoint{1.478423in}{2.269179in}}%
\pgfpathlineto{\pgfqpoint{1.480482in}{2.181780in}}%
\pgfpathlineto{\pgfqpoint{1.482540in}{2.172069in}}%
\pgfpathlineto{\pgfqpoint{1.486658in}{2.259468in}}%
\pgfpathlineto{\pgfqpoint{1.492835in}{2.230335in}}%
\pgfpathlineto{\pgfqpoint{1.494894in}{2.249757in}}%
\pgfpathlineto{\pgfqpoint{1.496953in}{2.152647in}}%
\pgfpathlineto{\pgfqpoint{1.499012in}{2.152647in}}%
\pgfpathlineto{\pgfqpoint{1.501071in}{2.094382in}}%
\pgfpathlineto{\pgfqpoint{1.507248in}{2.045827in}}%
\pgfpathlineto{\pgfqpoint{1.509307in}{2.133225in}}%
\pgfpathlineto{\pgfqpoint{1.511366in}{2.142936in}}%
\pgfpathlineto{\pgfqpoint{1.513425in}{2.142936in}}%
\pgfpathlineto{\pgfqpoint{1.515484in}{2.172069in}}%
\pgfpathlineto{\pgfqpoint{1.521660in}{2.191491in}}%
\pgfpathlineto{\pgfqpoint{1.523719in}{2.142936in}}%
\pgfpathlineto{\pgfqpoint{1.525778in}{2.172069in}}%
\pgfpathlineto{\pgfqpoint{1.529896in}{2.123514in}}%
\pgfpathlineto{\pgfqpoint{1.538132in}{2.181780in}}%
\pgfpathlineto{\pgfqpoint{1.540191in}{2.181780in}}%
\pgfpathlineto{\pgfqpoint{1.542250in}{2.201202in}}%
\pgfpathlineto{\pgfqpoint{1.544309in}{2.172069in}}%
\pgfpathlineto{\pgfqpoint{1.550486in}{2.162358in}}%
\pgfpathlineto{\pgfqpoint{1.552545in}{2.259468in}}%
\pgfpathlineto{\pgfqpoint{1.554604in}{2.269179in}}%
\pgfpathlineto{\pgfqpoint{1.556663in}{2.152647in}}%
\pgfpathlineto{\pgfqpoint{1.558721in}{2.104092in}}%
\pgfpathlineto{\pgfqpoint{1.564898in}{2.162358in}}%
\pgfpathlineto{\pgfqpoint{1.566957in}{2.094382in}}%
\pgfpathlineto{\pgfqpoint{1.569016in}{2.123514in}}%
\pgfpathlineto{\pgfqpoint{1.571075in}{2.094382in}}%
\pgfpathlineto{\pgfqpoint{1.573134in}{2.133225in}}%
\pgfpathlineto{\pgfqpoint{1.579311in}{2.074960in}}%
\pgfpathlineto{\pgfqpoint{1.581370in}{2.026405in}}%
\pgfpathlineto{\pgfqpoint{1.585488in}{2.026405in}}%
\pgfpathlineto{\pgfqpoint{1.587547in}{1.948717in}}%
\pgfpathlineto{\pgfqpoint{1.593724in}{2.006983in}}%
\pgfpathlineto{\pgfqpoint{1.595782in}{1.997272in}}%
\pgfpathlineto{\pgfqpoint{1.599900in}{2.055538in}}%
\pgfpathlineto{\pgfqpoint{1.601959in}{2.065249in}}%
\pgfpathlineto{\pgfqpoint{1.610195in}{2.016694in}}%
\pgfpathlineto{\pgfqpoint{1.612254in}{1.948717in}}%
\pgfpathlineto{\pgfqpoint{1.614313in}{1.997272in}}%
\pgfpathlineto{\pgfqpoint{1.616372in}{2.016694in}}%
\pgfpathlineto{\pgfqpoint{1.622549in}{2.006983in}}%
\pgfpathlineto{\pgfqpoint{1.624608in}{2.055538in}}%
\pgfpathlineto{\pgfqpoint{1.626667in}{2.026405in}}%
\pgfpathlineto{\pgfqpoint{1.628726in}{2.016694in}}%
\pgfpathlineto{\pgfqpoint{1.630785in}{2.084671in}}%
\pgfpathlineto{\pgfqpoint{1.636961in}{2.065249in}}%
\pgfpathlineto{\pgfqpoint{1.639020in}{2.036116in}}%
\pgfpathlineto{\pgfqpoint{1.643138in}{2.181780in}}%
\pgfpathlineto{\pgfqpoint{1.645197in}{2.172069in}}%
\pgfpathlineto{\pgfqpoint{1.653433in}{2.240046in}}%
\pgfpathlineto{\pgfqpoint{1.655492in}{2.288601in}}%
\pgfpathlineto{\pgfqpoint{1.657551in}{2.298312in}}%
\pgfpathlineto{\pgfqpoint{1.659610in}{2.376000in}}%
\pgfpathlineto{\pgfqpoint{1.665787in}{2.395422in}}%
\pgfpathlineto{\pgfqpoint{1.667846in}{2.366289in}}%
\pgfpathlineto{\pgfqpoint{1.671963in}{2.376000in}}%
\pgfpathlineto{\pgfqpoint{1.674022in}{2.317734in}}%
\pgfpathlineto{\pgfqpoint{1.680199in}{2.308023in}}%
\pgfpathlineto{\pgfqpoint{1.682258in}{2.308023in}}%
\pgfpathlineto{\pgfqpoint{1.684317in}{2.337156in}}%
\pgfpathlineto{\pgfqpoint{1.686376in}{2.327445in}}%
\pgfpathlineto{\pgfqpoint{1.688435in}{2.346867in}}%
\pgfpathlineto{\pgfqpoint{1.694612in}{2.346867in}}%
\pgfpathlineto{\pgfqpoint{1.696671in}{2.308023in}}%
\pgfpathlineto{\pgfqpoint{1.698730in}{2.308023in}}%
\pgfpathlineto{\pgfqpoint{1.702848in}{2.288601in}}%
\pgfpathlineto{\pgfqpoint{1.709024in}{2.298312in}}%
\pgfpathlineto{\pgfqpoint{1.711083in}{2.240046in}}%
\pgfpathlineto{\pgfqpoint{1.713142in}{2.278890in}}%
\pgfpathlineto{\pgfqpoint{1.715201in}{2.385711in}}%
\pgfpathlineto{\pgfqpoint{1.717260in}{2.356578in}}%
\pgfpathlineto{\pgfqpoint{1.723437in}{2.317734in}}%
\pgfpathlineto{\pgfqpoint{1.725496in}{2.327445in}}%
\pgfpathlineto{\pgfqpoint{1.727555in}{2.288601in}}%
\pgfpathlineto{\pgfqpoint{1.729614in}{2.327445in}}%
\pgfpathlineto{\pgfqpoint{1.731673in}{2.210913in}}%
\pgfpathlineto{\pgfqpoint{1.739909in}{2.356578in}}%
\pgfpathlineto{\pgfqpoint{1.741968in}{2.395422in}}%
\pgfpathlineto{\pgfqpoint{1.744027in}{2.376000in}}%
\pgfpathlineto{\pgfqpoint{1.746086in}{2.317734in}}%
\pgfpathlineto{\pgfqpoint{1.752262in}{2.317734in}}%
\pgfpathlineto{\pgfqpoint{1.756380in}{2.385711in}}%
\pgfpathlineto{\pgfqpoint{1.758439in}{2.376000in}}%
\pgfpathlineto{\pgfqpoint{1.766675in}{2.376000in}}%
\pgfpathlineto{\pgfqpoint{1.768734in}{2.453688in}}%
\pgfpathlineto{\pgfqpoint{1.770793in}{2.443977in}}%
\pgfpathlineto{\pgfqpoint{1.772852in}{2.405133in}}%
\pgfpathlineto{\pgfqpoint{1.781088in}{2.376000in}}%
\pgfpathlineto{\pgfqpoint{1.783147in}{2.376000in}}%
\pgfpathlineto{\pgfqpoint{1.787264in}{2.259468in}}%
\pgfpathlineto{\pgfqpoint{1.789323in}{2.220624in}}%
\pgfpathlineto{\pgfqpoint{1.795500in}{2.230335in}}%
\pgfpathlineto{\pgfqpoint{1.799618in}{2.162358in}}%
\pgfpathlineto{\pgfqpoint{1.801677in}{2.172069in}}%
\pgfpathlineto{\pgfqpoint{1.803736in}{2.113803in}}%
\pgfpathlineto{\pgfqpoint{1.811972in}{2.142936in}}%
\pgfpathlineto{\pgfqpoint{1.814031in}{2.094382in}}%
\pgfpathlineto{\pgfqpoint{1.816090in}{2.094382in}}%
\pgfpathlineto{\pgfqpoint{1.818149in}{2.142936in}}%
\pgfpathlineto{\pgfqpoint{1.824325in}{2.123514in}}%
\pgfpathlineto{\pgfqpoint{1.830502in}{2.055538in}}%
\pgfpathlineto{\pgfqpoint{1.832561in}{1.987561in}}%
\pgfpathlineto{\pgfqpoint{1.838738in}{2.036116in}}%
\pgfpathlineto{\pgfqpoint{1.840797in}{1.939006in}}%
\pgfpathlineto{\pgfqpoint{1.842856in}{1.929295in}}%
\pgfpathlineto{\pgfqpoint{1.844915in}{1.909873in}}%
\pgfpathlineto{\pgfqpoint{1.846974in}{1.909873in}}%
\pgfpathlineto{\pgfqpoint{1.853151in}{1.822474in}}%
\pgfpathlineto{\pgfqpoint{1.855210in}{1.812763in}}%
\pgfpathlineto{\pgfqpoint{1.857269in}{1.812763in}}%
\pgfpathlineto{\pgfqpoint{1.859328in}{1.773919in}}%
\pgfpathlineto{\pgfqpoint{1.861386in}{1.861318in}}%
\pgfpathlineto{\pgfqpoint{1.869622in}{1.890451in}}%
\pgfpathlineto{\pgfqpoint{1.871681in}{1.919584in}}%
\pgfpathlineto{\pgfqpoint{1.873740in}{1.871029in}}%
\pgfpathlineto{\pgfqpoint{1.875799in}{1.900162in}}%
\pgfpathlineto{\pgfqpoint{1.881976in}{1.909873in}}%
\pgfpathlineto{\pgfqpoint{1.886094in}{1.871029in}}%
\pgfpathlineto{\pgfqpoint{1.888153in}{1.822474in}}%
\pgfpathlineto{\pgfqpoint{1.890212in}{1.890451in}}%
\pgfpathlineto{\pgfqpoint{1.896389in}{1.880740in}}%
\pgfpathlineto{\pgfqpoint{1.898447in}{1.968139in}}%
\pgfpathlineto{\pgfqpoint{1.900506in}{1.997272in}}%
\pgfpathlineto{\pgfqpoint{1.902565in}{1.987561in}}%
\pgfpathlineto{\pgfqpoint{1.904624in}{2.036116in}}%
\pgfpathlineto{\pgfqpoint{1.910801in}{2.074960in}}%
\pgfpathlineto{\pgfqpoint{1.912860in}{1.997272in}}%
\pgfpathlineto{\pgfqpoint{1.919037in}{2.142936in}}%
\pgfpathlineto{\pgfqpoint{1.925214in}{2.142936in}}%
\pgfpathlineto{\pgfqpoint{1.927273in}{2.152647in}}%
\pgfpathlineto{\pgfqpoint{1.929332in}{2.065249in}}%
\pgfpathlineto{\pgfqpoint{1.931391in}{2.045827in}}%
\pgfpathlineto{\pgfqpoint{1.933450in}{1.997272in}}%
\pgfpathlineto{\pgfqpoint{1.939626in}{2.036116in}}%
\pgfpathlineto{\pgfqpoint{1.941685in}{2.074960in}}%
\pgfpathlineto{\pgfqpoint{1.943744in}{2.026405in}}%
\pgfpathlineto{\pgfqpoint{1.945803in}{2.045827in}}%
\pgfpathlineto{\pgfqpoint{1.954039in}{2.026405in}}%
\pgfpathlineto{\pgfqpoint{1.956098in}{1.948717in}}%
\pgfpathlineto{\pgfqpoint{1.958157in}{1.919584in}}%
\pgfpathlineto{\pgfqpoint{1.960216in}{1.871029in}}%
\pgfpathlineto{\pgfqpoint{1.962275in}{1.900162in}}%
\pgfpathlineto{\pgfqpoint{1.968452in}{1.880740in}}%
\pgfpathlineto{\pgfqpoint{1.970511in}{1.832185in}}%
\pgfpathlineto{\pgfqpoint{1.972570in}{1.861318in}}%
\pgfpathlineto{\pgfqpoint{1.974628in}{1.803052in}}%
\pgfpathlineto{\pgfqpoint{1.976687in}{1.822474in}}%
\pgfpathlineto{\pgfqpoint{1.982864in}{1.822474in}}%
\pgfpathlineto{\pgfqpoint{1.984923in}{1.880740in}}%
\pgfpathlineto{\pgfqpoint{1.986982in}{1.880740in}}%
\pgfpathlineto{\pgfqpoint{1.989041in}{1.919584in}}%
\pgfpathlineto{\pgfqpoint{1.991100in}{1.880740in}}%
\pgfpathlineto{\pgfqpoint{1.997277in}{1.900162in}}%
\pgfpathlineto{\pgfqpoint{1.999336in}{1.919584in}}%
\pgfpathlineto{\pgfqpoint{2.003454in}{2.006983in}}%
\pgfpathlineto{\pgfqpoint{2.005513in}{2.026405in}}%
\pgfpathlineto{\pgfqpoint{2.011689in}{2.036116in}}%
\pgfpathlineto{\pgfqpoint{2.013748in}{2.055538in}}%
\pgfpathlineto{\pgfqpoint{2.017866in}{1.939006in}}%
\pgfpathlineto{\pgfqpoint{2.019925in}{1.939006in}}%
\pgfpathlineto{\pgfqpoint{2.026102in}{1.977850in}}%
\pgfpathlineto{\pgfqpoint{2.028161in}{1.909873in}}%
\pgfpathlineto{\pgfqpoint{2.032279in}{1.861318in}}%
\pgfpathlineto{\pgfqpoint{2.034338in}{1.890451in}}%
\pgfpathlineto{\pgfqpoint{2.040515in}{1.861318in}}%
\pgfpathlineto{\pgfqpoint{2.044633in}{1.861318in}}%
\pgfpathlineto{\pgfqpoint{2.046692in}{1.900162in}}%
\pgfpathlineto{\pgfqpoint{2.048751in}{1.880740in}}%
\pgfpathlineto{\pgfqpoint{2.054927in}{1.919584in}}%
\pgfpathlineto{\pgfqpoint{2.056986in}{1.948717in}}%
\pgfpathlineto{\pgfqpoint{2.059045in}{2.065249in}}%
\pgfpathlineto{\pgfqpoint{2.061104in}{2.036116in}}%
\pgfpathlineto{\pgfqpoint{2.069340in}{2.036116in}}%
\pgfpathlineto{\pgfqpoint{2.071399in}{2.065249in}}%
\pgfpathlineto{\pgfqpoint{2.073458in}{2.055538in}}%
\pgfpathlineto{\pgfqpoint{2.075517in}{2.006983in}}%
\pgfpathlineto{\pgfqpoint{2.077576in}{2.045827in}}%
\pgfpathlineto{\pgfqpoint{2.085812in}{2.026405in}}%
\pgfpathlineto{\pgfqpoint{2.087870in}{2.045827in}}%
\pgfpathlineto{\pgfqpoint{2.089929in}{2.016694in}}%
\pgfpathlineto{\pgfqpoint{2.091988in}{1.890451in}}%
\pgfpathlineto{\pgfqpoint{2.098165in}{1.909873in}}%
\pgfpathlineto{\pgfqpoint{2.100224in}{1.890451in}}%
\pgfpathlineto{\pgfqpoint{2.102283in}{1.890451in}}%
\pgfpathlineto{\pgfqpoint{2.104342in}{1.880740in}}%
\pgfpathlineto{\pgfqpoint{2.106401in}{1.832185in}}%
\pgfpathlineto{\pgfqpoint{2.112578in}{1.803052in}}%
\pgfpathlineto{\pgfqpoint{2.114637in}{1.812763in}}%
\pgfpathlineto{\pgfqpoint{2.116696in}{1.764208in}}%
\pgfpathlineto{\pgfqpoint{2.118755in}{1.764208in}}%
\pgfpathlineto{\pgfqpoint{2.120814in}{1.793341in}}%
\pgfpathlineto{\pgfqpoint{2.126990in}{1.832185in}}%
\pgfpathlineto{\pgfqpoint{2.129049in}{1.880740in}}%
\pgfpathlineto{\pgfqpoint{2.131108in}{1.861318in}}%
\pgfpathlineto{\pgfqpoint{2.133167in}{1.909873in}}%
\pgfpathlineto{\pgfqpoint{2.135226in}{1.744786in}}%
\pgfpathlineto{\pgfqpoint{2.141403in}{1.667098in}}%
\pgfpathlineto{\pgfqpoint{2.143462in}{1.667098in}}%
\pgfpathlineto{\pgfqpoint{2.145521in}{1.696231in}}%
\pgfpathlineto{\pgfqpoint{2.149639in}{1.667098in}}%
\pgfpathlineto{\pgfqpoint{2.157875in}{1.608832in}}%
\pgfpathlineto{\pgfqpoint{2.159934in}{1.618543in}}%
\pgfpathlineto{\pgfqpoint{2.161993in}{1.637965in}}%
\pgfpathlineto{\pgfqpoint{2.164051in}{1.618543in}}%
\pgfpathlineto{\pgfqpoint{2.170228in}{1.696231in}}%
\pgfpathlineto{\pgfqpoint{2.172287in}{1.764208in}}%
\pgfpathlineto{\pgfqpoint{2.174346in}{1.735075in}}%
\pgfpathlineto{\pgfqpoint{2.176405in}{1.764208in}}%
\pgfpathlineto{\pgfqpoint{2.178464in}{1.812763in}}%
\pgfpathlineto{\pgfqpoint{2.184641in}{1.803052in}}%
\pgfpathlineto{\pgfqpoint{2.186700in}{1.783630in}}%
\pgfpathlineto{\pgfqpoint{2.188759in}{1.812763in}}%
\pgfpathlineto{\pgfqpoint{2.190818in}{1.773919in}}%
\pgfpathlineto{\pgfqpoint{2.192877in}{1.793341in}}%
\pgfpathlineto{\pgfqpoint{2.199054in}{1.812763in}}%
\pgfpathlineto{\pgfqpoint{2.201112in}{1.812763in}}%
\pgfpathlineto{\pgfqpoint{2.203171in}{1.764208in}}%
\pgfpathlineto{\pgfqpoint{2.205230in}{1.754497in}}%
\pgfpathlineto{\pgfqpoint{2.207289in}{1.696231in}}%
\pgfpathlineto{\pgfqpoint{2.215525in}{1.735075in}}%
\pgfpathlineto{\pgfqpoint{2.217584in}{1.735075in}}%
\pgfpathlineto{\pgfqpoint{2.219643in}{1.696231in}}%
\pgfpathlineto{\pgfqpoint{2.221702in}{1.793341in}}%
\pgfpathlineto{\pgfqpoint{2.227879in}{1.812763in}}%
\pgfpathlineto{\pgfqpoint{2.231997in}{1.735075in}}%
\pgfpathlineto{\pgfqpoint{2.234056in}{1.822474in}}%
\pgfpathlineto{\pgfqpoint{2.236115in}{1.764208in}}%
\pgfpathlineto{\pgfqpoint{2.244350in}{1.822474in}}%
\pgfpathlineto{\pgfqpoint{2.246409in}{1.812763in}}%
\pgfpathlineto{\pgfqpoint{2.248468in}{1.783630in}}%
\pgfpathlineto{\pgfqpoint{2.250527in}{1.832185in}}%
\pgfpathlineto{\pgfqpoint{2.256704in}{1.812763in}}%
\pgfpathlineto{\pgfqpoint{2.258763in}{1.812763in}}%
\pgfpathlineto{\pgfqpoint{2.260822in}{1.793341in}}%
\pgfpathlineto{\pgfqpoint{2.262881in}{1.822474in}}%
\pgfpathlineto{\pgfqpoint{2.264940in}{1.890451in}}%
\pgfpathlineto{\pgfqpoint{2.271117in}{1.841896in}}%
\pgfpathlineto{\pgfqpoint{2.273176in}{1.841896in}}%
\pgfpathlineto{\pgfqpoint{2.275235in}{1.851607in}}%
\pgfpathlineto{\pgfqpoint{2.277293in}{1.841896in}}%
\pgfpathlineto{\pgfqpoint{2.279352in}{1.861318in}}%
\pgfpathlineto{\pgfqpoint{2.289647in}{1.783630in}}%
\pgfpathlineto{\pgfqpoint{2.293765in}{1.890451in}}%
\pgfpathlineto{\pgfqpoint{2.299942in}{1.871029in}}%
\pgfpathlineto{\pgfqpoint{2.302001in}{1.919584in}}%
\pgfpathlineto{\pgfqpoint{2.304060in}{1.871029in}}%
\pgfpathlineto{\pgfqpoint{2.306119in}{1.861318in}}%
\pgfpathlineto{\pgfqpoint{2.308178in}{1.871029in}}%
\pgfpathlineto{\pgfqpoint{2.314355in}{1.880740in}}%
\pgfpathlineto{\pgfqpoint{2.316413in}{1.861318in}}%
\pgfpathlineto{\pgfqpoint{2.318472in}{1.861318in}}%
\pgfpathlineto{\pgfqpoint{2.322590in}{1.822474in}}%
\pgfpathlineto{\pgfqpoint{2.330826in}{1.783630in}}%
\pgfpathlineto{\pgfqpoint{2.332885in}{1.793341in}}%
\pgfpathlineto{\pgfqpoint{2.334944in}{1.783630in}}%
\pgfpathlineto{\pgfqpoint{2.337003in}{1.803052in}}%
\pgfpathlineto{\pgfqpoint{2.343180in}{1.841896in}}%
\pgfpathlineto{\pgfqpoint{2.349357in}{1.939006in}}%
\pgfpathlineto{\pgfqpoint{2.351416in}{1.919584in}}%
\pgfpathlineto{\pgfqpoint{2.361710in}{1.968139in}}%
\pgfpathlineto{\pgfqpoint{2.363769in}{1.929295in}}%
\pgfpathlineto{\pgfqpoint{2.365828in}{1.939006in}}%
\pgfpathlineto{\pgfqpoint{2.372005in}{1.919584in}}%
\pgfpathlineto{\pgfqpoint{2.374064in}{1.900162in}}%
\pgfpathlineto{\pgfqpoint{2.376123in}{1.900162in}}%
\pgfpathlineto{\pgfqpoint{2.378182in}{1.919584in}}%
\pgfpathlineto{\pgfqpoint{2.380241in}{1.909873in}}%
\pgfpathlineto{\pgfqpoint{2.386418in}{1.929295in}}%
\pgfpathlineto{\pgfqpoint{2.392594in}{1.987561in}}%
\pgfpathlineto{\pgfqpoint{2.394653in}{1.987561in}}%
\pgfpathlineto{\pgfqpoint{2.402889in}{1.958428in}}%
\pgfpathlineto{\pgfqpoint{2.404948in}{1.919584in}}%
\pgfpathlineto{\pgfqpoint{2.407007in}{1.919584in}}%
\pgfpathlineto{\pgfqpoint{2.409066in}{1.900162in}}%
\pgfpathlineto{\pgfqpoint{2.415243in}{1.948717in}}%
\pgfpathlineto{\pgfqpoint{2.417302in}{1.997272in}}%
\pgfpathlineto{\pgfqpoint{2.421420in}{2.210913in}}%
\pgfpathlineto{\pgfqpoint{2.431714in}{2.327445in}}%
\pgfpathlineto{\pgfqpoint{2.433773in}{2.327445in}}%
\pgfpathlineto{\pgfqpoint{2.437891in}{2.443977in}}%
\pgfpathlineto{\pgfqpoint{2.444068in}{2.434266in}}%
\pgfpathlineto{\pgfqpoint{2.446127in}{2.414844in}}%
\pgfpathlineto{\pgfqpoint{2.448186in}{2.473110in}}%
\pgfpathlineto{\pgfqpoint{2.452304in}{2.473110in}}%
\pgfpathlineto{\pgfqpoint{2.458481in}{2.443977in}}%
\pgfpathlineto{\pgfqpoint{2.460540in}{2.424555in}}%
\pgfpathlineto{\pgfqpoint{2.464658in}{2.541087in}}%
\pgfpathlineto{\pgfqpoint{2.466716in}{2.482821in}}%
\pgfpathlineto{\pgfqpoint{2.474952in}{2.482821in}}%
\pgfpathlineto{\pgfqpoint{2.477011in}{2.443977in}}%
\pgfpathlineto{\pgfqpoint{2.479070in}{2.473110in}}%
\pgfpathlineto{\pgfqpoint{2.481129in}{2.531376in}}%
\pgfpathlineto{\pgfqpoint{2.487306in}{2.541087in}}%
\pgfpathlineto{\pgfqpoint{2.489365in}{2.560509in}}%
\pgfpathlineto{\pgfqpoint{2.493483in}{2.735306in}}%
\pgfpathlineto{\pgfqpoint{2.495542in}{2.706173in}}%
\pgfpathlineto{\pgfqpoint{2.501719in}{2.667329in}}%
\pgfpathlineto{\pgfqpoint{2.503778in}{2.696462in}}%
\pgfpathlineto{\pgfqpoint{2.505836in}{2.677040in}}%
\pgfpathlineto{\pgfqpoint{2.509954in}{2.677040in}}%
\pgfpathlineto{\pgfqpoint{2.518190in}{2.706173in}}%
\pgfpathlineto{\pgfqpoint{2.524367in}{2.570220in}}%
\pgfpathlineto{\pgfqpoint{2.532603in}{2.579931in}}%
\pgfpathlineto{\pgfqpoint{2.534662in}{2.579931in}}%
\pgfpathlineto{\pgfqpoint{2.536721in}{2.502243in}}%
\pgfpathlineto{\pgfqpoint{2.538780in}{2.560509in}}%
\pgfpathlineto{\pgfqpoint{2.544956in}{2.531376in}}%
\pgfpathlineto{\pgfqpoint{2.549074in}{2.531376in}}%
\pgfpathlineto{\pgfqpoint{2.551133in}{2.511954in}}%
\pgfpathlineto{\pgfqpoint{2.553192in}{2.541087in}}%
\pgfpathlineto{\pgfqpoint{2.561428in}{2.482821in}}%
\pgfpathlineto{\pgfqpoint{2.563487in}{2.570220in}}%
\pgfpathlineto{\pgfqpoint{2.565546in}{2.609064in}}%
\pgfpathlineto{\pgfqpoint{2.573782in}{2.521665in}}%
\pgfpathlineto{\pgfqpoint{2.577900in}{2.628486in}}%
\pgfpathlineto{\pgfqpoint{2.579958in}{2.589642in}}%
\pgfpathlineto{\pgfqpoint{2.582017in}{2.579931in}}%
\pgfpathlineto{\pgfqpoint{2.588194in}{2.579931in}}%
\pgfpathlineto{\pgfqpoint{2.590253in}{2.541087in}}%
\pgfpathlineto{\pgfqpoint{2.592312in}{2.570220in}}%
\pgfpathlineto{\pgfqpoint{2.594371in}{2.560509in}}%
\pgfpathlineto{\pgfqpoint{2.596430in}{2.570220in}}%
\pgfpathlineto{\pgfqpoint{2.602607in}{2.502243in}}%
\pgfpathlineto{\pgfqpoint{2.604666in}{2.492532in}}%
\pgfpathlineto{\pgfqpoint{2.606725in}{2.453688in}}%
\pgfpathlineto{\pgfqpoint{2.608784in}{2.521665in}}%
\pgfpathlineto{\pgfqpoint{2.617020in}{2.560509in}}%
\pgfpathlineto{\pgfqpoint{2.621137in}{2.647908in}}%
\pgfpathlineto{\pgfqpoint{2.625255in}{2.560509in}}%
\pgfpathlineto{\pgfqpoint{2.633491in}{2.570220in}}%
\pgfpathlineto{\pgfqpoint{2.635550in}{2.560509in}}%
\pgfpathlineto{\pgfqpoint{2.639668in}{2.443977in}}%
\pgfpathlineto{\pgfqpoint{2.647904in}{2.531376in}}%
\pgfpathlineto{\pgfqpoint{2.649963in}{2.628486in}}%
\pgfpathlineto{\pgfqpoint{2.652022in}{2.667329in}}%
\pgfpathlineto{\pgfqpoint{2.654081in}{2.657618in}}%
\pgfpathlineto{\pgfqpoint{2.660257in}{2.657618in}}%
\pgfpathlineto{\pgfqpoint{2.664375in}{2.715884in}}%
\pgfpathlineto{\pgfqpoint{2.666434in}{2.764439in}}%
\pgfpathlineto{\pgfqpoint{2.668493in}{2.745017in}}%
\pgfpathlineto{\pgfqpoint{2.674670in}{2.774150in}}%
\pgfpathlineto{\pgfqpoint{2.676729in}{2.764439in}}%
\pgfpathlineto{\pgfqpoint{2.678788in}{2.657618in}}%
\pgfpathlineto{\pgfqpoint{2.680847in}{2.686751in}}%
\pgfpathlineto{\pgfqpoint{2.682906in}{2.667329in}}%
\pgfpathlineto{\pgfqpoint{2.689083in}{2.638197in}}%
\pgfpathlineto{\pgfqpoint{2.691142in}{2.599353in}}%
\pgfpathlineto{\pgfqpoint{2.693200in}{2.589642in}}%
\pgfpathlineto{\pgfqpoint{2.695259in}{2.589642in}}%
\pgfpathlineto{\pgfqpoint{2.697318in}{2.570220in}}%
\pgfpathlineto{\pgfqpoint{2.703495in}{2.570220in}}%
\pgfpathlineto{\pgfqpoint{2.705554in}{2.609064in}}%
\pgfpathlineto{\pgfqpoint{2.707613in}{2.570220in}}%
\pgfpathlineto{\pgfqpoint{2.709672in}{2.599353in}}%
\pgfpathlineto{\pgfqpoint{2.711731in}{2.570220in}}%
\pgfpathlineto{\pgfqpoint{2.717908in}{2.521665in}}%
\pgfpathlineto{\pgfqpoint{2.719967in}{2.521665in}}%
\pgfpathlineto{\pgfqpoint{2.722026in}{2.492532in}}%
\pgfpathlineto{\pgfqpoint{2.724085in}{2.511954in}}%
\pgfpathlineto{\pgfqpoint{2.726144in}{2.560509in}}%
\pgfpathlineto{\pgfqpoint{2.732320in}{2.550798in}}%
\pgfpathlineto{\pgfqpoint{2.734379in}{2.482821in}}%
\pgfpathlineto{\pgfqpoint{2.738497in}{2.414844in}}%
\pgfpathlineto{\pgfqpoint{2.746733in}{2.434266in}}%
\pgfpathlineto{\pgfqpoint{2.748792in}{2.356578in}}%
\pgfpathlineto{\pgfqpoint{2.752910in}{2.424555in}}%
\pgfpathlineto{\pgfqpoint{2.754969in}{2.414844in}}%
\pgfpathlineto{\pgfqpoint{2.761146in}{2.453688in}}%
\pgfpathlineto{\pgfqpoint{2.763205in}{2.511954in}}%
\pgfpathlineto{\pgfqpoint{2.767323in}{2.453688in}}%
\pgfpathlineto{\pgfqpoint{2.769381in}{2.453688in}}%
\pgfpathlineto{\pgfqpoint{2.775558in}{2.482821in}}%
\pgfpathlineto{\pgfqpoint{2.777617in}{2.453688in}}%
\pgfpathlineto{\pgfqpoint{2.779676in}{2.502243in}}%
\pgfpathlineto{\pgfqpoint{2.783794in}{2.531376in}}%
\pgfpathlineto{\pgfqpoint{2.789971in}{2.550798in}}%
\pgfpathlineto{\pgfqpoint{2.792030in}{2.579931in}}%
\pgfpathlineto{\pgfqpoint{2.794089in}{2.579931in}}%
\pgfpathlineto{\pgfqpoint{2.796148in}{2.570220in}}%
\pgfpathlineto{\pgfqpoint{2.798207in}{2.492532in}}%
\pgfpathlineto{\pgfqpoint{2.804384in}{2.502243in}}%
\pgfpathlineto{\pgfqpoint{2.806443in}{2.502243in}}%
\pgfpathlineto{\pgfqpoint{2.808501in}{2.405133in}}%
\pgfpathlineto{\pgfqpoint{2.812619in}{2.434266in}}%
\pgfpathlineto{\pgfqpoint{2.818796in}{2.443977in}}%
\pgfpathlineto{\pgfqpoint{2.820855in}{2.482821in}}%
\pgfpathlineto{\pgfqpoint{2.822914in}{2.434266in}}%
\pgfpathlineto{\pgfqpoint{2.824973in}{2.424555in}}%
\pgfpathlineto{\pgfqpoint{2.827032in}{2.434266in}}%
\pgfpathlineto{\pgfqpoint{2.837327in}{2.395422in}}%
\pgfpathlineto{\pgfqpoint{2.839386in}{2.405133in}}%
\pgfpathlineto{\pgfqpoint{2.841445in}{2.356578in}}%
\pgfpathlineto{\pgfqpoint{2.847621in}{2.385711in}}%
\pgfpathlineto{\pgfqpoint{2.849680in}{2.356578in}}%
\pgfpathlineto{\pgfqpoint{2.851739in}{2.385711in}}%
\pgfpathlineto{\pgfqpoint{2.853798in}{2.395422in}}%
\pgfpathlineto{\pgfqpoint{2.855857in}{2.414844in}}%
\pgfpathlineto{\pgfqpoint{2.862034in}{2.424555in}}%
\pgfpathlineto{\pgfqpoint{2.864093in}{2.434266in}}%
\pgfpathlineto{\pgfqpoint{2.866152in}{2.385711in}}%
\pgfpathlineto{\pgfqpoint{2.868211in}{2.405133in}}%
\pgfpathlineto{\pgfqpoint{2.870270in}{2.395422in}}%
\pgfpathlineto{\pgfqpoint{2.876447in}{2.443977in}}%
\pgfpathlineto{\pgfqpoint{2.878506in}{2.414844in}}%
\pgfpathlineto{\pgfqpoint{2.880565in}{2.424555in}}%
\pgfpathlineto{\pgfqpoint{2.882623in}{2.405133in}}%
\pgfpathlineto{\pgfqpoint{2.884682in}{2.414844in}}%
\pgfpathlineto{\pgfqpoint{2.890859in}{2.414844in}}%
\pgfpathlineto{\pgfqpoint{2.892918in}{2.473110in}}%
\pgfpathlineto{\pgfqpoint{2.894977in}{2.453688in}}%
\pgfpathlineto{\pgfqpoint{2.899095in}{2.531376in}}%
\pgfpathlineto{\pgfqpoint{2.905272in}{2.570220in}}%
\pgfpathlineto{\pgfqpoint{2.909390in}{2.560509in}}%
\pgfpathlineto{\pgfqpoint{2.913508in}{2.589642in}}%
\pgfpathlineto{\pgfqpoint{2.921743in}{2.560509in}}%
\pgfpathlineto{\pgfqpoint{2.923802in}{2.521665in}}%
\pgfpathlineto{\pgfqpoint{2.925861in}{2.531376in}}%
\pgfpathlineto{\pgfqpoint{2.927920in}{2.511954in}}%
\pgfpathlineto{\pgfqpoint{2.934097in}{2.502243in}}%
\pgfpathlineto{\pgfqpoint{2.936156in}{2.463399in}}%
\pgfpathlineto{\pgfqpoint{2.938215in}{2.473110in}}%
\pgfpathlineto{\pgfqpoint{2.942333in}{2.453688in}}%
\pgfpathlineto{\pgfqpoint{2.948510in}{2.473110in}}%
\pgfpathlineto{\pgfqpoint{2.950569in}{2.541087in}}%
\pgfpathlineto{\pgfqpoint{2.952628in}{2.473110in}}%
\pgfpathlineto{\pgfqpoint{2.954687in}{2.482821in}}%
\pgfpathlineto{\pgfqpoint{2.956746in}{2.473110in}}%
\pgfpathlineto{\pgfqpoint{2.962922in}{2.482821in}}%
\pgfpathlineto{\pgfqpoint{2.964981in}{2.443977in}}%
\pgfpathlineto{\pgfqpoint{2.967040in}{2.463399in}}%
\pgfpathlineto{\pgfqpoint{2.969099in}{2.434266in}}%
\pgfpathlineto{\pgfqpoint{2.971158in}{2.463399in}}%
\pgfpathlineto{\pgfqpoint{2.977335in}{2.453688in}}%
\pgfpathlineto{\pgfqpoint{2.979394in}{2.482821in}}%
\pgfpathlineto{\pgfqpoint{2.985571in}{2.385711in}}%
\pgfpathlineto{\pgfqpoint{2.991748in}{2.414844in}}%
\pgfpathlineto{\pgfqpoint{2.993807in}{2.473110in}}%
\pgfpathlineto{\pgfqpoint{2.997924in}{2.405133in}}%
\pgfpathlineto{\pgfqpoint{2.999983in}{2.414844in}}%
\pgfpathlineto{\pgfqpoint{3.006160in}{2.405133in}}%
\pgfpathlineto{\pgfqpoint{3.008219in}{2.443977in}}%
\pgfpathlineto{\pgfqpoint{3.010278in}{2.405133in}}%
\pgfpathlineto{\pgfqpoint{3.012337in}{2.424555in}}%
\pgfpathlineto{\pgfqpoint{3.020573in}{2.385711in}}%
\pgfpathlineto{\pgfqpoint{3.022632in}{2.346867in}}%
\pgfpathlineto{\pgfqpoint{3.024691in}{2.366289in}}%
\pgfpathlineto{\pgfqpoint{3.026750in}{2.346867in}}%
\pgfpathlineto{\pgfqpoint{3.028809in}{2.376000in}}%
\pgfpathlineto{\pgfqpoint{3.037044in}{2.298312in}}%
\pgfpathlineto{\pgfqpoint{3.039103in}{2.337156in}}%
\pgfpathlineto{\pgfqpoint{3.041162in}{2.278890in}}%
\pgfpathlineto{\pgfqpoint{3.043221in}{2.288601in}}%
\pgfpathlineto{\pgfqpoint{3.049398in}{2.356578in}}%
\pgfpathlineto{\pgfqpoint{3.053516in}{2.424555in}}%
\pgfpathlineto{\pgfqpoint{3.055575in}{2.434266in}}%
\pgfpathlineto{\pgfqpoint{3.057634in}{2.453688in}}%
\pgfpathlineto{\pgfqpoint{3.065870in}{2.482821in}}%
\pgfpathlineto{\pgfqpoint{3.067929in}{2.531376in}}%
\pgfpathlineto{\pgfqpoint{3.069988in}{2.531376in}}%
\pgfpathlineto{\pgfqpoint{3.078223in}{2.492532in}}%
\pgfpathlineto{\pgfqpoint{3.080282in}{2.511954in}}%
\pgfpathlineto{\pgfqpoint{3.082341in}{2.550798in}}%
\pgfpathlineto{\pgfqpoint{3.084400in}{2.531376in}}%
\pgfpathlineto{\pgfqpoint{3.086459in}{2.560509in}}%
\pgfpathlineto{\pgfqpoint{3.092636in}{2.579931in}}%
\pgfpathlineto{\pgfqpoint{3.094695in}{2.560509in}}%
\pgfpathlineto{\pgfqpoint{3.096754in}{2.560509in}}%
\pgfpathlineto{\pgfqpoint{3.100872in}{2.609064in}}%
\pgfpathlineto{\pgfqpoint{3.109108in}{2.589642in}}%
\pgfpathlineto{\pgfqpoint{3.113225in}{2.589642in}}%
\pgfpathlineto{\pgfqpoint{3.115284in}{2.550798in}}%
\pgfpathlineto{\pgfqpoint{3.123520in}{2.609064in}}%
\pgfpathlineto{\pgfqpoint{3.125579in}{2.628486in}}%
\pgfpathlineto{\pgfqpoint{3.127638in}{2.618775in}}%
\pgfpathlineto{\pgfqpoint{3.129697in}{2.667329in}}%
\pgfpathlineto{\pgfqpoint{3.135874in}{2.647908in}}%
\pgfpathlineto{\pgfqpoint{3.137933in}{2.686751in}}%
\pgfpathlineto{\pgfqpoint{3.142051in}{2.706173in}}%
\pgfpathlineto{\pgfqpoint{3.144110in}{2.667329in}}%
\pgfpathlineto{\pgfqpoint{3.150286in}{2.638197in}}%
\pgfpathlineto{\pgfqpoint{3.152345in}{2.647908in}}%
\pgfpathlineto{\pgfqpoint{3.154404in}{2.647908in}}%
\pgfpathlineto{\pgfqpoint{3.158522in}{2.628486in}}%
\pgfpathlineto{\pgfqpoint{3.166758in}{2.628486in}}%
\pgfpathlineto{\pgfqpoint{3.168817in}{2.647908in}}%
\pgfpathlineto{\pgfqpoint{3.170876in}{2.647908in}}%
\pgfpathlineto{\pgfqpoint{3.172935in}{2.696462in}}%
\pgfpathlineto{\pgfqpoint{3.179112in}{2.715884in}}%
\pgfpathlineto{\pgfqpoint{3.183230in}{2.677040in}}%
\pgfpathlineto{\pgfqpoint{3.185289in}{2.706173in}}%
\pgfpathlineto{\pgfqpoint{3.187347in}{2.696462in}}%
\pgfpathlineto{\pgfqpoint{3.193524in}{2.725595in}}%
\pgfpathlineto{\pgfqpoint{3.195583in}{2.745017in}}%
\pgfpathlineto{\pgfqpoint{3.197642in}{2.686751in}}%
\pgfpathlineto{\pgfqpoint{3.201760in}{2.706173in}}%
\pgfpathlineto{\pgfqpoint{3.207937in}{2.696462in}}%
\pgfpathlineto{\pgfqpoint{3.209996in}{2.706173in}}%
\pgfpathlineto{\pgfqpoint{3.212055in}{2.725595in}}%
\pgfpathlineto{\pgfqpoint{3.214114in}{2.774150in}}%
\pgfpathlineto{\pgfqpoint{3.216173in}{2.764439in}}%
\pgfpathlineto{\pgfqpoint{3.222350in}{2.783861in}}%
\pgfpathlineto{\pgfqpoint{3.224408in}{2.783861in}}%
\pgfpathlineto{\pgfqpoint{3.226467in}{2.745017in}}%
\pgfpathlineto{\pgfqpoint{3.228526in}{2.774150in}}%
\pgfpathlineto{\pgfqpoint{3.230585in}{2.774150in}}%
\pgfpathlineto{\pgfqpoint{3.236762in}{2.793572in}}%
\pgfpathlineto{\pgfqpoint{3.238821in}{2.812994in}}%
\pgfpathlineto{\pgfqpoint{3.240880in}{2.754728in}}%
\pgfpathlineto{\pgfqpoint{3.244998in}{2.793572in}}%
\pgfpathlineto{\pgfqpoint{3.251175in}{2.803283in}}%
\pgfpathlineto{\pgfqpoint{3.253234in}{2.861549in}}%
\pgfpathlineto{\pgfqpoint{3.255293in}{2.871260in}}%
\pgfpathlineto{\pgfqpoint{3.257352in}{2.890682in}}%
\pgfpathlineto{\pgfqpoint{3.259411in}{2.890682in}}%
\pgfpathlineto{\pgfqpoint{3.267646in}{2.880971in}}%
\pgfpathlineto{\pgfqpoint{3.269705in}{2.851838in}}%
\pgfpathlineto{\pgfqpoint{3.271764in}{2.861549in}}%
\pgfpathlineto{\pgfqpoint{3.273823in}{2.832416in}}%
\pgfpathlineto{\pgfqpoint{3.282059in}{2.880971in}}%
\pgfpathlineto{\pgfqpoint{3.284118in}{2.880971in}}%
\pgfpathlineto{\pgfqpoint{3.288236in}{2.919815in}}%
\pgfpathlineto{\pgfqpoint{3.294413in}{2.919815in}}%
\pgfpathlineto{\pgfqpoint{3.296472in}{2.958659in}}%
\pgfpathlineto{\pgfqpoint{3.298531in}{2.948948in}}%
\pgfpathlineto{\pgfqpoint{3.300589in}{2.948948in}}%
\pgfpathlineto{\pgfqpoint{3.302648in}{2.978081in}}%
\pgfpathlineto{\pgfqpoint{3.310884in}{2.987792in}}%
\pgfpathlineto{\pgfqpoint{3.317061in}{3.075191in}}%
\pgfpathlineto{\pgfqpoint{3.323238in}{3.084902in}}%
\pgfpathlineto{\pgfqpoint{3.325297in}{3.055769in}}%
\pgfpathlineto{\pgfqpoint{3.327356in}{3.055769in}}%
\pgfpathlineto{\pgfqpoint{3.329415in}{3.036347in}}%
\pgfpathlineto{\pgfqpoint{3.331474in}{3.094613in}}%
\pgfpathlineto{\pgfqpoint{3.337650in}{3.114035in}}%
\pgfpathlineto{\pgfqpoint{3.345886in}{3.201434in}}%
\pgfpathlineto{\pgfqpoint{3.352063in}{3.123746in}}%
\pgfpathlineto{\pgfqpoint{3.354122in}{3.143168in}}%
\pgfpathlineto{\pgfqpoint{3.356181in}{3.191723in}}%
\pgfpathlineto{\pgfqpoint{3.358240in}{3.191723in}}%
\pgfpathlineto{\pgfqpoint{3.360299in}{3.143168in}}%
\pgfpathlineto{\pgfqpoint{3.366476in}{3.182012in}}%
\pgfpathlineto{\pgfqpoint{3.368535in}{3.162590in}}%
\pgfpathlineto{\pgfqpoint{3.370594in}{3.269410in}}%
\pgfpathlineto{\pgfqpoint{3.372653in}{3.269410in}}%
\pgfpathlineto{\pgfqpoint{3.374711in}{3.249988in}}%
\pgfpathlineto{\pgfqpoint{3.382947in}{3.269410in}}%
\pgfpathlineto{\pgfqpoint{3.385006in}{3.308254in}}%
\pgfpathlineto{\pgfqpoint{3.389124in}{3.240277in}}%
\pgfpathlineto{\pgfqpoint{3.395301in}{3.220855in}}%
\pgfpathlineto{\pgfqpoint{3.397360in}{3.288832in}}%
\pgfpathlineto{\pgfqpoint{3.399419in}{3.269410in}}%
\pgfpathlineto{\pgfqpoint{3.401478in}{3.201434in}}%
\pgfpathlineto{\pgfqpoint{3.403537in}{3.249988in}}%
\pgfpathlineto{\pgfqpoint{3.409714in}{3.269410in}}%
\pgfpathlineto{\pgfqpoint{3.413831in}{3.269410in}}%
\pgfpathlineto{\pgfqpoint{3.415890in}{3.249988in}}%
\pgfpathlineto{\pgfqpoint{3.417949in}{3.269410in}}%
\pgfpathlineto{\pgfqpoint{3.424126in}{3.259699in}}%
\pgfpathlineto{\pgfqpoint{3.428244in}{3.230566in}}%
\pgfpathlineto{\pgfqpoint{3.430303in}{3.240277in}}%
\pgfpathlineto{\pgfqpoint{3.432362in}{3.269410in}}%
\pgfpathlineto{\pgfqpoint{3.438539in}{3.269410in}}%
\pgfpathlineto{\pgfqpoint{3.440598in}{3.308254in}}%
\pgfpathlineto{\pgfqpoint{3.442657in}{3.308254in}}%
\pgfpathlineto{\pgfqpoint{3.444716in}{3.249988in}}%
\pgfpathlineto{\pgfqpoint{3.446775in}{3.230566in}}%
\pgfpathlineto{\pgfqpoint{3.452951in}{3.259699in}}%
\pgfpathlineto{\pgfqpoint{3.455010in}{3.201434in}}%
\pgfpathlineto{\pgfqpoint{3.457069in}{3.211145in}}%
\pgfpathlineto{\pgfqpoint{3.459128in}{3.182012in}}%
\pgfpathlineto{\pgfqpoint{3.467364in}{3.172301in}}%
\pgfpathlineto{\pgfqpoint{3.469423in}{3.220855in}}%
\pgfpathlineto{\pgfqpoint{3.471482in}{3.230566in}}%
\pgfpathlineto{\pgfqpoint{3.473541in}{3.259699in}}%
\pgfpathlineto{\pgfqpoint{3.475600in}{3.201434in}}%
\pgfpathlineto{\pgfqpoint{3.481777in}{3.220855in}}%
\pgfpathlineto{\pgfqpoint{3.483836in}{3.240277in}}%
\pgfpathlineto{\pgfqpoint{3.485895in}{3.240277in}}%
\pgfpathlineto{\pgfqpoint{3.487954in}{3.288832in}}%
\pgfpathlineto{\pgfqpoint{3.490012in}{3.288832in}}%
\pgfpathlineto{\pgfqpoint{3.496189in}{3.308254in}}%
\pgfpathlineto{\pgfqpoint{3.498248in}{3.298543in}}%
\pgfpathlineto{\pgfqpoint{3.502366in}{3.385942in}}%
\pgfpathlineto{\pgfqpoint{3.504425in}{3.415075in}}%
\pgfpathlineto{\pgfqpoint{3.510602in}{3.444208in}}%
\pgfpathlineto{\pgfqpoint{3.512661in}{3.444208in}}%
\pgfpathlineto{\pgfqpoint{3.514720in}{3.453919in}}%
\pgfpathlineto{\pgfqpoint{3.518838in}{3.415075in}}%
\pgfpathlineto{\pgfqpoint{3.525015in}{3.405364in}}%
\pgfpathlineto{\pgfqpoint{3.527073in}{3.434497in}}%
\pgfpathlineto{\pgfqpoint{3.531191in}{3.395653in}}%
\pgfpathlineto{\pgfqpoint{3.539427in}{3.395653in}}%
\pgfpathlineto{\pgfqpoint{3.543545in}{3.453919in}}%
\pgfpathlineto{\pgfqpoint{3.545604in}{3.444208in}}%
\pgfpathlineto{\pgfqpoint{3.547663in}{3.453919in}}%
\pgfpathlineto{\pgfqpoint{3.553840in}{3.463630in}}%
\pgfpathlineto{\pgfqpoint{3.555899in}{3.531607in}}%
\pgfpathlineto{\pgfqpoint{3.557958in}{3.551029in}}%
\pgfpathlineto{\pgfqpoint{3.560017in}{3.551029in}}%
\pgfpathlineto{\pgfqpoint{3.562076in}{3.512185in}}%
\pgfpathlineto{\pgfqpoint{3.570311in}{3.512185in}}%
\pgfpathlineto{\pgfqpoint{3.572370in}{3.444208in}}%
\pgfpathlineto{\pgfqpoint{3.574429in}{3.434497in}}%
\pgfpathlineto{\pgfqpoint{3.576488in}{3.376231in}}%
\pgfpathlineto{\pgfqpoint{3.584724in}{3.201434in}}%
\pgfpathlineto{\pgfqpoint{3.586783in}{3.288832in}}%
\pgfpathlineto{\pgfqpoint{3.588842in}{3.298543in}}%
\pgfpathlineto{\pgfqpoint{3.590901in}{3.356809in}}%
\pgfpathlineto{\pgfqpoint{3.597078in}{3.395653in}}%
\pgfpathlineto{\pgfqpoint{3.599137in}{3.376231in}}%
\pgfpathlineto{\pgfqpoint{3.601196in}{3.424786in}}%
\pgfpathlineto{\pgfqpoint{3.603254in}{3.385942in}}%
\pgfpathlineto{\pgfqpoint{3.605313in}{3.385942in}}%
\pgfpathlineto{\pgfqpoint{3.613549in}{3.424786in}}%
\pgfpathlineto{\pgfqpoint{3.615608in}{3.463630in}}%
\pgfpathlineto{\pgfqpoint{3.617667in}{3.424786in}}%
\pgfpathlineto{\pgfqpoint{3.619726in}{3.424786in}}%
\pgfpathlineto{\pgfqpoint{3.625903in}{3.415075in}}%
\pgfpathlineto{\pgfqpoint{3.627962in}{3.385942in}}%
\pgfpathlineto{\pgfqpoint{3.630021in}{3.415075in}}%
\pgfpathlineto{\pgfqpoint{3.632080in}{3.385942in}}%
\pgfpathlineto{\pgfqpoint{3.634139in}{3.385942in}}%
\pgfpathlineto{\pgfqpoint{3.640315in}{3.366520in}}%
\pgfpathlineto{\pgfqpoint{3.642374in}{3.366520in}}%
\pgfpathlineto{\pgfqpoint{3.644433in}{3.327676in}}%
\pgfpathlineto{\pgfqpoint{3.646492in}{3.347098in}}%
\pgfpathlineto{\pgfqpoint{3.648551in}{3.347098in}}%
\pgfpathlineto{\pgfqpoint{3.654728in}{3.366520in}}%
\pgfpathlineto{\pgfqpoint{3.656787in}{3.337387in}}%
\pgfpathlineto{\pgfqpoint{3.660905in}{3.356809in}}%
\pgfpathlineto{\pgfqpoint{3.662964in}{3.327676in}}%
\pgfpathlineto{\pgfqpoint{3.671200in}{3.385942in}}%
\pgfpathlineto{\pgfqpoint{3.673259in}{3.356809in}}%
\pgfpathlineto{\pgfqpoint{3.675318in}{3.366520in}}%
\pgfpathlineto{\pgfqpoint{3.677377in}{3.347098in}}%
\pgfpathlineto{\pgfqpoint{3.685612in}{3.376231in}}%
\pgfpathlineto{\pgfqpoint{3.687671in}{3.385942in}}%
\pgfpathlineto{\pgfqpoint{3.689730in}{3.356809in}}%
\pgfpathlineto{\pgfqpoint{3.691789in}{3.385942in}}%
\pgfpathlineto{\pgfqpoint{3.697966in}{3.444208in}}%
\pgfpathlineto{\pgfqpoint{3.700025in}{3.444208in}}%
\pgfpathlineto{\pgfqpoint{3.702084in}{3.434497in}}%
\pgfpathlineto{\pgfqpoint{3.704143in}{3.473341in}}%
\pgfpathlineto{\pgfqpoint{3.706202in}{3.453919in}}%
\pgfpathlineto{\pgfqpoint{3.712379in}{3.463630in}}%
\pgfpathlineto{\pgfqpoint{3.714438in}{3.463630in}}%
\pgfpathlineto{\pgfqpoint{3.716496in}{3.483052in}}%
\pgfpathlineto{\pgfqpoint{3.720614in}{3.434497in}}%
\pgfpathlineto{\pgfqpoint{3.726791in}{3.415075in}}%
\pgfpathlineto{\pgfqpoint{3.728850in}{3.453919in}}%
\pgfpathlineto{\pgfqpoint{3.730909in}{3.444208in}}%
\pgfpathlineto{\pgfqpoint{3.732968in}{3.415075in}}%
\pgfpathlineto{\pgfqpoint{3.735027in}{3.366520in}}%
\pgfpathlineto{\pgfqpoint{3.741204in}{3.366520in}}%
\pgfpathlineto{\pgfqpoint{3.743263in}{3.385942in}}%
\pgfpathlineto{\pgfqpoint{3.745322in}{3.347098in}}%
\pgfpathlineto{\pgfqpoint{3.747381in}{3.366520in}}%
\pgfpathlineto{\pgfqpoint{3.749440in}{3.366520in}}%
\pgfpathlineto{\pgfqpoint{3.755616in}{3.317965in}}%
\pgfpathlineto{\pgfqpoint{3.757675in}{3.347098in}}%
\pgfpathlineto{\pgfqpoint{3.759734in}{3.317965in}}%
\pgfpathlineto{\pgfqpoint{3.761793in}{3.337387in}}%
\pgfpathlineto{\pgfqpoint{3.763852in}{3.337387in}}%
\pgfpathlineto{\pgfqpoint{3.770029in}{3.356809in}}%
\pgfpathlineto{\pgfqpoint{3.772088in}{3.385942in}}%
\pgfpathlineto{\pgfqpoint{3.774147in}{3.395653in}}%
\pgfpathlineto{\pgfqpoint{3.776206in}{3.366520in}}%
\pgfpathlineto{\pgfqpoint{3.778265in}{3.356809in}}%
\pgfpathlineto{\pgfqpoint{3.786501in}{3.395653in}}%
\pgfpathlineto{\pgfqpoint{3.790619in}{3.376231in}}%
\pgfpathlineto{\pgfqpoint{3.792677in}{3.434497in}}%
\pgfpathlineto{\pgfqpoint{3.798854in}{3.444208in}}%
\pgfpathlineto{\pgfqpoint{3.800913in}{3.483052in}}%
\pgfpathlineto{\pgfqpoint{3.805031in}{3.483052in}}%
\pgfpathlineto{\pgfqpoint{3.807090in}{3.512185in}}%
\pgfpathlineto{\pgfqpoint{3.813267in}{3.502474in}}%
\pgfpathlineto{\pgfqpoint{3.815326in}{3.551029in}}%
\pgfpathlineto{\pgfqpoint{3.817385in}{3.570451in}}%
\pgfpathlineto{\pgfqpoint{3.819444in}{3.570451in}}%
\pgfpathlineto{\pgfqpoint{3.821503in}{3.560740in}}%
\pgfpathlineto{\pgfqpoint{3.827680in}{3.570451in}}%
\pgfpathlineto{\pgfqpoint{3.829738in}{3.599584in}}%
\pgfpathlineto{\pgfqpoint{3.831797in}{3.570451in}}%
\pgfpathlineto{\pgfqpoint{3.833856in}{3.570451in}}%
\pgfpathlineto{\pgfqpoint{3.835915in}{3.551029in}}%
\pgfpathlineto{\pgfqpoint{3.842092in}{3.570451in}}%
\pgfpathlineto{\pgfqpoint{3.844151in}{3.551029in}}%
\pgfpathlineto{\pgfqpoint{3.846210in}{3.628717in}}%
\pgfpathlineto{\pgfqpoint{3.850328in}{3.677272in}}%
\pgfpathlineto{\pgfqpoint{3.858564in}{3.657850in}}%
\pgfpathlineto{\pgfqpoint{3.860623in}{3.657850in}}%
\pgfpathlineto{\pgfqpoint{3.862682in}{3.609295in}}%
\pgfpathlineto{\pgfqpoint{3.864741in}{3.609295in}}%
\pgfpathlineto{\pgfqpoint{3.870917in}{3.619006in}}%
\pgfpathlineto{\pgfqpoint{3.872976in}{3.628717in}}%
\pgfpathlineto{\pgfqpoint{3.875035in}{3.648139in}}%
\pgfpathlineto{\pgfqpoint{3.877094in}{3.638428in}}%
\pgfpathlineto{\pgfqpoint{3.879153in}{3.657850in}}%
\pgfpathlineto{\pgfqpoint{3.885330in}{3.657850in}}%
\pgfpathlineto{\pgfqpoint{3.887389in}{3.619006in}}%
\pgfpathlineto{\pgfqpoint{3.889448in}{3.551029in}}%
\pgfpathlineto{\pgfqpoint{3.891507in}{3.589873in}}%
\pgfpathlineto{\pgfqpoint{3.893566in}{3.521896in}}%
\pgfpathlineto{\pgfqpoint{3.899743in}{3.521896in}}%
\pgfpathlineto{\pgfqpoint{3.903861in}{3.589873in}}%
\pgfpathlineto{\pgfqpoint{3.905919in}{3.570451in}}%
\pgfpathlineto{\pgfqpoint{3.907978in}{3.648139in}}%
\pgfpathlineto{\pgfqpoint{3.914155in}{3.638428in}}%
\pgfpathlineto{\pgfqpoint{3.920332in}{3.696694in}}%
\pgfpathlineto{\pgfqpoint{3.922391in}{3.657850in}}%
\pgfpathlineto{\pgfqpoint{3.930627in}{3.599584in}}%
\pgfpathlineto{\pgfqpoint{3.932686in}{3.560740in}}%
\pgfpathlineto{\pgfqpoint{3.934745in}{3.551029in}}%
\pgfpathlineto{\pgfqpoint{3.936804in}{3.512185in}}%
\pgfpathlineto{\pgfqpoint{3.942980in}{3.483052in}}%
\pgfpathlineto{\pgfqpoint{3.947098in}{3.502474in}}%
\pgfpathlineto{\pgfqpoint{3.951216in}{3.492763in}}%
\pgfpathlineto{\pgfqpoint{3.957393in}{3.512185in}}%
\pgfpathlineto{\pgfqpoint{3.959452in}{3.502474in}}%
\pgfpathlineto{\pgfqpoint{3.965629in}{3.453919in}}%
\pgfpathlineto{\pgfqpoint{3.971806in}{3.444208in}}%
\pgfpathlineto{\pgfqpoint{3.973865in}{3.405364in}}%
\pgfpathlineto{\pgfqpoint{3.977983in}{3.366520in}}%
\pgfpathlineto{\pgfqpoint{3.980042in}{3.317965in}}%
\pgfpathlineto{\pgfqpoint{3.986218in}{3.327676in}}%
\pgfpathlineto{\pgfqpoint{3.990336in}{3.385942in}}%
\pgfpathlineto{\pgfqpoint{3.994454in}{3.347098in}}%
\pgfpathlineto{\pgfqpoint{4.000631in}{3.308254in}}%
\pgfpathlineto{\pgfqpoint{4.004749in}{3.240277in}}%
\pgfpathlineto{\pgfqpoint{4.006808in}{3.269410in}}%
\pgfpathlineto{\pgfqpoint{4.008867in}{3.259699in}}%
\pgfpathlineto{\pgfqpoint{4.015044in}{3.201434in}}%
\pgfpathlineto{\pgfqpoint{4.019161in}{3.288832in}}%
\pgfpathlineto{\pgfqpoint{4.023279in}{3.182012in}}%
\pgfpathlineto{\pgfqpoint{4.029456in}{3.133457in}}%
\pgfpathlineto{\pgfqpoint{4.033574in}{3.114035in}}%
\pgfpathlineto{\pgfqpoint{4.035633in}{2.997503in}}%
\pgfpathlineto{\pgfqpoint{4.037692in}{3.114035in}}%
\pgfpathlineto{\pgfqpoint{4.043869in}{3.152879in}}%
\pgfpathlineto{\pgfqpoint{4.045928in}{3.201434in}}%
\pgfpathlineto{\pgfqpoint{4.050046in}{3.182012in}}%
\pgfpathlineto{\pgfqpoint{4.052105in}{3.143168in}}%
\pgfpathlineto{\pgfqpoint{4.058281in}{3.152879in}}%
\pgfpathlineto{\pgfqpoint{4.060340in}{3.152879in}}%
\pgfpathlineto{\pgfqpoint{4.062399in}{3.162590in}}%
\pgfpathlineto{\pgfqpoint{4.066517in}{3.240277in}}%
\pgfpathlineto{\pgfqpoint{4.074753in}{3.191723in}}%
\pgfpathlineto{\pgfqpoint{4.076812in}{3.211145in}}%
\pgfpathlineto{\pgfqpoint{4.078871in}{3.172301in}}%
\pgfpathlineto{\pgfqpoint{4.080930in}{3.211145in}}%
\pgfpathlineto{\pgfqpoint{4.087107in}{3.201434in}}%
\pgfpathlineto{\pgfqpoint{4.089166in}{3.172301in}}%
\pgfpathlineto{\pgfqpoint{4.093284in}{3.055769in}}%
\pgfpathlineto{\pgfqpoint{4.095342in}{3.133457in}}%
\pgfpathlineto{\pgfqpoint{4.101519in}{3.152879in}}%
\pgfpathlineto{\pgfqpoint{4.109755in}{3.065480in}}%
\pgfpathlineto{\pgfqpoint{4.115932in}{3.094613in}}%
\pgfpathlineto{\pgfqpoint{4.117991in}{3.114035in}}%
\pgfpathlineto{\pgfqpoint{4.120050in}{3.152879in}}%
\pgfpathlineto{\pgfqpoint{4.122109in}{3.104324in}}%
\pgfpathlineto{\pgfqpoint{4.124168in}{3.114035in}}%
\pgfpathlineto{\pgfqpoint{4.132403in}{3.094613in}}%
\pgfpathlineto{\pgfqpoint{4.134462in}{3.094613in}}%
\pgfpathlineto{\pgfqpoint{4.136521in}{3.133457in}}%
\pgfpathlineto{\pgfqpoint{4.138580in}{3.094613in}}%
\pgfpathlineto{\pgfqpoint{4.144757in}{3.104324in}}%
\pgfpathlineto{\pgfqpoint{4.146816in}{3.075191in}}%
\pgfpathlineto{\pgfqpoint{4.152993in}{3.182012in}}%
\pgfpathlineto{\pgfqpoint{4.159170in}{3.152879in}}%
\pgfpathlineto{\pgfqpoint{4.161229in}{3.152879in}}%
\pgfpathlineto{\pgfqpoint{4.167406in}{3.046058in}}%
\pgfpathlineto{\pgfqpoint{4.173582in}{3.065480in}}%
\pgfpathlineto{\pgfqpoint{4.175641in}{3.036347in}}%
\pgfpathlineto{\pgfqpoint{4.179759in}{3.055769in}}%
\pgfpathlineto{\pgfqpoint{4.181818in}{3.026636in}}%
\pgfpathlineto{\pgfqpoint{4.187995in}{3.046058in}}%
\pgfpathlineto{\pgfqpoint{4.190054in}{3.046058in}}%
\pgfpathlineto{\pgfqpoint{4.192113in}{2.968370in}}%
\pgfpathlineto{\pgfqpoint{4.194172in}{2.968370in}}%
\pgfpathlineto{\pgfqpoint{4.196231in}{2.871260in}}%
\pgfpathlineto{\pgfqpoint{4.202408in}{2.842127in}}%
\pgfpathlineto{\pgfqpoint{4.204467in}{2.812994in}}%
\pgfpathlineto{\pgfqpoint{4.206526in}{2.812994in}}%
\pgfpathlineto{\pgfqpoint{4.210643in}{2.861549in}}%
\pgfpathlineto{\pgfqpoint{4.216820in}{2.939237in}}%
\pgfpathlineto{\pgfqpoint{4.218879in}{2.910104in}}%
\pgfpathlineto{\pgfqpoint{4.220938in}{2.948948in}}%
\pgfpathlineto{\pgfqpoint{4.222997in}{2.948948in}}%
\pgfpathlineto{\pgfqpoint{4.225056in}{2.939237in}}%
\pgfpathlineto{\pgfqpoint{4.231233in}{2.958659in}}%
\pgfpathlineto{\pgfqpoint{4.235351in}{2.910104in}}%
\pgfpathlineto{\pgfqpoint{4.237410in}{2.939237in}}%
\pgfpathlineto{\pgfqpoint{4.239469in}{3.007214in}}%
\pgfpathlineto{\pgfqpoint{4.245645in}{2.997503in}}%
\pgfpathlineto{\pgfqpoint{4.247704in}{3.036347in}}%
\pgfpathlineto{\pgfqpoint{4.249763in}{3.026636in}}%
\pgfpathlineto{\pgfqpoint{4.251822in}{3.007214in}}%
\pgfpathlineto{\pgfqpoint{4.260058in}{3.016925in}}%
\pgfpathlineto{\pgfqpoint{4.264176in}{2.948948in}}%
\pgfpathlineto{\pgfqpoint{4.266235in}{2.958659in}}%
\pgfpathlineto{\pgfqpoint{4.268294in}{2.919815in}}%
\pgfpathlineto{\pgfqpoint{4.274471in}{2.948948in}}%
\pgfpathlineto{\pgfqpoint{4.276530in}{2.910104in}}%
\pgfpathlineto{\pgfqpoint{4.280648in}{2.968370in}}%
\pgfpathlineto{\pgfqpoint{4.288883in}{2.929526in}}%
\pgfpathlineto{\pgfqpoint{4.290942in}{2.880971in}}%
\pgfpathlineto{\pgfqpoint{4.293001in}{2.910104in}}%
\pgfpathlineto{\pgfqpoint{4.295060in}{2.880971in}}%
\pgfpathlineto{\pgfqpoint{4.297119in}{2.890682in}}%
\pgfpathlineto{\pgfqpoint{4.303296in}{2.812994in}}%
\pgfpathlineto{\pgfqpoint{4.305355in}{2.832416in}}%
\pgfpathlineto{\pgfqpoint{4.307414in}{2.783861in}}%
\pgfpathlineto{\pgfqpoint{4.309473in}{2.812994in}}%
\pgfpathlineto{\pgfqpoint{4.311532in}{2.803283in}}%
\pgfpathlineto{\pgfqpoint{4.319768in}{2.861549in}}%
\pgfpathlineto{\pgfqpoint{4.321826in}{2.822705in}}%
\pgfpathlineto{\pgfqpoint{4.323885in}{2.745017in}}%
\pgfpathlineto{\pgfqpoint{4.325944in}{2.754728in}}%
\pgfpathlineto{\pgfqpoint{4.338298in}{2.667329in}}%
\pgfpathlineto{\pgfqpoint{4.340357in}{2.570220in}}%
\pgfpathlineto{\pgfqpoint{4.346534in}{2.473110in}}%
\pgfpathlineto{\pgfqpoint{4.348593in}{2.531376in}}%
\pgfpathlineto{\pgfqpoint{4.350652in}{2.502243in}}%
\pgfpathlineto{\pgfqpoint{4.352711in}{2.521665in}}%
\pgfpathlineto{\pgfqpoint{4.354770in}{2.492532in}}%
\pgfpathlineto{\pgfqpoint{4.363005in}{2.560509in}}%
\pgfpathlineto{\pgfqpoint{4.367123in}{2.482821in}}%
\pgfpathlineto{\pgfqpoint{4.369182in}{2.492532in}}%
\pgfpathlineto{\pgfqpoint{4.375359in}{2.492532in}}%
\pgfpathlineto{\pgfqpoint{4.377418in}{2.473110in}}%
\pgfpathlineto{\pgfqpoint{4.381536in}{2.385711in}}%
\pgfpathlineto{\pgfqpoint{4.383595in}{2.443977in}}%
\pgfpathlineto{\pgfqpoint{4.391831in}{2.376000in}}%
\pgfpathlineto{\pgfqpoint{4.393890in}{2.443977in}}%
\pgfpathlineto{\pgfqpoint{4.395949in}{2.405133in}}%
\pgfpathlineto{\pgfqpoint{4.398007in}{2.405133in}}%
\pgfpathlineto{\pgfqpoint{4.404184in}{2.434266in}}%
\pgfpathlineto{\pgfqpoint{4.406243in}{2.395422in}}%
\pgfpathlineto{\pgfqpoint{4.408302in}{2.385711in}}%
\pgfpathlineto{\pgfqpoint{4.412420in}{2.482821in}}%
\pgfpathlineto{\pgfqpoint{4.418597in}{2.502243in}}%
\pgfpathlineto{\pgfqpoint{4.420656in}{2.521665in}}%
\pgfpathlineto{\pgfqpoint{4.422715in}{2.463399in}}%
\pgfpathlineto{\pgfqpoint{4.424774in}{2.521665in}}%
\pgfpathlineto{\pgfqpoint{4.426833in}{2.502243in}}%
\pgfpathlineto{\pgfqpoint{4.433010in}{2.482821in}}%
\pgfpathlineto{\pgfqpoint{4.435068in}{2.521665in}}%
\pgfpathlineto{\pgfqpoint{4.439186in}{2.424555in}}%
\pgfpathlineto{\pgfqpoint{4.441245in}{2.443977in}}%
\pgfpathlineto{\pgfqpoint{4.447422in}{2.443977in}}%
\pgfpathlineto{\pgfqpoint{4.449481in}{2.473110in}}%
\pgfpathlineto{\pgfqpoint{4.451540in}{2.463399in}}%
\pgfpathlineto{\pgfqpoint{4.453599in}{2.492532in}}%
\pgfpathlineto{\pgfqpoint{4.455658in}{2.492532in}}%
\pgfpathlineto{\pgfqpoint{4.461835in}{2.482821in}}%
\pgfpathlineto{\pgfqpoint{4.465953in}{2.482821in}}%
\pgfpathlineto{\pgfqpoint{4.468012in}{2.327445in}}%
\pgfpathlineto{\pgfqpoint{4.470071in}{2.308023in}}%
\pgfpathlineto{\pgfqpoint{4.478306in}{2.181780in}}%
\pgfpathlineto{\pgfqpoint{4.480365in}{2.172069in}}%
\pgfpathlineto{\pgfqpoint{4.484483in}{2.220624in}}%
\pgfpathlineto{\pgfqpoint{4.490660in}{2.142936in}}%
\pgfpathlineto{\pgfqpoint{4.492719in}{2.220624in}}%
\pgfpathlineto{\pgfqpoint{4.498896in}{2.074960in}}%
\pgfpathlineto{\pgfqpoint{4.505073in}{2.123514in}}%
\pgfpathlineto{\pgfqpoint{4.507132in}{2.074960in}}%
\pgfpathlineto{\pgfqpoint{4.511249in}{2.152647in}}%
\pgfpathlineto{\pgfqpoint{4.513308in}{2.055538in}}%
\pgfpathlineto{\pgfqpoint{4.519485in}{2.084671in}}%
\pgfpathlineto{\pgfqpoint{4.523603in}{2.026405in}}%
\pgfpathlineto{\pgfqpoint{4.525662in}{2.055538in}}%
\pgfpathlineto{\pgfqpoint{4.535957in}{2.006983in}}%
\pgfpathlineto{\pgfqpoint{4.538016in}{1.977850in}}%
\pgfpathlineto{\pgfqpoint{4.540075in}{2.084671in}}%
\pgfpathlineto{\pgfqpoint{4.542134in}{2.074960in}}%
\pgfpathlineto{\pgfqpoint{4.548311in}{2.142936in}}%
\pgfpathlineto{\pgfqpoint{4.550369in}{2.230335in}}%
\pgfpathlineto{\pgfqpoint{4.552428in}{2.249757in}}%
\pgfpathlineto{\pgfqpoint{4.554487in}{2.298312in}}%
\pgfpathlineto{\pgfqpoint{4.556546in}{2.395422in}}%
\pgfpathlineto{\pgfqpoint{4.564782in}{2.308023in}}%
\pgfpathlineto{\pgfqpoint{4.566841in}{2.327445in}}%
\pgfpathlineto{\pgfqpoint{4.568900in}{2.308023in}}%
\pgfpathlineto{\pgfqpoint{4.570959in}{2.259468in}}%
\pgfpathlineto{\pgfqpoint{4.577136in}{2.240046in}}%
\pgfpathlineto{\pgfqpoint{4.579195in}{2.172069in}}%
\pgfpathlineto{\pgfqpoint{4.581254in}{2.249757in}}%
\pgfpathlineto{\pgfqpoint{4.583313in}{2.240046in}}%
\pgfpathlineto{\pgfqpoint{4.585372in}{2.210913in}}%
\pgfpathlineto{\pgfqpoint{4.591548in}{2.201202in}}%
\pgfpathlineto{\pgfqpoint{4.593607in}{2.162358in}}%
\pgfpathlineto{\pgfqpoint{4.597725in}{1.997272in}}%
\pgfpathlineto{\pgfqpoint{4.599784in}{1.997272in}}%
\pgfpathlineto{\pgfqpoint{4.605961in}{2.036116in}}%
\pgfpathlineto{\pgfqpoint{4.608020in}{2.016694in}}%
\pgfpathlineto{\pgfqpoint{4.610079in}{2.055538in}}%
\pgfpathlineto{\pgfqpoint{4.614197in}{2.240046in}}%
\pgfpathlineto{\pgfqpoint{4.622433in}{2.240046in}}%
\pgfpathlineto{\pgfqpoint{4.624491in}{2.220624in}}%
\pgfpathlineto{\pgfqpoint{4.626550in}{2.220624in}}%
\pgfpathlineto{\pgfqpoint{4.628609in}{2.210913in}}%
\pgfpathlineto{\pgfqpoint{4.634786in}{2.259468in}}%
\pgfpathlineto{\pgfqpoint{4.636845in}{2.249757in}}%
\pgfpathlineto{\pgfqpoint{4.638904in}{2.230335in}}%
\pgfpathlineto{\pgfqpoint{4.640963in}{2.230335in}}%
\pgfpathlineto{\pgfqpoint{4.643022in}{2.269179in}}%
\pgfpathlineto{\pgfqpoint{4.649199in}{2.308023in}}%
\pgfpathlineto{\pgfqpoint{4.651258in}{2.308023in}}%
\pgfpathlineto{\pgfqpoint{4.653317in}{2.259468in}}%
\pgfpathlineto{\pgfqpoint{4.655376in}{2.162358in}}%
\pgfpathlineto{\pgfqpoint{4.657435in}{2.201202in}}%
\pgfpathlineto{\pgfqpoint{4.663611in}{2.249757in}}%
\pgfpathlineto{\pgfqpoint{4.665670in}{2.308023in}}%
\pgfpathlineto{\pgfqpoint{4.667729in}{2.278890in}}%
\pgfpathlineto{\pgfqpoint{4.669788in}{2.385711in}}%
\pgfpathlineto{\pgfqpoint{4.671847in}{2.385711in}}%
\pgfpathlineto{\pgfqpoint{4.680083in}{2.376000in}}%
\pgfpathlineto{\pgfqpoint{4.684201in}{2.278890in}}%
\pgfpathlineto{\pgfqpoint{4.686260in}{2.298312in}}%
\pgfpathlineto{\pgfqpoint{4.692437in}{2.278890in}}%
\pgfpathlineto{\pgfqpoint{4.694496in}{2.278890in}}%
\pgfpathlineto{\pgfqpoint{4.696555in}{2.230335in}}%
\pgfpathlineto{\pgfqpoint{4.698614in}{2.269179in}}%
\pgfpathlineto{\pgfqpoint{4.706849in}{2.269179in}}%
\pgfpathlineto{\pgfqpoint{4.708908in}{2.230335in}}%
\pgfpathlineto{\pgfqpoint{4.710967in}{2.278890in}}%
\pgfpathlineto{\pgfqpoint{4.715085in}{2.269179in}}%
\pgfpathlineto{\pgfqpoint{4.721262in}{2.298312in}}%
\pgfpathlineto{\pgfqpoint{4.723321in}{2.191491in}}%
\pgfpathlineto{\pgfqpoint{4.725380in}{2.249757in}}%
\pgfpathlineto{\pgfqpoint{4.727439in}{2.269179in}}%
\pgfpathlineto{\pgfqpoint{4.729498in}{2.317734in}}%
\pgfpathlineto{\pgfqpoint{4.735675in}{2.317734in}}%
\pgfpathlineto{\pgfqpoint{4.737733in}{2.327445in}}%
\pgfpathlineto{\pgfqpoint{4.739792in}{2.288601in}}%
\pgfpathlineto{\pgfqpoint{4.741851in}{2.376000in}}%
\pgfpathlineto{\pgfqpoint{4.743910in}{2.308023in}}%
\pgfpathlineto{\pgfqpoint{4.750087in}{2.366289in}}%
\pgfpathlineto{\pgfqpoint{4.752146in}{2.356578in}}%
\pgfpathlineto{\pgfqpoint{4.754205in}{2.385711in}}%
\pgfpathlineto{\pgfqpoint{4.756264in}{2.376000in}}%
\pgfpathlineto{\pgfqpoint{4.758323in}{2.376000in}}%
\pgfpathlineto{\pgfqpoint{4.764500in}{2.395422in}}%
\pgfpathlineto{\pgfqpoint{4.766559in}{2.366289in}}%
\pgfpathlineto{\pgfqpoint{4.770677in}{2.366289in}}%
\pgfpathlineto{\pgfqpoint{4.772736in}{2.327445in}}%
\pgfpathlineto{\pgfqpoint{4.778912in}{2.327445in}}%
\pgfpathlineto{\pgfqpoint{4.780971in}{2.337156in}}%
\pgfpathlineto{\pgfqpoint{4.785089in}{2.317734in}}%
\pgfpathlineto{\pgfqpoint{4.787148in}{2.240046in}}%
\pgfpathlineto{\pgfqpoint{4.795384in}{2.269179in}}%
\pgfpathlineto{\pgfqpoint{4.797443in}{2.317734in}}%
\pgfpathlineto{\pgfqpoint{4.801561in}{2.278890in}}%
\pgfpathlineto{\pgfqpoint{4.807738in}{2.298312in}}%
\pgfpathlineto{\pgfqpoint{4.811856in}{2.249757in}}%
\pgfpathlineto{\pgfqpoint{4.813914in}{2.278890in}}%
\pgfpathlineto{\pgfqpoint{4.815973in}{2.278890in}}%
\pgfpathlineto{\pgfqpoint{4.824209in}{2.220624in}}%
\pgfpathlineto{\pgfqpoint{4.826268in}{2.220624in}}%
\pgfpathlineto{\pgfqpoint{4.828327in}{2.201202in}}%
\pgfpathlineto{\pgfqpoint{4.830386in}{2.162358in}}%
\pgfpathlineto{\pgfqpoint{4.836563in}{2.094382in}}%
\pgfpathlineto{\pgfqpoint{4.838622in}{2.123514in}}%
\pgfpathlineto{\pgfqpoint{4.840681in}{2.065249in}}%
\pgfpathlineto{\pgfqpoint{4.842740in}{2.045827in}}%
\pgfpathlineto{\pgfqpoint{4.844799in}{1.977850in}}%
\pgfpathlineto{\pgfqpoint{4.850976in}{2.006983in}}%
\pgfpathlineto{\pgfqpoint{4.855093in}{2.113803in}}%
\pgfpathlineto{\pgfqpoint{4.857152in}{2.104092in}}%
\pgfpathlineto{\pgfqpoint{4.859211in}{2.065249in}}%
\pgfpathlineto{\pgfqpoint{4.865388in}{2.036116in}}%
\pgfpathlineto{\pgfqpoint{4.867447in}{2.055538in}}%
\pgfpathlineto{\pgfqpoint{4.869506in}{2.104092in}}%
\pgfpathlineto{\pgfqpoint{4.873624in}{2.074960in}}%
\pgfpathlineto{\pgfqpoint{4.881860in}{2.045827in}}%
\pgfpathlineto{\pgfqpoint{4.883919in}{2.065249in}}%
\pgfpathlineto{\pgfqpoint{4.885978in}{2.026405in}}%
\pgfpathlineto{\pgfqpoint{4.888037in}{1.958428in}}%
\pgfpathlineto{\pgfqpoint{4.894213in}{1.871029in}}%
\pgfpathlineto{\pgfqpoint{4.896272in}{1.822474in}}%
\pgfpathlineto{\pgfqpoint{4.900390in}{1.773919in}}%
\pgfpathlineto{\pgfqpoint{4.902449in}{1.560277in}}%
\pgfpathlineto{\pgfqpoint{4.908626in}{1.550566in}}%
\pgfpathlineto{\pgfqpoint{4.910685in}{1.443746in}}%
\pgfpathlineto{\pgfqpoint{4.912744in}{1.424324in}}%
\pgfpathlineto{\pgfqpoint{4.916862in}{1.259237in}}%
\pgfpathlineto{\pgfqpoint{4.923039in}{1.142705in}}%
\pgfpathlineto{\pgfqpoint{4.925098in}{1.307792in}}%
\pgfpathlineto{\pgfqpoint{4.927156in}{1.336925in}}%
\pgfpathlineto{\pgfqpoint{4.929215in}{1.336925in}}%
\pgfpathlineto{\pgfqpoint{4.931274in}{1.375769in}}%
\pgfpathlineto{\pgfqpoint{4.937451in}{1.171838in}}%
\pgfpathlineto{\pgfqpoint{4.941569in}{1.463168in}}%
\pgfpathlineto{\pgfqpoint{4.945687in}{1.200971in}}%
\pgfpathlineto{\pgfqpoint{4.951864in}{1.065017in}}%
\pgfpathlineto{\pgfqpoint{4.953923in}{1.200971in}}%
\pgfpathlineto{\pgfqpoint{4.955982in}{1.239815in}}%
\pgfpathlineto{\pgfqpoint{4.958041in}{1.191260in}}%
\pgfpathlineto{\pgfqpoint{4.960100in}{1.094150in}}%
\pgfpathlineto{\pgfqpoint{4.966276in}{1.074728in}}%
\pgfpathlineto{\pgfqpoint{4.968335in}{1.055306in}}%
\pgfpathlineto{\pgfqpoint{4.970394in}{1.055306in}}%
\pgfpathlineto{\pgfqpoint{4.972453in}{1.074728in}}%
\pgfpathlineto{\pgfqpoint{4.974512in}{1.074728in}}%
\pgfpathlineto{\pgfqpoint{4.980689in}{1.123283in}}%
\pgfpathlineto{\pgfqpoint{4.982748in}{1.162127in}}%
\pgfpathlineto{\pgfqpoint{4.984807in}{1.152416in}}%
\pgfpathlineto{\pgfqpoint{4.986866in}{1.094150in}}%
\pgfpathlineto{\pgfqpoint{4.995102in}{1.123283in}}%
\pgfpathlineto{\pgfqpoint{4.997161in}{1.103861in}}%
\pgfpathlineto{\pgfqpoint{4.999220in}{1.026173in}}%
\pgfpathlineto{\pgfqpoint{5.003337in}{1.045595in}}%
\pgfpathlineto{\pgfqpoint{5.009514in}{1.035884in}}%
\pgfpathlineto{\pgfqpoint{5.011573in}{1.026173in}}%
\pgfpathlineto{\pgfqpoint{5.013632in}{1.055306in}}%
\pgfpathlineto{\pgfqpoint{5.015691in}{1.055306in}}%
\pgfpathlineto{\pgfqpoint{5.017750in}{1.045595in}}%
\pgfpathlineto{\pgfqpoint{5.023927in}{1.094150in}}%
\pgfpathlineto{\pgfqpoint{5.025986in}{1.055306in}}%
\pgfpathlineto{\pgfqpoint{5.028045in}{1.045595in}}%
\pgfpathlineto{\pgfqpoint{5.040399in}{1.045595in}}%
\pgfpathlineto{\pgfqpoint{5.042457in}{1.055306in}}%
\pgfpathlineto{\pgfqpoint{5.044516in}{0.977618in}}%
\pgfpathlineto{\pgfqpoint{5.046575in}{1.016462in}}%
\pgfpathlineto{\pgfqpoint{5.052752in}{1.045595in}}%
\pgfpathlineto{\pgfqpoint{5.058929in}{0.997040in}}%
\pgfpathlineto{\pgfqpoint{5.060988in}{0.997040in}}%
\pgfpathlineto{\pgfqpoint{5.067165in}{1.065017in}}%
\pgfpathlineto{\pgfqpoint{5.069224in}{1.035884in}}%
\pgfpathlineto{\pgfqpoint{5.071283in}{1.026173in}}%
\pgfpathlineto{\pgfqpoint{5.073342in}{1.035884in}}%
\pgfpathlineto{\pgfqpoint{5.075401in}{1.026173in}}%
\pgfpathlineto{\pgfqpoint{5.083636in}{1.035884in}}%
\pgfpathlineto{\pgfqpoint{5.085695in}{1.026173in}}%
\pgfpathlineto{\pgfqpoint{5.087754in}{1.026173in}}%
\pgfpathlineto{\pgfqpoint{5.089813in}{0.987329in}}%
\pgfpathlineto{\pgfqpoint{5.095990in}{0.997040in}}%
\pgfpathlineto{\pgfqpoint{5.098049in}{1.006751in}}%
\pgfpathlineto{\pgfqpoint{5.100108in}{1.065017in}}%
\pgfpathlineto{\pgfqpoint{5.102167in}{1.084439in}}%
\pgfpathlineto{\pgfqpoint{5.104226in}{1.152416in}}%
\pgfpathlineto{\pgfqpoint{5.110403in}{1.132994in}}%
\pgfpathlineto{\pgfqpoint{5.116579in}{1.006751in}}%
\pgfpathlineto{\pgfqpoint{5.118638in}{1.016462in}}%
\pgfpathlineto{\pgfqpoint{5.124815in}{1.016462in}}%
\pgfpathlineto{\pgfqpoint{5.126874in}{1.026173in}}%
\pgfpathlineto{\pgfqpoint{5.130992in}{1.026173in}}%
\pgfpathlineto{\pgfqpoint{5.133051in}{1.016462in}}%
\pgfpathlineto{\pgfqpoint{5.139228in}{1.026173in}}%
\pgfpathlineto{\pgfqpoint{5.141287in}{1.016462in}}%
\pgfpathlineto{\pgfqpoint{5.143346in}{1.016462in}}%
\pgfpathlineto{\pgfqpoint{5.145405in}{1.006751in}}%
\pgfpathlineto{\pgfqpoint{5.147464in}{0.987329in}}%
\pgfpathlineto{\pgfqpoint{5.153641in}{0.967908in}}%
\pgfpathlineto{\pgfqpoint{5.155699in}{0.977618in}}%
\pgfpathlineto{\pgfqpoint{5.157758in}{0.997040in}}%
\pgfpathlineto{\pgfqpoint{5.159817in}{0.977618in}}%
\pgfpathlineto{\pgfqpoint{5.168053in}{0.997040in}}%
\pgfpathlineto{\pgfqpoint{5.170112in}{0.977618in}}%
\pgfpathlineto{\pgfqpoint{5.172171in}{0.987329in}}%
\pgfpathlineto{\pgfqpoint{5.174230in}{0.967908in}}%
\pgfpathlineto{\pgfqpoint{5.176289in}{0.987329in}}%
\pgfpathlineto{\pgfqpoint{5.182466in}{0.987329in}}%
\pgfpathlineto{\pgfqpoint{5.184525in}{0.967908in}}%
\pgfpathlineto{\pgfqpoint{5.188643in}{0.967908in}}%
\pgfpathlineto{\pgfqpoint{5.190702in}{0.977618in}}%
\pgfpathlineto{\pgfqpoint{5.196878in}{0.977618in}}%
\pgfpathlineto{\pgfqpoint{5.198937in}{0.958197in}}%
\pgfpathlineto{\pgfqpoint{5.205114in}{0.958197in}}%
\pgfpathlineto{\pgfqpoint{5.211291in}{0.987329in}}%
\pgfpathlineto{\pgfqpoint{5.213350in}{0.948486in}}%
\pgfpathlineto{\pgfqpoint{5.215409in}{0.938775in}}%
\pgfpathlineto{\pgfqpoint{5.219527in}{0.899931in}}%
\pgfpathlineto{\pgfqpoint{5.225704in}{0.909642in}}%
\pgfpathlineto{\pgfqpoint{5.227763in}{0.880509in}}%
\pgfpathlineto{\pgfqpoint{5.229822in}{0.909642in}}%
\pgfpathlineto{\pgfqpoint{5.231880in}{0.899931in}}%
\pgfpathlineto{\pgfqpoint{5.233939in}{0.919353in}}%
\pgfpathlineto{\pgfqpoint{5.240116in}{0.929064in}}%
\pgfpathlineto{\pgfqpoint{5.246293in}{1.006751in}}%
\pgfpathlineto{\pgfqpoint{5.248352in}{0.977618in}}%
\pgfpathlineto{\pgfqpoint{5.254529in}{0.977618in}}%
\pgfpathlineto{\pgfqpoint{5.256588in}{0.958197in}}%
\pgfpathlineto{\pgfqpoint{5.258647in}{0.977618in}}%
\pgfpathlineto{\pgfqpoint{5.260706in}{0.948486in}}%
\pgfpathlineto{\pgfqpoint{5.262765in}{0.958197in}}%
\pgfpathlineto{\pgfqpoint{5.268941in}{0.967908in}}%
\pgfpathlineto{\pgfqpoint{5.271000in}{0.987329in}}%
\pgfpathlineto{\pgfqpoint{5.273059in}{0.967908in}}%
\pgfpathlineto{\pgfqpoint{5.275118in}{0.997040in}}%
\pgfpathlineto{\pgfqpoint{5.277177in}{0.967908in}}%
\pgfpathlineto{\pgfqpoint{5.283354in}{0.967908in}}%
\pgfpathlineto{\pgfqpoint{5.285413in}{0.948486in}}%
\pgfpathlineto{\pgfqpoint{5.287472in}{0.948486in}}%
\pgfpathlineto{\pgfqpoint{5.289531in}{0.929064in}}%
\pgfpathlineto{\pgfqpoint{5.291590in}{0.987329in}}%
\pgfpathlineto{\pgfqpoint{5.299826in}{0.967908in}}%
\pgfpathlineto{\pgfqpoint{5.301885in}{0.967908in}}%
\pgfpathlineto{\pgfqpoint{5.303944in}{0.948486in}}%
\pgfpathlineto{\pgfqpoint{5.306002in}{0.948486in}}%
\pgfpathlineto{\pgfqpoint{5.312179in}{0.958197in}}%
\pgfpathlineto{\pgfqpoint{5.314238in}{0.958197in}}%
\pgfpathlineto{\pgfqpoint{5.316297in}{0.967908in}}%
\pgfpathlineto{\pgfqpoint{5.318356in}{0.967908in}}%
\pgfpathlineto{\pgfqpoint{5.320415in}{0.977618in}}%
\pgfpathlineto{\pgfqpoint{5.326592in}{0.958197in}}%
\pgfpathlineto{\pgfqpoint{5.328651in}{0.958197in}}%
\pgfpathlineto{\pgfqpoint{5.330710in}{0.967908in}}%
\pgfpathlineto{\pgfqpoint{5.334828in}{0.948486in}}%
\pgfpathlineto{\pgfqpoint{5.341005in}{0.948486in}}%
\pgfpathlineto{\pgfqpoint{5.343064in}{0.929064in}}%
\pgfpathlineto{\pgfqpoint{5.345122in}{0.967908in}}%
\pgfpathlineto{\pgfqpoint{5.347181in}{0.958197in}}%
\pgfpathlineto{\pgfqpoint{5.349240in}{0.967908in}}%
\pgfpathlineto{\pgfqpoint{5.355417in}{1.016462in}}%
\pgfpathlineto{\pgfqpoint{5.357476in}{1.006751in}}%
\pgfpathlineto{\pgfqpoint{5.359535in}{1.035884in}}%
\pgfpathlineto{\pgfqpoint{5.361594in}{1.016462in}}%
\pgfpathlineto{\pgfqpoint{5.363653in}{1.026173in}}%
\pgfpathlineto{\pgfqpoint{5.373948in}{0.987329in}}%
\pgfpathlineto{\pgfqpoint{5.376007in}{1.006751in}}%
\pgfpathlineto{\pgfqpoint{5.378066in}{1.006751in}}%
\pgfpathlineto{\pgfqpoint{5.384242in}{1.026173in}}%
\pgfpathlineto{\pgfqpoint{5.386301in}{1.026173in}}%
\pgfpathlineto{\pgfqpoint{5.390419in}{1.065017in}}%
\pgfpathlineto{\pgfqpoint{5.392478in}{1.055306in}}%
\pgfpathlineto{\pgfqpoint{5.402773in}{1.026173in}}%
\pgfpathlineto{\pgfqpoint{5.404832in}{1.065017in}}%
\pgfpathlineto{\pgfqpoint{5.413068in}{1.065017in}}%
\pgfpathlineto{\pgfqpoint{5.415127in}{1.074728in}}%
\pgfpathlineto{\pgfqpoint{5.417186in}{1.016462in}}%
\pgfpathlineto{\pgfqpoint{5.419244in}{1.016462in}}%
\pgfpathlineto{\pgfqpoint{5.427480in}{1.123283in}}%
\pgfpathlineto{\pgfqpoint{5.429539in}{1.142705in}}%
\pgfpathlineto{\pgfqpoint{5.433657in}{1.084439in}}%
\pgfpathlineto{\pgfqpoint{5.435716in}{1.094150in}}%
\pgfpathlineto{\pgfqpoint{5.441893in}{1.094150in}}%
\pgfpathlineto{\pgfqpoint{5.443952in}{1.074728in}}%
\pgfpathlineto{\pgfqpoint{5.446011in}{1.084439in}}%
\pgfpathlineto{\pgfqpoint{5.450129in}{1.065017in}}%
\pgfpathlineto{\pgfqpoint{5.456306in}{1.074728in}}%
\pgfpathlineto{\pgfqpoint{5.460423in}{1.074728in}}%
\pgfpathlineto{\pgfqpoint{5.464541in}{1.055306in}}%
\pgfpathlineto{\pgfqpoint{5.470718in}{1.045595in}}%
\pgfpathlineto{\pgfqpoint{5.472777in}{1.103861in}}%
\pgfpathlineto{\pgfqpoint{5.474836in}{1.103861in}}%
\pgfpathlineto{\pgfqpoint{5.476895in}{1.084439in}}%
\pgfpathlineto{\pgfqpoint{5.478954in}{1.103861in}}%
\pgfpathlineto{\pgfqpoint{5.487190in}{1.074728in}}%
\pgfpathlineto{\pgfqpoint{5.489249in}{1.094150in}}%
\pgfpathlineto{\pgfqpoint{5.493367in}{1.055306in}}%
\pgfpathlineto{\pgfqpoint{5.503661in}{1.055306in}}%
\pgfpathlineto{\pgfqpoint{5.507779in}{1.074728in}}%
\pgfpathlineto{\pgfqpoint{5.513956in}{1.074728in}}%
\pgfpathlineto{\pgfqpoint{5.516015in}{1.055306in}}%
\pgfpathlineto{\pgfqpoint{5.518074in}{1.065017in}}%
\pgfpathlineto{\pgfqpoint{5.520133in}{1.055306in}}%
\pgfpathlineto{\pgfqpoint{5.528369in}{1.065017in}}%
\pgfpathlineto{\pgfqpoint{5.530428in}{1.055306in}}%
\pgfpathlineto{\pgfqpoint{5.532487in}{1.055306in}}%
\pgfpathlineto{\pgfqpoint{5.534545in}{1.045595in}}%
\pgfpathlineto{\pgfqpoint{5.534545in}{1.045595in}}%
\pgfusepath{stroke}%
\end{pgfscope}%
\begin{pgfscope}%
\pgfpathrectangle{\pgfqpoint{0.800000in}{0.528000in}}{\pgfqpoint{4.960000in}{3.696000in}}%
\pgfusepath{clip}%
\pgfsetrectcap%
\pgfsetroundjoin%
\pgfsetlinewidth{1.003750pt}%
\definecolor{currentstroke}{rgb}{0.501961,0.501961,0.501961}%
\pgfsetstrokecolor{currentstroke}%
\pgfsetstrokeopacity{0.900000}%
\pgfsetdash{}{0pt}%
\pgfpathmoveto{\pgfqpoint{1.025455in}{2.560509in}}%
\pgfpathlineto{\pgfqpoint{1.031631in}{2.492532in}}%
\pgfpathlineto{\pgfqpoint{1.033690in}{2.424555in}}%
\pgfpathlineto{\pgfqpoint{1.035749in}{2.405133in}}%
\pgfpathlineto{\pgfqpoint{1.037808in}{2.453688in}}%
\pgfpathlineto{\pgfqpoint{1.039867in}{2.395422in}}%
\pgfpathlineto{\pgfqpoint{1.048103in}{2.317734in}}%
\pgfpathlineto{\pgfqpoint{1.050162in}{2.269179in}}%
\pgfpathlineto{\pgfqpoint{1.052221in}{2.181780in}}%
\pgfpathlineto{\pgfqpoint{1.054280in}{2.249757in}}%
\pgfpathlineto{\pgfqpoint{1.062516in}{2.259468in}}%
\pgfpathlineto{\pgfqpoint{1.066633in}{2.337156in}}%
\pgfpathlineto{\pgfqpoint{1.068692in}{2.269179in}}%
\pgfpathlineto{\pgfqpoint{1.074869in}{2.288601in}}%
\pgfpathlineto{\pgfqpoint{1.076928in}{2.269179in}}%
\pgfpathlineto{\pgfqpoint{1.078987in}{2.181780in}}%
\pgfpathlineto{\pgfqpoint{1.081046in}{2.240046in}}%
\pgfpathlineto{\pgfqpoint{1.083105in}{2.142936in}}%
\pgfpathlineto{\pgfqpoint{1.089282in}{2.142936in}}%
\pgfpathlineto{\pgfqpoint{1.091341in}{2.249757in}}%
\pgfpathlineto{\pgfqpoint{1.095459in}{2.269179in}}%
\pgfpathlineto{\pgfqpoint{1.097518in}{2.414844in}}%
\pgfpathlineto{\pgfqpoint{1.103694in}{2.424555in}}%
\pgfpathlineto{\pgfqpoint{1.105753in}{2.463399in}}%
\pgfpathlineto{\pgfqpoint{1.107812in}{2.473110in}}%
\pgfpathlineto{\pgfqpoint{1.109871in}{2.453688in}}%
\pgfpathlineto{\pgfqpoint{1.120166in}{2.589642in}}%
\pgfpathlineto{\pgfqpoint{1.122225in}{2.502243in}}%
\pgfpathlineto{\pgfqpoint{1.124284in}{2.560509in}}%
\pgfpathlineto{\pgfqpoint{1.126343in}{2.579931in}}%
\pgfpathlineto{\pgfqpoint{1.132520in}{2.521665in}}%
\pgfpathlineto{\pgfqpoint{1.134579in}{2.434266in}}%
\pgfpathlineto{\pgfqpoint{1.136638in}{2.424555in}}%
\pgfpathlineto{\pgfqpoint{1.138697in}{2.502243in}}%
\pgfpathlineto{\pgfqpoint{1.140756in}{2.463399in}}%
\pgfpathlineto{\pgfqpoint{1.146932in}{2.531376in}}%
\pgfpathlineto{\pgfqpoint{1.148991in}{2.579931in}}%
\pgfpathlineto{\pgfqpoint{1.151050in}{2.570220in}}%
\pgfpathlineto{\pgfqpoint{1.153109in}{2.541087in}}%
\pgfpathlineto{\pgfqpoint{1.155168in}{2.677040in}}%
\pgfpathlineto{\pgfqpoint{1.161345in}{2.628486in}}%
\pgfpathlineto{\pgfqpoint{1.163404in}{2.579931in}}%
\pgfpathlineto{\pgfqpoint{1.167522in}{2.550798in}}%
\pgfpathlineto{\pgfqpoint{1.169581in}{2.570220in}}%
\pgfpathlineto{\pgfqpoint{1.175758in}{2.541087in}}%
\pgfpathlineto{\pgfqpoint{1.177817in}{2.511954in}}%
\pgfpathlineto{\pgfqpoint{1.179875in}{2.376000in}}%
\pgfpathlineto{\pgfqpoint{1.181934in}{2.434266in}}%
\pgfpathlineto{\pgfqpoint{1.183993in}{2.376000in}}%
\pgfpathlineto{\pgfqpoint{1.190170in}{2.356578in}}%
\pgfpathlineto{\pgfqpoint{1.192229in}{2.327445in}}%
\pgfpathlineto{\pgfqpoint{1.194288in}{2.376000in}}%
\pgfpathlineto{\pgfqpoint{1.196347in}{2.453688in}}%
\pgfpathlineto{\pgfqpoint{1.198406in}{2.385711in}}%
\pgfpathlineto{\pgfqpoint{1.204583in}{2.385711in}}%
\pgfpathlineto{\pgfqpoint{1.206642in}{2.356578in}}%
\pgfpathlineto{\pgfqpoint{1.208701in}{2.298312in}}%
\pgfpathlineto{\pgfqpoint{1.210760in}{2.337156in}}%
\pgfpathlineto{\pgfqpoint{1.212819in}{2.249757in}}%
\pgfpathlineto{\pgfqpoint{1.218995in}{2.317734in}}%
\pgfpathlineto{\pgfqpoint{1.221054in}{2.308023in}}%
\pgfpathlineto{\pgfqpoint{1.223113in}{2.327445in}}%
\pgfpathlineto{\pgfqpoint{1.225172in}{2.376000in}}%
\pgfpathlineto{\pgfqpoint{1.227231in}{2.376000in}}%
\pgfpathlineto{\pgfqpoint{1.233408in}{2.356578in}}%
\pgfpathlineto{\pgfqpoint{1.235467in}{2.317734in}}%
\pgfpathlineto{\pgfqpoint{1.237526in}{2.308023in}}%
\pgfpathlineto{\pgfqpoint{1.241644in}{2.278890in}}%
\pgfpathlineto{\pgfqpoint{1.247821in}{2.298312in}}%
\pgfpathlineto{\pgfqpoint{1.249880in}{2.317734in}}%
\pgfpathlineto{\pgfqpoint{1.251939in}{2.395422in}}%
\pgfpathlineto{\pgfqpoint{1.253998in}{2.346867in}}%
\pgfpathlineto{\pgfqpoint{1.256056in}{2.327445in}}%
\pgfpathlineto{\pgfqpoint{1.262233in}{2.337156in}}%
\pgfpathlineto{\pgfqpoint{1.266351in}{2.443977in}}%
\pgfpathlineto{\pgfqpoint{1.268410in}{2.434266in}}%
\pgfpathlineto{\pgfqpoint{1.270469in}{2.511954in}}%
\pgfpathlineto{\pgfqpoint{1.276646in}{2.541087in}}%
\pgfpathlineto{\pgfqpoint{1.278705in}{2.560509in}}%
\pgfpathlineto{\pgfqpoint{1.280764in}{2.609064in}}%
\pgfpathlineto{\pgfqpoint{1.284882in}{2.531376in}}%
\pgfpathlineto{\pgfqpoint{1.291059in}{2.638197in}}%
\pgfpathlineto{\pgfqpoint{1.295176in}{2.638197in}}%
\pgfpathlineto{\pgfqpoint{1.299294in}{2.511954in}}%
\pgfpathlineto{\pgfqpoint{1.305471in}{2.589642in}}%
\pgfpathlineto{\pgfqpoint{1.307530in}{2.647908in}}%
\pgfpathlineto{\pgfqpoint{1.309589in}{2.618775in}}%
\pgfpathlineto{\pgfqpoint{1.311648in}{2.560509in}}%
\pgfpathlineto{\pgfqpoint{1.313707in}{2.599353in}}%
\pgfpathlineto{\pgfqpoint{1.321943in}{2.541087in}}%
\pgfpathlineto{\pgfqpoint{1.324002in}{2.541087in}}%
\pgfpathlineto{\pgfqpoint{1.326061in}{2.531376in}}%
\pgfpathlineto{\pgfqpoint{1.328120in}{2.502243in}}%
\pgfpathlineto{\pgfqpoint{1.334296in}{2.570220in}}%
\pgfpathlineto{\pgfqpoint{1.338414in}{2.754728in}}%
\pgfpathlineto{\pgfqpoint{1.340473in}{2.686751in}}%
\pgfpathlineto{\pgfqpoint{1.342532in}{2.793572in}}%
\pgfpathlineto{\pgfqpoint{1.348709in}{2.764439in}}%
\pgfpathlineto{\pgfqpoint{1.350768in}{2.793572in}}%
\pgfpathlineto{\pgfqpoint{1.352827in}{2.851838in}}%
\pgfpathlineto{\pgfqpoint{1.354886in}{2.774150in}}%
\pgfpathlineto{\pgfqpoint{1.356945in}{2.783861in}}%
\pgfpathlineto{\pgfqpoint{1.363122in}{2.745017in}}%
\pgfpathlineto{\pgfqpoint{1.367240in}{2.686751in}}%
\pgfpathlineto{\pgfqpoint{1.369298in}{2.715884in}}%
\pgfpathlineto{\pgfqpoint{1.371357in}{2.628486in}}%
\pgfpathlineto{\pgfqpoint{1.379593in}{2.774150in}}%
\pgfpathlineto{\pgfqpoint{1.381652in}{2.754728in}}%
\pgfpathlineto{\pgfqpoint{1.383711in}{2.774150in}}%
\pgfpathlineto{\pgfqpoint{1.385770in}{2.832416in}}%
\pgfpathlineto{\pgfqpoint{1.391947in}{2.686751in}}%
\pgfpathlineto{\pgfqpoint{1.394006in}{2.706173in}}%
\pgfpathlineto{\pgfqpoint{1.396065in}{2.774150in}}%
\pgfpathlineto{\pgfqpoint{1.398124in}{2.725595in}}%
\pgfpathlineto{\pgfqpoint{1.408418in}{2.618775in}}%
\pgfpathlineto{\pgfqpoint{1.410477in}{2.560509in}}%
\pgfpathlineto{\pgfqpoint{1.414595in}{2.754728in}}%
\pgfpathlineto{\pgfqpoint{1.420772in}{2.774150in}}%
\pgfpathlineto{\pgfqpoint{1.424890in}{2.686751in}}%
\pgfpathlineto{\pgfqpoint{1.426949in}{2.706173in}}%
\pgfpathlineto{\pgfqpoint{1.429008in}{2.706173in}}%
\pgfpathlineto{\pgfqpoint{1.435185in}{2.745017in}}%
\pgfpathlineto{\pgfqpoint{1.437244in}{2.715884in}}%
\pgfpathlineto{\pgfqpoint{1.439303in}{2.706173in}}%
\pgfpathlineto{\pgfqpoint{1.441362in}{2.667329in}}%
\pgfpathlineto{\pgfqpoint{1.443421in}{2.657618in}}%
\pgfpathlineto{\pgfqpoint{1.449597in}{2.599353in}}%
\pgfpathlineto{\pgfqpoint{1.453715in}{2.657618in}}%
\pgfpathlineto{\pgfqpoint{1.455774in}{2.657618in}}%
\pgfpathlineto{\pgfqpoint{1.457833in}{2.570220in}}%
\pgfpathlineto{\pgfqpoint{1.464010in}{2.531376in}}%
\pgfpathlineto{\pgfqpoint{1.468128in}{2.657618in}}%
\pgfpathlineto{\pgfqpoint{1.472246in}{2.570220in}}%
\pgfpathlineto{\pgfqpoint{1.478423in}{2.618775in}}%
\pgfpathlineto{\pgfqpoint{1.480482in}{2.531376in}}%
\pgfpathlineto{\pgfqpoint{1.482540in}{2.521665in}}%
\pgfpathlineto{\pgfqpoint{1.484599in}{2.579931in}}%
\pgfpathlineto{\pgfqpoint{1.486658in}{2.599353in}}%
\pgfpathlineto{\pgfqpoint{1.492835in}{2.560509in}}%
\pgfpathlineto{\pgfqpoint{1.494894in}{2.589642in}}%
\pgfpathlineto{\pgfqpoint{1.496953in}{2.492532in}}%
\pgfpathlineto{\pgfqpoint{1.499012in}{2.482821in}}%
\pgfpathlineto{\pgfqpoint{1.501071in}{2.434266in}}%
\pgfpathlineto{\pgfqpoint{1.507248in}{2.395422in}}%
\pgfpathlineto{\pgfqpoint{1.509307in}{2.502243in}}%
\pgfpathlineto{\pgfqpoint{1.511366in}{2.541087in}}%
\pgfpathlineto{\pgfqpoint{1.515484in}{2.560509in}}%
\pgfpathlineto{\pgfqpoint{1.521660in}{2.579931in}}%
\pgfpathlineto{\pgfqpoint{1.523719in}{2.531376in}}%
\pgfpathlineto{\pgfqpoint{1.525778in}{2.560509in}}%
\pgfpathlineto{\pgfqpoint{1.527837in}{2.541087in}}%
\pgfpathlineto{\pgfqpoint{1.529896in}{2.492532in}}%
\pgfpathlineto{\pgfqpoint{1.538132in}{2.560509in}}%
\pgfpathlineto{\pgfqpoint{1.540191in}{2.550798in}}%
\pgfpathlineto{\pgfqpoint{1.542250in}{2.570220in}}%
\pgfpathlineto{\pgfqpoint{1.544309in}{2.531376in}}%
\pgfpathlineto{\pgfqpoint{1.550486in}{2.521665in}}%
\pgfpathlineto{\pgfqpoint{1.552545in}{2.628486in}}%
\pgfpathlineto{\pgfqpoint{1.554604in}{2.647908in}}%
\pgfpathlineto{\pgfqpoint{1.558721in}{2.473110in}}%
\pgfpathlineto{\pgfqpoint{1.564898in}{2.541087in}}%
\pgfpathlineto{\pgfqpoint{1.566957in}{2.473110in}}%
\pgfpathlineto{\pgfqpoint{1.569016in}{2.492532in}}%
\pgfpathlineto{\pgfqpoint{1.571075in}{2.482821in}}%
\pgfpathlineto{\pgfqpoint{1.573134in}{2.511954in}}%
\pgfpathlineto{\pgfqpoint{1.579311in}{2.443977in}}%
\pgfpathlineto{\pgfqpoint{1.581370in}{2.385711in}}%
\pgfpathlineto{\pgfqpoint{1.583429in}{2.395422in}}%
\pgfpathlineto{\pgfqpoint{1.585488in}{2.395422in}}%
\pgfpathlineto{\pgfqpoint{1.587547in}{2.317734in}}%
\pgfpathlineto{\pgfqpoint{1.593724in}{2.385711in}}%
\pgfpathlineto{\pgfqpoint{1.595782in}{2.366289in}}%
\pgfpathlineto{\pgfqpoint{1.601959in}{2.443977in}}%
\pgfpathlineto{\pgfqpoint{1.610195in}{2.385711in}}%
\pgfpathlineto{\pgfqpoint{1.612254in}{2.308023in}}%
\pgfpathlineto{\pgfqpoint{1.614313in}{2.366289in}}%
\pgfpathlineto{\pgfqpoint{1.616372in}{2.376000in}}%
\pgfpathlineto{\pgfqpoint{1.622549in}{2.376000in}}%
\pgfpathlineto{\pgfqpoint{1.624608in}{2.424555in}}%
\pgfpathlineto{\pgfqpoint{1.626667in}{2.385711in}}%
\pgfpathlineto{\pgfqpoint{1.628726in}{2.385711in}}%
\pgfpathlineto{\pgfqpoint{1.630785in}{2.453688in}}%
\pgfpathlineto{\pgfqpoint{1.636961in}{2.424555in}}%
\pgfpathlineto{\pgfqpoint{1.639020in}{2.395422in}}%
\pgfpathlineto{\pgfqpoint{1.643138in}{2.541087in}}%
\pgfpathlineto{\pgfqpoint{1.645197in}{2.521665in}}%
\pgfpathlineto{\pgfqpoint{1.651374in}{2.570220in}}%
\pgfpathlineto{\pgfqpoint{1.655492in}{2.628486in}}%
\pgfpathlineto{\pgfqpoint{1.657551in}{2.638197in}}%
\pgfpathlineto{\pgfqpoint{1.659610in}{2.725595in}}%
\pgfpathlineto{\pgfqpoint{1.665787in}{2.745017in}}%
\pgfpathlineto{\pgfqpoint{1.667846in}{2.715884in}}%
\pgfpathlineto{\pgfqpoint{1.671963in}{2.725595in}}%
\pgfpathlineto{\pgfqpoint{1.674022in}{2.686751in}}%
\pgfpathlineto{\pgfqpoint{1.680199in}{2.657618in}}%
\pgfpathlineto{\pgfqpoint{1.682258in}{2.657618in}}%
\pgfpathlineto{\pgfqpoint{1.684317in}{2.677040in}}%
\pgfpathlineto{\pgfqpoint{1.686376in}{2.657618in}}%
\pgfpathlineto{\pgfqpoint{1.688435in}{2.677040in}}%
\pgfpathlineto{\pgfqpoint{1.694612in}{2.667329in}}%
\pgfpathlineto{\pgfqpoint{1.696671in}{2.647908in}}%
\pgfpathlineto{\pgfqpoint{1.698730in}{2.647908in}}%
\pgfpathlineto{\pgfqpoint{1.702848in}{2.628486in}}%
\pgfpathlineto{\pgfqpoint{1.709024in}{2.628486in}}%
\pgfpathlineto{\pgfqpoint{1.711083in}{2.570220in}}%
\pgfpathlineto{\pgfqpoint{1.713142in}{2.609064in}}%
\pgfpathlineto{\pgfqpoint{1.715201in}{2.735306in}}%
\pgfpathlineto{\pgfqpoint{1.717260in}{2.696462in}}%
\pgfpathlineto{\pgfqpoint{1.723437in}{2.657618in}}%
\pgfpathlineto{\pgfqpoint{1.725496in}{2.657618in}}%
\pgfpathlineto{\pgfqpoint{1.727555in}{2.628486in}}%
\pgfpathlineto{\pgfqpoint{1.729614in}{2.667329in}}%
\pgfpathlineto{\pgfqpoint{1.731673in}{2.550798in}}%
\pgfpathlineto{\pgfqpoint{1.741968in}{2.745017in}}%
\pgfpathlineto{\pgfqpoint{1.746086in}{2.638197in}}%
\pgfpathlineto{\pgfqpoint{1.752262in}{2.638197in}}%
\pgfpathlineto{\pgfqpoint{1.756380in}{2.706173in}}%
\pgfpathlineto{\pgfqpoint{1.758439in}{2.696462in}}%
\pgfpathlineto{\pgfqpoint{1.766675in}{2.686751in}}%
\pgfpathlineto{\pgfqpoint{1.768734in}{2.754728in}}%
\pgfpathlineto{\pgfqpoint{1.770793in}{2.774150in}}%
\pgfpathlineto{\pgfqpoint{1.772852in}{2.725595in}}%
\pgfpathlineto{\pgfqpoint{1.781088in}{2.696462in}}%
\pgfpathlineto{\pgfqpoint{1.783147in}{2.696462in}}%
\pgfpathlineto{\pgfqpoint{1.787264in}{2.579931in}}%
\pgfpathlineto{\pgfqpoint{1.789323in}{2.550798in}}%
\pgfpathlineto{\pgfqpoint{1.795500in}{2.579931in}}%
\pgfpathlineto{\pgfqpoint{1.799618in}{2.492532in}}%
\pgfpathlineto{\pgfqpoint{1.801677in}{2.511954in}}%
\pgfpathlineto{\pgfqpoint{1.803736in}{2.434266in}}%
\pgfpathlineto{\pgfqpoint{1.811972in}{2.463399in}}%
\pgfpathlineto{\pgfqpoint{1.814031in}{2.405133in}}%
\pgfpathlineto{\pgfqpoint{1.816090in}{2.414844in}}%
\pgfpathlineto{\pgfqpoint{1.818149in}{2.453688in}}%
\pgfpathlineto{\pgfqpoint{1.824325in}{2.434266in}}%
\pgfpathlineto{\pgfqpoint{1.826384in}{2.405133in}}%
\pgfpathlineto{\pgfqpoint{1.828443in}{2.405133in}}%
\pgfpathlineto{\pgfqpoint{1.830502in}{2.395422in}}%
\pgfpathlineto{\pgfqpoint{1.832561in}{2.317734in}}%
\pgfpathlineto{\pgfqpoint{1.838738in}{2.366289in}}%
\pgfpathlineto{\pgfqpoint{1.840797in}{2.259468in}}%
\pgfpathlineto{\pgfqpoint{1.842856in}{2.259468in}}%
\pgfpathlineto{\pgfqpoint{1.844915in}{2.249757in}}%
\pgfpathlineto{\pgfqpoint{1.846974in}{2.230335in}}%
\pgfpathlineto{\pgfqpoint{1.853151in}{2.133225in}}%
\pgfpathlineto{\pgfqpoint{1.857269in}{2.113803in}}%
\pgfpathlineto{\pgfqpoint{1.859328in}{2.045827in}}%
\pgfpathlineto{\pgfqpoint{1.861386in}{2.152647in}}%
\pgfpathlineto{\pgfqpoint{1.869622in}{2.181780in}}%
\pgfpathlineto{\pgfqpoint{1.871681in}{2.220624in}}%
\pgfpathlineto{\pgfqpoint{1.873740in}{2.162358in}}%
\pgfpathlineto{\pgfqpoint{1.875799in}{2.181780in}}%
\pgfpathlineto{\pgfqpoint{1.881976in}{2.191491in}}%
\pgfpathlineto{\pgfqpoint{1.884035in}{2.162358in}}%
\pgfpathlineto{\pgfqpoint{1.886094in}{2.172069in}}%
\pgfpathlineto{\pgfqpoint{1.888153in}{2.123514in}}%
\pgfpathlineto{\pgfqpoint{1.890212in}{2.201202in}}%
\pgfpathlineto{\pgfqpoint{1.896389in}{2.172069in}}%
\pgfpathlineto{\pgfqpoint{1.898447in}{2.269179in}}%
\pgfpathlineto{\pgfqpoint{1.900506in}{2.298312in}}%
\pgfpathlineto{\pgfqpoint{1.902565in}{2.278890in}}%
\pgfpathlineto{\pgfqpoint{1.904624in}{2.337156in}}%
\pgfpathlineto{\pgfqpoint{1.910801in}{2.366289in}}%
\pgfpathlineto{\pgfqpoint{1.912860in}{2.288601in}}%
\pgfpathlineto{\pgfqpoint{1.919037in}{2.434266in}}%
\pgfpathlineto{\pgfqpoint{1.925214in}{2.424555in}}%
\pgfpathlineto{\pgfqpoint{1.927273in}{2.424555in}}%
\pgfpathlineto{\pgfqpoint{1.929332in}{2.366289in}}%
\pgfpathlineto{\pgfqpoint{1.931391in}{2.346867in}}%
\pgfpathlineto{\pgfqpoint{1.933450in}{2.308023in}}%
\pgfpathlineto{\pgfqpoint{1.939626in}{2.346867in}}%
\pgfpathlineto{\pgfqpoint{1.941685in}{2.385711in}}%
\pgfpathlineto{\pgfqpoint{1.943744in}{2.317734in}}%
\pgfpathlineto{\pgfqpoint{1.945803in}{2.346867in}}%
\pgfpathlineto{\pgfqpoint{1.954039in}{2.327445in}}%
\pgfpathlineto{\pgfqpoint{1.956098in}{2.240046in}}%
\pgfpathlineto{\pgfqpoint{1.958157in}{2.249757in}}%
\pgfpathlineto{\pgfqpoint{1.960216in}{2.191491in}}%
\pgfpathlineto{\pgfqpoint{1.962275in}{2.210913in}}%
\pgfpathlineto{\pgfqpoint{1.968452in}{2.181780in}}%
\pgfpathlineto{\pgfqpoint{1.970511in}{2.142936in}}%
\pgfpathlineto{\pgfqpoint{1.972570in}{2.172069in}}%
\pgfpathlineto{\pgfqpoint{1.974628in}{2.113803in}}%
\pgfpathlineto{\pgfqpoint{1.976687in}{2.123514in}}%
\pgfpathlineto{\pgfqpoint{1.982864in}{2.133225in}}%
\pgfpathlineto{\pgfqpoint{1.984923in}{2.191491in}}%
\pgfpathlineto{\pgfqpoint{1.986982in}{2.181780in}}%
\pgfpathlineto{\pgfqpoint{1.989041in}{2.220624in}}%
\pgfpathlineto{\pgfqpoint{1.991100in}{2.172069in}}%
\pgfpathlineto{\pgfqpoint{1.997277in}{2.191491in}}%
\pgfpathlineto{\pgfqpoint{1.999336in}{2.220624in}}%
\pgfpathlineto{\pgfqpoint{2.001395in}{2.278890in}}%
\pgfpathlineto{\pgfqpoint{2.005513in}{2.317734in}}%
\pgfpathlineto{\pgfqpoint{2.011689in}{2.337156in}}%
\pgfpathlineto{\pgfqpoint{2.013748in}{2.366289in}}%
\pgfpathlineto{\pgfqpoint{2.017866in}{2.249757in}}%
\pgfpathlineto{\pgfqpoint{2.019925in}{2.249757in}}%
\pgfpathlineto{\pgfqpoint{2.026102in}{2.288601in}}%
\pgfpathlineto{\pgfqpoint{2.028161in}{2.220624in}}%
\pgfpathlineto{\pgfqpoint{2.032279in}{2.172069in}}%
\pgfpathlineto{\pgfqpoint{2.034338in}{2.201202in}}%
\pgfpathlineto{\pgfqpoint{2.040515in}{2.162358in}}%
\pgfpathlineto{\pgfqpoint{2.042574in}{2.172069in}}%
\pgfpathlineto{\pgfqpoint{2.044633in}{2.162358in}}%
\pgfpathlineto{\pgfqpoint{2.046692in}{2.191491in}}%
\pgfpathlineto{\pgfqpoint{2.048751in}{2.162358in}}%
\pgfpathlineto{\pgfqpoint{2.056986in}{2.220624in}}%
\pgfpathlineto{\pgfqpoint{2.059045in}{2.337156in}}%
\pgfpathlineto{\pgfqpoint{2.063163in}{2.298312in}}%
\pgfpathlineto{\pgfqpoint{2.069340in}{2.298312in}}%
\pgfpathlineto{\pgfqpoint{2.071399in}{2.327445in}}%
\pgfpathlineto{\pgfqpoint{2.073458in}{2.337156in}}%
\pgfpathlineto{\pgfqpoint{2.075517in}{2.298312in}}%
\pgfpathlineto{\pgfqpoint{2.077576in}{2.317734in}}%
\pgfpathlineto{\pgfqpoint{2.085812in}{2.308023in}}%
\pgfpathlineto{\pgfqpoint{2.087870in}{2.317734in}}%
\pgfpathlineto{\pgfqpoint{2.089929in}{2.278890in}}%
\pgfpathlineto{\pgfqpoint{2.091988in}{2.152647in}}%
\pgfpathlineto{\pgfqpoint{2.098165in}{2.181780in}}%
\pgfpathlineto{\pgfqpoint{2.100224in}{2.162358in}}%
\pgfpathlineto{\pgfqpoint{2.102283in}{2.162358in}}%
\pgfpathlineto{\pgfqpoint{2.104342in}{2.142936in}}%
\pgfpathlineto{\pgfqpoint{2.106401in}{2.094382in}}%
\pgfpathlineto{\pgfqpoint{2.112578in}{2.074960in}}%
\pgfpathlineto{\pgfqpoint{2.114637in}{2.074960in}}%
\pgfpathlineto{\pgfqpoint{2.116696in}{2.036116in}}%
\pgfpathlineto{\pgfqpoint{2.118755in}{2.026405in}}%
\pgfpathlineto{\pgfqpoint{2.120814in}{2.065249in}}%
\pgfpathlineto{\pgfqpoint{2.126990in}{2.104092in}}%
\pgfpathlineto{\pgfqpoint{2.129049in}{2.142936in}}%
\pgfpathlineto{\pgfqpoint{2.131108in}{2.142936in}}%
\pgfpathlineto{\pgfqpoint{2.133167in}{2.191491in}}%
\pgfpathlineto{\pgfqpoint{2.135226in}{2.006983in}}%
\pgfpathlineto{\pgfqpoint{2.141403in}{1.919584in}}%
\pgfpathlineto{\pgfqpoint{2.143462in}{1.919584in}}%
\pgfpathlineto{\pgfqpoint{2.145521in}{1.958428in}}%
\pgfpathlineto{\pgfqpoint{2.147580in}{1.948717in}}%
\pgfpathlineto{\pgfqpoint{2.157875in}{1.851607in}}%
\pgfpathlineto{\pgfqpoint{2.161993in}{1.871029in}}%
\pgfpathlineto{\pgfqpoint{2.164051in}{1.851607in}}%
\pgfpathlineto{\pgfqpoint{2.170228in}{1.929295in}}%
\pgfpathlineto{\pgfqpoint{2.172287in}{2.006983in}}%
\pgfpathlineto{\pgfqpoint{2.174346in}{1.977850in}}%
\pgfpathlineto{\pgfqpoint{2.178464in}{2.074960in}}%
\pgfpathlineto{\pgfqpoint{2.184641in}{2.055538in}}%
\pgfpathlineto{\pgfqpoint{2.186700in}{2.036116in}}%
\pgfpathlineto{\pgfqpoint{2.188759in}{2.065249in}}%
\pgfpathlineto{\pgfqpoint{2.190818in}{2.036116in}}%
\pgfpathlineto{\pgfqpoint{2.192877in}{2.055538in}}%
\pgfpathlineto{\pgfqpoint{2.199054in}{2.065249in}}%
\pgfpathlineto{\pgfqpoint{2.201112in}{2.055538in}}%
\pgfpathlineto{\pgfqpoint{2.203171in}{2.006983in}}%
\pgfpathlineto{\pgfqpoint{2.205230in}{2.006983in}}%
\pgfpathlineto{\pgfqpoint{2.207289in}{1.948717in}}%
\pgfpathlineto{\pgfqpoint{2.213466in}{1.987561in}}%
\pgfpathlineto{\pgfqpoint{2.215525in}{2.016694in}}%
\pgfpathlineto{\pgfqpoint{2.217584in}{2.006983in}}%
\pgfpathlineto{\pgfqpoint{2.219643in}{1.968139in}}%
\pgfpathlineto{\pgfqpoint{2.221702in}{2.065249in}}%
\pgfpathlineto{\pgfqpoint{2.227879in}{2.074960in}}%
\pgfpathlineto{\pgfqpoint{2.231997in}{1.997272in}}%
\pgfpathlineto{\pgfqpoint{2.234056in}{2.074960in}}%
\pgfpathlineto{\pgfqpoint{2.236115in}{2.016694in}}%
\pgfpathlineto{\pgfqpoint{2.244350in}{2.074960in}}%
\pgfpathlineto{\pgfqpoint{2.246409in}{2.065249in}}%
\pgfpathlineto{\pgfqpoint{2.248468in}{2.036116in}}%
\pgfpathlineto{\pgfqpoint{2.250527in}{2.084671in}}%
\pgfpathlineto{\pgfqpoint{2.256704in}{2.055538in}}%
\pgfpathlineto{\pgfqpoint{2.260822in}{2.055538in}}%
\pgfpathlineto{\pgfqpoint{2.262881in}{2.084671in}}%
\pgfpathlineto{\pgfqpoint{2.264940in}{2.142936in}}%
\pgfpathlineto{\pgfqpoint{2.271117in}{2.084671in}}%
\pgfpathlineto{\pgfqpoint{2.275235in}{2.104092in}}%
\pgfpathlineto{\pgfqpoint{2.277293in}{2.094382in}}%
\pgfpathlineto{\pgfqpoint{2.279352in}{2.123514in}}%
\pgfpathlineto{\pgfqpoint{2.289647in}{2.045827in}}%
\pgfpathlineto{\pgfqpoint{2.293765in}{2.162358in}}%
\pgfpathlineto{\pgfqpoint{2.299942in}{2.152647in}}%
\pgfpathlineto{\pgfqpoint{2.302001in}{2.210913in}}%
\pgfpathlineto{\pgfqpoint{2.304060in}{2.172069in}}%
\pgfpathlineto{\pgfqpoint{2.306119in}{2.162358in}}%
\pgfpathlineto{\pgfqpoint{2.308178in}{2.162358in}}%
\pgfpathlineto{\pgfqpoint{2.314355in}{2.172069in}}%
\pgfpathlineto{\pgfqpoint{2.322590in}{2.094382in}}%
\pgfpathlineto{\pgfqpoint{2.328767in}{2.065249in}}%
\pgfpathlineto{\pgfqpoint{2.330826in}{2.045827in}}%
\pgfpathlineto{\pgfqpoint{2.332885in}{2.065249in}}%
\pgfpathlineto{\pgfqpoint{2.334944in}{2.045827in}}%
\pgfpathlineto{\pgfqpoint{2.337003in}{2.074960in}}%
\pgfpathlineto{\pgfqpoint{2.343180in}{2.113803in}}%
\pgfpathlineto{\pgfqpoint{2.347298in}{2.191491in}}%
\pgfpathlineto{\pgfqpoint{2.349357in}{2.230335in}}%
\pgfpathlineto{\pgfqpoint{2.351416in}{2.201202in}}%
\pgfpathlineto{\pgfqpoint{2.359651in}{2.230335in}}%
\pgfpathlineto{\pgfqpoint{2.361710in}{2.249757in}}%
\pgfpathlineto{\pgfqpoint{2.363769in}{2.210913in}}%
\pgfpathlineto{\pgfqpoint{2.365828in}{2.230335in}}%
\pgfpathlineto{\pgfqpoint{2.372005in}{2.210913in}}%
\pgfpathlineto{\pgfqpoint{2.374064in}{2.191491in}}%
\pgfpathlineto{\pgfqpoint{2.376123in}{2.191491in}}%
\pgfpathlineto{\pgfqpoint{2.378182in}{2.201202in}}%
\pgfpathlineto{\pgfqpoint{2.380241in}{2.181780in}}%
\pgfpathlineto{\pgfqpoint{2.386418in}{2.210913in}}%
\pgfpathlineto{\pgfqpoint{2.388477in}{2.210913in}}%
\pgfpathlineto{\pgfqpoint{2.390535in}{2.240046in}}%
\pgfpathlineto{\pgfqpoint{2.392594in}{2.288601in}}%
\pgfpathlineto{\pgfqpoint{2.394653in}{2.278890in}}%
\pgfpathlineto{\pgfqpoint{2.400830in}{2.269179in}}%
\pgfpathlineto{\pgfqpoint{2.402889in}{2.259468in}}%
\pgfpathlineto{\pgfqpoint{2.404948in}{2.220624in}}%
\pgfpathlineto{\pgfqpoint{2.407007in}{2.230335in}}%
\pgfpathlineto{\pgfqpoint{2.409066in}{2.201202in}}%
\pgfpathlineto{\pgfqpoint{2.415243in}{2.249757in}}%
\pgfpathlineto{\pgfqpoint{2.417302in}{2.298312in}}%
\pgfpathlineto{\pgfqpoint{2.419361in}{2.482821in}}%
\pgfpathlineto{\pgfqpoint{2.421420in}{2.560509in}}%
\pgfpathlineto{\pgfqpoint{2.431714in}{2.667329in}}%
\pgfpathlineto{\pgfqpoint{2.433773in}{2.667329in}}%
\pgfpathlineto{\pgfqpoint{2.437891in}{2.774150in}}%
\pgfpathlineto{\pgfqpoint{2.444068in}{2.764439in}}%
\pgfpathlineto{\pgfqpoint{2.446127in}{2.754728in}}%
\pgfpathlineto{\pgfqpoint{2.448186in}{2.803283in}}%
\pgfpathlineto{\pgfqpoint{2.452304in}{2.812994in}}%
\pgfpathlineto{\pgfqpoint{2.460540in}{2.754728in}}%
\pgfpathlineto{\pgfqpoint{2.464658in}{2.880971in}}%
\pgfpathlineto{\pgfqpoint{2.466716in}{2.832416in}}%
\pgfpathlineto{\pgfqpoint{2.472893in}{2.822705in}}%
\pgfpathlineto{\pgfqpoint{2.474952in}{2.812994in}}%
\pgfpathlineto{\pgfqpoint{2.477011in}{2.774150in}}%
\pgfpathlineto{\pgfqpoint{2.481129in}{2.890682in}}%
\pgfpathlineto{\pgfqpoint{2.489365in}{2.890682in}}%
\pgfpathlineto{\pgfqpoint{2.493483in}{3.046058in}}%
\pgfpathlineto{\pgfqpoint{2.495542in}{3.036347in}}%
\pgfpathlineto{\pgfqpoint{2.501719in}{2.978081in}}%
\pgfpathlineto{\pgfqpoint{2.503778in}{3.007214in}}%
\pgfpathlineto{\pgfqpoint{2.505836in}{2.978081in}}%
\pgfpathlineto{\pgfqpoint{2.507895in}{2.987792in}}%
\pgfpathlineto{\pgfqpoint{2.509954in}{2.978081in}}%
\pgfpathlineto{\pgfqpoint{2.518190in}{2.997503in}}%
\pgfpathlineto{\pgfqpoint{2.520249in}{2.948948in}}%
\pgfpathlineto{\pgfqpoint{2.522308in}{2.929526in}}%
\pgfpathlineto{\pgfqpoint{2.524367in}{2.880971in}}%
\pgfpathlineto{\pgfqpoint{2.532603in}{2.890682in}}%
\pgfpathlineto{\pgfqpoint{2.534662in}{2.890682in}}%
\pgfpathlineto{\pgfqpoint{2.536721in}{2.812994in}}%
\pgfpathlineto{\pgfqpoint{2.538780in}{2.861549in}}%
\pgfpathlineto{\pgfqpoint{2.544956in}{2.812994in}}%
\pgfpathlineto{\pgfqpoint{2.549074in}{2.812994in}}%
\pgfpathlineto{\pgfqpoint{2.551133in}{2.803283in}}%
\pgfpathlineto{\pgfqpoint{2.553192in}{2.832416in}}%
\pgfpathlineto{\pgfqpoint{2.561428in}{2.774150in}}%
\pgfpathlineto{\pgfqpoint{2.563487in}{2.871260in}}%
\pgfpathlineto{\pgfqpoint{2.565546in}{2.910104in}}%
\pgfpathlineto{\pgfqpoint{2.567605in}{2.910104in}}%
\pgfpathlineto{\pgfqpoint{2.573782in}{2.822705in}}%
\pgfpathlineto{\pgfqpoint{2.577900in}{2.958659in}}%
\pgfpathlineto{\pgfqpoint{2.582017in}{2.910104in}}%
\pgfpathlineto{\pgfqpoint{2.588194in}{2.910104in}}%
\pgfpathlineto{\pgfqpoint{2.590253in}{2.871260in}}%
\pgfpathlineto{\pgfqpoint{2.592312in}{2.900393in}}%
\pgfpathlineto{\pgfqpoint{2.596430in}{2.900393in}}%
\pgfpathlineto{\pgfqpoint{2.606725in}{2.774150in}}%
\pgfpathlineto{\pgfqpoint{2.608784in}{2.832416in}}%
\pgfpathlineto{\pgfqpoint{2.610843in}{2.851838in}}%
\pgfpathlineto{\pgfqpoint{2.617020in}{2.871260in}}%
\pgfpathlineto{\pgfqpoint{2.621137in}{2.958659in}}%
\pgfpathlineto{\pgfqpoint{2.623196in}{2.890682in}}%
\pgfpathlineto{\pgfqpoint{2.625255in}{2.861549in}}%
\pgfpathlineto{\pgfqpoint{2.633491in}{2.871260in}}%
\pgfpathlineto{\pgfqpoint{2.635550in}{2.861549in}}%
\pgfpathlineto{\pgfqpoint{2.637609in}{2.832416in}}%
\pgfpathlineto{\pgfqpoint{2.639668in}{2.754728in}}%
\pgfpathlineto{\pgfqpoint{2.647904in}{2.822705in}}%
\pgfpathlineto{\pgfqpoint{2.649963in}{2.919815in}}%
\pgfpathlineto{\pgfqpoint{2.652022in}{2.948948in}}%
\pgfpathlineto{\pgfqpoint{2.660257in}{2.948948in}}%
\pgfpathlineto{\pgfqpoint{2.662316in}{2.968370in}}%
\pgfpathlineto{\pgfqpoint{2.666434in}{3.055769in}}%
\pgfpathlineto{\pgfqpoint{2.668493in}{3.026636in}}%
\pgfpathlineto{\pgfqpoint{2.674670in}{3.055769in}}%
\pgfpathlineto{\pgfqpoint{2.676729in}{3.046058in}}%
\pgfpathlineto{\pgfqpoint{2.678788in}{2.939237in}}%
\pgfpathlineto{\pgfqpoint{2.680847in}{2.968370in}}%
\pgfpathlineto{\pgfqpoint{2.682906in}{2.939237in}}%
\pgfpathlineto{\pgfqpoint{2.689083in}{2.910104in}}%
\pgfpathlineto{\pgfqpoint{2.693200in}{2.851838in}}%
\pgfpathlineto{\pgfqpoint{2.695259in}{2.861549in}}%
\pgfpathlineto{\pgfqpoint{2.697318in}{2.851838in}}%
\pgfpathlineto{\pgfqpoint{2.703495in}{2.832416in}}%
\pgfpathlineto{\pgfqpoint{2.705554in}{2.880971in}}%
\pgfpathlineto{\pgfqpoint{2.707613in}{2.842127in}}%
\pgfpathlineto{\pgfqpoint{2.709672in}{2.880971in}}%
\pgfpathlineto{\pgfqpoint{2.711731in}{2.851838in}}%
\pgfpathlineto{\pgfqpoint{2.717908in}{2.793572in}}%
\pgfpathlineto{\pgfqpoint{2.719967in}{2.793572in}}%
\pgfpathlineto{\pgfqpoint{2.722026in}{2.774150in}}%
\pgfpathlineto{\pgfqpoint{2.724085in}{2.783861in}}%
\pgfpathlineto{\pgfqpoint{2.726144in}{2.832416in}}%
\pgfpathlineto{\pgfqpoint{2.732320in}{2.812994in}}%
\pgfpathlineto{\pgfqpoint{2.734379in}{2.745017in}}%
\pgfpathlineto{\pgfqpoint{2.736438in}{2.725595in}}%
\pgfpathlineto{\pgfqpoint{2.738497in}{2.686751in}}%
\pgfpathlineto{\pgfqpoint{2.746733in}{2.706173in}}%
\pgfpathlineto{\pgfqpoint{2.748792in}{2.618775in}}%
\pgfpathlineto{\pgfqpoint{2.752910in}{2.696462in}}%
\pgfpathlineto{\pgfqpoint{2.754969in}{2.686751in}}%
\pgfpathlineto{\pgfqpoint{2.761146in}{2.725595in}}%
\pgfpathlineto{\pgfqpoint{2.763205in}{2.783861in}}%
\pgfpathlineto{\pgfqpoint{2.767323in}{2.735306in}}%
\pgfpathlineto{\pgfqpoint{2.769381in}{2.735306in}}%
\pgfpathlineto{\pgfqpoint{2.775558in}{2.764439in}}%
\pgfpathlineto{\pgfqpoint{2.777617in}{2.725595in}}%
\pgfpathlineto{\pgfqpoint{2.781735in}{2.803283in}}%
\pgfpathlineto{\pgfqpoint{2.783794in}{2.803283in}}%
\pgfpathlineto{\pgfqpoint{2.789971in}{2.822705in}}%
\pgfpathlineto{\pgfqpoint{2.792030in}{2.851838in}}%
\pgfpathlineto{\pgfqpoint{2.794089in}{2.851838in}}%
\pgfpathlineto{\pgfqpoint{2.796148in}{2.832416in}}%
\pgfpathlineto{\pgfqpoint{2.798207in}{2.764439in}}%
\pgfpathlineto{\pgfqpoint{2.804384in}{2.774150in}}%
\pgfpathlineto{\pgfqpoint{2.806443in}{2.764439in}}%
\pgfpathlineto{\pgfqpoint{2.808501in}{2.667329in}}%
\pgfpathlineto{\pgfqpoint{2.812619in}{2.686751in}}%
\pgfpathlineto{\pgfqpoint{2.818796in}{2.696462in}}%
\pgfpathlineto{\pgfqpoint{2.820855in}{2.735306in}}%
\pgfpathlineto{\pgfqpoint{2.822914in}{2.706173in}}%
\pgfpathlineto{\pgfqpoint{2.824973in}{2.696462in}}%
\pgfpathlineto{\pgfqpoint{2.827032in}{2.696462in}}%
\pgfpathlineto{\pgfqpoint{2.835268in}{2.657618in}}%
\pgfpathlineto{\pgfqpoint{2.839386in}{2.657618in}}%
\pgfpathlineto{\pgfqpoint{2.841445in}{2.599353in}}%
\pgfpathlineto{\pgfqpoint{2.847621in}{2.628486in}}%
\pgfpathlineto{\pgfqpoint{2.849680in}{2.589642in}}%
\pgfpathlineto{\pgfqpoint{2.851739in}{2.628486in}}%
\pgfpathlineto{\pgfqpoint{2.853798in}{2.638197in}}%
\pgfpathlineto{\pgfqpoint{2.855857in}{2.657618in}}%
\pgfpathlineto{\pgfqpoint{2.864093in}{2.657618in}}%
\pgfpathlineto{\pgfqpoint{2.866152in}{2.599353in}}%
\pgfpathlineto{\pgfqpoint{2.868211in}{2.618775in}}%
\pgfpathlineto{\pgfqpoint{2.870270in}{2.609064in}}%
\pgfpathlineto{\pgfqpoint{2.876447in}{2.657618in}}%
\pgfpathlineto{\pgfqpoint{2.878506in}{2.628486in}}%
\pgfpathlineto{\pgfqpoint{2.880565in}{2.638197in}}%
\pgfpathlineto{\pgfqpoint{2.882623in}{2.618775in}}%
\pgfpathlineto{\pgfqpoint{2.884682in}{2.618775in}}%
\pgfpathlineto{\pgfqpoint{2.890859in}{2.609064in}}%
\pgfpathlineto{\pgfqpoint{2.892918in}{2.677040in}}%
\pgfpathlineto{\pgfqpoint{2.894977in}{2.686751in}}%
\pgfpathlineto{\pgfqpoint{2.899095in}{2.774150in}}%
\pgfpathlineto{\pgfqpoint{2.905272in}{2.822705in}}%
\pgfpathlineto{\pgfqpoint{2.909390in}{2.803283in}}%
\pgfpathlineto{\pgfqpoint{2.911449in}{2.842127in}}%
\pgfpathlineto{\pgfqpoint{2.913508in}{2.851838in}}%
\pgfpathlineto{\pgfqpoint{2.919685in}{2.832416in}}%
\pgfpathlineto{\pgfqpoint{2.921743in}{2.812994in}}%
\pgfpathlineto{\pgfqpoint{2.923802in}{2.774150in}}%
\pgfpathlineto{\pgfqpoint{2.925861in}{2.793572in}}%
\pgfpathlineto{\pgfqpoint{2.927920in}{2.764439in}}%
\pgfpathlineto{\pgfqpoint{2.934097in}{2.754728in}}%
\pgfpathlineto{\pgfqpoint{2.936156in}{2.715884in}}%
\pgfpathlineto{\pgfqpoint{2.938215in}{2.725595in}}%
\pgfpathlineto{\pgfqpoint{2.940274in}{2.715884in}}%
\pgfpathlineto{\pgfqpoint{2.942333in}{2.686751in}}%
\pgfpathlineto{\pgfqpoint{2.948510in}{2.706173in}}%
\pgfpathlineto{\pgfqpoint{2.950569in}{2.783861in}}%
\pgfpathlineto{\pgfqpoint{2.952628in}{2.725595in}}%
\pgfpathlineto{\pgfqpoint{2.954687in}{2.754728in}}%
\pgfpathlineto{\pgfqpoint{2.956746in}{2.735306in}}%
\pgfpathlineto{\pgfqpoint{2.962922in}{2.745017in}}%
\pgfpathlineto{\pgfqpoint{2.964981in}{2.706173in}}%
\pgfpathlineto{\pgfqpoint{2.967040in}{2.715884in}}%
\pgfpathlineto{\pgfqpoint{2.969099in}{2.686751in}}%
\pgfpathlineto{\pgfqpoint{2.971158in}{2.715884in}}%
\pgfpathlineto{\pgfqpoint{2.977335in}{2.706173in}}%
\pgfpathlineto{\pgfqpoint{2.979394in}{2.735306in}}%
\pgfpathlineto{\pgfqpoint{2.985571in}{2.638197in}}%
\pgfpathlineto{\pgfqpoint{2.991748in}{2.677040in}}%
\pgfpathlineto{\pgfqpoint{2.993807in}{2.725595in}}%
\pgfpathlineto{\pgfqpoint{2.997924in}{2.647908in}}%
\pgfpathlineto{\pgfqpoint{2.999983in}{2.647908in}}%
\pgfpathlineto{\pgfqpoint{3.006160in}{2.638197in}}%
\pgfpathlineto{\pgfqpoint{3.008219in}{2.677040in}}%
\pgfpathlineto{\pgfqpoint{3.010278in}{2.628486in}}%
\pgfpathlineto{\pgfqpoint{3.012337in}{2.647908in}}%
\pgfpathlineto{\pgfqpoint{3.014396in}{2.638197in}}%
\pgfpathlineto{\pgfqpoint{3.020573in}{2.628486in}}%
\pgfpathlineto{\pgfqpoint{3.022632in}{2.599353in}}%
\pgfpathlineto{\pgfqpoint{3.024691in}{2.609064in}}%
\pgfpathlineto{\pgfqpoint{3.026750in}{2.589642in}}%
\pgfpathlineto{\pgfqpoint{3.028809in}{2.628486in}}%
\pgfpathlineto{\pgfqpoint{3.037044in}{2.541087in}}%
\pgfpathlineto{\pgfqpoint{3.039103in}{2.570220in}}%
\pgfpathlineto{\pgfqpoint{3.041162in}{2.521665in}}%
\pgfpathlineto{\pgfqpoint{3.043221in}{2.531376in}}%
\pgfpathlineto{\pgfqpoint{3.053516in}{2.647908in}}%
\pgfpathlineto{\pgfqpoint{3.055575in}{2.647908in}}%
\pgfpathlineto{\pgfqpoint{3.057634in}{2.677040in}}%
\pgfpathlineto{\pgfqpoint{3.065870in}{2.706173in}}%
\pgfpathlineto{\pgfqpoint{3.067929in}{2.754728in}}%
\pgfpathlineto{\pgfqpoint{3.078223in}{2.706173in}}%
\pgfpathlineto{\pgfqpoint{3.080282in}{2.715884in}}%
\pgfpathlineto{\pgfqpoint{3.082341in}{2.774150in}}%
\pgfpathlineto{\pgfqpoint{3.084400in}{2.764439in}}%
\pgfpathlineto{\pgfqpoint{3.086459in}{2.793572in}}%
\pgfpathlineto{\pgfqpoint{3.092636in}{2.803283in}}%
\pgfpathlineto{\pgfqpoint{3.094695in}{2.783861in}}%
\pgfpathlineto{\pgfqpoint{3.096754in}{2.783861in}}%
\pgfpathlineto{\pgfqpoint{3.100872in}{2.832416in}}%
\pgfpathlineto{\pgfqpoint{3.109108in}{2.812994in}}%
\pgfpathlineto{\pgfqpoint{3.113225in}{2.793572in}}%
\pgfpathlineto{\pgfqpoint{3.115284in}{2.754728in}}%
\pgfpathlineto{\pgfqpoint{3.121461in}{2.783861in}}%
\pgfpathlineto{\pgfqpoint{3.123520in}{2.783861in}}%
\pgfpathlineto{\pgfqpoint{3.125579in}{2.822705in}}%
\pgfpathlineto{\pgfqpoint{3.127638in}{2.812994in}}%
\pgfpathlineto{\pgfqpoint{3.129697in}{2.871260in}}%
\pgfpathlineto{\pgfqpoint{3.135874in}{2.851838in}}%
\pgfpathlineto{\pgfqpoint{3.139992in}{2.910104in}}%
\pgfpathlineto{\pgfqpoint{3.142051in}{2.929526in}}%
\pgfpathlineto{\pgfqpoint{3.144110in}{2.890682in}}%
\pgfpathlineto{\pgfqpoint{3.150286in}{2.851838in}}%
\pgfpathlineto{\pgfqpoint{3.152345in}{2.861549in}}%
\pgfpathlineto{\pgfqpoint{3.156463in}{2.842127in}}%
\pgfpathlineto{\pgfqpoint{3.158522in}{2.822705in}}%
\pgfpathlineto{\pgfqpoint{3.164699in}{2.803283in}}%
\pgfpathlineto{\pgfqpoint{3.166758in}{2.803283in}}%
\pgfpathlineto{\pgfqpoint{3.170876in}{2.832416in}}%
\pgfpathlineto{\pgfqpoint{3.172935in}{2.900393in}}%
\pgfpathlineto{\pgfqpoint{3.179112in}{2.900393in}}%
\pgfpathlineto{\pgfqpoint{3.181171in}{2.890682in}}%
\pgfpathlineto{\pgfqpoint{3.183230in}{2.842127in}}%
\pgfpathlineto{\pgfqpoint{3.185289in}{2.880971in}}%
\pgfpathlineto{\pgfqpoint{3.187347in}{2.861549in}}%
\pgfpathlineto{\pgfqpoint{3.195583in}{2.900393in}}%
\pgfpathlineto{\pgfqpoint{3.197642in}{2.851838in}}%
\pgfpathlineto{\pgfqpoint{3.201760in}{2.861549in}}%
\pgfpathlineto{\pgfqpoint{3.207937in}{2.842127in}}%
\pgfpathlineto{\pgfqpoint{3.214114in}{2.939237in}}%
\pgfpathlineto{\pgfqpoint{3.216173in}{2.910104in}}%
\pgfpathlineto{\pgfqpoint{3.222350in}{2.919815in}}%
\pgfpathlineto{\pgfqpoint{3.224408in}{2.910104in}}%
\pgfpathlineto{\pgfqpoint{3.226467in}{2.880971in}}%
\pgfpathlineto{\pgfqpoint{3.228526in}{2.919815in}}%
\pgfpathlineto{\pgfqpoint{3.230585in}{2.919815in}}%
\pgfpathlineto{\pgfqpoint{3.236762in}{2.929526in}}%
\pgfpathlineto{\pgfqpoint{3.238821in}{2.948948in}}%
\pgfpathlineto{\pgfqpoint{3.240880in}{2.890682in}}%
\pgfpathlineto{\pgfqpoint{3.244998in}{2.910104in}}%
\pgfpathlineto{\pgfqpoint{3.251175in}{2.929526in}}%
\pgfpathlineto{\pgfqpoint{3.253234in}{2.997503in}}%
\pgfpathlineto{\pgfqpoint{3.255293in}{3.026636in}}%
\pgfpathlineto{\pgfqpoint{3.257352in}{3.016925in}}%
\pgfpathlineto{\pgfqpoint{3.259411in}{3.026636in}}%
\pgfpathlineto{\pgfqpoint{3.267646in}{3.007214in}}%
\pgfpathlineto{\pgfqpoint{3.269705in}{2.968370in}}%
\pgfpathlineto{\pgfqpoint{3.271764in}{2.987792in}}%
\pgfpathlineto{\pgfqpoint{3.273823in}{2.958659in}}%
\pgfpathlineto{\pgfqpoint{3.282059in}{3.007214in}}%
\pgfpathlineto{\pgfqpoint{3.284118in}{2.997503in}}%
\pgfpathlineto{\pgfqpoint{3.286177in}{3.007214in}}%
\pgfpathlineto{\pgfqpoint{3.288236in}{3.026636in}}%
\pgfpathlineto{\pgfqpoint{3.294413in}{3.036347in}}%
\pgfpathlineto{\pgfqpoint{3.296472in}{3.084902in}}%
\pgfpathlineto{\pgfqpoint{3.298531in}{3.094613in}}%
\pgfpathlineto{\pgfqpoint{3.300589in}{3.084902in}}%
\pgfpathlineto{\pgfqpoint{3.302648in}{3.104324in}}%
\pgfpathlineto{\pgfqpoint{3.310884in}{3.104324in}}%
\pgfpathlineto{\pgfqpoint{3.317061in}{3.191723in}}%
\pgfpathlineto{\pgfqpoint{3.323238in}{3.211145in}}%
\pgfpathlineto{\pgfqpoint{3.325297in}{3.172301in}}%
\pgfpathlineto{\pgfqpoint{3.327356in}{3.191723in}}%
\pgfpathlineto{\pgfqpoint{3.329415in}{3.172301in}}%
\pgfpathlineto{\pgfqpoint{3.331474in}{3.220855in}}%
\pgfpathlineto{\pgfqpoint{3.337650in}{3.249988in}}%
\pgfpathlineto{\pgfqpoint{3.341768in}{3.279121in}}%
\pgfpathlineto{\pgfqpoint{3.345886in}{3.376231in}}%
\pgfpathlineto{\pgfqpoint{3.352063in}{3.298543in}}%
\pgfpathlineto{\pgfqpoint{3.354122in}{3.317965in}}%
\pgfpathlineto{\pgfqpoint{3.356181in}{3.366520in}}%
\pgfpathlineto{\pgfqpoint{3.358240in}{3.376231in}}%
\pgfpathlineto{\pgfqpoint{3.360299in}{3.337387in}}%
\pgfpathlineto{\pgfqpoint{3.366476in}{3.385942in}}%
\pgfpathlineto{\pgfqpoint{3.368535in}{3.356809in}}%
\pgfpathlineto{\pgfqpoint{3.370594in}{3.453919in}}%
\pgfpathlineto{\pgfqpoint{3.372653in}{3.444208in}}%
\pgfpathlineto{\pgfqpoint{3.374711in}{3.424786in}}%
\pgfpathlineto{\pgfqpoint{3.382947in}{3.424786in}}%
\pgfpathlineto{\pgfqpoint{3.385006in}{3.473341in}}%
\pgfpathlineto{\pgfqpoint{3.387065in}{3.453919in}}%
\pgfpathlineto{\pgfqpoint{3.389124in}{3.405364in}}%
\pgfpathlineto{\pgfqpoint{3.395301in}{3.385942in}}%
\pgfpathlineto{\pgfqpoint{3.397360in}{3.444208in}}%
\pgfpathlineto{\pgfqpoint{3.399419in}{3.415075in}}%
\pgfpathlineto{\pgfqpoint{3.401478in}{3.356809in}}%
\pgfpathlineto{\pgfqpoint{3.403537in}{3.405364in}}%
\pgfpathlineto{\pgfqpoint{3.409714in}{3.424786in}}%
\pgfpathlineto{\pgfqpoint{3.413831in}{3.424786in}}%
\pgfpathlineto{\pgfqpoint{3.415890in}{3.405364in}}%
\pgfpathlineto{\pgfqpoint{3.417949in}{3.434497in}}%
\pgfpathlineto{\pgfqpoint{3.424126in}{3.405364in}}%
\pgfpathlineto{\pgfqpoint{3.428244in}{3.366520in}}%
\pgfpathlineto{\pgfqpoint{3.430303in}{3.376231in}}%
\pgfpathlineto{\pgfqpoint{3.432362in}{3.395653in}}%
\pgfpathlineto{\pgfqpoint{3.438539in}{3.395653in}}%
\pgfpathlineto{\pgfqpoint{3.440598in}{3.434497in}}%
\pgfpathlineto{\pgfqpoint{3.442657in}{3.434497in}}%
\pgfpathlineto{\pgfqpoint{3.444716in}{3.376231in}}%
\pgfpathlineto{\pgfqpoint{3.446775in}{3.356809in}}%
\pgfpathlineto{\pgfqpoint{3.452951in}{3.395653in}}%
\pgfpathlineto{\pgfqpoint{3.455010in}{3.317965in}}%
\pgfpathlineto{\pgfqpoint{3.457069in}{3.337387in}}%
\pgfpathlineto{\pgfqpoint{3.459128in}{3.298543in}}%
\pgfpathlineto{\pgfqpoint{3.467364in}{3.288832in}}%
\pgfpathlineto{\pgfqpoint{3.469423in}{3.347098in}}%
\pgfpathlineto{\pgfqpoint{3.471482in}{3.347098in}}%
\pgfpathlineto{\pgfqpoint{3.473541in}{3.376231in}}%
\pgfpathlineto{\pgfqpoint{3.475600in}{3.317965in}}%
\pgfpathlineto{\pgfqpoint{3.481777in}{3.337387in}}%
\pgfpathlineto{\pgfqpoint{3.483836in}{3.356809in}}%
\pgfpathlineto{\pgfqpoint{3.485895in}{3.337387in}}%
\pgfpathlineto{\pgfqpoint{3.487954in}{3.395653in}}%
\pgfpathlineto{\pgfqpoint{3.490012in}{3.385942in}}%
\pgfpathlineto{\pgfqpoint{3.496189in}{3.395653in}}%
\pgfpathlineto{\pgfqpoint{3.498248in}{3.385942in}}%
\pgfpathlineto{\pgfqpoint{3.502366in}{3.483052in}}%
\pgfpathlineto{\pgfqpoint{3.504425in}{3.521896in}}%
\pgfpathlineto{\pgfqpoint{3.512661in}{3.560740in}}%
\pgfpathlineto{\pgfqpoint{3.514720in}{3.580162in}}%
\pgfpathlineto{\pgfqpoint{3.518838in}{3.531607in}}%
\pgfpathlineto{\pgfqpoint{3.525015in}{3.521896in}}%
\pgfpathlineto{\pgfqpoint{3.527073in}{3.541318in}}%
\pgfpathlineto{\pgfqpoint{3.529132in}{3.531607in}}%
\pgfpathlineto{\pgfqpoint{3.531191in}{3.512185in}}%
\pgfpathlineto{\pgfqpoint{3.539427in}{3.512185in}}%
\pgfpathlineto{\pgfqpoint{3.543545in}{3.570451in}}%
\pgfpathlineto{\pgfqpoint{3.545604in}{3.551029in}}%
\pgfpathlineto{\pgfqpoint{3.547663in}{3.551029in}}%
\pgfpathlineto{\pgfqpoint{3.553840in}{3.570451in}}%
\pgfpathlineto{\pgfqpoint{3.555899in}{3.648139in}}%
\pgfpathlineto{\pgfqpoint{3.557958in}{3.657850in}}%
\pgfpathlineto{\pgfqpoint{3.560017in}{3.677272in}}%
\pgfpathlineto{\pgfqpoint{3.562076in}{3.628717in}}%
\pgfpathlineto{\pgfqpoint{3.570311in}{3.628717in}}%
\pgfpathlineto{\pgfqpoint{3.572370in}{3.560740in}}%
\pgfpathlineto{\pgfqpoint{3.574429in}{3.541318in}}%
\pgfpathlineto{\pgfqpoint{3.576488in}{3.492763in}}%
\pgfpathlineto{\pgfqpoint{3.584724in}{3.327676in}}%
\pgfpathlineto{\pgfqpoint{3.586783in}{3.405364in}}%
\pgfpathlineto{\pgfqpoint{3.588842in}{3.395653in}}%
\pgfpathlineto{\pgfqpoint{3.590901in}{3.463630in}}%
\pgfpathlineto{\pgfqpoint{3.597078in}{3.502474in}}%
\pgfpathlineto{\pgfqpoint{3.599137in}{3.483052in}}%
\pgfpathlineto{\pgfqpoint{3.601196in}{3.541318in}}%
\pgfpathlineto{\pgfqpoint{3.603254in}{3.492763in}}%
\pgfpathlineto{\pgfqpoint{3.605313in}{3.492763in}}%
\pgfpathlineto{\pgfqpoint{3.611490in}{3.521896in}}%
\pgfpathlineto{\pgfqpoint{3.613549in}{3.521896in}}%
\pgfpathlineto{\pgfqpoint{3.615608in}{3.560740in}}%
\pgfpathlineto{\pgfqpoint{3.617667in}{3.512185in}}%
\pgfpathlineto{\pgfqpoint{3.619726in}{3.502474in}}%
\pgfpathlineto{\pgfqpoint{3.625903in}{3.502474in}}%
\pgfpathlineto{\pgfqpoint{3.627962in}{3.453919in}}%
\pgfpathlineto{\pgfqpoint{3.630021in}{3.502474in}}%
\pgfpathlineto{\pgfqpoint{3.632080in}{3.473341in}}%
\pgfpathlineto{\pgfqpoint{3.634139in}{3.473341in}}%
\pgfpathlineto{\pgfqpoint{3.640315in}{3.444208in}}%
\pgfpathlineto{\pgfqpoint{3.642374in}{3.453919in}}%
\pgfpathlineto{\pgfqpoint{3.644433in}{3.405364in}}%
\pgfpathlineto{\pgfqpoint{3.646492in}{3.424786in}}%
\pgfpathlineto{\pgfqpoint{3.648551in}{3.424786in}}%
\pgfpathlineto{\pgfqpoint{3.654728in}{3.444208in}}%
\pgfpathlineto{\pgfqpoint{3.656787in}{3.405364in}}%
\pgfpathlineto{\pgfqpoint{3.660905in}{3.415075in}}%
\pgfpathlineto{\pgfqpoint{3.662964in}{3.395653in}}%
\pgfpathlineto{\pgfqpoint{3.671200in}{3.444208in}}%
\pgfpathlineto{\pgfqpoint{3.673259in}{3.434497in}}%
\pgfpathlineto{\pgfqpoint{3.675318in}{3.444208in}}%
\pgfpathlineto{\pgfqpoint{3.677377in}{3.415075in}}%
\pgfpathlineto{\pgfqpoint{3.685612in}{3.444208in}}%
\pgfpathlineto{\pgfqpoint{3.687671in}{3.453919in}}%
\pgfpathlineto{\pgfqpoint{3.689730in}{3.424786in}}%
\pgfpathlineto{\pgfqpoint{3.691789in}{3.463630in}}%
\pgfpathlineto{\pgfqpoint{3.697966in}{3.531607in}}%
\pgfpathlineto{\pgfqpoint{3.702084in}{3.512185in}}%
\pgfpathlineto{\pgfqpoint{3.704143in}{3.560740in}}%
\pgfpathlineto{\pgfqpoint{3.706202in}{3.531607in}}%
\pgfpathlineto{\pgfqpoint{3.712379in}{3.551029in}}%
\pgfpathlineto{\pgfqpoint{3.714438in}{3.531607in}}%
\pgfpathlineto{\pgfqpoint{3.716496in}{3.570451in}}%
\pgfpathlineto{\pgfqpoint{3.720614in}{3.521896in}}%
\pgfpathlineto{\pgfqpoint{3.726791in}{3.502474in}}%
\pgfpathlineto{\pgfqpoint{3.728850in}{3.531607in}}%
\pgfpathlineto{\pgfqpoint{3.730909in}{3.531607in}}%
\pgfpathlineto{\pgfqpoint{3.732968in}{3.502474in}}%
\pgfpathlineto{\pgfqpoint{3.735027in}{3.434497in}}%
\pgfpathlineto{\pgfqpoint{3.741204in}{3.434497in}}%
\pgfpathlineto{\pgfqpoint{3.743263in}{3.453919in}}%
\pgfpathlineto{\pgfqpoint{3.745322in}{3.424786in}}%
\pgfpathlineto{\pgfqpoint{3.747381in}{3.434497in}}%
\pgfpathlineto{\pgfqpoint{3.749440in}{3.434497in}}%
\pgfpathlineto{\pgfqpoint{3.755616in}{3.385942in}}%
\pgfpathlineto{\pgfqpoint{3.757675in}{3.415075in}}%
\pgfpathlineto{\pgfqpoint{3.759734in}{3.385942in}}%
\pgfpathlineto{\pgfqpoint{3.761793in}{3.395653in}}%
\pgfpathlineto{\pgfqpoint{3.763852in}{3.395653in}}%
\pgfpathlineto{\pgfqpoint{3.770029in}{3.424786in}}%
\pgfpathlineto{\pgfqpoint{3.772088in}{3.453919in}}%
\pgfpathlineto{\pgfqpoint{3.774147in}{3.463630in}}%
\pgfpathlineto{\pgfqpoint{3.776206in}{3.434497in}}%
\pgfpathlineto{\pgfqpoint{3.778265in}{3.424786in}}%
\pgfpathlineto{\pgfqpoint{3.786501in}{3.463630in}}%
\pgfpathlineto{\pgfqpoint{3.788560in}{3.463630in}}%
\pgfpathlineto{\pgfqpoint{3.790619in}{3.444208in}}%
\pgfpathlineto{\pgfqpoint{3.792677in}{3.502474in}}%
\pgfpathlineto{\pgfqpoint{3.798854in}{3.502474in}}%
\pgfpathlineto{\pgfqpoint{3.800913in}{3.551029in}}%
\pgfpathlineto{\pgfqpoint{3.802972in}{3.541318in}}%
\pgfpathlineto{\pgfqpoint{3.805031in}{3.541318in}}%
\pgfpathlineto{\pgfqpoint{3.807090in}{3.570451in}}%
\pgfpathlineto{\pgfqpoint{3.813267in}{3.570451in}}%
\pgfpathlineto{\pgfqpoint{3.817385in}{3.648139in}}%
\pgfpathlineto{\pgfqpoint{3.819444in}{3.638428in}}%
\pgfpathlineto{\pgfqpoint{3.821503in}{3.638428in}}%
\pgfpathlineto{\pgfqpoint{3.827680in}{3.648139in}}%
\pgfpathlineto{\pgfqpoint{3.829738in}{3.667561in}}%
\pgfpathlineto{\pgfqpoint{3.831797in}{3.628717in}}%
\pgfpathlineto{\pgfqpoint{3.833856in}{3.628717in}}%
\pgfpathlineto{\pgfqpoint{3.835915in}{3.619006in}}%
\pgfpathlineto{\pgfqpoint{3.842092in}{3.648139in}}%
\pgfpathlineto{\pgfqpoint{3.844151in}{3.619006in}}%
\pgfpathlineto{\pgfqpoint{3.846210in}{3.706405in}}%
\pgfpathlineto{\pgfqpoint{3.850328in}{3.784092in}}%
\pgfpathlineto{\pgfqpoint{3.858564in}{3.754960in}}%
\pgfpathlineto{\pgfqpoint{3.860623in}{3.754960in}}%
\pgfpathlineto{\pgfqpoint{3.862682in}{3.696694in}}%
\pgfpathlineto{\pgfqpoint{3.864741in}{3.696694in}}%
\pgfpathlineto{\pgfqpoint{3.870917in}{3.706405in}}%
\pgfpathlineto{\pgfqpoint{3.872976in}{3.706405in}}%
\pgfpathlineto{\pgfqpoint{3.875035in}{3.735538in}}%
\pgfpathlineto{\pgfqpoint{3.877094in}{3.716116in}}%
\pgfpathlineto{\pgfqpoint{3.879153in}{3.745249in}}%
\pgfpathlineto{\pgfqpoint{3.885330in}{3.735538in}}%
\pgfpathlineto{\pgfqpoint{3.887389in}{3.706405in}}%
\pgfpathlineto{\pgfqpoint{3.889448in}{3.628717in}}%
\pgfpathlineto{\pgfqpoint{3.891507in}{3.677272in}}%
\pgfpathlineto{\pgfqpoint{3.893566in}{3.609295in}}%
\pgfpathlineto{\pgfqpoint{3.899743in}{3.609295in}}%
\pgfpathlineto{\pgfqpoint{3.903861in}{3.677272in}}%
\pgfpathlineto{\pgfqpoint{3.905919in}{3.667561in}}%
\pgfpathlineto{\pgfqpoint{3.907978in}{3.735538in}}%
\pgfpathlineto{\pgfqpoint{3.914155in}{3.725827in}}%
\pgfpathlineto{\pgfqpoint{3.920332in}{3.774382in}}%
\pgfpathlineto{\pgfqpoint{3.922391in}{3.735538in}}%
\pgfpathlineto{\pgfqpoint{3.930627in}{3.677272in}}%
\pgfpathlineto{\pgfqpoint{3.936804in}{3.599584in}}%
\pgfpathlineto{\pgfqpoint{3.942980in}{3.580162in}}%
\pgfpathlineto{\pgfqpoint{3.945039in}{3.580162in}}%
\pgfpathlineto{\pgfqpoint{3.947098in}{3.589873in}}%
\pgfpathlineto{\pgfqpoint{3.951216in}{3.580162in}}%
\pgfpathlineto{\pgfqpoint{3.957393in}{3.589873in}}%
\pgfpathlineto{\pgfqpoint{3.959452in}{3.589873in}}%
\pgfpathlineto{\pgfqpoint{3.961511in}{3.580162in}}%
\pgfpathlineto{\pgfqpoint{3.965629in}{3.531607in}}%
\pgfpathlineto{\pgfqpoint{3.971806in}{3.512185in}}%
\pgfpathlineto{\pgfqpoint{3.973865in}{3.453919in}}%
\pgfpathlineto{\pgfqpoint{3.980042in}{3.385942in}}%
\pgfpathlineto{\pgfqpoint{3.986218in}{3.385942in}}%
\pgfpathlineto{\pgfqpoint{3.990336in}{3.453919in}}%
\pgfpathlineto{\pgfqpoint{3.992395in}{3.444208in}}%
\pgfpathlineto{\pgfqpoint{3.994454in}{3.424786in}}%
\pgfpathlineto{\pgfqpoint{4.000631in}{3.385942in}}%
\pgfpathlineto{\pgfqpoint{4.002690in}{3.356809in}}%
\pgfpathlineto{\pgfqpoint{4.004749in}{3.308254in}}%
\pgfpathlineto{\pgfqpoint{4.006808in}{3.337387in}}%
\pgfpathlineto{\pgfqpoint{4.008867in}{3.337387in}}%
\pgfpathlineto{\pgfqpoint{4.015044in}{3.279121in}}%
\pgfpathlineto{\pgfqpoint{4.019161in}{3.356809in}}%
\pgfpathlineto{\pgfqpoint{4.023279in}{3.249988in}}%
\pgfpathlineto{\pgfqpoint{4.033574in}{3.182012in}}%
\pgfpathlineto{\pgfqpoint{4.035633in}{3.065480in}}%
\pgfpathlineto{\pgfqpoint{4.037692in}{3.182012in}}%
\pgfpathlineto{\pgfqpoint{4.043869in}{3.220855in}}%
\pgfpathlineto{\pgfqpoint{4.045928in}{3.249988in}}%
\pgfpathlineto{\pgfqpoint{4.047987in}{3.259699in}}%
\pgfpathlineto{\pgfqpoint{4.050046in}{3.249988in}}%
\pgfpathlineto{\pgfqpoint{4.052105in}{3.220855in}}%
\pgfpathlineto{\pgfqpoint{4.058281in}{3.220855in}}%
\pgfpathlineto{\pgfqpoint{4.062399in}{3.240277in}}%
\pgfpathlineto{\pgfqpoint{4.066517in}{3.317965in}}%
\pgfpathlineto{\pgfqpoint{4.074753in}{3.269410in}}%
\pgfpathlineto{\pgfqpoint{4.076812in}{3.279121in}}%
\pgfpathlineto{\pgfqpoint{4.078871in}{3.240277in}}%
\pgfpathlineto{\pgfqpoint{4.080930in}{3.279121in}}%
\pgfpathlineto{\pgfqpoint{4.087107in}{3.269410in}}%
\pgfpathlineto{\pgfqpoint{4.093284in}{3.133457in}}%
\pgfpathlineto{\pgfqpoint{4.095342in}{3.211145in}}%
\pgfpathlineto{\pgfqpoint{4.101519in}{3.240277in}}%
\pgfpathlineto{\pgfqpoint{4.105637in}{3.211145in}}%
\pgfpathlineto{\pgfqpoint{4.107696in}{3.162590in}}%
\pgfpathlineto{\pgfqpoint{4.109755in}{3.152879in}}%
\pgfpathlineto{\pgfqpoint{4.115932in}{3.182012in}}%
\pgfpathlineto{\pgfqpoint{4.120050in}{3.230566in}}%
\pgfpathlineto{\pgfqpoint{4.122109in}{3.182012in}}%
\pgfpathlineto{\pgfqpoint{4.124168in}{3.191723in}}%
\pgfpathlineto{\pgfqpoint{4.132403in}{3.172301in}}%
\pgfpathlineto{\pgfqpoint{4.134462in}{3.172301in}}%
\pgfpathlineto{\pgfqpoint{4.136521in}{3.211145in}}%
\pgfpathlineto{\pgfqpoint{4.138580in}{3.172301in}}%
\pgfpathlineto{\pgfqpoint{4.144757in}{3.191723in}}%
\pgfpathlineto{\pgfqpoint{4.146816in}{3.162590in}}%
\pgfpathlineto{\pgfqpoint{4.152993in}{3.288832in}}%
\pgfpathlineto{\pgfqpoint{4.159170in}{3.249988in}}%
\pgfpathlineto{\pgfqpoint{4.161229in}{3.249988in}}%
\pgfpathlineto{\pgfqpoint{4.167406in}{3.133457in}}%
\pgfpathlineto{\pgfqpoint{4.173582in}{3.152879in}}%
\pgfpathlineto{\pgfqpoint{4.175641in}{3.123746in}}%
\pgfpathlineto{\pgfqpoint{4.177700in}{3.133457in}}%
\pgfpathlineto{\pgfqpoint{4.179759in}{3.152879in}}%
\pgfpathlineto{\pgfqpoint{4.181818in}{3.114035in}}%
\pgfpathlineto{\pgfqpoint{4.187995in}{3.133457in}}%
\pgfpathlineto{\pgfqpoint{4.190054in}{3.133457in}}%
\pgfpathlineto{\pgfqpoint{4.192113in}{3.065480in}}%
\pgfpathlineto{\pgfqpoint{4.194172in}{3.065480in}}%
\pgfpathlineto{\pgfqpoint{4.196231in}{2.968370in}}%
\pgfpathlineto{\pgfqpoint{4.202408in}{2.948948in}}%
\pgfpathlineto{\pgfqpoint{4.204467in}{2.919815in}}%
\pgfpathlineto{\pgfqpoint{4.206526in}{2.910104in}}%
\pgfpathlineto{\pgfqpoint{4.208584in}{2.919815in}}%
\pgfpathlineto{\pgfqpoint{4.210643in}{2.939237in}}%
\pgfpathlineto{\pgfqpoint{4.216820in}{3.026636in}}%
\pgfpathlineto{\pgfqpoint{4.218879in}{3.007214in}}%
\pgfpathlineto{\pgfqpoint{4.220938in}{3.046058in}}%
\pgfpathlineto{\pgfqpoint{4.225056in}{3.026636in}}%
\pgfpathlineto{\pgfqpoint{4.231233in}{3.046058in}}%
\pgfpathlineto{\pgfqpoint{4.235351in}{2.997503in}}%
\pgfpathlineto{\pgfqpoint{4.237410in}{3.026636in}}%
\pgfpathlineto{\pgfqpoint{4.239469in}{3.094613in}}%
\pgfpathlineto{\pgfqpoint{4.245645in}{3.084902in}}%
\pgfpathlineto{\pgfqpoint{4.247704in}{3.123746in}}%
\pgfpathlineto{\pgfqpoint{4.249763in}{3.114035in}}%
\pgfpathlineto{\pgfqpoint{4.251822in}{3.084902in}}%
\pgfpathlineto{\pgfqpoint{4.260058in}{3.114035in}}%
\pgfpathlineto{\pgfqpoint{4.262117in}{3.084902in}}%
\pgfpathlineto{\pgfqpoint{4.264176in}{3.036347in}}%
\pgfpathlineto{\pgfqpoint{4.266235in}{3.046058in}}%
\pgfpathlineto{\pgfqpoint{4.268294in}{3.016925in}}%
\pgfpathlineto{\pgfqpoint{4.274471in}{3.046058in}}%
\pgfpathlineto{\pgfqpoint{4.276530in}{3.016925in}}%
\pgfpathlineto{\pgfqpoint{4.280648in}{3.065480in}}%
\pgfpathlineto{\pgfqpoint{4.288883in}{3.026636in}}%
\pgfpathlineto{\pgfqpoint{4.290942in}{2.978081in}}%
\pgfpathlineto{\pgfqpoint{4.293001in}{3.007214in}}%
\pgfpathlineto{\pgfqpoint{4.295060in}{2.968370in}}%
\pgfpathlineto{\pgfqpoint{4.297119in}{2.997503in}}%
\pgfpathlineto{\pgfqpoint{4.303296in}{2.910104in}}%
\pgfpathlineto{\pgfqpoint{4.305355in}{2.929526in}}%
\pgfpathlineto{\pgfqpoint{4.307414in}{2.880971in}}%
\pgfpathlineto{\pgfqpoint{4.309473in}{2.910104in}}%
\pgfpathlineto{\pgfqpoint{4.311532in}{2.900393in}}%
\pgfpathlineto{\pgfqpoint{4.317709in}{2.929526in}}%
\pgfpathlineto{\pgfqpoint{4.319768in}{2.958659in}}%
\pgfpathlineto{\pgfqpoint{4.321826in}{2.910104in}}%
\pgfpathlineto{\pgfqpoint{4.323885in}{2.832416in}}%
\pgfpathlineto{\pgfqpoint{4.325944in}{2.851838in}}%
\pgfpathlineto{\pgfqpoint{4.334180in}{2.793572in}}%
\pgfpathlineto{\pgfqpoint{4.336239in}{2.793572in}}%
\pgfpathlineto{\pgfqpoint{4.338298in}{2.754728in}}%
\pgfpathlineto{\pgfqpoint{4.340357in}{2.667329in}}%
\pgfpathlineto{\pgfqpoint{4.346534in}{2.589642in}}%
\pgfpathlineto{\pgfqpoint{4.348593in}{2.647908in}}%
\pgfpathlineto{\pgfqpoint{4.350652in}{2.638197in}}%
\pgfpathlineto{\pgfqpoint{4.352711in}{2.647908in}}%
\pgfpathlineto{\pgfqpoint{4.354770in}{2.609064in}}%
\pgfpathlineto{\pgfqpoint{4.360946in}{2.667329in}}%
\pgfpathlineto{\pgfqpoint{4.363005in}{2.667329in}}%
\pgfpathlineto{\pgfqpoint{4.367123in}{2.599353in}}%
\pgfpathlineto{\pgfqpoint{4.375359in}{2.599353in}}%
\pgfpathlineto{\pgfqpoint{4.381536in}{2.511954in}}%
\pgfpathlineto{\pgfqpoint{4.383595in}{2.570220in}}%
\pgfpathlineto{\pgfqpoint{4.391831in}{2.502243in}}%
\pgfpathlineto{\pgfqpoint{4.393890in}{2.560509in}}%
\pgfpathlineto{\pgfqpoint{4.395949in}{2.521665in}}%
\pgfpathlineto{\pgfqpoint{4.398007in}{2.511954in}}%
\pgfpathlineto{\pgfqpoint{4.404184in}{2.541087in}}%
\pgfpathlineto{\pgfqpoint{4.406243in}{2.492532in}}%
\pgfpathlineto{\pgfqpoint{4.408302in}{2.473110in}}%
\pgfpathlineto{\pgfqpoint{4.412420in}{2.570220in}}%
\pgfpathlineto{\pgfqpoint{4.418597in}{2.579931in}}%
\pgfpathlineto{\pgfqpoint{4.420656in}{2.599353in}}%
\pgfpathlineto{\pgfqpoint{4.422715in}{2.570220in}}%
\pgfpathlineto{\pgfqpoint{4.424774in}{2.628486in}}%
\pgfpathlineto{\pgfqpoint{4.426833in}{2.618775in}}%
\pgfpathlineto{\pgfqpoint{4.433010in}{2.599353in}}%
\pgfpathlineto{\pgfqpoint{4.435068in}{2.638197in}}%
\pgfpathlineto{\pgfqpoint{4.439186in}{2.531376in}}%
\pgfpathlineto{\pgfqpoint{4.441245in}{2.550798in}}%
\pgfpathlineto{\pgfqpoint{4.447422in}{2.560509in}}%
\pgfpathlineto{\pgfqpoint{4.449481in}{2.579931in}}%
\pgfpathlineto{\pgfqpoint{4.451540in}{2.570220in}}%
\pgfpathlineto{\pgfqpoint{4.453599in}{2.589642in}}%
\pgfpathlineto{\pgfqpoint{4.455658in}{2.589642in}}%
\pgfpathlineto{\pgfqpoint{4.461835in}{2.570220in}}%
\pgfpathlineto{\pgfqpoint{4.463894in}{2.570220in}}%
\pgfpathlineto{\pgfqpoint{4.465953in}{2.560509in}}%
\pgfpathlineto{\pgfqpoint{4.468012in}{2.414844in}}%
\pgfpathlineto{\pgfqpoint{4.470071in}{2.395422in}}%
\pgfpathlineto{\pgfqpoint{4.476247in}{2.278890in}}%
\pgfpathlineto{\pgfqpoint{4.478306in}{2.269179in}}%
\pgfpathlineto{\pgfqpoint{4.480365in}{2.249757in}}%
\pgfpathlineto{\pgfqpoint{4.484483in}{2.298312in}}%
\pgfpathlineto{\pgfqpoint{4.490660in}{2.210913in}}%
\pgfpathlineto{\pgfqpoint{4.492719in}{2.269179in}}%
\pgfpathlineto{\pgfqpoint{4.496837in}{2.123514in}}%
\pgfpathlineto{\pgfqpoint{4.505073in}{2.191491in}}%
\pgfpathlineto{\pgfqpoint{4.507132in}{2.142936in}}%
\pgfpathlineto{\pgfqpoint{4.509191in}{2.191491in}}%
\pgfpathlineto{\pgfqpoint{4.511249in}{2.210913in}}%
\pgfpathlineto{\pgfqpoint{4.513308in}{2.113803in}}%
\pgfpathlineto{\pgfqpoint{4.519485in}{2.142936in}}%
\pgfpathlineto{\pgfqpoint{4.521544in}{2.094382in}}%
\pgfpathlineto{\pgfqpoint{4.523603in}{2.074960in}}%
\pgfpathlineto{\pgfqpoint{4.525662in}{2.113803in}}%
\pgfpathlineto{\pgfqpoint{4.527721in}{2.104092in}}%
\pgfpathlineto{\pgfqpoint{4.535957in}{2.074960in}}%
\pgfpathlineto{\pgfqpoint{4.538016in}{2.055538in}}%
\pgfpathlineto{\pgfqpoint{4.540075in}{2.162358in}}%
\pgfpathlineto{\pgfqpoint{4.542134in}{2.152647in}}%
\pgfpathlineto{\pgfqpoint{4.548311in}{2.220624in}}%
\pgfpathlineto{\pgfqpoint{4.550369in}{2.308023in}}%
\pgfpathlineto{\pgfqpoint{4.552428in}{2.327445in}}%
\pgfpathlineto{\pgfqpoint{4.554487in}{2.366289in}}%
\pgfpathlineto{\pgfqpoint{4.556546in}{2.473110in}}%
\pgfpathlineto{\pgfqpoint{4.564782in}{2.395422in}}%
\pgfpathlineto{\pgfqpoint{4.566841in}{2.405133in}}%
\pgfpathlineto{\pgfqpoint{4.568900in}{2.376000in}}%
\pgfpathlineto{\pgfqpoint{4.570959in}{2.327445in}}%
\pgfpathlineto{\pgfqpoint{4.577136in}{2.298312in}}%
\pgfpathlineto{\pgfqpoint{4.579195in}{2.230335in}}%
\pgfpathlineto{\pgfqpoint{4.581254in}{2.308023in}}%
\pgfpathlineto{\pgfqpoint{4.583313in}{2.298312in}}%
\pgfpathlineto{\pgfqpoint{4.585372in}{2.269179in}}%
\pgfpathlineto{\pgfqpoint{4.591548in}{2.269179in}}%
\pgfpathlineto{\pgfqpoint{4.593607in}{2.240046in}}%
\pgfpathlineto{\pgfqpoint{4.599784in}{2.084671in}}%
\pgfpathlineto{\pgfqpoint{4.605961in}{2.123514in}}%
\pgfpathlineto{\pgfqpoint{4.608020in}{2.104092in}}%
\pgfpathlineto{\pgfqpoint{4.612138in}{2.220624in}}%
\pgfpathlineto{\pgfqpoint{4.614197in}{2.327445in}}%
\pgfpathlineto{\pgfqpoint{4.622433in}{2.327445in}}%
\pgfpathlineto{\pgfqpoint{4.624491in}{2.298312in}}%
\pgfpathlineto{\pgfqpoint{4.626550in}{2.308023in}}%
\pgfpathlineto{\pgfqpoint{4.628609in}{2.308023in}}%
\pgfpathlineto{\pgfqpoint{4.634786in}{2.346867in}}%
\pgfpathlineto{\pgfqpoint{4.638904in}{2.317734in}}%
\pgfpathlineto{\pgfqpoint{4.640963in}{2.317734in}}%
\pgfpathlineto{\pgfqpoint{4.643022in}{2.356578in}}%
\pgfpathlineto{\pgfqpoint{4.649199in}{2.395422in}}%
\pgfpathlineto{\pgfqpoint{4.651258in}{2.385711in}}%
\pgfpathlineto{\pgfqpoint{4.653317in}{2.337156in}}%
\pgfpathlineto{\pgfqpoint{4.655376in}{2.249757in}}%
\pgfpathlineto{\pgfqpoint{4.657435in}{2.278890in}}%
\pgfpathlineto{\pgfqpoint{4.663611in}{2.337156in}}%
\pgfpathlineto{\pgfqpoint{4.665670in}{2.414844in}}%
\pgfpathlineto{\pgfqpoint{4.667729in}{2.376000in}}%
\pgfpathlineto{\pgfqpoint{4.669788in}{2.482821in}}%
\pgfpathlineto{\pgfqpoint{4.671847in}{2.502243in}}%
\pgfpathlineto{\pgfqpoint{4.680083in}{2.482821in}}%
\pgfpathlineto{\pgfqpoint{4.684201in}{2.376000in}}%
\pgfpathlineto{\pgfqpoint{4.686260in}{2.395422in}}%
\pgfpathlineto{\pgfqpoint{4.692437in}{2.376000in}}%
\pgfpathlineto{\pgfqpoint{4.694496in}{2.356578in}}%
\pgfpathlineto{\pgfqpoint{4.696555in}{2.308023in}}%
\pgfpathlineto{\pgfqpoint{4.698614in}{2.356578in}}%
\pgfpathlineto{\pgfqpoint{4.706849in}{2.356578in}}%
\pgfpathlineto{\pgfqpoint{4.708908in}{2.327445in}}%
\pgfpathlineto{\pgfqpoint{4.710967in}{2.356578in}}%
\pgfpathlineto{\pgfqpoint{4.715085in}{2.376000in}}%
\pgfpathlineto{\pgfqpoint{4.721262in}{2.414844in}}%
\pgfpathlineto{\pgfqpoint{4.723321in}{2.298312in}}%
\pgfpathlineto{\pgfqpoint{4.725380in}{2.356578in}}%
\pgfpathlineto{\pgfqpoint{4.727439in}{2.376000in}}%
\pgfpathlineto{\pgfqpoint{4.729498in}{2.424555in}}%
\pgfpathlineto{\pgfqpoint{4.735675in}{2.414844in}}%
\pgfpathlineto{\pgfqpoint{4.737733in}{2.424555in}}%
\pgfpathlineto{\pgfqpoint{4.739792in}{2.385711in}}%
\pgfpathlineto{\pgfqpoint{4.741851in}{2.482821in}}%
\pgfpathlineto{\pgfqpoint{4.743910in}{2.405133in}}%
\pgfpathlineto{\pgfqpoint{4.750087in}{2.463399in}}%
\pgfpathlineto{\pgfqpoint{4.752146in}{2.463399in}}%
\pgfpathlineto{\pgfqpoint{4.754205in}{2.502243in}}%
\pgfpathlineto{\pgfqpoint{4.756264in}{2.482821in}}%
\pgfpathlineto{\pgfqpoint{4.758323in}{2.482821in}}%
\pgfpathlineto{\pgfqpoint{4.764500in}{2.502243in}}%
\pgfpathlineto{\pgfqpoint{4.766559in}{2.473110in}}%
\pgfpathlineto{\pgfqpoint{4.770677in}{2.492532in}}%
\pgfpathlineto{\pgfqpoint{4.772736in}{2.443977in}}%
\pgfpathlineto{\pgfqpoint{4.778912in}{2.453688in}}%
\pgfpathlineto{\pgfqpoint{4.780971in}{2.473110in}}%
\pgfpathlineto{\pgfqpoint{4.785089in}{2.434266in}}%
\pgfpathlineto{\pgfqpoint{4.787148in}{2.356578in}}%
\pgfpathlineto{\pgfqpoint{4.793325in}{2.366289in}}%
\pgfpathlineto{\pgfqpoint{4.795384in}{2.385711in}}%
\pgfpathlineto{\pgfqpoint{4.797443in}{2.424555in}}%
\pgfpathlineto{\pgfqpoint{4.799502in}{2.414844in}}%
\pgfpathlineto{\pgfqpoint{4.801561in}{2.385711in}}%
\pgfpathlineto{\pgfqpoint{4.807738in}{2.405133in}}%
\pgfpathlineto{\pgfqpoint{4.811856in}{2.356578in}}%
\pgfpathlineto{\pgfqpoint{4.815973in}{2.385711in}}%
\pgfpathlineto{\pgfqpoint{4.826268in}{2.327445in}}%
\pgfpathlineto{\pgfqpoint{4.836563in}{2.172069in}}%
\pgfpathlineto{\pgfqpoint{4.838622in}{2.210913in}}%
\pgfpathlineto{\pgfqpoint{4.840681in}{2.162358in}}%
\pgfpathlineto{\pgfqpoint{4.842740in}{2.142936in}}%
\pgfpathlineto{\pgfqpoint{4.844799in}{2.074960in}}%
\pgfpathlineto{\pgfqpoint{4.850976in}{2.104092in}}%
\pgfpathlineto{\pgfqpoint{4.855093in}{2.220624in}}%
\pgfpathlineto{\pgfqpoint{4.857152in}{2.210913in}}%
\pgfpathlineto{\pgfqpoint{4.859211in}{2.162358in}}%
\pgfpathlineto{\pgfqpoint{4.865388in}{2.123514in}}%
\pgfpathlineto{\pgfqpoint{4.867447in}{2.152647in}}%
\pgfpathlineto{\pgfqpoint{4.869506in}{2.201202in}}%
\pgfpathlineto{\pgfqpoint{4.873624in}{2.162358in}}%
\pgfpathlineto{\pgfqpoint{4.881860in}{2.133225in}}%
\pgfpathlineto{\pgfqpoint{4.883919in}{2.152647in}}%
\pgfpathlineto{\pgfqpoint{4.888037in}{2.045827in}}%
\pgfpathlineto{\pgfqpoint{4.894213in}{1.958428in}}%
\pgfpathlineto{\pgfqpoint{4.896272in}{1.909873in}}%
\pgfpathlineto{\pgfqpoint{4.898331in}{1.909873in}}%
\pgfpathlineto{\pgfqpoint{4.900390in}{1.880740in}}%
\pgfpathlineto{\pgfqpoint{4.902449in}{1.696231in}}%
\pgfpathlineto{\pgfqpoint{4.908626in}{1.676809in}}%
\pgfpathlineto{\pgfqpoint{4.910685in}{1.579699in}}%
\pgfpathlineto{\pgfqpoint{4.912744in}{1.569988in}}%
\pgfpathlineto{\pgfqpoint{4.916862in}{1.366058in}}%
\pgfpathlineto{\pgfqpoint{4.923039in}{1.239815in}}%
\pgfpathlineto{\pgfqpoint{4.925098in}{1.404902in}}%
\pgfpathlineto{\pgfqpoint{4.931274in}{1.560277in}}%
\pgfpathlineto{\pgfqpoint{4.937451in}{1.346636in}}%
\pgfpathlineto{\pgfqpoint{4.941569in}{1.744786in}}%
\pgfpathlineto{\pgfqpoint{4.943628in}{1.667098in}}%
\pgfpathlineto{\pgfqpoint{4.945687in}{1.492301in}}%
\pgfpathlineto{\pgfqpoint{4.951864in}{1.307792in}}%
\pgfpathlineto{\pgfqpoint{4.953923in}{1.424324in}}%
\pgfpathlineto{\pgfqpoint{4.955982in}{1.443746in}}%
\pgfpathlineto{\pgfqpoint{4.958041in}{1.395191in}}%
\pgfpathlineto{\pgfqpoint{4.960100in}{1.278659in}}%
\pgfpathlineto{\pgfqpoint{4.966276in}{1.249526in}}%
\pgfpathlineto{\pgfqpoint{4.968335in}{1.230104in}}%
\pgfpathlineto{\pgfqpoint{4.970394in}{1.191260in}}%
\pgfpathlineto{\pgfqpoint{4.972453in}{1.210682in}}%
\pgfpathlineto{\pgfqpoint{4.974512in}{1.200971in}}%
\pgfpathlineto{\pgfqpoint{4.980689in}{1.259237in}}%
\pgfpathlineto{\pgfqpoint{4.982748in}{1.317503in}}%
\pgfpathlineto{\pgfqpoint{4.984807in}{1.327214in}}%
\pgfpathlineto{\pgfqpoint{4.986866in}{1.278659in}}%
\pgfpathlineto{\pgfqpoint{4.995102in}{1.307792in}}%
\pgfpathlineto{\pgfqpoint{4.997161in}{1.288370in}}%
\pgfpathlineto{\pgfqpoint{4.999220in}{1.171838in}}%
\pgfpathlineto{\pgfqpoint{5.001279in}{1.181549in}}%
\pgfpathlineto{\pgfqpoint{5.003337in}{1.210682in}}%
\pgfpathlineto{\pgfqpoint{5.009514in}{1.191260in}}%
\pgfpathlineto{\pgfqpoint{5.011573in}{1.162127in}}%
\pgfpathlineto{\pgfqpoint{5.013632in}{1.200971in}}%
\pgfpathlineto{\pgfqpoint{5.015691in}{1.191260in}}%
\pgfpathlineto{\pgfqpoint{5.017750in}{1.191260in}}%
\pgfpathlineto{\pgfqpoint{5.023927in}{1.239815in}}%
\pgfpathlineto{\pgfqpoint{5.025986in}{1.200971in}}%
\pgfpathlineto{\pgfqpoint{5.028045in}{1.200971in}}%
\pgfpathlineto{\pgfqpoint{5.030104in}{1.210682in}}%
\pgfpathlineto{\pgfqpoint{5.032163in}{1.210682in}}%
\pgfpathlineto{\pgfqpoint{5.038340in}{1.200971in}}%
\pgfpathlineto{\pgfqpoint{5.040399in}{1.210682in}}%
\pgfpathlineto{\pgfqpoint{5.042457in}{1.249526in}}%
\pgfpathlineto{\pgfqpoint{5.044516in}{1.171838in}}%
\pgfpathlineto{\pgfqpoint{5.046575in}{1.210682in}}%
\pgfpathlineto{\pgfqpoint{5.052752in}{1.239815in}}%
\pgfpathlineto{\pgfqpoint{5.058929in}{1.162127in}}%
\pgfpathlineto{\pgfqpoint{5.060988in}{1.171838in}}%
\pgfpathlineto{\pgfqpoint{5.067165in}{1.249526in}}%
\pgfpathlineto{\pgfqpoint{5.069224in}{1.220393in}}%
\pgfpathlineto{\pgfqpoint{5.071283in}{1.210682in}}%
\pgfpathlineto{\pgfqpoint{5.073342in}{1.210682in}}%
\pgfpathlineto{\pgfqpoint{5.075401in}{1.191260in}}%
\pgfpathlineto{\pgfqpoint{5.083636in}{1.210682in}}%
\pgfpathlineto{\pgfqpoint{5.085695in}{1.200971in}}%
\pgfpathlineto{\pgfqpoint{5.087754in}{1.220393in}}%
\pgfpathlineto{\pgfqpoint{5.089813in}{1.181549in}}%
\pgfpathlineto{\pgfqpoint{5.095990in}{1.181549in}}%
\pgfpathlineto{\pgfqpoint{5.098049in}{1.200971in}}%
\pgfpathlineto{\pgfqpoint{5.104226in}{1.385480in}}%
\pgfpathlineto{\pgfqpoint{5.110403in}{1.366058in}}%
\pgfpathlineto{\pgfqpoint{5.112462in}{1.327214in}}%
\pgfpathlineto{\pgfqpoint{5.116579in}{1.191260in}}%
\pgfpathlineto{\pgfqpoint{5.118638in}{1.220393in}}%
\pgfpathlineto{\pgfqpoint{5.124815in}{1.220393in}}%
\pgfpathlineto{\pgfqpoint{5.126874in}{1.239815in}}%
\pgfpathlineto{\pgfqpoint{5.133051in}{1.210682in}}%
\pgfpathlineto{\pgfqpoint{5.139228in}{1.220393in}}%
\pgfpathlineto{\pgfqpoint{5.141287in}{1.220393in}}%
\pgfpathlineto{\pgfqpoint{5.143346in}{1.200971in}}%
\pgfpathlineto{\pgfqpoint{5.145405in}{1.210682in}}%
\pgfpathlineto{\pgfqpoint{5.147464in}{1.171838in}}%
\pgfpathlineto{\pgfqpoint{5.153641in}{1.152416in}}%
\pgfpathlineto{\pgfqpoint{5.157758in}{1.200971in}}%
\pgfpathlineto{\pgfqpoint{5.159817in}{1.181549in}}%
\pgfpathlineto{\pgfqpoint{5.168053in}{1.191260in}}%
\pgfpathlineto{\pgfqpoint{5.170112in}{1.162127in}}%
\pgfpathlineto{\pgfqpoint{5.172171in}{1.171838in}}%
\pgfpathlineto{\pgfqpoint{5.174230in}{1.142705in}}%
\pgfpathlineto{\pgfqpoint{5.176289in}{1.171838in}}%
\pgfpathlineto{\pgfqpoint{5.182466in}{1.162127in}}%
\pgfpathlineto{\pgfqpoint{5.184525in}{1.152416in}}%
\pgfpathlineto{\pgfqpoint{5.186584in}{1.152416in}}%
\pgfpathlineto{\pgfqpoint{5.188643in}{1.142705in}}%
\pgfpathlineto{\pgfqpoint{5.190702in}{1.152416in}}%
\pgfpathlineto{\pgfqpoint{5.196878in}{1.152416in}}%
\pgfpathlineto{\pgfqpoint{5.198937in}{1.132994in}}%
\pgfpathlineto{\pgfqpoint{5.200996in}{1.132994in}}%
\pgfpathlineto{\pgfqpoint{5.203055in}{1.123283in}}%
\pgfpathlineto{\pgfqpoint{5.205114in}{1.123283in}}%
\pgfpathlineto{\pgfqpoint{5.211291in}{1.142705in}}%
\pgfpathlineto{\pgfqpoint{5.219527in}{1.074728in}}%
\pgfpathlineto{\pgfqpoint{5.225704in}{1.084439in}}%
\pgfpathlineto{\pgfqpoint{5.227763in}{1.045595in}}%
\pgfpathlineto{\pgfqpoint{5.229822in}{1.074728in}}%
\pgfpathlineto{\pgfqpoint{5.231880in}{1.074728in}}%
\pgfpathlineto{\pgfqpoint{5.233939in}{1.094150in}}%
\pgfpathlineto{\pgfqpoint{5.240116in}{1.103861in}}%
\pgfpathlineto{\pgfqpoint{5.246293in}{1.200971in}}%
\pgfpathlineto{\pgfqpoint{5.248352in}{1.191260in}}%
\pgfpathlineto{\pgfqpoint{5.254529in}{1.171838in}}%
\pgfpathlineto{\pgfqpoint{5.256588in}{1.152416in}}%
\pgfpathlineto{\pgfqpoint{5.258647in}{1.162127in}}%
\pgfpathlineto{\pgfqpoint{5.260706in}{1.142705in}}%
\pgfpathlineto{\pgfqpoint{5.262765in}{1.142705in}}%
\pgfpathlineto{\pgfqpoint{5.268941in}{1.152416in}}%
\pgfpathlineto{\pgfqpoint{5.275118in}{1.210682in}}%
\pgfpathlineto{\pgfqpoint{5.277177in}{1.191260in}}%
\pgfpathlineto{\pgfqpoint{5.283354in}{1.181549in}}%
\pgfpathlineto{\pgfqpoint{5.285413in}{1.142705in}}%
\pgfpathlineto{\pgfqpoint{5.287472in}{1.132994in}}%
\pgfpathlineto{\pgfqpoint{5.289531in}{1.113572in}}%
\pgfpathlineto{\pgfqpoint{5.291590in}{1.181549in}}%
\pgfpathlineto{\pgfqpoint{5.299826in}{1.152416in}}%
\pgfpathlineto{\pgfqpoint{5.301885in}{1.162127in}}%
\pgfpathlineto{\pgfqpoint{5.306002in}{1.132994in}}%
\pgfpathlineto{\pgfqpoint{5.312179in}{1.142705in}}%
\pgfpathlineto{\pgfqpoint{5.314238in}{1.142705in}}%
\pgfpathlineto{\pgfqpoint{5.316297in}{1.152416in}}%
\pgfpathlineto{\pgfqpoint{5.318356in}{1.152416in}}%
\pgfpathlineto{\pgfqpoint{5.320415in}{1.162127in}}%
\pgfpathlineto{\pgfqpoint{5.326592in}{1.142705in}}%
\pgfpathlineto{\pgfqpoint{5.332769in}{1.142705in}}%
\pgfpathlineto{\pgfqpoint{5.334828in}{1.132994in}}%
\pgfpathlineto{\pgfqpoint{5.341005in}{1.142705in}}%
\pgfpathlineto{\pgfqpoint{5.343064in}{1.123283in}}%
\pgfpathlineto{\pgfqpoint{5.345122in}{1.152416in}}%
\pgfpathlineto{\pgfqpoint{5.347181in}{1.142705in}}%
\pgfpathlineto{\pgfqpoint{5.355417in}{1.230104in}}%
\pgfpathlineto{\pgfqpoint{5.357476in}{1.210682in}}%
\pgfpathlineto{\pgfqpoint{5.359535in}{1.239815in}}%
\pgfpathlineto{\pgfqpoint{5.361594in}{1.220393in}}%
\pgfpathlineto{\pgfqpoint{5.363653in}{1.230104in}}%
\pgfpathlineto{\pgfqpoint{5.373948in}{1.191260in}}%
\pgfpathlineto{\pgfqpoint{5.378066in}{1.210682in}}%
\pgfpathlineto{\pgfqpoint{5.384242in}{1.220393in}}%
\pgfpathlineto{\pgfqpoint{5.386301in}{1.249526in}}%
\pgfpathlineto{\pgfqpoint{5.388360in}{1.259237in}}%
\pgfpathlineto{\pgfqpoint{5.390419in}{1.298081in}}%
\pgfpathlineto{\pgfqpoint{5.398655in}{1.259237in}}%
\pgfpathlineto{\pgfqpoint{5.400714in}{1.239815in}}%
\pgfpathlineto{\pgfqpoint{5.402773in}{1.239815in}}%
\pgfpathlineto{\pgfqpoint{5.406891in}{1.317503in}}%
\pgfpathlineto{\pgfqpoint{5.413068in}{1.307792in}}%
\pgfpathlineto{\pgfqpoint{5.415127in}{1.327214in}}%
\pgfpathlineto{\pgfqpoint{5.417186in}{1.230104in}}%
\pgfpathlineto{\pgfqpoint{5.419244in}{1.239815in}}%
\pgfpathlineto{\pgfqpoint{5.421303in}{1.268948in}}%
\pgfpathlineto{\pgfqpoint{5.429539in}{1.395191in}}%
\pgfpathlineto{\pgfqpoint{5.433657in}{1.317503in}}%
\pgfpathlineto{\pgfqpoint{5.435716in}{1.327214in}}%
\pgfpathlineto{\pgfqpoint{5.441893in}{1.336925in}}%
\pgfpathlineto{\pgfqpoint{5.443952in}{1.307792in}}%
\pgfpathlineto{\pgfqpoint{5.446011in}{1.317503in}}%
\pgfpathlineto{\pgfqpoint{5.450129in}{1.298081in}}%
\pgfpathlineto{\pgfqpoint{5.460423in}{1.327214in}}%
\pgfpathlineto{\pgfqpoint{5.464541in}{1.288370in}}%
\pgfpathlineto{\pgfqpoint{5.470718in}{1.298081in}}%
\pgfpathlineto{\pgfqpoint{5.472777in}{1.356347in}}%
\pgfpathlineto{\pgfqpoint{5.474836in}{1.366058in}}%
\pgfpathlineto{\pgfqpoint{5.476895in}{1.346636in}}%
\pgfpathlineto{\pgfqpoint{5.478954in}{1.375769in}}%
\pgfpathlineto{\pgfqpoint{5.485131in}{1.346636in}}%
\pgfpathlineto{\pgfqpoint{5.487190in}{1.327214in}}%
\pgfpathlineto{\pgfqpoint{5.489249in}{1.356347in}}%
\pgfpathlineto{\pgfqpoint{5.493367in}{1.307792in}}%
\pgfpathlineto{\pgfqpoint{5.499543in}{1.307792in}}%
\pgfpathlineto{\pgfqpoint{5.501602in}{1.317503in}}%
\pgfpathlineto{\pgfqpoint{5.503661in}{1.317503in}}%
\pgfpathlineto{\pgfqpoint{5.507779in}{1.346636in}}%
\pgfpathlineto{\pgfqpoint{5.513956in}{1.336925in}}%
\pgfpathlineto{\pgfqpoint{5.516015in}{1.317503in}}%
\pgfpathlineto{\pgfqpoint{5.518074in}{1.346636in}}%
\pgfpathlineto{\pgfqpoint{5.520133in}{1.336925in}}%
\pgfpathlineto{\pgfqpoint{5.528369in}{1.327214in}}%
\pgfpathlineto{\pgfqpoint{5.530428in}{1.336925in}}%
\pgfpathlineto{\pgfqpoint{5.532487in}{1.336925in}}%
\pgfpathlineto{\pgfqpoint{5.534545in}{1.327214in}}%
\pgfpathlineto{\pgfqpoint{5.534545in}{1.327214in}}%
\pgfusepath{stroke}%
\end{pgfscope}%
\begin{pgfscope}%
\pgfpathrectangle{\pgfqpoint{0.800000in}{0.528000in}}{\pgfqpoint{4.960000in}{3.696000in}}%
\pgfusepath{clip}%
\pgfsetrectcap%
\pgfsetroundjoin%
\pgfsetlinewidth{1.003750pt}%
\definecolor{currentstroke}{rgb}{0.501961,0.501961,0.501961}%
\pgfsetstrokecolor{currentstroke}%
\pgfsetstrokeopacity{0.900000}%
\pgfsetdash{}{0pt}%
\pgfpathmoveto{\pgfqpoint{1.025455in}{2.754728in}}%
\pgfpathlineto{\pgfqpoint{1.031631in}{2.677040in}}%
\pgfpathlineto{\pgfqpoint{1.033690in}{2.609064in}}%
\pgfpathlineto{\pgfqpoint{1.035749in}{2.599353in}}%
\pgfpathlineto{\pgfqpoint{1.037808in}{2.667329in}}%
\pgfpathlineto{\pgfqpoint{1.039867in}{2.618775in}}%
\pgfpathlineto{\pgfqpoint{1.048103in}{2.550798in}}%
\pgfpathlineto{\pgfqpoint{1.050162in}{2.502243in}}%
\pgfpathlineto{\pgfqpoint{1.052221in}{2.414844in}}%
\pgfpathlineto{\pgfqpoint{1.054280in}{2.473110in}}%
\pgfpathlineto{\pgfqpoint{1.062516in}{2.463399in}}%
\pgfpathlineto{\pgfqpoint{1.066633in}{2.541087in}}%
\pgfpathlineto{\pgfqpoint{1.068692in}{2.453688in}}%
\pgfpathlineto{\pgfqpoint{1.074869in}{2.473110in}}%
\pgfpathlineto{\pgfqpoint{1.076928in}{2.473110in}}%
\pgfpathlineto{\pgfqpoint{1.078987in}{2.376000in}}%
\pgfpathlineto{\pgfqpoint{1.081046in}{2.414844in}}%
\pgfpathlineto{\pgfqpoint{1.083105in}{2.327445in}}%
\pgfpathlineto{\pgfqpoint{1.089282in}{2.327445in}}%
\pgfpathlineto{\pgfqpoint{1.091341in}{2.434266in}}%
\pgfpathlineto{\pgfqpoint{1.095459in}{2.473110in}}%
\pgfpathlineto{\pgfqpoint{1.097518in}{2.589642in}}%
\pgfpathlineto{\pgfqpoint{1.103694in}{2.599353in}}%
\pgfpathlineto{\pgfqpoint{1.105753in}{2.647908in}}%
\pgfpathlineto{\pgfqpoint{1.109871in}{2.628486in}}%
\pgfpathlineto{\pgfqpoint{1.120166in}{2.774150in}}%
\pgfpathlineto{\pgfqpoint{1.122225in}{2.706173in}}%
\pgfpathlineto{\pgfqpoint{1.126343in}{2.764439in}}%
\pgfpathlineto{\pgfqpoint{1.132520in}{2.696462in}}%
\pgfpathlineto{\pgfqpoint{1.134579in}{2.628486in}}%
\pgfpathlineto{\pgfqpoint{1.136638in}{2.599353in}}%
\pgfpathlineto{\pgfqpoint{1.138697in}{2.667329in}}%
\pgfpathlineto{\pgfqpoint{1.140756in}{2.638197in}}%
\pgfpathlineto{\pgfqpoint{1.151050in}{2.754728in}}%
\pgfpathlineto{\pgfqpoint{1.153109in}{2.745017in}}%
\pgfpathlineto{\pgfqpoint{1.155168in}{2.871260in}}%
\pgfpathlineto{\pgfqpoint{1.161345in}{2.832416in}}%
\pgfpathlineto{\pgfqpoint{1.165463in}{2.745017in}}%
\pgfpathlineto{\pgfqpoint{1.167522in}{2.735306in}}%
\pgfpathlineto{\pgfqpoint{1.169581in}{2.764439in}}%
\pgfpathlineto{\pgfqpoint{1.175758in}{2.735306in}}%
\pgfpathlineto{\pgfqpoint{1.177817in}{2.696462in}}%
\pgfpathlineto{\pgfqpoint{1.179875in}{2.570220in}}%
\pgfpathlineto{\pgfqpoint{1.181934in}{2.618775in}}%
\pgfpathlineto{\pgfqpoint{1.183993in}{2.570220in}}%
\pgfpathlineto{\pgfqpoint{1.190170in}{2.560509in}}%
\pgfpathlineto{\pgfqpoint{1.192229in}{2.521665in}}%
\pgfpathlineto{\pgfqpoint{1.194288in}{2.570220in}}%
\pgfpathlineto{\pgfqpoint{1.196347in}{2.647908in}}%
\pgfpathlineto{\pgfqpoint{1.198406in}{2.589642in}}%
\pgfpathlineto{\pgfqpoint{1.204583in}{2.599353in}}%
\pgfpathlineto{\pgfqpoint{1.206642in}{2.579931in}}%
\pgfpathlineto{\pgfqpoint{1.208701in}{2.511954in}}%
\pgfpathlineto{\pgfqpoint{1.210760in}{2.560509in}}%
\pgfpathlineto{\pgfqpoint{1.212819in}{2.492532in}}%
\pgfpathlineto{\pgfqpoint{1.218995in}{2.560509in}}%
\pgfpathlineto{\pgfqpoint{1.221054in}{2.531376in}}%
\pgfpathlineto{\pgfqpoint{1.223113in}{2.560509in}}%
\pgfpathlineto{\pgfqpoint{1.225172in}{2.609064in}}%
\pgfpathlineto{\pgfqpoint{1.227231in}{2.599353in}}%
\pgfpathlineto{\pgfqpoint{1.233408in}{2.579931in}}%
\pgfpathlineto{\pgfqpoint{1.235467in}{2.541087in}}%
\pgfpathlineto{\pgfqpoint{1.237526in}{2.550798in}}%
\pgfpathlineto{\pgfqpoint{1.239585in}{2.541087in}}%
\pgfpathlineto{\pgfqpoint{1.241644in}{2.511954in}}%
\pgfpathlineto{\pgfqpoint{1.247821in}{2.541087in}}%
\pgfpathlineto{\pgfqpoint{1.249880in}{2.560509in}}%
\pgfpathlineto{\pgfqpoint{1.251939in}{2.628486in}}%
\pgfpathlineto{\pgfqpoint{1.256056in}{2.570220in}}%
\pgfpathlineto{\pgfqpoint{1.262233in}{2.579931in}}%
\pgfpathlineto{\pgfqpoint{1.266351in}{2.696462in}}%
\pgfpathlineto{\pgfqpoint{1.268410in}{2.686751in}}%
\pgfpathlineto{\pgfqpoint{1.270469in}{2.754728in}}%
\pgfpathlineto{\pgfqpoint{1.276646in}{2.793572in}}%
\pgfpathlineto{\pgfqpoint{1.278705in}{2.822705in}}%
\pgfpathlineto{\pgfqpoint{1.280764in}{2.880971in}}%
\pgfpathlineto{\pgfqpoint{1.282823in}{2.812994in}}%
\pgfpathlineto{\pgfqpoint{1.284882in}{2.793572in}}%
\pgfpathlineto{\pgfqpoint{1.291059in}{2.910104in}}%
\pgfpathlineto{\pgfqpoint{1.293117in}{2.910104in}}%
\pgfpathlineto{\pgfqpoint{1.295176in}{2.900393in}}%
\pgfpathlineto{\pgfqpoint{1.297235in}{2.861549in}}%
\pgfpathlineto{\pgfqpoint{1.299294in}{2.774150in}}%
\pgfpathlineto{\pgfqpoint{1.307530in}{2.900393in}}%
\pgfpathlineto{\pgfqpoint{1.309589in}{2.890682in}}%
\pgfpathlineto{\pgfqpoint{1.311648in}{2.822705in}}%
\pgfpathlineto{\pgfqpoint{1.313707in}{2.842127in}}%
\pgfpathlineto{\pgfqpoint{1.321943in}{2.774150in}}%
\pgfpathlineto{\pgfqpoint{1.324002in}{2.774150in}}%
\pgfpathlineto{\pgfqpoint{1.328120in}{2.754728in}}%
\pgfpathlineto{\pgfqpoint{1.334296in}{2.822705in}}%
\pgfpathlineto{\pgfqpoint{1.338414in}{3.007214in}}%
\pgfpathlineto{\pgfqpoint{1.340473in}{2.939237in}}%
\pgfpathlineto{\pgfqpoint{1.342532in}{3.036347in}}%
\pgfpathlineto{\pgfqpoint{1.348709in}{3.016925in}}%
\pgfpathlineto{\pgfqpoint{1.350768in}{3.046058in}}%
\pgfpathlineto{\pgfqpoint{1.352827in}{3.123746in}}%
\pgfpathlineto{\pgfqpoint{1.354886in}{3.016925in}}%
\pgfpathlineto{\pgfqpoint{1.356945in}{3.016925in}}%
\pgfpathlineto{\pgfqpoint{1.363122in}{2.987792in}}%
\pgfpathlineto{\pgfqpoint{1.365181in}{2.948948in}}%
\pgfpathlineto{\pgfqpoint{1.367240in}{2.948948in}}%
\pgfpathlineto{\pgfqpoint{1.369298in}{2.978081in}}%
\pgfpathlineto{\pgfqpoint{1.371357in}{2.890682in}}%
\pgfpathlineto{\pgfqpoint{1.379593in}{3.046058in}}%
\pgfpathlineto{\pgfqpoint{1.381652in}{3.007214in}}%
\pgfpathlineto{\pgfqpoint{1.383711in}{3.026636in}}%
\pgfpathlineto{\pgfqpoint{1.385770in}{3.114035in}}%
\pgfpathlineto{\pgfqpoint{1.391947in}{2.958659in}}%
\pgfpathlineto{\pgfqpoint{1.394006in}{2.978081in}}%
\pgfpathlineto{\pgfqpoint{1.396065in}{3.055769in}}%
\pgfpathlineto{\pgfqpoint{1.398124in}{3.026636in}}%
\pgfpathlineto{\pgfqpoint{1.408418in}{2.900393in}}%
\pgfpathlineto{\pgfqpoint{1.410477in}{2.851838in}}%
\pgfpathlineto{\pgfqpoint{1.414595in}{3.046058in}}%
\pgfpathlineto{\pgfqpoint{1.420772in}{3.065480in}}%
\pgfpathlineto{\pgfqpoint{1.422831in}{3.036347in}}%
\pgfpathlineto{\pgfqpoint{1.424890in}{2.987792in}}%
\pgfpathlineto{\pgfqpoint{1.426949in}{2.987792in}}%
\pgfpathlineto{\pgfqpoint{1.429008in}{2.968370in}}%
\pgfpathlineto{\pgfqpoint{1.435185in}{3.007214in}}%
\pgfpathlineto{\pgfqpoint{1.443421in}{2.900393in}}%
\pgfpathlineto{\pgfqpoint{1.449597in}{2.861549in}}%
\pgfpathlineto{\pgfqpoint{1.453715in}{2.919815in}}%
\pgfpathlineto{\pgfqpoint{1.455774in}{2.910104in}}%
\pgfpathlineto{\pgfqpoint{1.457833in}{2.832416in}}%
\pgfpathlineto{\pgfqpoint{1.464010in}{2.793572in}}%
\pgfpathlineto{\pgfqpoint{1.468128in}{2.910104in}}%
\pgfpathlineto{\pgfqpoint{1.472246in}{2.812994in}}%
\pgfpathlineto{\pgfqpoint{1.478423in}{2.871260in}}%
\pgfpathlineto{\pgfqpoint{1.480482in}{2.783861in}}%
\pgfpathlineto{\pgfqpoint{1.482540in}{2.774150in}}%
\pgfpathlineto{\pgfqpoint{1.484599in}{2.822705in}}%
\pgfpathlineto{\pgfqpoint{1.486658in}{2.832416in}}%
\pgfpathlineto{\pgfqpoint{1.492835in}{2.793572in}}%
\pgfpathlineto{\pgfqpoint{1.494894in}{2.832416in}}%
\pgfpathlineto{\pgfqpoint{1.496953in}{2.754728in}}%
\pgfpathlineto{\pgfqpoint{1.501071in}{2.686751in}}%
\pgfpathlineto{\pgfqpoint{1.507248in}{2.647908in}}%
\pgfpathlineto{\pgfqpoint{1.511366in}{2.812994in}}%
\pgfpathlineto{\pgfqpoint{1.513425in}{2.812994in}}%
\pgfpathlineto{\pgfqpoint{1.515484in}{2.822705in}}%
\pgfpathlineto{\pgfqpoint{1.521660in}{2.842127in}}%
\pgfpathlineto{\pgfqpoint{1.523719in}{2.803283in}}%
\pgfpathlineto{\pgfqpoint{1.525778in}{2.832416in}}%
\pgfpathlineto{\pgfqpoint{1.527837in}{2.812994in}}%
\pgfpathlineto{\pgfqpoint{1.529896in}{2.764439in}}%
\pgfpathlineto{\pgfqpoint{1.542250in}{2.861549in}}%
\pgfpathlineto{\pgfqpoint{1.544309in}{2.832416in}}%
\pgfpathlineto{\pgfqpoint{1.550486in}{2.812994in}}%
\pgfpathlineto{\pgfqpoint{1.552545in}{2.910104in}}%
\pgfpathlineto{\pgfqpoint{1.554604in}{2.929526in}}%
\pgfpathlineto{\pgfqpoint{1.558721in}{2.764439in}}%
\pgfpathlineto{\pgfqpoint{1.564898in}{2.832416in}}%
\pgfpathlineto{\pgfqpoint{1.566957in}{2.774150in}}%
\pgfpathlineto{\pgfqpoint{1.569016in}{2.793572in}}%
\pgfpathlineto{\pgfqpoint{1.571075in}{2.764439in}}%
\pgfpathlineto{\pgfqpoint{1.573134in}{2.803283in}}%
\pgfpathlineto{\pgfqpoint{1.579311in}{2.735306in}}%
\pgfpathlineto{\pgfqpoint{1.581370in}{2.686751in}}%
\pgfpathlineto{\pgfqpoint{1.583429in}{2.696462in}}%
\pgfpathlineto{\pgfqpoint{1.585488in}{2.686751in}}%
\pgfpathlineto{\pgfqpoint{1.587547in}{2.628486in}}%
\pgfpathlineto{\pgfqpoint{1.593724in}{2.706173in}}%
\pgfpathlineto{\pgfqpoint{1.595782in}{2.686751in}}%
\pgfpathlineto{\pgfqpoint{1.599900in}{2.754728in}}%
\pgfpathlineto{\pgfqpoint{1.601959in}{2.754728in}}%
\pgfpathlineto{\pgfqpoint{1.610195in}{2.696462in}}%
\pgfpathlineto{\pgfqpoint{1.612254in}{2.628486in}}%
\pgfpathlineto{\pgfqpoint{1.614313in}{2.677040in}}%
\pgfpathlineto{\pgfqpoint{1.622549in}{2.677040in}}%
\pgfpathlineto{\pgfqpoint{1.624608in}{2.715884in}}%
\pgfpathlineto{\pgfqpoint{1.626667in}{2.677040in}}%
\pgfpathlineto{\pgfqpoint{1.628726in}{2.677040in}}%
\pgfpathlineto{\pgfqpoint{1.630785in}{2.725595in}}%
\pgfpathlineto{\pgfqpoint{1.636961in}{2.706173in}}%
\pgfpathlineto{\pgfqpoint{1.639020in}{2.686751in}}%
\pgfpathlineto{\pgfqpoint{1.641079in}{2.735306in}}%
\pgfpathlineto{\pgfqpoint{1.643138in}{2.822705in}}%
\pgfpathlineto{\pgfqpoint{1.645197in}{2.793572in}}%
\pgfpathlineto{\pgfqpoint{1.651374in}{2.832416in}}%
\pgfpathlineto{\pgfqpoint{1.655492in}{2.880971in}}%
\pgfpathlineto{\pgfqpoint{1.657551in}{2.890682in}}%
\pgfpathlineto{\pgfqpoint{1.659610in}{2.968370in}}%
\pgfpathlineto{\pgfqpoint{1.665787in}{2.987792in}}%
\pgfpathlineto{\pgfqpoint{1.667846in}{2.948948in}}%
\pgfpathlineto{\pgfqpoint{1.671963in}{2.948948in}}%
\pgfpathlineto{\pgfqpoint{1.674022in}{2.910104in}}%
\pgfpathlineto{\pgfqpoint{1.680199in}{2.900393in}}%
\pgfpathlineto{\pgfqpoint{1.682258in}{2.880971in}}%
\pgfpathlineto{\pgfqpoint{1.684317in}{2.900393in}}%
\pgfpathlineto{\pgfqpoint{1.686376in}{2.871260in}}%
\pgfpathlineto{\pgfqpoint{1.688435in}{2.890682in}}%
\pgfpathlineto{\pgfqpoint{1.694612in}{2.880971in}}%
\pgfpathlineto{\pgfqpoint{1.698730in}{2.861549in}}%
\pgfpathlineto{\pgfqpoint{1.702848in}{2.851838in}}%
\pgfpathlineto{\pgfqpoint{1.709024in}{2.842127in}}%
\pgfpathlineto{\pgfqpoint{1.711083in}{2.783861in}}%
\pgfpathlineto{\pgfqpoint{1.713142in}{2.812994in}}%
\pgfpathlineto{\pgfqpoint{1.715201in}{2.958659in}}%
\pgfpathlineto{\pgfqpoint{1.717260in}{2.910104in}}%
\pgfpathlineto{\pgfqpoint{1.723437in}{2.861549in}}%
\pgfpathlineto{\pgfqpoint{1.725496in}{2.871260in}}%
\pgfpathlineto{\pgfqpoint{1.727555in}{2.851838in}}%
\pgfpathlineto{\pgfqpoint{1.729614in}{2.871260in}}%
\pgfpathlineto{\pgfqpoint{1.731673in}{2.764439in}}%
\pgfpathlineto{\pgfqpoint{1.741968in}{2.929526in}}%
\pgfpathlineto{\pgfqpoint{1.746086in}{2.822705in}}%
\pgfpathlineto{\pgfqpoint{1.752262in}{2.832416in}}%
\pgfpathlineto{\pgfqpoint{1.756380in}{2.900393in}}%
\pgfpathlineto{\pgfqpoint{1.758439in}{2.880971in}}%
\pgfpathlineto{\pgfqpoint{1.766675in}{2.871260in}}%
\pgfpathlineto{\pgfqpoint{1.768734in}{2.948948in}}%
\pgfpathlineto{\pgfqpoint{1.770793in}{2.939237in}}%
\pgfpathlineto{\pgfqpoint{1.772852in}{2.900393in}}%
\pgfpathlineto{\pgfqpoint{1.781088in}{2.871260in}}%
\pgfpathlineto{\pgfqpoint{1.783147in}{2.880971in}}%
\pgfpathlineto{\pgfqpoint{1.785205in}{2.812994in}}%
\pgfpathlineto{\pgfqpoint{1.789323in}{2.764439in}}%
\pgfpathlineto{\pgfqpoint{1.795500in}{2.803283in}}%
\pgfpathlineto{\pgfqpoint{1.799618in}{2.715884in}}%
\pgfpathlineto{\pgfqpoint{1.801677in}{2.735306in}}%
\pgfpathlineto{\pgfqpoint{1.803736in}{2.667329in}}%
\pgfpathlineto{\pgfqpoint{1.811972in}{2.696462in}}%
\pgfpathlineto{\pgfqpoint{1.814031in}{2.647908in}}%
\pgfpathlineto{\pgfqpoint{1.816090in}{2.657618in}}%
\pgfpathlineto{\pgfqpoint{1.818149in}{2.706173in}}%
\pgfpathlineto{\pgfqpoint{1.826384in}{2.647908in}}%
\pgfpathlineto{\pgfqpoint{1.828443in}{2.657618in}}%
\pgfpathlineto{\pgfqpoint{1.830502in}{2.638197in}}%
\pgfpathlineto{\pgfqpoint{1.832561in}{2.579931in}}%
\pgfpathlineto{\pgfqpoint{1.838738in}{2.609064in}}%
\pgfpathlineto{\pgfqpoint{1.840797in}{2.511954in}}%
\pgfpathlineto{\pgfqpoint{1.842856in}{2.521665in}}%
\pgfpathlineto{\pgfqpoint{1.846974in}{2.502243in}}%
\pgfpathlineto{\pgfqpoint{1.853151in}{2.395422in}}%
\pgfpathlineto{\pgfqpoint{1.855210in}{2.385711in}}%
\pgfpathlineto{\pgfqpoint{1.857269in}{2.356578in}}%
\pgfpathlineto{\pgfqpoint{1.859328in}{2.278890in}}%
\pgfpathlineto{\pgfqpoint{1.861386in}{2.385711in}}%
\pgfpathlineto{\pgfqpoint{1.869622in}{2.424555in}}%
\pgfpathlineto{\pgfqpoint{1.871681in}{2.453688in}}%
\pgfpathlineto{\pgfqpoint{1.873740in}{2.395422in}}%
\pgfpathlineto{\pgfqpoint{1.875799in}{2.405133in}}%
\pgfpathlineto{\pgfqpoint{1.881976in}{2.414844in}}%
\pgfpathlineto{\pgfqpoint{1.884035in}{2.385711in}}%
\pgfpathlineto{\pgfqpoint{1.886094in}{2.395422in}}%
\pgfpathlineto{\pgfqpoint{1.888153in}{2.356578in}}%
\pgfpathlineto{\pgfqpoint{1.890212in}{2.405133in}}%
\pgfpathlineto{\pgfqpoint{1.896389in}{2.385711in}}%
\pgfpathlineto{\pgfqpoint{1.898447in}{2.473110in}}%
\pgfpathlineto{\pgfqpoint{1.900506in}{2.482821in}}%
\pgfpathlineto{\pgfqpoint{1.902565in}{2.473110in}}%
\pgfpathlineto{\pgfqpoint{1.904624in}{2.521665in}}%
\pgfpathlineto{\pgfqpoint{1.910801in}{2.550798in}}%
\pgfpathlineto{\pgfqpoint{1.912860in}{2.473110in}}%
\pgfpathlineto{\pgfqpoint{1.914919in}{2.541087in}}%
\pgfpathlineto{\pgfqpoint{1.916978in}{2.570220in}}%
\pgfpathlineto{\pgfqpoint{1.919037in}{2.618775in}}%
\pgfpathlineto{\pgfqpoint{1.925214in}{2.609064in}}%
\pgfpathlineto{\pgfqpoint{1.927273in}{2.609064in}}%
\pgfpathlineto{\pgfqpoint{1.933450in}{2.521665in}}%
\pgfpathlineto{\pgfqpoint{1.941685in}{2.579931in}}%
\pgfpathlineto{\pgfqpoint{1.943744in}{2.521665in}}%
\pgfpathlineto{\pgfqpoint{1.945803in}{2.550798in}}%
\pgfpathlineto{\pgfqpoint{1.954039in}{2.531376in}}%
\pgfpathlineto{\pgfqpoint{1.956098in}{2.453688in}}%
\pgfpathlineto{\pgfqpoint{1.958157in}{2.473110in}}%
\pgfpathlineto{\pgfqpoint{1.960216in}{2.424555in}}%
\pgfpathlineto{\pgfqpoint{1.962275in}{2.434266in}}%
\pgfpathlineto{\pgfqpoint{1.968452in}{2.424555in}}%
\pgfpathlineto{\pgfqpoint{1.970511in}{2.376000in}}%
\pgfpathlineto{\pgfqpoint{1.972570in}{2.405133in}}%
\pgfpathlineto{\pgfqpoint{1.974628in}{2.346867in}}%
\pgfpathlineto{\pgfqpoint{1.976687in}{2.366289in}}%
\pgfpathlineto{\pgfqpoint{1.982864in}{2.376000in}}%
\pgfpathlineto{\pgfqpoint{1.984923in}{2.434266in}}%
\pgfpathlineto{\pgfqpoint{1.986982in}{2.414844in}}%
\pgfpathlineto{\pgfqpoint{1.989041in}{2.443977in}}%
\pgfpathlineto{\pgfqpoint{1.991100in}{2.405133in}}%
\pgfpathlineto{\pgfqpoint{1.999336in}{2.434266in}}%
\pgfpathlineto{\pgfqpoint{2.003454in}{2.521665in}}%
\pgfpathlineto{\pgfqpoint{2.005513in}{2.531376in}}%
\pgfpathlineto{\pgfqpoint{2.011689in}{2.550798in}}%
\pgfpathlineto{\pgfqpoint{2.013748in}{2.579931in}}%
\pgfpathlineto{\pgfqpoint{2.015807in}{2.511954in}}%
\pgfpathlineto{\pgfqpoint{2.017866in}{2.482821in}}%
\pgfpathlineto{\pgfqpoint{2.019925in}{2.473110in}}%
\pgfpathlineto{\pgfqpoint{2.026102in}{2.521665in}}%
\pgfpathlineto{\pgfqpoint{2.028161in}{2.453688in}}%
\pgfpathlineto{\pgfqpoint{2.032279in}{2.405133in}}%
\pgfpathlineto{\pgfqpoint{2.034338in}{2.434266in}}%
\pgfpathlineto{\pgfqpoint{2.040515in}{2.414844in}}%
\pgfpathlineto{\pgfqpoint{2.042574in}{2.414844in}}%
\pgfpathlineto{\pgfqpoint{2.044633in}{2.376000in}}%
\pgfpathlineto{\pgfqpoint{2.046692in}{2.395422in}}%
\pgfpathlineto{\pgfqpoint{2.048751in}{2.356578in}}%
\pgfpathlineto{\pgfqpoint{2.056986in}{2.405133in}}%
\pgfpathlineto{\pgfqpoint{2.059045in}{2.511954in}}%
\pgfpathlineto{\pgfqpoint{2.061104in}{2.492532in}}%
\pgfpathlineto{\pgfqpoint{2.063163in}{2.492532in}}%
\pgfpathlineto{\pgfqpoint{2.069340in}{2.482821in}}%
\pgfpathlineto{\pgfqpoint{2.073458in}{2.511954in}}%
\pgfpathlineto{\pgfqpoint{2.075517in}{2.473110in}}%
\pgfpathlineto{\pgfqpoint{2.077576in}{2.492532in}}%
\pgfpathlineto{\pgfqpoint{2.085812in}{2.482821in}}%
\pgfpathlineto{\pgfqpoint{2.087870in}{2.492532in}}%
\pgfpathlineto{\pgfqpoint{2.089929in}{2.453688in}}%
\pgfpathlineto{\pgfqpoint{2.091988in}{2.356578in}}%
\pgfpathlineto{\pgfqpoint{2.098165in}{2.376000in}}%
\pgfpathlineto{\pgfqpoint{2.102283in}{2.356578in}}%
\pgfpathlineto{\pgfqpoint{2.106401in}{2.288601in}}%
\pgfpathlineto{\pgfqpoint{2.112578in}{2.269179in}}%
\pgfpathlineto{\pgfqpoint{2.114637in}{2.269179in}}%
\pgfpathlineto{\pgfqpoint{2.118755in}{2.220624in}}%
\pgfpathlineto{\pgfqpoint{2.120814in}{2.269179in}}%
\pgfpathlineto{\pgfqpoint{2.126990in}{2.317734in}}%
\pgfpathlineto{\pgfqpoint{2.129049in}{2.356578in}}%
\pgfpathlineto{\pgfqpoint{2.131108in}{2.337156in}}%
\pgfpathlineto{\pgfqpoint{2.133167in}{2.385711in}}%
\pgfpathlineto{\pgfqpoint{2.135226in}{2.220624in}}%
\pgfpathlineto{\pgfqpoint{2.141403in}{2.113803in}}%
\pgfpathlineto{\pgfqpoint{2.143462in}{2.113803in}}%
\pgfpathlineto{\pgfqpoint{2.145521in}{2.152647in}}%
\pgfpathlineto{\pgfqpoint{2.147580in}{2.142936in}}%
\pgfpathlineto{\pgfqpoint{2.149639in}{2.113803in}}%
\pgfpathlineto{\pgfqpoint{2.157875in}{2.026405in}}%
\pgfpathlineto{\pgfqpoint{2.159934in}{2.036116in}}%
\pgfpathlineto{\pgfqpoint{2.161993in}{2.055538in}}%
\pgfpathlineto{\pgfqpoint{2.164051in}{2.026405in}}%
\pgfpathlineto{\pgfqpoint{2.170228in}{2.084671in}}%
\pgfpathlineto{\pgfqpoint{2.172287in}{2.181780in}}%
\pgfpathlineto{\pgfqpoint{2.174346in}{2.133225in}}%
\pgfpathlineto{\pgfqpoint{2.178464in}{2.249757in}}%
\pgfpathlineto{\pgfqpoint{2.184641in}{2.240046in}}%
\pgfpathlineto{\pgfqpoint{2.186700in}{2.210913in}}%
\pgfpathlineto{\pgfqpoint{2.188759in}{2.240046in}}%
\pgfpathlineto{\pgfqpoint{2.190818in}{2.220624in}}%
\pgfpathlineto{\pgfqpoint{2.192877in}{2.220624in}}%
\pgfpathlineto{\pgfqpoint{2.199054in}{2.230335in}}%
\pgfpathlineto{\pgfqpoint{2.201112in}{2.220624in}}%
\pgfpathlineto{\pgfqpoint{2.203171in}{2.172069in}}%
\pgfpathlineto{\pgfqpoint{2.205230in}{2.172069in}}%
\pgfpathlineto{\pgfqpoint{2.207289in}{2.113803in}}%
\pgfpathlineto{\pgfqpoint{2.213466in}{2.162358in}}%
\pgfpathlineto{\pgfqpoint{2.215525in}{2.201202in}}%
\pgfpathlineto{\pgfqpoint{2.217584in}{2.201202in}}%
\pgfpathlineto{\pgfqpoint{2.219643in}{2.162358in}}%
\pgfpathlineto{\pgfqpoint{2.221702in}{2.240046in}}%
\pgfpathlineto{\pgfqpoint{2.227879in}{2.240046in}}%
\pgfpathlineto{\pgfqpoint{2.231997in}{2.152647in}}%
\pgfpathlineto{\pgfqpoint{2.234056in}{2.220624in}}%
\pgfpathlineto{\pgfqpoint{2.236115in}{2.162358in}}%
\pgfpathlineto{\pgfqpoint{2.244350in}{2.220624in}}%
\pgfpathlineto{\pgfqpoint{2.246409in}{2.210913in}}%
\pgfpathlineto{\pgfqpoint{2.248468in}{2.181780in}}%
\pgfpathlineto{\pgfqpoint{2.250527in}{2.230335in}}%
\pgfpathlineto{\pgfqpoint{2.256704in}{2.201202in}}%
\pgfpathlineto{\pgfqpoint{2.258763in}{2.201202in}}%
\pgfpathlineto{\pgfqpoint{2.260822in}{2.210913in}}%
\pgfpathlineto{\pgfqpoint{2.262881in}{2.230335in}}%
\pgfpathlineto{\pgfqpoint{2.264940in}{2.269179in}}%
\pgfpathlineto{\pgfqpoint{2.271117in}{2.220624in}}%
\pgfpathlineto{\pgfqpoint{2.273176in}{2.220624in}}%
\pgfpathlineto{\pgfqpoint{2.275235in}{2.230335in}}%
\pgfpathlineto{\pgfqpoint{2.277293in}{2.220624in}}%
\pgfpathlineto{\pgfqpoint{2.279352in}{2.249757in}}%
\pgfpathlineto{\pgfqpoint{2.289647in}{2.191491in}}%
\pgfpathlineto{\pgfqpoint{2.293765in}{2.317734in}}%
\pgfpathlineto{\pgfqpoint{2.299942in}{2.327445in}}%
\pgfpathlineto{\pgfqpoint{2.302001in}{2.376000in}}%
\pgfpathlineto{\pgfqpoint{2.304060in}{2.346867in}}%
\pgfpathlineto{\pgfqpoint{2.306119in}{2.356578in}}%
\pgfpathlineto{\pgfqpoint{2.308178in}{2.346867in}}%
\pgfpathlineto{\pgfqpoint{2.314355in}{2.346867in}}%
\pgfpathlineto{\pgfqpoint{2.316413in}{2.337156in}}%
\pgfpathlineto{\pgfqpoint{2.320531in}{2.278890in}}%
\pgfpathlineto{\pgfqpoint{2.328767in}{2.240046in}}%
\pgfpathlineto{\pgfqpoint{2.330826in}{2.210913in}}%
\pgfpathlineto{\pgfqpoint{2.332885in}{2.220624in}}%
\pgfpathlineto{\pgfqpoint{2.334944in}{2.210913in}}%
\pgfpathlineto{\pgfqpoint{2.337003in}{2.249757in}}%
\pgfpathlineto{\pgfqpoint{2.343180in}{2.278890in}}%
\pgfpathlineto{\pgfqpoint{2.347298in}{2.366289in}}%
\pgfpathlineto{\pgfqpoint{2.349357in}{2.395422in}}%
\pgfpathlineto{\pgfqpoint{2.351416in}{2.376000in}}%
\pgfpathlineto{\pgfqpoint{2.359651in}{2.414844in}}%
\pgfpathlineto{\pgfqpoint{2.361710in}{2.434266in}}%
\pgfpathlineto{\pgfqpoint{2.363769in}{2.395422in}}%
\pgfpathlineto{\pgfqpoint{2.365828in}{2.443977in}}%
\pgfpathlineto{\pgfqpoint{2.372005in}{2.414844in}}%
\pgfpathlineto{\pgfqpoint{2.374064in}{2.395422in}}%
\pgfpathlineto{\pgfqpoint{2.376123in}{2.405133in}}%
\pgfpathlineto{\pgfqpoint{2.378182in}{2.405133in}}%
\pgfpathlineto{\pgfqpoint{2.380241in}{2.385711in}}%
\pgfpathlineto{\pgfqpoint{2.386418in}{2.414844in}}%
\pgfpathlineto{\pgfqpoint{2.388477in}{2.414844in}}%
\pgfpathlineto{\pgfqpoint{2.390535in}{2.434266in}}%
\pgfpathlineto{\pgfqpoint{2.392594in}{2.492532in}}%
\pgfpathlineto{\pgfqpoint{2.394653in}{2.502243in}}%
\pgfpathlineto{\pgfqpoint{2.402889in}{2.473110in}}%
\pgfpathlineto{\pgfqpoint{2.404948in}{2.453688in}}%
\pgfpathlineto{\pgfqpoint{2.407007in}{2.463399in}}%
\pgfpathlineto{\pgfqpoint{2.409066in}{2.434266in}}%
\pgfpathlineto{\pgfqpoint{2.415243in}{2.473110in}}%
\pgfpathlineto{\pgfqpoint{2.417302in}{2.521665in}}%
\pgfpathlineto{\pgfqpoint{2.419361in}{2.706173in}}%
\pgfpathlineto{\pgfqpoint{2.421420in}{2.783861in}}%
\pgfpathlineto{\pgfqpoint{2.429655in}{2.861549in}}%
\pgfpathlineto{\pgfqpoint{2.431714in}{2.861549in}}%
\pgfpathlineto{\pgfqpoint{2.433773in}{2.851838in}}%
\pgfpathlineto{\pgfqpoint{2.437891in}{2.968370in}}%
\pgfpathlineto{\pgfqpoint{2.444068in}{2.958659in}}%
\pgfpathlineto{\pgfqpoint{2.446127in}{2.939237in}}%
\pgfpathlineto{\pgfqpoint{2.448186in}{2.987792in}}%
\pgfpathlineto{\pgfqpoint{2.452304in}{2.987792in}}%
\pgfpathlineto{\pgfqpoint{2.460540in}{2.929526in}}%
\pgfpathlineto{\pgfqpoint{2.464658in}{3.075191in}}%
\pgfpathlineto{\pgfqpoint{2.466716in}{3.026636in}}%
\pgfpathlineto{\pgfqpoint{2.472893in}{3.016925in}}%
\pgfpathlineto{\pgfqpoint{2.474952in}{3.016925in}}%
\pgfpathlineto{\pgfqpoint{2.477011in}{2.968370in}}%
\pgfpathlineto{\pgfqpoint{2.481129in}{3.094613in}}%
\pgfpathlineto{\pgfqpoint{2.487306in}{3.114035in}}%
\pgfpathlineto{\pgfqpoint{2.489365in}{3.104324in}}%
\pgfpathlineto{\pgfqpoint{2.493483in}{3.220855in}}%
\pgfpathlineto{\pgfqpoint{2.495542in}{3.220855in}}%
\pgfpathlineto{\pgfqpoint{2.501719in}{3.162590in}}%
\pgfpathlineto{\pgfqpoint{2.503778in}{3.191723in}}%
\pgfpathlineto{\pgfqpoint{2.505836in}{3.172301in}}%
\pgfpathlineto{\pgfqpoint{2.509954in}{3.172301in}}%
\pgfpathlineto{\pgfqpoint{2.518190in}{3.191723in}}%
\pgfpathlineto{\pgfqpoint{2.520249in}{3.133457in}}%
\pgfpathlineto{\pgfqpoint{2.522308in}{3.114035in}}%
\pgfpathlineto{\pgfqpoint{2.524367in}{3.075191in}}%
\pgfpathlineto{\pgfqpoint{2.532603in}{3.075191in}}%
\pgfpathlineto{\pgfqpoint{2.534662in}{3.084902in}}%
\pgfpathlineto{\pgfqpoint{2.536721in}{2.997503in}}%
\pgfpathlineto{\pgfqpoint{2.538780in}{3.046058in}}%
\pgfpathlineto{\pgfqpoint{2.544956in}{3.007214in}}%
\pgfpathlineto{\pgfqpoint{2.549074in}{3.007214in}}%
\pgfpathlineto{\pgfqpoint{2.551133in}{2.987792in}}%
\pgfpathlineto{\pgfqpoint{2.553192in}{3.026636in}}%
\pgfpathlineto{\pgfqpoint{2.561428in}{2.958659in}}%
\pgfpathlineto{\pgfqpoint{2.565546in}{3.094613in}}%
\pgfpathlineto{\pgfqpoint{2.567605in}{3.104324in}}%
\pgfpathlineto{\pgfqpoint{2.573782in}{3.036347in}}%
\pgfpathlineto{\pgfqpoint{2.577900in}{3.152879in}}%
\pgfpathlineto{\pgfqpoint{2.582017in}{3.114035in}}%
\pgfpathlineto{\pgfqpoint{2.588194in}{3.114035in}}%
\pgfpathlineto{\pgfqpoint{2.590253in}{3.075191in}}%
\pgfpathlineto{\pgfqpoint{2.592312in}{3.104324in}}%
\pgfpathlineto{\pgfqpoint{2.594371in}{3.104324in}}%
\pgfpathlineto{\pgfqpoint{2.596430in}{3.114035in}}%
\pgfpathlineto{\pgfqpoint{2.604666in}{3.026636in}}%
\pgfpathlineto{\pgfqpoint{2.606725in}{2.968370in}}%
\pgfpathlineto{\pgfqpoint{2.608784in}{3.026636in}}%
\pgfpathlineto{\pgfqpoint{2.610843in}{3.036347in}}%
\pgfpathlineto{\pgfqpoint{2.617020in}{3.055769in}}%
\pgfpathlineto{\pgfqpoint{2.621137in}{3.133457in}}%
\pgfpathlineto{\pgfqpoint{2.625255in}{3.046058in}}%
\pgfpathlineto{\pgfqpoint{2.633491in}{3.055769in}}%
\pgfpathlineto{\pgfqpoint{2.635550in}{3.046058in}}%
\pgfpathlineto{\pgfqpoint{2.637609in}{3.007214in}}%
\pgfpathlineto{\pgfqpoint{2.639668in}{2.939237in}}%
\pgfpathlineto{\pgfqpoint{2.645845in}{2.987792in}}%
\pgfpathlineto{\pgfqpoint{2.647904in}{2.987792in}}%
\pgfpathlineto{\pgfqpoint{2.649963in}{3.084902in}}%
\pgfpathlineto{\pgfqpoint{2.652022in}{3.114035in}}%
\pgfpathlineto{\pgfqpoint{2.660257in}{3.114035in}}%
\pgfpathlineto{\pgfqpoint{2.662316in}{3.143168in}}%
\pgfpathlineto{\pgfqpoint{2.666434in}{3.220855in}}%
\pgfpathlineto{\pgfqpoint{2.668493in}{3.201434in}}%
\pgfpathlineto{\pgfqpoint{2.674670in}{3.240277in}}%
\pgfpathlineto{\pgfqpoint{2.676729in}{3.220855in}}%
\pgfpathlineto{\pgfqpoint{2.678788in}{3.133457in}}%
\pgfpathlineto{\pgfqpoint{2.680847in}{3.152879in}}%
\pgfpathlineto{\pgfqpoint{2.682906in}{3.123746in}}%
\pgfpathlineto{\pgfqpoint{2.689083in}{3.094613in}}%
\pgfpathlineto{\pgfqpoint{2.693200in}{3.026636in}}%
\pgfpathlineto{\pgfqpoint{2.695259in}{3.036347in}}%
\pgfpathlineto{\pgfqpoint{2.697318in}{3.026636in}}%
\pgfpathlineto{\pgfqpoint{2.703495in}{3.007214in}}%
\pgfpathlineto{\pgfqpoint{2.705554in}{3.046058in}}%
\pgfpathlineto{\pgfqpoint{2.707613in}{3.016925in}}%
\pgfpathlineto{\pgfqpoint{2.709672in}{3.046058in}}%
\pgfpathlineto{\pgfqpoint{2.717908in}{2.978081in}}%
\pgfpathlineto{\pgfqpoint{2.719967in}{2.987792in}}%
\pgfpathlineto{\pgfqpoint{2.722026in}{2.968370in}}%
\pgfpathlineto{\pgfqpoint{2.724085in}{2.968370in}}%
\pgfpathlineto{\pgfqpoint{2.726144in}{3.007214in}}%
\pgfpathlineto{\pgfqpoint{2.732320in}{2.997503in}}%
\pgfpathlineto{\pgfqpoint{2.738497in}{2.871260in}}%
\pgfpathlineto{\pgfqpoint{2.746733in}{2.890682in}}%
\pgfpathlineto{\pgfqpoint{2.748792in}{2.812994in}}%
\pgfpathlineto{\pgfqpoint{2.752910in}{2.871260in}}%
\pgfpathlineto{\pgfqpoint{2.754969in}{2.871260in}}%
\pgfpathlineto{\pgfqpoint{2.761146in}{2.910104in}}%
\pgfpathlineto{\pgfqpoint{2.763205in}{2.978081in}}%
\pgfpathlineto{\pgfqpoint{2.767323in}{2.929526in}}%
\pgfpathlineto{\pgfqpoint{2.769381in}{2.919815in}}%
\pgfpathlineto{\pgfqpoint{2.775558in}{2.958659in}}%
\pgfpathlineto{\pgfqpoint{2.777617in}{2.919815in}}%
\pgfpathlineto{\pgfqpoint{2.781735in}{2.987792in}}%
\pgfpathlineto{\pgfqpoint{2.783794in}{2.987792in}}%
\pgfpathlineto{\pgfqpoint{2.789971in}{3.016925in}}%
\pgfpathlineto{\pgfqpoint{2.792030in}{3.046058in}}%
\pgfpathlineto{\pgfqpoint{2.794089in}{3.036347in}}%
\pgfpathlineto{\pgfqpoint{2.796148in}{3.016925in}}%
\pgfpathlineto{\pgfqpoint{2.798207in}{2.958659in}}%
\pgfpathlineto{\pgfqpoint{2.804384in}{2.968370in}}%
\pgfpathlineto{\pgfqpoint{2.806443in}{2.958659in}}%
\pgfpathlineto{\pgfqpoint{2.808501in}{2.851838in}}%
\pgfpathlineto{\pgfqpoint{2.810560in}{2.861549in}}%
\pgfpathlineto{\pgfqpoint{2.812619in}{2.861549in}}%
\pgfpathlineto{\pgfqpoint{2.818796in}{2.880971in}}%
\pgfpathlineto{\pgfqpoint{2.820855in}{2.919815in}}%
\pgfpathlineto{\pgfqpoint{2.822914in}{2.890682in}}%
\pgfpathlineto{\pgfqpoint{2.824973in}{2.880971in}}%
\pgfpathlineto{\pgfqpoint{2.827032in}{2.880971in}}%
\pgfpathlineto{\pgfqpoint{2.835268in}{2.842127in}}%
\pgfpathlineto{\pgfqpoint{2.839386in}{2.842127in}}%
\pgfpathlineto{\pgfqpoint{2.841445in}{2.783861in}}%
\pgfpathlineto{\pgfqpoint{2.847621in}{2.812994in}}%
\pgfpathlineto{\pgfqpoint{2.849680in}{2.774150in}}%
\pgfpathlineto{\pgfqpoint{2.851739in}{2.812994in}}%
\pgfpathlineto{\pgfqpoint{2.853798in}{2.822705in}}%
\pgfpathlineto{\pgfqpoint{2.855857in}{2.842127in}}%
\pgfpathlineto{\pgfqpoint{2.864093in}{2.842127in}}%
\pgfpathlineto{\pgfqpoint{2.866152in}{2.783861in}}%
\pgfpathlineto{\pgfqpoint{2.868211in}{2.793572in}}%
\pgfpathlineto{\pgfqpoint{2.870270in}{2.793572in}}%
\pgfpathlineto{\pgfqpoint{2.876447in}{2.822705in}}%
\pgfpathlineto{\pgfqpoint{2.878506in}{2.793572in}}%
\pgfpathlineto{\pgfqpoint{2.880565in}{2.793572in}}%
\pgfpathlineto{\pgfqpoint{2.882623in}{2.783861in}}%
\pgfpathlineto{\pgfqpoint{2.884682in}{2.783861in}}%
\pgfpathlineto{\pgfqpoint{2.890859in}{2.774150in}}%
\pgfpathlineto{\pgfqpoint{2.892918in}{2.842127in}}%
\pgfpathlineto{\pgfqpoint{2.894977in}{2.851838in}}%
\pgfpathlineto{\pgfqpoint{2.899095in}{2.939237in}}%
\pgfpathlineto{\pgfqpoint{2.905272in}{2.978081in}}%
\pgfpathlineto{\pgfqpoint{2.909390in}{2.958659in}}%
\pgfpathlineto{\pgfqpoint{2.913508in}{3.016925in}}%
\pgfpathlineto{\pgfqpoint{2.919685in}{3.007214in}}%
\pgfpathlineto{\pgfqpoint{2.921743in}{2.997503in}}%
\pgfpathlineto{\pgfqpoint{2.923802in}{2.958659in}}%
\pgfpathlineto{\pgfqpoint{2.925861in}{2.978081in}}%
\pgfpathlineto{\pgfqpoint{2.927920in}{2.958659in}}%
\pgfpathlineto{\pgfqpoint{2.934097in}{2.939237in}}%
\pgfpathlineto{\pgfqpoint{2.936156in}{2.900393in}}%
\pgfpathlineto{\pgfqpoint{2.940274in}{2.900393in}}%
\pgfpathlineto{\pgfqpoint{2.942333in}{2.871260in}}%
\pgfpathlineto{\pgfqpoint{2.948510in}{2.890682in}}%
\pgfpathlineto{\pgfqpoint{2.950569in}{2.958659in}}%
\pgfpathlineto{\pgfqpoint{2.952628in}{2.919815in}}%
\pgfpathlineto{\pgfqpoint{2.954687in}{2.948948in}}%
\pgfpathlineto{\pgfqpoint{2.956746in}{2.929526in}}%
\pgfpathlineto{\pgfqpoint{2.962922in}{2.929526in}}%
\pgfpathlineto{\pgfqpoint{2.964981in}{2.890682in}}%
\pgfpathlineto{\pgfqpoint{2.967040in}{2.900393in}}%
\pgfpathlineto{\pgfqpoint{2.969099in}{2.871260in}}%
\pgfpathlineto{\pgfqpoint{2.971158in}{2.900393in}}%
\pgfpathlineto{\pgfqpoint{2.977335in}{2.890682in}}%
\pgfpathlineto{\pgfqpoint{2.979394in}{2.919815in}}%
\pgfpathlineto{\pgfqpoint{2.983512in}{2.832416in}}%
\pgfpathlineto{\pgfqpoint{2.985571in}{2.822705in}}%
\pgfpathlineto{\pgfqpoint{2.991748in}{2.851838in}}%
\pgfpathlineto{\pgfqpoint{2.993807in}{2.900393in}}%
\pgfpathlineto{\pgfqpoint{2.997924in}{2.822705in}}%
\pgfpathlineto{\pgfqpoint{2.999983in}{2.822705in}}%
\pgfpathlineto{\pgfqpoint{3.006160in}{2.812994in}}%
\pgfpathlineto{\pgfqpoint{3.008219in}{2.851838in}}%
\pgfpathlineto{\pgfqpoint{3.010278in}{2.803283in}}%
\pgfpathlineto{\pgfqpoint{3.012337in}{2.822705in}}%
\pgfpathlineto{\pgfqpoint{3.014396in}{2.803283in}}%
\pgfpathlineto{\pgfqpoint{3.020573in}{2.793572in}}%
\pgfpathlineto{\pgfqpoint{3.022632in}{2.764439in}}%
\pgfpathlineto{\pgfqpoint{3.024691in}{2.783861in}}%
\pgfpathlineto{\pgfqpoint{3.026750in}{2.754728in}}%
\pgfpathlineto{\pgfqpoint{3.028809in}{2.793572in}}%
\pgfpathlineto{\pgfqpoint{3.037044in}{2.706173in}}%
\pgfpathlineto{\pgfqpoint{3.039103in}{2.735306in}}%
\pgfpathlineto{\pgfqpoint{3.041162in}{2.686751in}}%
\pgfpathlineto{\pgfqpoint{3.043221in}{2.696462in}}%
\pgfpathlineto{\pgfqpoint{3.053516in}{2.832416in}}%
\pgfpathlineto{\pgfqpoint{3.057634in}{2.832416in}}%
\pgfpathlineto{\pgfqpoint{3.065870in}{2.871260in}}%
\pgfpathlineto{\pgfqpoint{3.067929in}{2.910104in}}%
\pgfpathlineto{\pgfqpoint{3.072046in}{2.890682in}}%
\pgfpathlineto{\pgfqpoint{3.078223in}{2.851838in}}%
\pgfpathlineto{\pgfqpoint{3.080282in}{2.871260in}}%
\pgfpathlineto{\pgfqpoint{3.082341in}{2.939237in}}%
\pgfpathlineto{\pgfqpoint{3.084400in}{2.939237in}}%
\pgfpathlineto{\pgfqpoint{3.086459in}{2.958659in}}%
\pgfpathlineto{\pgfqpoint{3.092636in}{2.968370in}}%
\pgfpathlineto{\pgfqpoint{3.094695in}{2.958659in}}%
\pgfpathlineto{\pgfqpoint{3.096754in}{2.958659in}}%
\pgfpathlineto{\pgfqpoint{3.100872in}{2.997503in}}%
\pgfpathlineto{\pgfqpoint{3.109108in}{2.978081in}}%
\pgfpathlineto{\pgfqpoint{3.111166in}{2.978081in}}%
\pgfpathlineto{\pgfqpoint{3.113225in}{2.958659in}}%
\pgfpathlineto{\pgfqpoint{3.115284in}{2.910104in}}%
\pgfpathlineto{\pgfqpoint{3.121461in}{2.929526in}}%
\pgfpathlineto{\pgfqpoint{3.123520in}{2.929526in}}%
\pgfpathlineto{\pgfqpoint{3.125579in}{2.968370in}}%
\pgfpathlineto{\pgfqpoint{3.127638in}{2.958659in}}%
\pgfpathlineto{\pgfqpoint{3.129697in}{3.016925in}}%
\pgfpathlineto{\pgfqpoint{3.135874in}{3.007214in}}%
\pgfpathlineto{\pgfqpoint{3.139992in}{3.065480in}}%
\pgfpathlineto{\pgfqpoint{3.142051in}{3.084902in}}%
\pgfpathlineto{\pgfqpoint{3.144110in}{3.046058in}}%
\pgfpathlineto{\pgfqpoint{3.150286in}{2.997503in}}%
\pgfpathlineto{\pgfqpoint{3.152345in}{3.007214in}}%
\pgfpathlineto{\pgfqpoint{3.154404in}{2.997503in}}%
\pgfpathlineto{\pgfqpoint{3.158522in}{2.968370in}}%
\pgfpathlineto{\pgfqpoint{3.164699in}{2.948948in}}%
\pgfpathlineto{\pgfqpoint{3.168817in}{2.948948in}}%
\pgfpathlineto{\pgfqpoint{3.170876in}{2.958659in}}%
\pgfpathlineto{\pgfqpoint{3.172935in}{3.026636in}}%
\pgfpathlineto{\pgfqpoint{3.179112in}{3.026636in}}%
\pgfpathlineto{\pgfqpoint{3.181171in}{3.007214in}}%
\pgfpathlineto{\pgfqpoint{3.183230in}{2.958659in}}%
\pgfpathlineto{\pgfqpoint{3.185289in}{2.997503in}}%
\pgfpathlineto{\pgfqpoint{3.187347in}{2.978081in}}%
\pgfpathlineto{\pgfqpoint{3.193524in}{2.997503in}}%
\pgfpathlineto{\pgfqpoint{3.195583in}{2.987792in}}%
\pgfpathlineto{\pgfqpoint{3.197642in}{2.948948in}}%
\pgfpathlineto{\pgfqpoint{3.201760in}{2.968370in}}%
\pgfpathlineto{\pgfqpoint{3.207937in}{2.948948in}}%
\pgfpathlineto{\pgfqpoint{3.212055in}{2.997503in}}%
\pgfpathlineto{\pgfqpoint{3.214114in}{3.046058in}}%
\pgfpathlineto{\pgfqpoint{3.216173in}{2.997503in}}%
\pgfpathlineto{\pgfqpoint{3.222350in}{2.997503in}}%
\pgfpathlineto{\pgfqpoint{3.224408in}{2.987792in}}%
\pgfpathlineto{\pgfqpoint{3.226467in}{2.958659in}}%
\pgfpathlineto{\pgfqpoint{3.228526in}{2.997503in}}%
\pgfpathlineto{\pgfqpoint{3.230585in}{3.007214in}}%
\pgfpathlineto{\pgfqpoint{3.236762in}{3.016925in}}%
\pgfpathlineto{\pgfqpoint{3.238821in}{3.026636in}}%
\pgfpathlineto{\pgfqpoint{3.240880in}{2.987792in}}%
\pgfpathlineto{\pgfqpoint{3.242939in}{2.978081in}}%
\pgfpathlineto{\pgfqpoint{3.244998in}{2.978081in}}%
\pgfpathlineto{\pgfqpoint{3.251175in}{3.016925in}}%
\pgfpathlineto{\pgfqpoint{3.253234in}{3.084902in}}%
\pgfpathlineto{\pgfqpoint{3.255293in}{3.114035in}}%
\pgfpathlineto{\pgfqpoint{3.257352in}{3.104324in}}%
\pgfpathlineto{\pgfqpoint{3.259411in}{3.104324in}}%
\pgfpathlineto{\pgfqpoint{3.267646in}{3.094613in}}%
\pgfpathlineto{\pgfqpoint{3.269705in}{3.046058in}}%
\pgfpathlineto{\pgfqpoint{3.271764in}{3.055769in}}%
\pgfpathlineto{\pgfqpoint{3.273823in}{3.026636in}}%
\pgfpathlineto{\pgfqpoint{3.282059in}{3.084902in}}%
\pgfpathlineto{\pgfqpoint{3.284118in}{3.065480in}}%
\pgfpathlineto{\pgfqpoint{3.288236in}{3.094613in}}%
\pgfpathlineto{\pgfqpoint{3.294413in}{3.114035in}}%
\pgfpathlineto{\pgfqpoint{3.296472in}{3.172301in}}%
\pgfpathlineto{\pgfqpoint{3.298531in}{3.172301in}}%
\pgfpathlineto{\pgfqpoint{3.300589in}{3.162590in}}%
\pgfpathlineto{\pgfqpoint{3.302648in}{3.172301in}}%
\pgfpathlineto{\pgfqpoint{3.310884in}{3.162590in}}%
\pgfpathlineto{\pgfqpoint{3.312943in}{3.191723in}}%
\pgfpathlineto{\pgfqpoint{3.315002in}{3.240277in}}%
\pgfpathlineto{\pgfqpoint{3.317061in}{3.259699in}}%
\pgfpathlineto{\pgfqpoint{3.323238in}{3.279121in}}%
\pgfpathlineto{\pgfqpoint{3.325297in}{3.249988in}}%
\pgfpathlineto{\pgfqpoint{3.327356in}{3.269410in}}%
\pgfpathlineto{\pgfqpoint{3.329415in}{3.249988in}}%
\pgfpathlineto{\pgfqpoint{3.331474in}{3.279121in}}%
\pgfpathlineto{\pgfqpoint{3.337650in}{3.317965in}}%
\pgfpathlineto{\pgfqpoint{3.339709in}{3.347098in}}%
\pgfpathlineto{\pgfqpoint{3.341768in}{3.337387in}}%
\pgfpathlineto{\pgfqpoint{3.345886in}{3.453919in}}%
\pgfpathlineto{\pgfqpoint{3.352063in}{3.385942in}}%
\pgfpathlineto{\pgfqpoint{3.354122in}{3.405364in}}%
\pgfpathlineto{\pgfqpoint{3.356181in}{3.453919in}}%
\pgfpathlineto{\pgfqpoint{3.358240in}{3.463630in}}%
\pgfpathlineto{\pgfqpoint{3.360299in}{3.444208in}}%
\pgfpathlineto{\pgfqpoint{3.366476in}{3.473341in}}%
\pgfpathlineto{\pgfqpoint{3.368535in}{3.444208in}}%
\pgfpathlineto{\pgfqpoint{3.370594in}{3.521896in}}%
\pgfpathlineto{\pgfqpoint{3.372653in}{3.512185in}}%
\pgfpathlineto{\pgfqpoint{3.374711in}{3.483052in}}%
\pgfpathlineto{\pgfqpoint{3.382947in}{3.492763in}}%
\pgfpathlineto{\pgfqpoint{3.385006in}{3.551029in}}%
\pgfpathlineto{\pgfqpoint{3.387065in}{3.531607in}}%
\pgfpathlineto{\pgfqpoint{3.389124in}{3.492763in}}%
\pgfpathlineto{\pgfqpoint{3.395301in}{3.473341in}}%
\pgfpathlineto{\pgfqpoint{3.397360in}{3.512185in}}%
\pgfpathlineto{\pgfqpoint{3.399419in}{3.483052in}}%
\pgfpathlineto{\pgfqpoint{3.401478in}{3.424786in}}%
\pgfpathlineto{\pgfqpoint{3.403537in}{3.473341in}}%
\pgfpathlineto{\pgfqpoint{3.409714in}{3.492763in}}%
\pgfpathlineto{\pgfqpoint{3.411773in}{3.492763in}}%
\pgfpathlineto{\pgfqpoint{3.413831in}{3.502474in}}%
\pgfpathlineto{\pgfqpoint{3.415890in}{3.473341in}}%
\pgfpathlineto{\pgfqpoint{3.417949in}{3.512185in}}%
\pgfpathlineto{\pgfqpoint{3.424126in}{3.483052in}}%
\pgfpathlineto{\pgfqpoint{3.428244in}{3.424786in}}%
\pgfpathlineto{\pgfqpoint{3.430303in}{3.434497in}}%
\pgfpathlineto{\pgfqpoint{3.432362in}{3.463630in}}%
\pgfpathlineto{\pgfqpoint{3.438539in}{3.463630in}}%
\pgfpathlineto{\pgfqpoint{3.440598in}{3.502474in}}%
\pgfpathlineto{\pgfqpoint{3.442657in}{3.502474in}}%
\pgfpathlineto{\pgfqpoint{3.444716in}{3.444208in}}%
\pgfpathlineto{\pgfqpoint{3.446775in}{3.434497in}}%
\pgfpathlineto{\pgfqpoint{3.452951in}{3.463630in}}%
\pgfpathlineto{\pgfqpoint{3.455010in}{3.395653in}}%
\pgfpathlineto{\pgfqpoint{3.457069in}{3.385942in}}%
\pgfpathlineto{\pgfqpoint{3.459128in}{3.356809in}}%
\pgfpathlineto{\pgfqpoint{3.467364in}{3.347098in}}%
\pgfpathlineto{\pgfqpoint{3.469423in}{3.405364in}}%
\pgfpathlineto{\pgfqpoint{3.471482in}{3.405364in}}%
\pgfpathlineto{\pgfqpoint{3.473541in}{3.444208in}}%
\pgfpathlineto{\pgfqpoint{3.475600in}{3.385942in}}%
\pgfpathlineto{\pgfqpoint{3.481777in}{3.395653in}}%
\pgfpathlineto{\pgfqpoint{3.483836in}{3.415075in}}%
\pgfpathlineto{\pgfqpoint{3.485895in}{3.405364in}}%
\pgfpathlineto{\pgfqpoint{3.487954in}{3.444208in}}%
\pgfpathlineto{\pgfqpoint{3.490012in}{3.434497in}}%
\pgfpathlineto{\pgfqpoint{3.496189in}{3.444208in}}%
\pgfpathlineto{\pgfqpoint{3.498248in}{3.434497in}}%
\pgfpathlineto{\pgfqpoint{3.504425in}{3.570451in}}%
\pgfpathlineto{\pgfqpoint{3.510602in}{3.589873in}}%
\pgfpathlineto{\pgfqpoint{3.514720in}{3.638428in}}%
\pgfpathlineto{\pgfqpoint{3.518838in}{3.570451in}}%
\pgfpathlineto{\pgfqpoint{3.525015in}{3.560740in}}%
\pgfpathlineto{\pgfqpoint{3.527073in}{3.580162in}}%
\pgfpathlineto{\pgfqpoint{3.529132in}{3.580162in}}%
\pgfpathlineto{\pgfqpoint{3.531191in}{3.551029in}}%
\pgfpathlineto{\pgfqpoint{3.533250in}{3.560740in}}%
\pgfpathlineto{\pgfqpoint{3.539427in}{3.560740in}}%
\pgfpathlineto{\pgfqpoint{3.543545in}{3.609295in}}%
\pgfpathlineto{\pgfqpoint{3.545604in}{3.580162in}}%
\pgfpathlineto{\pgfqpoint{3.547663in}{3.580162in}}%
\pgfpathlineto{\pgfqpoint{3.553840in}{3.609295in}}%
\pgfpathlineto{\pgfqpoint{3.555899in}{3.686983in}}%
\pgfpathlineto{\pgfqpoint{3.557958in}{3.696694in}}%
\pgfpathlineto{\pgfqpoint{3.560017in}{3.716116in}}%
\pgfpathlineto{\pgfqpoint{3.562076in}{3.667561in}}%
\pgfpathlineto{\pgfqpoint{3.570311in}{3.667561in}}%
\pgfpathlineto{\pgfqpoint{3.576488in}{3.541318in}}%
\pgfpathlineto{\pgfqpoint{3.584724in}{3.385942in}}%
\pgfpathlineto{\pgfqpoint{3.586783in}{3.453919in}}%
\pgfpathlineto{\pgfqpoint{3.588842in}{3.444208in}}%
\pgfpathlineto{\pgfqpoint{3.590901in}{3.502474in}}%
\pgfpathlineto{\pgfqpoint{3.597078in}{3.551029in}}%
\pgfpathlineto{\pgfqpoint{3.599137in}{3.531607in}}%
\pgfpathlineto{\pgfqpoint{3.601196in}{3.580162in}}%
\pgfpathlineto{\pgfqpoint{3.603254in}{3.541318in}}%
\pgfpathlineto{\pgfqpoint{3.605313in}{3.541318in}}%
\pgfpathlineto{\pgfqpoint{3.611490in}{3.570451in}}%
\pgfpathlineto{\pgfqpoint{3.613549in}{3.570451in}}%
\pgfpathlineto{\pgfqpoint{3.615608in}{3.589873in}}%
\pgfpathlineto{\pgfqpoint{3.617667in}{3.551029in}}%
\pgfpathlineto{\pgfqpoint{3.619726in}{3.541318in}}%
\pgfpathlineto{\pgfqpoint{3.625903in}{3.531607in}}%
\pgfpathlineto{\pgfqpoint{3.627962in}{3.502474in}}%
\pgfpathlineto{\pgfqpoint{3.630021in}{3.541318in}}%
\pgfpathlineto{\pgfqpoint{3.632080in}{3.512185in}}%
\pgfpathlineto{\pgfqpoint{3.634139in}{3.512185in}}%
\pgfpathlineto{\pgfqpoint{3.640315in}{3.483052in}}%
\pgfpathlineto{\pgfqpoint{3.642374in}{3.492763in}}%
\pgfpathlineto{\pgfqpoint{3.644433in}{3.444208in}}%
\pgfpathlineto{\pgfqpoint{3.648551in}{3.463630in}}%
\pgfpathlineto{\pgfqpoint{3.654728in}{3.483052in}}%
\pgfpathlineto{\pgfqpoint{3.656787in}{3.444208in}}%
\pgfpathlineto{\pgfqpoint{3.660905in}{3.453919in}}%
\pgfpathlineto{\pgfqpoint{3.662964in}{3.434497in}}%
\pgfpathlineto{\pgfqpoint{3.671200in}{3.483052in}}%
\pgfpathlineto{\pgfqpoint{3.673259in}{3.463630in}}%
\pgfpathlineto{\pgfqpoint{3.675318in}{3.463630in}}%
\pgfpathlineto{\pgfqpoint{3.677377in}{3.444208in}}%
\pgfpathlineto{\pgfqpoint{3.685612in}{3.473341in}}%
\pgfpathlineto{\pgfqpoint{3.687671in}{3.492763in}}%
\pgfpathlineto{\pgfqpoint{3.689730in}{3.453919in}}%
\pgfpathlineto{\pgfqpoint{3.691789in}{3.502474in}}%
\pgfpathlineto{\pgfqpoint{3.697966in}{3.570451in}}%
\pgfpathlineto{\pgfqpoint{3.702084in}{3.551029in}}%
\pgfpathlineto{\pgfqpoint{3.704143in}{3.589873in}}%
\pgfpathlineto{\pgfqpoint{3.706202in}{3.570451in}}%
\pgfpathlineto{\pgfqpoint{3.712379in}{3.589873in}}%
\pgfpathlineto{\pgfqpoint{3.714438in}{3.570451in}}%
\pgfpathlineto{\pgfqpoint{3.716496in}{3.609295in}}%
\pgfpathlineto{\pgfqpoint{3.720614in}{3.560740in}}%
\pgfpathlineto{\pgfqpoint{3.726791in}{3.551029in}}%
\pgfpathlineto{\pgfqpoint{3.728850in}{3.589873in}}%
\pgfpathlineto{\pgfqpoint{3.732968in}{3.541318in}}%
\pgfpathlineto{\pgfqpoint{3.735027in}{3.483052in}}%
\pgfpathlineto{\pgfqpoint{3.741204in}{3.492763in}}%
\pgfpathlineto{\pgfqpoint{3.743263in}{3.502474in}}%
\pgfpathlineto{\pgfqpoint{3.745322in}{3.473341in}}%
\pgfpathlineto{\pgfqpoint{3.747381in}{3.483052in}}%
\pgfpathlineto{\pgfqpoint{3.749440in}{3.483052in}}%
\pgfpathlineto{\pgfqpoint{3.755616in}{3.434497in}}%
\pgfpathlineto{\pgfqpoint{3.757675in}{3.463630in}}%
\pgfpathlineto{\pgfqpoint{3.759734in}{3.434497in}}%
\pgfpathlineto{\pgfqpoint{3.763852in}{3.434497in}}%
\pgfpathlineto{\pgfqpoint{3.770029in}{3.463630in}}%
\pgfpathlineto{\pgfqpoint{3.772088in}{3.492763in}}%
\pgfpathlineto{\pgfqpoint{3.774147in}{3.502474in}}%
\pgfpathlineto{\pgfqpoint{3.776206in}{3.473341in}}%
\pgfpathlineto{\pgfqpoint{3.778265in}{3.473341in}}%
\pgfpathlineto{\pgfqpoint{3.786501in}{3.512185in}}%
\pgfpathlineto{\pgfqpoint{3.788560in}{3.512185in}}%
\pgfpathlineto{\pgfqpoint{3.790619in}{3.492763in}}%
\pgfpathlineto{\pgfqpoint{3.792677in}{3.551029in}}%
\pgfpathlineto{\pgfqpoint{3.798854in}{3.551029in}}%
\pgfpathlineto{\pgfqpoint{3.800913in}{3.589873in}}%
\pgfpathlineto{\pgfqpoint{3.802972in}{3.580162in}}%
\pgfpathlineto{\pgfqpoint{3.805031in}{3.580162in}}%
\pgfpathlineto{\pgfqpoint{3.807090in}{3.599584in}}%
\pgfpathlineto{\pgfqpoint{3.813267in}{3.599584in}}%
\pgfpathlineto{\pgfqpoint{3.817385in}{3.686983in}}%
\pgfpathlineto{\pgfqpoint{3.819444in}{3.677272in}}%
\pgfpathlineto{\pgfqpoint{3.821503in}{3.677272in}}%
\pgfpathlineto{\pgfqpoint{3.827680in}{3.686983in}}%
\pgfpathlineto{\pgfqpoint{3.829738in}{3.706405in}}%
\pgfpathlineto{\pgfqpoint{3.831797in}{3.667561in}}%
\pgfpathlineto{\pgfqpoint{3.833856in}{3.667561in}}%
\pgfpathlineto{\pgfqpoint{3.835915in}{3.657850in}}%
\pgfpathlineto{\pgfqpoint{3.842092in}{3.696694in}}%
\pgfpathlineto{\pgfqpoint{3.844151in}{3.657850in}}%
\pgfpathlineto{\pgfqpoint{3.846210in}{3.754960in}}%
\pgfpathlineto{\pgfqpoint{3.850328in}{3.832647in}}%
\pgfpathlineto{\pgfqpoint{3.858564in}{3.813225in}}%
\pgfpathlineto{\pgfqpoint{3.860623in}{3.822936in}}%
\pgfpathlineto{\pgfqpoint{3.862682in}{3.745249in}}%
\pgfpathlineto{\pgfqpoint{3.864741in}{3.754960in}}%
\pgfpathlineto{\pgfqpoint{3.870917in}{3.764671in}}%
\pgfpathlineto{\pgfqpoint{3.872976in}{3.764671in}}%
\pgfpathlineto{\pgfqpoint{3.875035in}{3.793803in}}%
\pgfpathlineto{\pgfqpoint{3.877094in}{3.774382in}}%
\pgfpathlineto{\pgfqpoint{3.879153in}{3.803514in}}%
\pgfpathlineto{\pgfqpoint{3.885330in}{3.803514in}}%
\pgfpathlineto{\pgfqpoint{3.887389in}{3.774382in}}%
\pgfpathlineto{\pgfqpoint{3.889448in}{3.706405in}}%
\pgfpathlineto{\pgfqpoint{3.891507in}{3.745249in}}%
\pgfpathlineto{\pgfqpoint{3.893566in}{3.686983in}}%
\pgfpathlineto{\pgfqpoint{3.899743in}{3.686983in}}%
\pgfpathlineto{\pgfqpoint{3.903861in}{3.754960in}}%
\pgfpathlineto{\pgfqpoint{3.905919in}{3.745249in}}%
\pgfpathlineto{\pgfqpoint{3.907978in}{3.822936in}}%
\pgfpathlineto{\pgfqpoint{3.914155in}{3.803514in}}%
\pgfpathlineto{\pgfqpoint{3.916214in}{3.822936in}}%
\pgfpathlineto{\pgfqpoint{3.918273in}{3.822936in}}%
\pgfpathlineto{\pgfqpoint{3.920332in}{3.842358in}}%
\pgfpathlineto{\pgfqpoint{3.922391in}{3.793803in}}%
\pgfpathlineto{\pgfqpoint{3.930627in}{3.745249in}}%
\pgfpathlineto{\pgfqpoint{3.936804in}{3.686983in}}%
\pgfpathlineto{\pgfqpoint{3.942980in}{3.667561in}}%
\pgfpathlineto{\pgfqpoint{3.947098in}{3.667561in}}%
\pgfpathlineto{\pgfqpoint{3.951216in}{3.657850in}}%
\pgfpathlineto{\pgfqpoint{3.957393in}{3.677272in}}%
\pgfpathlineto{\pgfqpoint{3.959452in}{3.667561in}}%
\pgfpathlineto{\pgfqpoint{3.961511in}{3.667561in}}%
\pgfpathlineto{\pgfqpoint{3.965629in}{3.619006in}}%
\pgfpathlineto{\pgfqpoint{3.971806in}{3.589873in}}%
\pgfpathlineto{\pgfqpoint{3.973865in}{3.521896in}}%
\pgfpathlineto{\pgfqpoint{3.980042in}{3.463630in}}%
\pgfpathlineto{\pgfqpoint{3.986218in}{3.463630in}}%
\pgfpathlineto{\pgfqpoint{3.990336in}{3.521896in}}%
\pgfpathlineto{\pgfqpoint{3.992395in}{3.521896in}}%
\pgfpathlineto{\pgfqpoint{3.994454in}{3.502474in}}%
\pgfpathlineto{\pgfqpoint{4.000631in}{3.473341in}}%
\pgfpathlineto{\pgfqpoint{4.004749in}{3.385942in}}%
\pgfpathlineto{\pgfqpoint{4.006808in}{3.405364in}}%
\pgfpathlineto{\pgfqpoint{4.008867in}{3.405364in}}%
\pgfpathlineto{\pgfqpoint{4.015044in}{3.356809in}}%
\pgfpathlineto{\pgfqpoint{4.019161in}{3.424786in}}%
\pgfpathlineto{\pgfqpoint{4.023279in}{3.337387in}}%
\pgfpathlineto{\pgfqpoint{4.029456in}{3.308254in}}%
\pgfpathlineto{\pgfqpoint{4.033574in}{3.279121in}}%
\pgfpathlineto{\pgfqpoint{4.035633in}{3.182012in}}%
\pgfpathlineto{\pgfqpoint{4.037692in}{3.288832in}}%
\pgfpathlineto{\pgfqpoint{4.043869in}{3.317965in}}%
\pgfpathlineto{\pgfqpoint{4.045928in}{3.347098in}}%
\pgfpathlineto{\pgfqpoint{4.047987in}{3.356809in}}%
\pgfpathlineto{\pgfqpoint{4.050046in}{3.356809in}}%
\pgfpathlineto{\pgfqpoint{4.052105in}{3.327676in}}%
\pgfpathlineto{\pgfqpoint{4.058281in}{3.327676in}}%
\pgfpathlineto{\pgfqpoint{4.062399in}{3.347098in}}%
\pgfpathlineto{\pgfqpoint{4.064458in}{3.366520in}}%
\pgfpathlineto{\pgfqpoint{4.066517in}{3.405364in}}%
\pgfpathlineto{\pgfqpoint{4.074753in}{3.356809in}}%
\pgfpathlineto{\pgfqpoint{4.076812in}{3.376231in}}%
\pgfpathlineto{\pgfqpoint{4.078871in}{3.337387in}}%
\pgfpathlineto{\pgfqpoint{4.080930in}{3.376231in}}%
\pgfpathlineto{\pgfqpoint{4.087107in}{3.366520in}}%
\pgfpathlineto{\pgfqpoint{4.091225in}{3.317965in}}%
\pgfpathlineto{\pgfqpoint{4.093284in}{3.249988in}}%
\pgfpathlineto{\pgfqpoint{4.095342in}{3.317965in}}%
\pgfpathlineto{\pgfqpoint{4.101519in}{3.347098in}}%
\pgfpathlineto{\pgfqpoint{4.105637in}{3.317965in}}%
\pgfpathlineto{\pgfqpoint{4.107696in}{3.269410in}}%
\pgfpathlineto{\pgfqpoint{4.109755in}{3.249988in}}%
\pgfpathlineto{\pgfqpoint{4.115932in}{3.269410in}}%
\pgfpathlineto{\pgfqpoint{4.120050in}{3.327676in}}%
\pgfpathlineto{\pgfqpoint{4.122109in}{3.279121in}}%
\pgfpathlineto{\pgfqpoint{4.124168in}{3.279121in}}%
\pgfpathlineto{\pgfqpoint{4.132403in}{3.269410in}}%
\pgfpathlineto{\pgfqpoint{4.134462in}{3.269410in}}%
\pgfpathlineto{\pgfqpoint{4.136521in}{3.308254in}}%
\pgfpathlineto{\pgfqpoint{4.138580in}{3.269410in}}%
\pgfpathlineto{\pgfqpoint{4.144757in}{3.288832in}}%
\pgfpathlineto{\pgfqpoint{4.146816in}{3.259699in}}%
\pgfpathlineto{\pgfqpoint{4.150934in}{3.347098in}}%
\pgfpathlineto{\pgfqpoint{4.152993in}{3.376231in}}%
\pgfpathlineto{\pgfqpoint{4.159170in}{3.337387in}}%
\pgfpathlineto{\pgfqpoint{4.161229in}{3.337387in}}%
\pgfpathlineto{\pgfqpoint{4.163288in}{3.308254in}}%
\pgfpathlineto{\pgfqpoint{4.165347in}{3.259699in}}%
\pgfpathlineto{\pgfqpoint{4.167406in}{3.240277in}}%
\pgfpathlineto{\pgfqpoint{4.173582in}{3.259699in}}%
\pgfpathlineto{\pgfqpoint{4.175641in}{3.230566in}}%
\pgfpathlineto{\pgfqpoint{4.177700in}{3.230566in}}%
\pgfpathlineto{\pgfqpoint{4.179759in}{3.249988in}}%
\pgfpathlineto{\pgfqpoint{4.181818in}{3.211145in}}%
\pgfpathlineto{\pgfqpoint{4.187995in}{3.220855in}}%
\pgfpathlineto{\pgfqpoint{4.190054in}{3.230566in}}%
\pgfpathlineto{\pgfqpoint{4.192113in}{3.162590in}}%
\pgfpathlineto{\pgfqpoint{4.194172in}{3.162590in}}%
\pgfpathlineto{\pgfqpoint{4.196231in}{3.065480in}}%
\pgfpathlineto{\pgfqpoint{4.202408in}{3.055769in}}%
\pgfpathlineto{\pgfqpoint{4.206526in}{3.016925in}}%
\pgfpathlineto{\pgfqpoint{4.208584in}{3.016925in}}%
\pgfpathlineto{\pgfqpoint{4.210643in}{3.036347in}}%
\pgfpathlineto{\pgfqpoint{4.216820in}{3.114035in}}%
\pgfpathlineto{\pgfqpoint{4.218879in}{3.104324in}}%
\pgfpathlineto{\pgfqpoint{4.220938in}{3.143168in}}%
\pgfpathlineto{\pgfqpoint{4.225056in}{3.123746in}}%
\pgfpathlineto{\pgfqpoint{4.231233in}{3.143168in}}%
\pgfpathlineto{\pgfqpoint{4.233292in}{3.133457in}}%
\pgfpathlineto{\pgfqpoint{4.235351in}{3.104324in}}%
\pgfpathlineto{\pgfqpoint{4.237410in}{3.133457in}}%
\pgfpathlineto{\pgfqpoint{4.239469in}{3.182012in}}%
\pgfpathlineto{\pgfqpoint{4.245645in}{3.172301in}}%
\pgfpathlineto{\pgfqpoint{4.247704in}{3.220855in}}%
\pgfpathlineto{\pgfqpoint{4.249763in}{3.211145in}}%
\pgfpathlineto{\pgfqpoint{4.251822in}{3.191723in}}%
\pgfpathlineto{\pgfqpoint{4.260058in}{3.211145in}}%
\pgfpathlineto{\pgfqpoint{4.262117in}{3.191723in}}%
\pgfpathlineto{\pgfqpoint{4.264176in}{3.152879in}}%
\pgfpathlineto{\pgfqpoint{4.266235in}{3.162590in}}%
\pgfpathlineto{\pgfqpoint{4.268294in}{3.133457in}}%
\pgfpathlineto{\pgfqpoint{4.274471in}{3.162590in}}%
\pgfpathlineto{\pgfqpoint{4.276530in}{3.133457in}}%
\pgfpathlineto{\pgfqpoint{4.278589in}{3.143168in}}%
\pgfpathlineto{\pgfqpoint{4.280648in}{3.172301in}}%
\pgfpathlineto{\pgfqpoint{4.288883in}{3.133457in}}%
\pgfpathlineto{\pgfqpoint{4.290942in}{3.075191in}}%
\pgfpathlineto{\pgfqpoint{4.293001in}{3.114035in}}%
\pgfpathlineto{\pgfqpoint{4.295060in}{3.075191in}}%
\pgfpathlineto{\pgfqpoint{4.297119in}{3.094613in}}%
\pgfpathlineto{\pgfqpoint{4.303296in}{3.026636in}}%
\pgfpathlineto{\pgfqpoint{4.305355in}{3.046058in}}%
\pgfpathlineto{\pgfqpoint{4.307414in}{2.997503in}}%
\pgfpathlineto{\pgfqpoint{4.309473in}{3.026636in}}%
\pgfpathlineto{\pgfqpoint{4.311532in}{3.016925in}}%
\pgfpathlineto{\pgfqpoint{4.317709in}{3.036347in}}%
\pgfpathlineto{\pgfqpoint{4.319768in}{3.055769in}}%
\pgfpathlineto{\pgfqpoint{4.321826in}{3.016925in}}%
\pgfpathlineto{\pgfqpoint{4.323885in}{2.939237in}}%
\pgfpathlineto{\pgfqpoint{4.325944in}{2.948948in}}%
\pgfpathlineto{\pgfqpoint{4.336239in}{2.880971in}}%
\pgfpathlineto{\pgfqpoint{4.338298in}{2.851838in}}%
\pgfpathlineto{\pgfqpoint{4.340357in}{2.774150in}}%
\pgfpathlineto{\pgfqpoint{4.346534in}{2.706173in}}%
\pgfpathlineto{\pgfqpoint{4.348593in}{2.754728in}}%
\pgfpathlineto{\pgfqpoint{4.352711in}{2.754728in}}%
\pgfpathlineto{\pgfqpoint{4.354770in}{2.725595in}}%
\pgfpathlineto{\pgfqpoint{4.360946in}{2.783861in}}%
\pgfpathlineto{\pgfqpoint{4.363005in}{2.783861in}}%
\pgfpathlineto{\pgfqpoint{4.369182in}{2.725595in}}%
\pgfpathlineto{\pgfqpoint{4.375359in}{2.725595in}}%
\pgfpathlineto{\pgfqpoint{4.381536in}{2.647908in}}%
\pgfpathlineto{\pgfqpoint{4.383595in}{2.706173in}}%
\pgfpathlineto{\pgfqpoint{4.391831in}{2.638197in}}%
\pgfpathlineto{\pgfqpoint{4.393890in}{2.686751in}}%
\pgfpathlineto{\pgfqpoint{4.395949in}{2.647908in}}%
\pgfpathlineto{\pgfqpoint{4.398007in}{2.638197in}}%
\pgfpathlineto{\pgfqpoint{4.404184in}{2.667329in}}%
\pgfpathlineto{\pgfqpoint{4.406243in}{2.618775in}}%
\pgfpathlineto{\pgfqpoint{4.408302in}{2.599353in}}%
\pgfpathlineto{\pgfqpoint{4.412420in}{2.677040in}}%
\pgfpathlineto{\pgfqpoint{4.418597in}{2.686751in}}%
\pgfpathlineto{\pgfqpoint{4.420656in}{2.706173in}}%
\pgfpathlineto{\pgfqpoint{4.422715in}{2.706173in}}%
\pgfpathlineto{\pgfqpoint{4.424774in}{2.764439in}}%
\pgfpathlineto{\pgfqpoint{4.433010in}{2.725595in}}%
\pgfpathlineto{\pgfqpoint{4.435068in}{2.764439in}}%
\pgfpathlineto{\pgfqpoint{4.437127in}{2.696462in}}%
\pgfpathlineto{\pgfqpoint{4.439186in}{2.677040in}}%
\pgfpathlineto{\pgfqpoint{4.441245in}{2.686751in}}%
\pgfpathlineto{\pgfqpoint{4.447422in}{2.686751in}}%
\pgfpathlineto{\pgfqpoint{4.449481in}{2.715884in}}%
\pgfpathlineto{\pgfqpoint{4.451540in}{2.686751in}}%
\pgfpathlineto{\pgfqpoint{4.453599in}{2.715884in}}%
\pgfpathlineto{\pgfqpoint{4.455658in}{2.715884in}}%
\pgfpathlineto{\pgfqpoint{4.461835in}{2.696462in}}%
\pgfpathlineto{\pgfqpoint{4.463894in}{2.696462in}}%
\pgfpathlineto{\pgfqpoint{4.465953in}{2.657618in}}%
\pgfpathlineto{\pgfqpoint{4.468012in}{2.541087in}}%
\pgfpathlineto{\pgfqpoint{4.476247in}{2.395422in}}%
\pgfpathlineto{\pgfqpoint{4.480365in}{2.356578in}}%
\pgfpathlineto{\pgfqpoint{4.482424in}{2.366289in}}%
\pgfpathlineto{\pgfqpoint{4.484483in}{2.385711in}}%
\pgfpathlineto{\pgfqpoint{4.490660in}{2.298312in}}%
\pgfpathlineto{\pgfqpoint{4.492719in}{2.327445in}}%
\pgfpathlineto{\pgfqpoint{4.496837in}{2.172069in}}%
\pgfpathlineto{\pgfqpoint{4.498896in}{2.201202in}}%
\pgfpathlineto{\pgfqpoint{4.505073in}{2.249757in}}%
\pgfpathlineto{\pgfqpoint{4.507132in}{2.201202in}}%
\pgfpathlineto{\pgfqpoint{4.511249in}{2.269179in}}%
\pgfpathlineto{\pgfqpoint{4.513308in}{2.172069in}}%
\pgfpathlineto{\pgfqpoint{4.519485in}{2.191491in}}%
\pgfpathlineto{\pgfqpoint{4.521544in}{2.142936in}}%
\pgfpathlineto{\pgfqpoint{4.523603in}{2.123514in}}%
\pgfpathlineto{\pgfqpoint{4.525662in}{2.152647in}}%
\pgfpathlineto{\pgfqpoint{4.527721in}{2.152647in}}%
\pgfpathlineto{\pgfqpoint{4.535957in}{2.123514in}}%
\pgfpathlineto{\pgfqpoint{4.538016in}{2.123514in}}%
\pgfpathlineto{\pgfqpoint{4.540075in}{2.220624in}}%
\pgfpathlineto{\pgfqpoint{4.542134in}{2.201202in}}%
\pgfpathlineto{\pgfqpoint{4.548311in}{2.278890in}}%
\pgfpathlineto{\pgfqpoint{4.550369in}{2.366289in}}%
\pgfpathlineto{\pgfqpoint{4.554487in}{2.434266in}}%
\pgfpathlineto{\pgfqpoint{4.556546in}{2.541087in}}%
\pgfpathlineto{\pgfqpoint{4.568900in}{2.434266in}}%
\pgfpathlineto{\pgfqpoint{4.570959in}{2.385711in}}%
\pgfpathlineto{\pgfqpoint{4.577136in}{2.366289in}}%
\pgfpathlineto{\pgfqpoint{4.579195in}{2.288601in}}%
\pgfpathlineto{\pgfqpoint{4.581254in}{2.376000in}}%
\pgfpathlineto{\pgfqpoint{4.583313in}{2.346867in}}%
\pgfpathlineto{\pgfqpoint{4.585372in}{2.337156in}}%
\pgfpathlineto{\pgfqpoint{4.591548in}{2.327445in}}%
\pgfpathlineto{\pgfqpoint{4.593607in}{2.298312in}}%
\pgfpathlineto{\pgfqpoint{4.599784in}{2.172069in}}%
\pgfpathlineto{\pgfqpoint{4.605961in}{2.210913in}}%
\pgfpathlineto{\pgfqpoint{4.608020in}{2.191491in}}%
\pgfpathlineto{\pgfqpoint{4.610079in}{2.240046in}}%
\pgfpathlineto{\pgfqpoint{4.614197in}{2.405133in}}%
\pgfpathlineto{\pgfqpoint{4.622433in}{2.414844in}}%
\pgfpathlineto{\pgfqpoint{4.624491in}{2.395422in}}%
\pgfpathlineto{\pgfqpoint{4.626550in}{2.405133in}}%
\pgfpathlineto{\pgfqpoint{4.628609in}{2.405133in}}%
\pgfpathlineto{\pgfqpoint{4.634786in}{2.443977in}}%
\pgfpathlineto{\pgfqpoint{4.638904in}{2.414844in}}%
\pgfpathlineto{\pgfqpoint{4.640963in}{2.414844in}}%
\pgfpathlineto{\pgfqpoint{4.643022in}{2.443977in}}%
\pgfpathlineto{\pgfqpoint{4.649199in}{2.492532in}}%
\pgfpathlineto{\pgfqpoint{4.651258in}{2.482821in}}%
\pgfpathlineto{\pgfqpoint{4.655376in}{2.337156in}}%
\pgfpathlineto{\pgfqpoint{4.657435in}{2.376000in}}%
\pgfpathlineto{\pgfqpoint{4.663611in}{2.434266in}}%
\pgfpathlineto{\pgfqpoint{4.665670in}{2.502243in}}%
\pgfpathlineto{\pgfqpoint{4.667729in}{2.453688in}}%
\pgfpathlineto{\pgfqpoint{4.669788in}{2.560509in}}%
\pgfpathlineto{\pgfqpoint{4.671847in}{2.579931in}}%
\pgfpathlineto{\pgfqpoint{4.680083in}{2.560509in}}%
\pgfpathlineto{\pgfqpoint{4.684201in}{2.463399in}}%
\pgfpathlineto{\pgfqpoint{4.686260in}{2.482821in}}%
\pgfpathlineto{\pgfqpoint{4.692437in}{2.453688in}}%
\pgfpathlineto{\pgfqpoint{4.694496in}{2.434266in}}%
\pgfpathlineto{\pgfqpoint{4.696555in}{2.376000in}}%
\pgfpathlineto{\pgfqpoint{4.698614in}{2.414844in}}%
\pgfpathlineto{\pgfqpoint{4.700672in}{2.414844in}}%
\pgfpathlineto{\pgfqpoint{4.706849in}{2.405133in}}%
\pgfpathlineto{\pgfqpoint{4.708908in}{2.385711in}}%
\pgfpathlineto{\pgfqpoint{4.710967in}{2.414844in}}%
\pgfpathlineto{\pgfqpoint{4.715085in}{2.424555in}}%
\pgfpathlineto{\pgfqpoint{4.721262in}{2.473110in}}%
\pgfpathlineto{\pgfqpoint{4.723321in}{2.366289in}}%
\pgfpathlineto{\pgfqpoint{4.727439in}{2.443977in}}%
\pgfpathlineto{\pgfqpoint{4.729498in}{2.482821in}}%
\pgfpathlineto{\pgfqpoint{4.735675in}{2.473110in}}%
\pgfpathlineto{\pgfqpoint{4.737733in}{2.492532in}}%
\pgfpathlineto{\pgfqpoint{4.739792in}{2.434266in}}%
\pgfpathlineto{\pgfqpoint{4.741851in}{2.541087in}}%
\pgfpathlineto{\pgfqpoint{4.743910in}{2.463399in}}%
\pgfpathlineto{\pgfqpoint{4.750087in}{2.531376in}}%
\pgfpathlineto{\pgfqpoint{4.752146in}{2.531376in}}%
\pgfpathlineto{\pgfqpoint{4.754205in}{2.560509in}}%
\pgfpathlineto{\pgfqpoint{4.758323in}{2.560509in}}%
\pgfpathlineto{\pgfqpoint{4.764500in}{2.570220in}}%
\pgfpathlineto{\pgfqpoint{4.766559in}{2.541087in}}%
\pgfpathlineto{\pgfqpoint{4.770677in}{2.541087in}}%
\pgfpathlineto{\pgfqpoint{4.772736in}{2.521665in}}%
\pgfpathlineto{\pgfqpoint{4.778912in}{2.541087in}}%
\pgfpathlineto{\pgfqpoint{4.780971in}{2.560509in}}%
\pgfpathlineto{\pgfqpoint{4.785089in}{2.521665in}}%
\pgfpathlineto{\pgfqpoint{4.787148in}{2.443977in}}%
\pgfpathlineto{\pgfqpoint{4.793325in}{2.453688in}}%
\pgfpathlineto{\pgfqpoint{4.795384in}{2.473110in}}%
\pgfpathlineto{\pgfqpoint{4.797443in}{2.511954in}}%
\pgfpathlineto{\pgfqpoint{4.801561in}{2.473110in}}%
\pgfpathlineto{\pgfqpoint{4.807738in}{2.492532in}}%
\pgfpathlineto{\pgfqpoint{4.811856in}{2.434266in}}%
\pgfpathlineto{\pgfqpoint{4.815973in}{2.482821in}}%
\pgfpathlineto{\pgfqpoint{4.826268in}{2.414844in}}%
\pgfpathlineto{\pgfqpoint{4.836563in}{2.259468in}}%
\pgfpathlineto{\pgfqpoint{4.838622in}{2.298312in}}%
\pgfpathlineto{\pgfqpoint{4.844799in}{2.162358in}}%
\pgfpathlineto{\pgfqpoint{4.850976in}{2.191491in}}%
\pgfpathlineto{\pgfqpoint{4.855093in}{2.308023in}}%
\pgfpathlineto{\pgfqpoint{4.857152in}{2.298312in}}%
\pgfpathlineto{\pgfqpoint{4.859211in}{2.240046in}}%
\pgfpathlineto{\pgfqpoint{4.865388in}{2.210913in}}%
\pgfpathlineto{\pgfqpoint{4.869506in}{2.269179in}}%
\pgfpathlineto{\pgfqpoint{4.871565in}{2.259468in}}%
\pgfpathlineto{\pgfqpoint{4.873624in}{2.240046in}}%
\pgfpathlineto{\pgfqpoint{4.881860in}{2.201202in}}%
\pgfpathlineto{\pgfqpoint{4.883919in}{2.210913in}}%
\pgfpathlineto{\pgfqpoint{4.896272in}{1.987561in}}%
\pgfpathlineto{\pgfqpoint{4.898331in}{1.987561in}}%
\pgfpathlineto{\pgfqpoint{4.900390in}{1.958428in}}%
\pgfpathlineto{\pgfqpoint{4.902449in}{1.793341in}}%
\pgfpathlineto{\pgfqpoint{4.908626in}{1.764208in}}%
\pgfpathlineto{\pgfqpoint{4.910685in}{1.686520in}}%
\pgfpathlineto{\pgfqpoint{4.912744in}{1.686520in}}%
\pgfpathlineto{\pgfqpoint{4.914803in}{1.589410in}}%
\pgfpathlineto{\pgfqpoint{4.916862in}{1.414613in}}%
\pgfpathlineto{\pgfqpoint{4.923039in}{1.220393in}}%
\pgfpathlineto{\pgfqpoint{4.925098in}{1.434035in}}%
\pgfpathlineto{\pgfqpoint{4.931274in}{1.608832in}}%
\pgfpathlineto{\pgfqpoint{4.937451in}{1.404902in}}%
\pgfpathlineto{\pgfqpoint{4.941569in}{1.841896in}}%
\pgfpathlineto{\pgfqpoint{4.943628in}{1.783630in}}%
\pgfpathlineto{\pgfqpoint{4.945687in}{1.589410in}}%
\pgfpathlineto{\pgfqpoint{4.951864in}{1.434035in}}%
\pgfpathlineto{\pgfqpoint{4.955982in}{1.550566in}}%
\pgfpathlineto{\pgfqpoint{4.958041in}{1.502012in}}%
\pgfpathlineto{\pgfqpoint{4.960100in}{1.395191in}}%
\pgfpathlineto{\pgfqpoint{4.966276in}{1.375769in}}%
\pgfpathlineto{\pgfqpoint{4.968335in}{1.375769in}}%
\pgfpathlineto{\pgfqpoint{4.970394in}{1.298081in}}%
\pgfpathlineto{\pgfqpoint{4.972453in}{1.307792in}}%
\pgfpathlineto{\pgfqpoint{4.974512in}{1.298081in}}%
\pgfpathlineto{\pgfqpoint{4.980689in}{1.346636in}}%
\pgfpathlineto{\pgfqpoint{4.982748in}{1.424324in}}%
\pgfpathlineto{\pgfqpoint{4.984807in}{1.443746in}}%
\pgfpathlineto{\pgfqpoint{4.986866in}{1.404902in}}%
\pgfpathlineto{\pgfqpoint{4.995102in}{1.434035in}}%
\pgfpathlineto{\pgfqpoint{4.997161in}{1.434035in}}%
\pgfpathlineto{\pgfqpoint{4.999220in}{1.307792in}}%
\pgfpathlineto{\pgfqpoint{5.001279in}{1.288370in}}%
\pgfpathlineto{\pgfqpoint{5.003337in}{1.327214in}}%
\pgfpathlineto{\pgfqpoint{5.009514in}{1.307792in}}%
\pgfpathlineto{\pgfqpoint{5.011573in}{1.259237in}}%
\pgfpathlineto{\pgfqpoint{5.013632in}{1.307792in}}%
\pgfpathlineto{\pgfqpoint{5.017750in}{1.278659in}}%
\pgfpathlineto{\pgfqpoint{5.023927in}{1.346636in}}%
\pgfpathlineto{\pgfqpoint{5.025986in}{1.298081in}}%
\pgfpathlineto{\pgfqpoint{5.030104in}{1.317503in}}%
\pgfpathlineto{\pgfqpoint{5.038340in}{1.317503in}}%
\pgfpathlineto{\pgfqpoint{5.040399in}{1.336925in}}%
\pgfpathlineto{\pgfqpoint{5.042457in}{1.395191in}}%
\pgfpathlineto{\pgfqpoint{5.044516in}{1.307792in}}%
\pgfpathlineto{\pgfqpoint{5.046575in}{1.366058in}}%
\pgfpathlineto{\pgfqpoint{5.052752in}{1.404902in}}%
\pgfpathlineto{\pgfqpoint{5.058929in}{1.307792in}}%
\pgfpathlineto{\pgfqpoint{5.060988in}{1.317503in}}%
\pgfpathlineto{\pgfqpoint{5.067165in}{1.404902in}}%
\pgfpathlineto{\pgfqpoint{5.071283in}{1.356347in}}%
\pgfpathlineto{\pgfqpoint{5.073342in}{1.356347in}}%
\pgfpathlineto{\pgfqpoint{5.075401in}{1.336925in}}%
\pgfpathlineto{\pgfqpoint{5.083636in}{1.366058in}}%
\pgfpathlineto{\pgfqpoint{5.085695in}{1.356347in}}%
\pgfpathlineto{\pgfqpoint{5.087754in}{1.375769in}}%
\pgfpathlineto{\pgfqpoint{5.089813in}{1.327214in}}%
\pgfpathlineto{\pgfqpoint{5.095990in}{1.336925in}}%
\pgfpathlineto{\pgfqpoint{5.098049in}{1.356347in}}%
\pgfpathlineto{\pgfqpoint{5.104226in}{1.579699in}}%
\pgfpathlineto{\pgfqpoint{5.110403in}{1.550566in}}%
\pgfpathlineto{\pgfqpoint{5.112462in}{1.511723in}}%
\pgfpathlineto{\pgfqpoint{5.116579in}{1.336925in}}%
\pgfpathlineto{\pgfqpoint{5.118638in}{1.385480in}}%
\pgfpathlineto{\pgfqpoint{5.124815in}{1.385480in}}%
\pgfpathlineto{\pgfqpoint{5.126874in}{1.424324in}}%
\pgfpathlineto{\pgfqpoint{5.128933in}{1.414613in}}%
\pgfpathlineto{\pgfqpoint{5.130992in}{1.385480in}}%
\pgfpathlineto{\pgfqpoint{5.133051in}{1.375769in}}%
\pgfpathlineto{\pgfqpoint{5.139228in}{1.385480in}}%
\pgfpathlineto{\pgfqpoint{5.141287in}{1.395191in}}%
\pgfpathlineto{\pgfqpoint{5.143346in}{1.366058in}}%
\pgfpathlineto{\pgfqpoint{5.145405in}{1.356347in}}%
\pgfpathlineto{\pgfqpoint{5.147464in}{1.317503in}}%
\pgfpathlineto{\pgfqpoint{5.153641in}{1.317503in}}%
\pgfpathlineto{\pgfqpoint{5.157758in}{1.366058in}}%
\pgfpathlineto{\pgfqpoint{5.159817in}{1.356347in}}%
\pgfpathlineto{\pgfqpoint{5.168053in}{1.366058in}}%
\pgfpathlineto{\pgfqpoint{5.170112in}{1.327214in}}%
\pgfpathlineto{\pgfqpoint{5.172171in}{1.346636in}}%
\pgfpathlineto{\pgfqpoint{5.174230in}{1.298081in}}%
\pgfpathlineto{\pgfqpoint{5.176289in}{1.327214in}}%
\pgfpathlineto{\pgfqpoint{5.182466in}{1.317503in}}%
\pgfpathlineto{\pgfqpoint{5.184525in}{1.307792in}}%
\pgfpathlineto{\pgfqpoint{5.186584in}{1.317503in}}%
\pgfpathlineto{\pgfqpoint{5.188643in}{1.298081in}}%
\pgfpathlineto{\pgfqpoint{5.190702in}{1.317503in}}%
\pgfpathlineto{\pgfqpoint{5.198937in}{1.288370in}}%
\pgfpathlineto{\pgfqpoint{5.203055in}{1.268948in}}%
\pgfpathlineto{\pgfqpoint{5.205114in}{1.268948in}}%
\pgfpathlineto{\pgfqpoint{5.211291in}{1.298081in}}%
\pgfpathlineto{\pgfqpoint{5.213350in}{1.268948in}}%
\pgfpathlineto{\pgfqpoint{5.215409in}{1.259237in}}%
\pgfpathlineto{\pgfqpoint{5.217468in}{1.230104in}}%
\pgfpathlineto{\pgfqpoint{5.219527in}{1.230104in}}%
\pgfpathlineto{\pgfqpoint{5.225704in}{1.239815in}}%
\pgfpathlineto{\pgfqpoint{5.227763in}{1.200971in}}%
\pgfpathlineto{\pgfqpoint{5.229822in}{1.230104in}}%
\pgfpathlineto{\pgfqpoint{5.231880in}{1.230104in}}%
\pgfpathlineto{\pgfqpoint{5.233939in}{1.249526in}}%
\pgfpathlineto{\pgfqpoint{5.240116in}{1.268948in}}%
\pgfpathlineto{\pgfqpoint{5.244234in}{1.366058in}}%
\pgfpathlineto{\pgfqpoint{5.246293in}{1.385480in}}%
\pgfpathlineto{\pgfqpoint{5.248352in}{1.385480in}}%
\pgfpathlineto{\pgfqpoint{5.254529in}{1.366058in}}%
\pgfpathlineto{\pgfqpoint{5.256588in}{1.346636in}}%
\pgfpathlineto{\pgfqpoint{5.258647in}{1.356347in}}%
\pgfpathlineto{\pgfqpoint{5.260706in}{1.327214in}}%
\pgfpathlineto{\pgfqpoint{5.262765in}{1.317503in}}%
\pgfpathlineto{\pgfqpoint{5.268941in}{1.327214in}}%
\pgfpathlineto{\pgfqpoint{5.271000in}{1.366058in}}%
\pgfpathlineto{\pgfqpoint{5.273059in}{1.366058in}}%
\pgfpathlineto{\pgfqpoint{5.275118in}{1.414613in}}%
\pgfpathlineto{\pgfqpoint{5.277177in}{1.414613in}}%
\pgfpathlineto{\pgfqpoint{5.283354in}{1.395191in}}%
\pgfpathlineto{\pgfqpoint{5.287472in}{1.336925in}}%
\pgfpathlineto{\pgfqpoint{5.289531in}{1.307792in}}%
\pgfpathlineto{\pgfqpoint{5.291590in}{1.395191in}}%
\pgfpathlineto{\pgfqpoint{5.299826in}{1.366058in}}%
\pgfpathlineto{\pgfqpoint{5.301885in}{1.385480in}}%
\pgfpathlineto{\pgfqpoint{5.303944in}{1.356347in}}%
\pgfpathlineto{\pgfqpoint{5.306002in}{1.346636in}}%
\pgfpathlineto{\pgfqpoint{5.312179in}{1.356347in}}%
\pgfpathlineto{\pgfqpoint{5.314238in}{1.356347in}}%
\pgfpathlineto{\pgfqpoint{5.316297in}{1.366058in}}%
\pgfpathlineto{\pgfqpoint{5.318356in}{1.366058in}}%
\pgfpathlineto{\pgfqpoint{5.320415in}{1.375769in}}%
\pgfpathlineto{\pgfqpoint{5.326592in}{1.356347in}}%
\pgfpathlineto{\pgfqpoint{5.330710in}{1.356347in}}%
\pgfpathlineto{\pgfqpoint{5.334828in}{1.336925in}}%
\pgfpathlineto{\pgfqpoint{5.341005in}{1.346636in}}%
\pgfpathlineto{\pgfqpoint{5.343064in}{1.336925in}}%
\pgfpathlineto{\pgfqpoint{5.345122in}{1.366058in}}%
\pgfpathlineto{\pgfqpoint{5.347181in}{1.356347in}}%
\pgfpathlineto{\pgfqpoint{5.349240in}{1.375769in}}%
\pgfpathlineto{\pgfqpoint{5.355417in}{1.453457in}}%
\pgfpathlineto{\pgfqpoint{5.357476in}{1.434035in}}%
\pgfpathlineto{\pgfqpoint{5.359535in}{1.482590in}}%
\pgfpathlineto{\pgfqpoint{5.361594in}{1.453457in}}%
\pgfpathlineto{\pgfqpoint{5.363653in}{1.463168in}}%
\pgfpathlineto{\pgfqpoint{5.373948in}{1.404902in}}%
\pgfpathlineto{\pgfqpoint{5.376007in}{1.414613in}}%
\pgfpathlineto{\pgfqpoint{5.378066in}{1.434035in}}%
\pgfpathlineto{\pgfqpoint{5.384242in}{1.453457in}}%
\pgfpathlineto{\pgfqpoint{5.390419in}{1.540855in}}%
\pgfpathlineto{\pgfqpoint{5.392478in}{1.521434in}}%
\pgfpathlineto{\pgfqpoint{5.402773in}{1.463168in}}%
\pgfpathlineto{\pgfqpoint{5.406891in}{1.550566in}}%
\pgfpathlineto{\pgfqpoint{5.413068in}{1.540855in}}%
\pgfpathlineto{\pgfqpoint{5.415127in}{1.569988in}}%
\pgfpathlineto{\pgfqpoint{5.417186in}{1.453457in}}%
\pgfpathlineto{\pgfqpoint{5.419244in}{1.463168in}}%
\pgfpathlineto{\pgfqpoint{5.429539in}{1.647676in}}%
\pgfpathlineto{\pgfqpoint{5.433657in}{1.550566in}}%
\pgfpathlineto{\pgfqpoint{5.435716in}{1.560277in}}%
\pgfpathlineto{\pgfqpoint{5.441893in}{1.579699in}}%
\pgfpathlineto{\pgfqpoint{5.443952in}{1.540855in}}%
\pgfpathlineto{\pgfqpoint{5.446011in}{1.550566in}}%
\pgfpathlineto{\pgfqpoint{5.450129in}{1.502012in}}%
\pgfpathlineto{\pgfqpoint{5.456306in}{1.531145in}}%
\pgfpathlineto{\pgfqpoint{5.458364in}{1.550566in}}%
\pgfpathlineto{\pgfqpoint{5.460423in}{1.550566in}}%
\pgfpathlineto{\pgfqpoint{5.464541in}{1.511723in}}%
\pgfpathlineto{\pgfqpoint{5.470718in}{1.511723in}}%
\pgfpathlineto{\pgfqpoint{5.472777in}{1.589410in}}%
\pgfpathlineto{\pgfqpoint{5.474836in}{1.618543in}}%
\pgfpathlineto{\pgfqpoint{5.476895in}{1.589410in}}%
\pgfpathlineto{\pgfqpoint{5.478954in}{1.637965in}}%
\pgfpathlineto{\pgfqpoint{5.485131in}{1.608832in}}%
\pgfpathlineto{\pgfqpoint{5.487190in}{1.589410in}}%
\pgfpathlineto{\pgfqpoint{5.489249in}{1.618543in}}%
\pgfpathlineto{\pgfqpoint{5.493367in}{1.569988in}}%
\pgfpathlineto{\pgfqpoint{5.499543in}{1.569988in}}%
\pgfpathlineto{\pgfqpoint{5.501602in}{1.589410in}}%
\pgfpathlineto{\pgfqpoint{5.503661in}{1.589410in}}%
\pgfpathlineto{\pgfqpoint{5.507779in}{1.618543in}}%
\pgfpathlineto{\pgfqpoint{5.513956in}{1.618543in}}%
\pgfpathlineto{\pgfqpoint{5.516015in}{1.599121in}}%
\pgfpathlineto{\pgfqpoint{5.518074in}{1.628254in}}%
\pgfpathlineto{\pgfqpoint{5.520133in}{1.608832in}}%
\pgfpathlineto{\pgfqpoint{5.530428in}{1.608832in}}%
\pgfpathlineto{\pgfqpoint{5.532487in}{1.599121in}}%
\pgfpathlineto{\pgfqpoint{5.534545in}{1.599121in}}%
\pgfpathlineto{\pgfqpoint{5.534545in}{1.599121in}}%
\pgfusepath{stroke}%
\end{pgfscope}%
\begin{pgfscope}%
\pgfpathrectangle{\pgfqpoint{0.800000in}{0.528000in}}{\pgfqpoint{4.960000in}{3.696000in}}%
\pgfusepath{clip}%
\pgfsetrectcap%
\pgfsetroundjoin%
\pgfsetlinewidth{1.003750pt}%
\definecolor{currentstroke}{rgb}{0.501961,0.501961,0.501961}%
\pgfsetstrokecolor{currentstroke}%
\pgfsetstrokeopacity{0.900000}%
\pgfsetdash{}{0pt}%
\pgfpathmoveto{\pgfqpoint{1.025455in}{3.036347in}}%
\pgfpathlineto{\pgfqpoint{1.031631in}{2.948948in}}%
\pgfpathlineto{\pgfqpoint{1.033690in}{2.880971in}}%
\pgfpathlineto{\pgfqpoint{1.035749in}{2.880971in}}%
\pgfpathlineto{\pgfqpoint{1.037808in}{2.958659in}}%
\pgfpathlineto{\pgfqpoint{1.039867in}{2.919815in}}%
\pgfpathlineto{\pgfqpoint{1.046044in}{2.861549in}}%
\pgfpathlineto{\pgfqpoint{1.048103in}{2.871260in}}%
\pgfpathlineto{\pgfqpoint{1.050162in}{2.832416in}}%
\pgfpathlineto{\pgfqpoint{1.052221in}{2.754728in}}%
\pgfpathlineto{\pgfqpoint{1.054280in}{2.803283in}}%
\pgfpathlineto{\pgfqpoint{1.062516in}{2.783861in}}%
\pgfpathlineto{\pgfqpoint{1.064575in}{2.832416in}}%
\pgfpathlineto{\pgfqpoint{1.066633in}{2.842127in}}%
\pgfpathlineto{\pgfqpoint{1.068692in}{2.754728in}}%
\pgfpathlineto{\pgfqpoint{1.076928in}{2.783861in}}%
\pgfpathlineto{\pgfqpoint{1.078987in}{2.686751in}}%
\pgfpathlineto{\pgfqpoint{1.081046in}{2.745017in}}%
\pgfpathlineto{\pgfqpoint{1.083105in}{2.677040in}}%
\pgfpathlineto{\pgfqpoint{1.089282in}{2.706173in}}%
\pgfpathlineto{\pgfqpoint{1.091341in}{2.793572in}}%
\pgfpathlineto{\pgfqpoint{1.093400in}{2.803283in}}%
\pgfpathlineto{\pgfqpoint{1.095459in}{2.832416in}}%
\pgfpathlineto{\pgfqpoint{1.097518in}{2.910104in}}%
\pgfpathlineto{\pgfqpoint{1.103694in}{2.910104in}}%
\pgfpathlineto{\pgfqpoint{1.105753in}{2.958659in}}%
\pgfpathlineto{\pgfqpoint{1.109871in}{2.958659in}}%
\pgfpathlineto{\pgfqpoint{1.111930in}{3.016925in}}%
\pgfpathlineto{\pgfqpoint{1.120166in}{3.114035in}}%
\pgfpathlineto{\pgfqpoint{1.122225in}{3.084902in}}%
\pgfpathlineto{\pgfqpoint{1.124284in}{3.123746in}}%
\pgfpathlineto{\pgfqpoint{1.126343in}{3.123746in}}%
\pgfpathlineto{\pgfqpoint{1.132520in}{3.065480in}}%
\pgfpathlineto{\pgfqpoint{1.136638in}{2.978081in}}%
\pgfpathlineto{\pgfqpoint{1.138697in}{3.016925in}}%
\pgfpathlineto{\pgfqpoint{1.140756in}{3.007214in}}%
\pgfpathlineto{\pgfqpoint{1.148991in}{3.114035in}}%
\pgfpathlineto{\pgfqpoint{1.153109in}{3.114035in}}%
\pgfpathlineto{\pgfqpoint{1.155168in}{3.249988in}}%
\pgfpathlineto{\pgfqpoint{1.161345in}{3.201434in}}%
\pgfpathlineto{\pgfqpoint{1.165463in}{3.094613in}}%
\pgfpathlineto{\pgfqpoint{1.167522in}{3.094613in}}%
\pgfpathlineto{\pgfqpoint{1.169581in}{3.104324in}}%
\pgfpathlineto{\pgfqpoint{1.175758in}{3.075191in}}%
\pgfpathlineto{\pgfqpoint{1.177817in}{3.026636in}}%
\pgfpathlineto{\pgfqpoint{1.179875in}{2.939237in}}%
\pgfpathlineto{\pgfqpoint{1.181934in}{2.958659in}}%
\pgfpathlineto{\pgfqpoint{1.183993in}{2.919815in}}%
\pgfpathlineto{\pgfqpoint{1.190170in}{2.919815in}}%
\pgfpathlineto{\pgfqpoint{1.192229in}{2.871260in}}%
\pgfpathlineto{\pgfqpoint{1.194288in}{2.910104in}}%
\pgfpathlineto{\pgfqpoint{1.196347in}{2.997503in}}%
\pgfpathlineto{\pgfqpoint{1.198406in}{2.919815in}}%
\pgfpathlineto{\pgfqpoint{1.204583in}{2.948948in}}%
\pgfpathlineto{\pgfqpoint{1.206642in}{2.939237in}}%
\pgfpathlineto{\pgfqpoint{1.208701in}{2.861549in}}%
\pgfpathlineto{\pgfqpoint{1.210760in}{2.919815in}}%
\pgfpathlineto{\pgfqpoint{1.212819in}{2.871260in}}%
\pgfpathlineto{\pgfqpoint{1.218995in}{2.939237in}}%
\pgfpathlineto{\pgfqpoint{1.221054in}{2.900393in}}%
\pgfpathlineto{\pgfqpoint{1.223113in}{2.910104in}}%
\pgfpathlineto{\pgfqpoint{1.225172in}{2.978081in}}%
\pgfpathlineto{\pgfqpoint{1.227231in}{2.958659in}}%
\pgfpathlineto{\pgfqpoint{1.233408in}{2.958659in}}%
\pgfpathlineto{\pgfqpoint{1.235467in}{2.919815in}}%
\pgfpathlineto{\pgfqpoint{1.239585in}{2.939237in}}%
\pgfpathlineto{\pgfqpoint{1.241644in}{2.890682in}}%
\pgfpathlineto{\pgfqpoint{1.249880in}{2.958659in}}%
\pgfpathlineto{\pgfqpoint{1.251939in}{3.046058in}}%
\pgfpathlineto{\pgfqpoint{1.256056in}{2.987792in}}%
\pgfpathlineto{\pgfqpoint{1.262233in}{2.987792in}}%
\pgfpathlineto{\pgfqpoint{1.266351in}{3.114035in}}%
\pgfpathlineto{\pgfqpoint{1.268410in}{3.114035in}}%
\pgfpathlineto{\pgfqpoint{1.270469in}{3.191723in}}%
\pgfpathlineto{\pgfqpoint{1.278705in}{3.259699in}}%
\pgfpathlineto{\pgfqpoint{1.280764in}{3.337387in}}%
\pgfpathlineto{\pgfqpoint{1.282823in}{3.249988in}}%
\pgfpathlineto{\pgfqpoint{1.284882in}{3.249988in}}%
\pgfpathlineto{\pgfqpoint{1.291059in}{3.376231in}}%
\pgfpathlineto{\pgfqpoint{1.293117in}{3.366520in}}%
\pgfpathlineto{\pgfqpoint{1.295176in}{3.405364in}}%
\pgfpathlineto{\pgfqpoint{1.297235in}{3.376231in}}%
\pgfpathlineto{\pgfqpoint{1.299294in}{3.279121in}}%
\pgfpathlineto{\pgfqpoint{1.307530in}{3.405364in}}%
\pgfpathlineto{\pgfqpoint{1.309589in}{3.405364in}}%
\pgfpathlineto{\pgfqpoint{1.311648in}{3.327676in}}%
\pgfpathlineto{\pgfqpoint{1.313707in}{3.347098in}}%
\pgfpathlineto{\pgfqpoint{1.321943in}{3.259699in}}%
\pgfpathlineto{\pgfqpoint{1.324002in}{3.249988in}}%
\pgfpathlineto{\pgfqpoint{1.326061in}{3.259699in}}%
\pgfpathlineto{\pgfqpoint{1.328120in}{3.249988in}}%
\pgfpathlineto{\pgfqpoint{1.334296in}{3.308254in}}%
\pgfpathlineto{\pgfqpoint{1.338414in}{3.483052in}}%
\pgfpathlineto{\pgfqpoint{1.340473in}{3.395653in}}%
\pgfpathlineto{\pgfqpoint{1.342532in}{3.483052in}}%
\pgfpathlineto{\pgfqpoint{1.348709in}{3.473341in}}%
\pgfpathlineto{\pgfqpoint{1.350768in}{3.502474in}}%
\pgfpathlineto{\pgfqpoint{1.352827in}{3.570451in}}%
\pgfpathlineto{\pgfqpoint{1.354886in}{3.453919in}}%
\pgfpathlineto{\pgfqpoint{1.356945in}{3.453919in}}%
\pgfpathlineto{\pgfqpoint{1.363122in}{3.444208in}}%
\pgfpathlineto{\pgfqpoint{1.365181in}{3.405364in}}%
\pgfpathlineto{\pgfqpoint{1.369298in}{3.473341in}}%
\pgfpathlineto{\pgfqpoint{1.371357in}{3.376231in}}%
\pgfpathlineto{\pgfqpoint{1.379593in}{3.531607in}}%
\pgfpathlineto{\pgfqpoint{1.381652in}{3.483052in}}%
\pgfpathlineto{\pgfqpoint{1.383711in}{3.492763in}}%
\pgfpathlineto{\pgfqpoint{1.385770in}{3.589873in}}%
\pgfpathlineto{\pgfqpoint{1.391947in}{3.434497in}}%
\pgfpathlineto{\pgfqpoint{1.394006in}{3.444208in}}%
\pgfpathlineto{\pgfqpoint{1.396065in}{3.531607in}}%
\pgfpathlineto{\pgfqpoint{1.398124in}{3.512185in}}%
\pgfpathlineto{\pgfqpoint{1.408418in}{3.356809in}}%
\pgfpathlineto{\pgfqpoint{1.410477in}{3.308254in}}%
\pgfpathlineto{\pgfqpoint{1.414595in}{3.521896in}}%
\pgfpathlineto{\pgfqpoint{1.420772in}{3.531607in}}%
\pgfpathlineto{\pgfqpoint{1.422831in}{3.512185in}}%
\pgfpathlineto{\pgfqpoint{1.424890in}{3.444208in}}%
\pgfpathlineto{\pgfqpoint{1.426949in}{3.424786in}}%
\pgfpathlineto{\pgfqpoint{1.429008in}{3.385942in}}%
\pgfpathlineto{\pgfqpoint{1.435185in}{3.405364in}}%
\pgfpathlineto{\pgfqpoint{1.437244in}{3.385942in}}%
\pgfpathlineto{\pgfqpoint{1.443421in}{3.288832in}}%
\pgfpathlineto{\pgfqpoint{1.449597in}{3.259699in}}%
\pgfpathlineto{\pgfqpoint{1.453715in}{3.308254in}}%
\pgfpathlineto{\pgfqpoint{1.455774in}{3.279121in}}%
\pgfpathlineto{\pgfqpoint{1.457833in}{3.230566in}}%
\pgfpathlineto{\pgfqpoint{1.464010in}{3.172301in}}%
\pgfpathlineto{\pgfqpoint{1.468128in}{3.259699in}}%
\pgfpathlineto{\pgfqpoint{1.472246in}{3.143168in}}%
\pgfpathlineto{\pgfqpoint{1.478423in}{3.201434in}}%
\pgfpathlineto{\pgfqpoint{1.480482in}{3.123746in}}%
\pgfpathlineto{\pgfqpoint{1.484599in}{3.162590in}}%
\pgfpathlineto{\pgfqpoint{1.486658in}{3.162590in}}%
\pgfpathlineto{\pgfqpoint{1.492835in}{3.133457in}}%
\pgfpathlineto{\pgfqpoint{1.494894in}{3.182012in}}%
\pgfpathlineto{\pgfqpoint{1.499012in}{3.075191in}}%
\pgfpathlineto{\pgfqpoint{1.501071in}{3.065480in}}%
\pgfpathlineto{\pgfqpoint{1.507248in}{3.046058in}}%
\pgfpathlineto{\pgfqpoint{1.511366in}{3.259699in}}%
\pgfpathlineto{\pgfqpoint{1.513425in}{3.230566in}}%
\pgfpathlineto{\pgfqpoint{1.515484in}{3.230566in}}%
\pgfpathlineto{\pgfqpoint{1.521660in}{3.259699in}}%
\pgfpathlineto{\pgfqpoint{1.523719in}{3.240277in}}%
\pgfpathlineto{\pgfqpoint{1.525778in}{3.279121in}}%
\pgfpathlineto{\pgfqpoint{1.527837in}{3.259699in}}%
\pgfpathlineto{\pgfqpoint{1.529896in}{3.201434in}}%
\pgfpathlineto{\pgfqpoint{1.538132in}{3.279121in}}%
\pgfpathlineto{\pgfqpoint{1.540191in}{3.259699in}}%
\pgfpathlineto{\pgfqpoint{1.542250in}{3.279121in}}%
\pgfpathlineto{\pgfqpoint{1.544309in}{3.249988in}}%
\pgfpathlineto{\pgfqpoint{1.550486in}{3.240277in}}%
\pgfpathlineto{\pgfqpoint{1.552545in}{3.347098in}}%
\pgfpathlineto{\pgfqpoint{1.554604in}{3.366520in}}%
\pgfpathlineto{\pgfqpoint{1.556663in}{3.308254in}}%
\pgfpathlineto{\pgfqpoint{1.558721in}{3.201434in}}%
\pgfpathlineto{\pgfqpoint{1.564898in}{3.288832in}}%
\pgfpathlineto{\pgfqpoint{1.566957in}{3.220855in}}%
\pgfpathlineto{\pgfqpoint{1.569016in}{3.220855in}}%
\pgfpathlineto{\pgfqpoint{1.571075in}{3.172301in}}%
\pgfpathlineto{\pgfqpoint{1.573134in}{3.220855in}}%
\pgfpathlineto{\pgfqpoint{1.581370in}{3.104324in}}%
\pgfpathlineto{\pgfqpoint{1.583429in}{3.133457in}}%
\pgfpathlineto{\pgfqpoint{1.585488in}{3.114035in}}%
\pgfpathlineto{\pgfqpoint{1.587547in}{3.065480in}}%
\pgfpathlineto{\pgfqpoint{1.593724in}{3.143168in}}%
\pgfpathlineto{\pgfqpoint{1.595782in}{3.123746in}}%
\pgfpathlineto{\pgfqpoint{1.597841in}{3.133457in}}%
\pgfpathlineto{\pgfqpoint{1.599900in}{3.182012in}}%
\pgfpathlineto{\pgfqpoint{1.601959in}{3.172301in}}%
\pgfpathlineto{\pgfqpoint{1.610195in}{3.114035in}}%
\pgfpathlineto{\pgfqpoint{1.612254in}{3.046058in}}%
\pgfpathlineto{\pgfqpoint{1.614313in}{3.084902in}}%
\pgfpathlineto{\pgfqpoint{1.616372in}{3.084902in}}%
\pgfpathlineto{\pgfqpoint{1.622549in}{3.104324in}}%
\pgfpathlineto{\pgfqpoint{1.624608in}{3.152879in}}%
\pgfpathlineto{\pgfqpoint{1.626667in}{3.104324in}}%
\pgfpathlineto{\pgfqpoint{1.628726in}{3.104324in}}%
\pgfpathlineto{\pgfqpoint{1.630785in}{3.162590in}}%
\pgfpathlineto{\pgfqpoint{1.639020in}{3.104324in}}%
\pgfpathlineto{\pgfqpoint{1.641079in}{3.123746in}}%
\pgfpathlineto{\pgfqpoint{1.643138in}{3.220855in}}%
\pgfpathlineto{\pgfqpoint{1.645197in}{3.191723in}}%
\pgfpathlineto{\pgfqpoint{1.651374in}{3.220855in}}%
\pgfpathlineto{\pgfqpoint{1.653433in}{3.269410in}}%
\pgfpathlineto{\pgfqpoint{1.655492in}{3.279121in}}%
\pgfpathlineto{\pgfqpoint{1.657551in}{3.298543in}}%
\pgfpathlineto{\pgfqpoint{1.659610in}{3.376231in}}%
\pgfpathlineto{\pgfqpoint{1.665787in}{3.405364in}}%
\pgfpathlineto{\pgfqpoint{1.667846in}{3.376231in}}%
\pgfpathlineto{\pgfqpoint{1.671963in}{3.366520in}}%
\pgfpathlineto{\pgfqpoint{1.674022in}{3.337387in}}%
\pgfpathlineto{\pgfqpoint{1.680199in}{3.337387in}}%
\pgfpathlineto{\pgfqpoint{1.682258in}{3.308254in}}%
\pgfpathlineto{\pgfqpoint{1.684317in}{3.317965in}}%
\pgfpathlineto{\pgfqpoint{1.686376in}{3.279121in}}%
\pgfpathlineto{\pgfqpoint{1.688435in}{3.298543in}}%
\pgfpathlineto{\pgfqpoint{1.696671in}{3.269410in}}%
\pgfpathlineto{\pgfqpoint{1.698730in}{3.259699in}}%
\pgfpathlineto{\pgfqpoint{1.702848in}{3.259699in}}%
\pgfpathlineto{\pgfqpoint{1.709024in}{3.249988in}}%
\pgfpathlineto{\pgfqpoint{1.711083in}{3.172301in}}%
\pgfpathlineto{\pgfqpoint{1.713142in}{3.172301in}}%
\pgfpathlineto{\pgfqpoint{1.715201in}{3.337387in}}%
\pgfpathlineto{\pgfqpoint{1.717260in}{3.269410in}}%
\pgfpathlineto{\pgfqpoint{1.723437in}{3.211145in}}%
\pgfpathlineto{\pgfqpoint{1.725496in}{3.230566in}}%
\pgfpathlineto{\pgfqpoint{1.727555in}{3.211145in}}%
\pgfpathlineto{\pgfqpoint{1.729614in}{3.220855in}}%
\pgfpathlineto{\pgfqpoint{1.731673in}{3.114035in}}%
\pgfpathlineto{\pgfqpoint{1.741968in}{3.269410in}}%
\pgfpathlineto{\pgfqpoint{1.744027in}{3.191723in}}%
\pgfpathlineto{\pgfqpoint{1.746086in}{3.162590in}}%
\pgfpathlineto{\pgfqpoint{1.752262in}{3.172301in}}%
\pgfpathlineto{\pgfqpoint{1.756380in}{3.259699in}}%
\pgfpathlineto{\pgfqpoint{1.758439in}{3.230566in}}%
\pgfpathlineto{\pgfqpoint{1.766675in}{3.211145in}}%
\pgfpathlineto{\pgfqpoint{1.768734in}{3.308254in}}%
\pgfpathlineto{\pgfqpoint{1.770793in}{3.308254in}}%
\pgfpathlineto{\pgfqpoint{1.772852in}{3.288832in}}%
\pgfpathlineto{\pgfqpoint{1.781088in}{3.259699in}}%
\pgfpathlineto{\pgfqpoint{1.783147in}{3.288832in}}%
\pgfpathlineto{\pgfqpoint{1.785205in}{3.211145in}}%
\pgfpathlineto{\pgfqpoint{1.787264in}{3.182012in}}%
\pgfpathlineto{\pgfqpoint{1.789323in}{3.172301in}}%
\pgfpathlineto{\pgfqpoint{1.795500in}{3.211145in}}%
\pgfpathlineto{\pgfqpoint{1.799618in}{3.094613in}}%
\pgfpathlineto{\pgfqpoint{1.801677in}{3.133457in}}%
\pgfpathlineto{\pgfqpoint{1.803736in}{3.065480in}}%
\pgfpathlineto{\pgfqpoint{1.811972in}{3.075191in}}%
\pgfpathlineto{\pgfqpoint{1.814031in}{3.036347in}}%
\pgfpathlineto{\pgfqpoint{1.816090in}{3.046058in}}%
\pgfpathlineto{\pgfqpoint{1.818149in}{3.084902in}}%
\pgfpathlineto{\pgfqpoint{1.826384in}{3.036347in}}%
\pgfpathlineto{\pgfqpoint{1.828443in}{3.046058in}}%
\pgfpathlineto{\pgfqpoint{1.830502in}{3.036347in}}%
\pgfpathlineto{\pgfqpoint{1.832561in}{2.987792in}}%
\pgfpathlineto{\pgfqpoint{1.838738in}{3.007214in}}%
\pgfpathlineto{\pgfqpoint{1.840797in}{2.900393in}}%
\pgfpathlineto{\pgfqpoint{1.842856in}{2.929526in}}%
\pgfpathlineto{\pgfqpoint{1.844915in}{2.919815in}}%
\pgfpathlineto{\pgfqpoint{1.846974in}{2.900393in}}%
\pgfpathlineto{\pgfqpoint{1.853151in}{2.803283in}}%
\pgfpathlineto{\pgfqpoint{1.855210in}{2.793572in}}%
\pgfpathlineto{\pgfqpoint{1.857269in}{2.764439in}}%
\pgfpathlineto{\pgfqpoint{1.859328in}{2.696462in}}%
\pgfpathlineto{\pgfqpoint{1.861386in}{2.783861in}}%
\pgfpathlineto{\pgfqpoint{1.869622in}{2.822705in}}%
\pgfpathlineto{\pgfqpoint{1.871681in}{2.871260in}}%
\pgfpathlineto{\pgfqpoint{1.873740in}{2.803283in}}%
\pgfpathlineto{\pgfqpoint{1.875799in}{2.803283in}}%
\pgfpathlineto{\pgfqpoint{1.881976in}{2.812994in}}%
\pgfpathlineto{\pgfqpoint{1.884035in}{2.793572in}}%
\pgfpathlineto{\pgfqpoint{1.886094in}{2.793572in}}%
\pgfpathlineto{\pgfqpoint{1.888153in}{2.774150in}}%
\pgfpathlineto{\pgfqpoint{1.890212in}{2.832416in}}%
\pgfpathlineto{\pgfqpoint{1.896389in}{2.822705in}}%
\pgfpathlineto{\pgfqpoint{1.898447in}{2.910104in}}%
\pgfpathlineto{\pgfqpoint{1.900506in}{2.900393in}}%
\pgfpathlineto{\pgfqpoint{1.902565in}{2.861549in}}%
\pgfpathlineto{\pgfqpoint{1.904624in}{2.919815in}}%
\pgfpathlineto{\pgfqpoint{1.910801in}{2.929526in}}%
\pgfpathlineto{\pgfqpoint{1.912860in}{2.851838in}}%
\pgfpathlineto{\pgfqpoint{1.914919in}{2.900393in}}%
\pgfpathlineto{\pgfqpoint{1.916978in}{2.919815in}}%
\pgfpathlineto{\pgfqpoint{1.919037in}{2.968370in}}%
\pgfpathlineto{\pgfqpoint{1.925214in}{2.958659in}}%
\pgfpathlineto{\pgfqpoint{1.927273in}{2.958659in}}%
\pgfpathlineto{\pgfqpoint{1.929332in}{2.948948in}}%
\pgfpathlineto{\pgfqpoint{1.933450in}{2.890682in}}%
\pgfpathlineto{\pgfqpoint{1.941685in}{2.948948in}}%
\pgfpathlineto{\pgfqpoint{1.943744in}{2.880971in}}%
\pgfpathlineto{\pgfqpoint{1.945803in}{2.910104in}}%
\pgfpathlineto{\pgfqpoint{1.954039in}{2.890682in}}%
\pgfpathlineto{\pgfqpoint{1.956098in}{2.832416in}}%
\pgfpathlineto{\pgfqpoint{1.958157in}{2.871260in}}%
\pgfpathlineto{\pgfqpoint{1.960216in}{2.832416in}}%
\pgfpathlineto{\pgfqpoint{1.962275in}{2.832416in}}%
\pgfpathlineto{\pgfqpoint{1.968452in}{2.822705in}}%
\pgfpathlineto{\pgfqpoint{1.970511in}{2.764439in}}%
\pgfpathlineto{\pgfqpoint{1.972570in}{2.803283in}}%
\pgfpathlineto{\pgfqpoint{1.974628in}{2.735306in}}%
\pgfpathlineto{\pgfqpoint{1.976687in}{2.764439in}}%
\pgfpathlineto{\pgfqpoint{1.982864in}{2.774150in}}%
\pgfpathlineto{\pgfqpoint{1.984923in}{2.812994in}}%
\pgfpathlineto{\pgfqpoint{1.986982in}{2.793572in}}%
\pgfpathlineto{\pgfqpoint{1.989041in}{2.812994in}}%
\pgfpathlineto{\pgfqpoint{1.991100in}{2.774150in}}%
\pgfpathlineto{\pgfqpoint{1.997277in}{2.803283in}}%
\pgfpathlineto{\pgfqpoint{1.999336in}{2.822705in}}%
\pgfpathlineto{\pgfqpoint{2.003454in}{2.919815in}}%
\pgfpathlineto{\pgfqpoint{2.005513in}{2.929526in}}%
\pgfpathlineto{\pgfqpoint{2.011689in}{2.948948in}}%
\pgfpathlineto{\pgfqpoint{2.013748in}{2.978081in}}%
\pgfpathlineto{\pgfqpoint{2.017866in}{2.900393in}}%
\pgfpathlineto{\pgfqpoint{2.019925in}{2.890682in}}%
\pgfpathlineto{\pgfqpoint{2.026102in}{2.939237in}}%
\pgfpathlineto{\pgfqpoint{2.028161in}{2.871260in}}%
\pgfpathlineto{\pgfqpoint{2.030220in}{2.851838in}}%
\pgfpathlineto{\pgfqpoint{2.032279in}{2.803283in}}%
\pgfpathlineto{\pgfqpoint{2.034338in}{2.832416in}}%
\pgfpathlineto{\pgfqpoint{2.040515in}{2.812994in}}%
\pgfpathlineto{\pgfqpoint{2.042574in}{2.812994in}}%
\pgfpathlineto{\pgfqpoint{2.044633in}{2.783861in}}%
\pgfpathlineto{\pgfqpoint{2.046692in}{2.812994in}}%
\pgfpathlineto{\pgfqpoint{2.048751in}{2.774150in}}%
\pgfpathlineto{\pgfqpoint{2.054927in}{2.812994in}}%
\pgfpathlineto{\pgfqpoint{2.056986in}{2.812994in}}%
\pgfpathlineto{\pgfqpoint{2.059045in}{2.900393in}}%
\pgfpathlineto{\pgfqpoint{2.061104in}{2.871260in}}%
\pgfpathlineto{\pgfqpoint{2.063163in}{2.871260in}}%
\pgfpathlineto{\pgfqpoint{2.069340in}{2.861549in}}%
\pgfpathlineto{\pgfqpoint{2.073458in}{2.900393in}}%
\pgfpathlineto{\pgfqpoint{2.075517in}{2.871260in}}%
\pgfpathlineto{\pgfqpoint{2.077576in}{2.880971in}}%
\pgfpathlineto{\pgfqpoint{2.085812in}{2.861549in}}%
\pgfpathlineto{\pgfqpoint{2.087870in}{2.851838in}}%
\pgfpathlineto{\pgfqpoint{2.089929in}{2.803283in}}%
\pgfpathlineto{\pgfqpoint{2.091988in}{2.725595in}}%
\pgfpathlineto{\pgfqpoint{2.098165in}{2.754728in}}%
\pgfpathlineto{\pgfqpoint{2.104342in}{2.686751in}}%
\pgfpathlineto{\pgfqpoint{2.106401in}{2.657618in}}%
\pgfpathlineto{\pgfqpoint{2.112578in}{2.647908in}}%
\pgfpathlineto{\pgfqpoint{2.116696in}{2.628486in}}%
\pgfpathlineto{\pgfqpoint{2.118755in}{2.599353in}}%
\pgfpathlineto{\pgfqpoint{2.120814in}{2.628486in}}%
\pgfpathlineto{\pgfqpoint{2.126990in}{2.667329in}}%
\pgfpathlineto{\pgfqpoint{2.129049in}{2.706173in}}%
\pgfpathlineto{\pgfqpoint{2.131108in}{2.696462in}}%
\pgfpathlineto{\pgfqpoint{2.133167in}{2.754728in}}%
\pgfpathlineto{\pgfqpoint{2.135226in}{2.599353in}}%
\pgfpathlineto{\pgfqpoint{2.141403in}{2.473110in}}%
\pgfpathlineto{\pgfqpoint{2.143462in}{2.473110in}}%
\pgfpathlineto{\pgfqpoint{2.145521in}{2.502243in}}%
\pgfpathlineto{\pgfqpoint{2.147580in}{2.502243in}}%
\pgfpathlineto{\pgfqpoint{2.149639in}{2.453688in}}%
\pgfpathlineto{\pgfqpoint{2.157875in}{2.366289in}}%
\pgfpathlineto{\pgfqpoint{2.161993in}{2.366289in}}%
\pgfpathlineto{\pgfqpoint{2.164051in}{2.337156in}}%
\pgfpathlineto{\pgfqpoint{2.170228in}{2.376000in}}%
\pgfpathlineto{\pgfqpoint{2.172287in}{2.463399in}}%
\pgfpathlineto{\pgfqpoint{2.174346in}{2.414844in}}%
\pgfpathlineto{\pgfqpoint{2.178464in}{2.541087in}}%
\pgfpathlineto{\pgfqpoint{2.184641in}{2.541087in}}%
\pgfpathlineto{\pgfqpoint{2.186700in}{2.521665in}}%
\pgfpathlineto{\pgfqpoint{2.188759in}{2.550798in}}%
\pgfpathlineto{\pgfqpoint{2.190818in}{2.541087in}}%
\pgfpathlineto{\pgfqpoint{2.199054in}{2.541087in}}%
\pgfpathlineto{\pgfqpoint{2.201112in}{2.531376in}}%
\pgfpathlineto{\pgfqpoint{2.203171in}{2.482821in}}%
\pgfpathlineto{\pgfqpoint{2.205230in}{2.473110in}}%
\pgfpathlineto{\pgfqpoint{2.207289in}{2.424555in}}%
\pgfpathlineto{\pgfqpoint{2.213466in}{2.482821in}}%
\pgfpathlineto{\pgfqpoint{2.215525in}{2.521665in}}%
\pgfpathlineto{\pgfqpoint{2.217584in}{2.521665in}}%
\pgfpathlineto{\pgfqpoint{2.219643in}{2.482821in}}%
\pgfpathlineto{\pgfqpoint{2.221702in}{2.550798in}}%
\pgfpathlineto{\pgfqpoint{2.227879in}{2.550798in}}%
\pgfpathlineto{\pgfqpoint{2.231997in}{2.473110in}}%
\pgfpathlineto{\pgfqpoint{2.234056in}{2.531376in}}%
\pgfpathlineto{\pgfqpoint{2.236115in}{2.492532in}}%
\pgfpathlineto{\pgfqpoint{2.244350in}{2.560509in}}%
\pgfpathlineto{\pgfqpoint{2.248468in}{2.531376in}}%
\pgfpathlineto{\pgfqpoint{2.250527in}{2.570220in}}%
\pgfpathlineto{\pgfqpoint{2.256704in}{2.521665in}}%
\pgfpathlineto{\pgfqpoint{2.258763in}{2.521665in}}%
\pgfpathlineto{\pgfqpoint{2.260822in}{2.531376in}}%
\pgfpathlineto{\pgfqpoint{2.262881in}{2.550798in}}%
\pgfpathlineto{\pgfqpoint{2.264940in}{2.599353in}}%
\pgfpathlineto{\pgfqpoint{2.271117in}{2.541087in}}%
\pgfpathlineto{\pgfqpoint{2.273176in}{2.550798in}}%
\pgfpathlineto{\pgfqpoint{2.275235in}{2.541087in}}%
\pgfpathlineto{\pgfqpoint{2.277293in}{2.541087in}}%
\pgfpathlineto{\pgfqpoint{2.279352in}{2.589642in}}%
\pgfpathlineto{\pgfqpoint{2.289647in}{2.531376in}}%
\pgfpathlineto{\pgfqpoint{2.293765in}{2.686751in}}%
\pgfpathlineto{\pgfqpoint{2.299942in}{2.686751in}}%
\pgfpathlineto{\pgfqpoint{2.302001in}{2.754728in}}%
\pgfpathlineto{\pgfqpoint{2.304060in}{2.735306in}}%
\pgfpathlineto{\pgfqpoint{2.306119in}{2.764439in}}%
\pgfpathlineto{\pgfqpoint{2.308178in}{2.735306in}}%
\pgfpathlineto{\pgfqpoint{2.314355in}{2.735306in}}%
\pgfpathlineto{\pgfqpoint{2.316413in}{2.725595in}}%
\pgfpathlineto{\pgfqpoint{2.320531in}{2.657618in}}%
\pgfpathlineto{\pgfqpoint{2.322590in}{2.657618in}}%
\pgfpathlineto{\pgfqpoint{2.328767in}{2.638197in}}%
\pgfpathlineto{\pgfqpoint{2.330826in}{2.599353in}}%
\pgfpathlineto{\pgfqpoint{2.332885in}{2.599353in}}%
\pgfpathlineto{\pgfqpoint{2.334944in}{2.589642in}}%
\pgfpathlineto{\pgfqpoint{2.337003in}{2.628486in}}%
\pgfpathlineto{\pgfqpoint{2.343180in}{2.647908in}}%
\pgfpathlineto{\pgfqpoint{2.345239in}{2.715884in}}%
\pgfpathlineto{\pgfqpoint{2.349357in}{2.774150in}}%
\pgfpathlineto{\pgfqpoint{2.351416in}{2.774150in}}%
\pgfpathlineto{\pgfqpoint{2.359651in}{2.803283in}}%
\pgfpathlineto{\pgfqpoint{2.361710in}{2.822705in}}%
\pgfpathlineto{\pgfqpoint{2.363769in}{2.783861in}}%
\pgfpathlineto{\pgfqpoint{2.365828in}{2.851838in}}%
\pgfpathlineto{\pgfqpoint{2.374064in}{2.812994in}}%
\pgfpathlineto{\pgfqpoint{2.376123in}{2.812994in}}%
\pgfpathlineto{\pgfqpoint{2.378182in}{2.803283in}}%
\pgfpathlineto{\pgfqpoint{2.380241in}{2.783861in}}%
\pgfpathlineto{\pgfqpoint{2.386418in}{2.812994in}}%
\pgfpathlineto{\pgfqpoint{2.388477in}{2.803283in}}%
\pgfpathlineto{\pgfqpoint{2.390535in}{2.832416in}}%
\pgfpathlineto{\pgfqpoint{2.392594in}{2.890682in}}%
\pgfpathlineto{\pgfqpoint{2.394653in}{2.900393in}}%
\pgfpathlineto{\pgfqpoint{2.402889in}{2.871260in}}%
\pgfpathlineto{\pgfqpoint{2.404948in}{2.851838in}}%
\pgfpathlineto{\pgfqpoint{2.407007in}{2.880971in}}%
\pgfpathlineto{\pgfqpoint{2.409066in}{2.851838in}}%
\pgfpathlineto{\pgfqpoint{2.415243in}{2.890682in}}%
\pgfpathlineto{\pgfqpoint{2.417302in}{2.919815in}}%
\pgfpathlineto{\pgfqpoint{2.419361in}{3.143168in}}%
\pgfpathlineto{\pgfqpoint{2.421420in}{3.201434in}}%
\pgfpathlineto{\pgfqpoint{2.429655in}{3.269410in}}%
\pgfpathlineto{\pgfqpoint{2.431714in}{3.259699in}}%
\pgfpathlineto{\pgfqpoint{2.433773in}{3.230566in}}%
\pgfpathlineto{\pgfqpoint{2.435832in}{3.308254in}}%
\pgfpathlineto{\pgfqpoint{2.437891in}{3.317965in}}%
\pgfpathlineto{\pgfqpoint{2.444068in}{3.308254in}}%
\pgfpathlineto{\pgfqpoint{2.446127in}{3.308254in}}%
\pgfpathlineto{\pgfqpoint{2.448186in}{3.327676in}}%
\pgfpathlineto{\pgfqpoint{2.452304in}{3.327676in}}%
\pgfpathlineto{\pgfqpoint{2.458481in}{3.298543in}}%
\pgfpathlineto{\pgfqpoint{2.460540in}{3.279121in}}%
\pgfpathlineto{\pgfqpoint{2.464658in}{3.434497in}}%
\pgfpathlineto{\pgfqpoint{2.466716in}{3.395653in}}%
\pgfpathlineto{\pgfqpoint{2.472893in}{3.376231in}}%
\pgfpathlineto{\pgfqpoint{2.474952in}{3.385942in}}%
\pgfpathlineto{\pgfqpoint{2.477011in}{3.347098in}}%
\pgfpathlineto{\pgfqpoint{2.481129in}{3.483052in}}%
\pgfpathlineto{\pgfqpoint{2.487306in}{3.473341in}}%
\pgfpathlineto{\pgfqpoint{2.489365in}{3.463630in}}%
\pgfpathlineto{\pgfqpoint{2.491424in}{3.473341in}}%
\pgfpathlineto{\pgfqpoint{2.495542in}{3.521896in}}%
\pgfpathlineto{\pgfqpoint{2.501719in}{3.463630in}}%
\pgfpathlineto{\pgfqpoint{2.503778in}{3.492763in}}%
\pgfpathlineto{\pgfqpoint{2.505836in}{3.473341in}}%
\pgfpathlineto{\pgfqpoint{2.509954in}{3.473341in}}%
\pgfpathlineto{\pgfqpoint{2.518190in}{3.492763in}}%
\pgfpathlineto{\pgfqpoint{2.520249in}{3.444208in}}%
\pgfpathlineto{\pgfqpoint{2.522308in}{3.434497in}}%
\pgfpathlineto{\pgfqpoint{2.524367in}{3.405364in}}%
\pgfpathlineto{\pgfqpoint{2.532603in}{3.395653in}}%
\pgfpathlineto{\pgfqpoint{2.534662in}{3.395653in}}%
\pgfpathlineto{\pgfqpoint{2.536721in}{3.308254in}}%
\pgfpathlineto{\pgfqpoint{2.538780in}{3.347098in}}%
\pgfpathlineto{\pgfqpoint{2.544956in}{3.308254in}}%
\pgfpathlineto{\pgfqpoint{2.547015in}{3.308254in}}%
\pgfpathlineto{\pgfqpoint{2.549074in}{3.298543in}}%
\pgfpathlineto{\pgfqpoint{2.551133in}{3.298543in}}%
\pgfpathlineto{\pgfqpoint{2.553192in}{3.327676in}}%
\pgfpathlineto{\pgfqpoint{2.561428in}{3.279121in}}%
\pgfpathlineto{\pgfqpoint{2.563487in}{3.356809in}}%
\pgfpathlineto{\pgfqpoint{2.567605in}{3.405364in}}%
\pgfpathlineto{\pgfqpoint{2.573782in}{3.337387in}}%
\pgfpathlineto{\pgfqpoint{2.577900in}{3.453919in}}%
\pgfpathlineto{\pgfqpoint{2.582017in}{3.415075in}}%
\pgfpathlineto{\pgfqpoint{2.588194in}{3.434497in}}%
\pgfpathlineto{\pgfqpoint{2.590253in}{3.395653in}}%
\pgfpathlineto{\pgfqpoint{2.592312in}{3.415075in}}%
\pgfpathlineto{\pgfqpoint{2.594371in}{3.415075in}}%
\pgfpathlineto{\pgfqpoint{2.596430in}{3.434497in}}%
\pgfpathlineto{\pgfqpoint{2.604666in}{3.356809in}}%
\pgfpathlineto{\pgfqpoint{2.606725in}{3.298543in}}%
\pgfpathlineto{\pgfqpoint{2.608784in}{3.356809in}}%
\pgfpathlineto{\pgfqpoint{2.610843in}{3.366520in}}%
\pgfpathlineto{\pgfqpoint{2.617020in}{3.385942in}}%
\pgfpathlineto{\pgfqpoint{2.621137in}{3.453919in}}%
\pgfpathlineto{\pgfqpoint{2.625255in}{3.395653in}}%
\pgfpathlineto{\pgfqpoint{2.635550in}{3.395653in}}%
\pgfpathlineto{\pgfqpoint{2.637609in}{3.366520in}}%
\pgfpathlineto{\pgfqpoint{2.639668in}{3.308254in}}%
\pgfpathlineto{\pgfqpoint{2.645845in}{3.337387in}}%
\pgfpathlineto{\pgfqpoint{2.647904in}{3.317965in}}%
\pgfpathlineto{\pgfqpoint{2.649963in}{3.424786in}}%
\pgfpathlineto{\pgfqpoint{2.652022in}{3.453919in}}%
\pgfpathlineto{\pgfqpoint{2.654081in}{3.444208in}}%
\pgfpathlineto{\pgfqpoint{2.660257in}{3.453919in}}%
\pgfpathlineto{\pgfqpoint{2.662316in}{3.463630in}}%
\pgfpathlineto{\pgfqpoint{2.666434in}{3.551029in}}%
\pgfpathlineto{\pgfqpoint{2.668493in}{3.551029in}}%
\pgfpathlineto{\pgfqpoint{2.674670in}{3.580162in}}%
\pgfpathlineto{\pgfqpoint{2.676729in}{3.551029in}}%
\pgfpathlineto{\pgfqpoint{2.678788in}{3.483052in}}%
\pgfpathlineto{\pgfqpoint{2.680847in}{3.502474in}}%
\pgfpathlineto{\pgfqpoint{2.682906in}{3.473341in}}%
\pgfpathlineto{\pgfqpoint{2.689083in}{3.444208in}}%
\pgfpathlineto{\pgfqpoint{2.693200in}{3.376231in}}%
\pgfpathlineto{\pgfqpoint{2.695259in}{3.376231in}}%
\pgfpathlineto{\pgfqpoint{2.697318in}{3.356809in}}%
\pgfpathlineto{\pgfqpoint{2.703495in}{3.347098in}}%
\pgfpathlineto{\pgfqpoint{2.705554in}{3.385942in}}%
\pgfpathlineto{\pgfqpoint{2.707613in}{3.356809in}}%
\pgfpathlineto{\pgfqpoint{2.709672in}{3.395653in}}%
\pgfpathlineto{\pgfqpoint{2.717908in}{3.327676in}}%
\pgfpathlineto{\pgfqpoint{2.719967in}{3.337387in}}%
\pgfpathlineto{\pgfqpoint{2.722026in}{3.327676in}}%
\pgfpathlineto{\pgfqpoint{2.724085in}{3.337387in}}%
\pgfpathlineto{\pgfqpoint{2.726144in}{3.356809in}}%
\pgfpathlineto{\pgfqpoint{2.732320in}{3.337387in}}%
\pgfpathlineto{\pgfqpoint{2.734379in}{3.288832in}}%
\pgfpathlineto{\pgfqpoint{2.738497in}{3.240277in}}%
\pgfpathlineto{\pgfqpoint{2.746733in}{3.269410in}}%
\pgfpathlineto{\pgfqpoint{2.748792in}{3.182012in}}%
\pgfpathlineto{\pgfqpoint{2.752910in}{3.230566in}}%
\pgfpathlineto{\pgfqpoint{2.754969in}{3.230566in}}%
\pgfpathlineto{\pgfqpoint{2.761146in}{3.269410in}}%
\pgfpathlineto{\pgfqpoint{2.763205in}{3.327676in}}%
\pgfpathlineto{\pgfqpoint{2.767323in}{3.298543in}}%
\pgfpathlineto{\pgfqpoint{2.769381in}{3.288832in}}%
\pgfpathlineto{\pgfqpoint{2.775558in}{3.327676in}}%
\pgfpathlineto{\pgfqpoint{2.777617in}{3.298543in}}%
\pgfpathlineto{\pgfqpoint{2.783794in}{3.347098in}}%
\pgfpathlineto{\pgfqpoint{2.789971in}{3.376231in}}%
\pgfpathlineto{\pgfqpoint{2.792030in}{3.405364in}}%
\pgfpathlineto{\pgfqpoint{2.794089in}{3.405364in}}%
\pgfpathlineto{\pgfqpoint{2.796148in}{3.395653in}}%
\pgfpathlineto{\pgfqpoint{2.798207in}{3.356809in}}%
\pgfpathlineto{\pgfqpoint{2.804384in}{3.376231in}}%
\pgfpathlineto{\pgfqpoint{2.806443in}{3.356809in}}%
\pgfpathlineto{\pgfqpoint{2.808501in}{3.269410in}}%
\pgfpathlineto{\pgfqpoint{2.812619in}{3.249988in}}%
\pgfpathlineto{\pgfqpoint{2.818796in}{3.259699in}}%
\pgfpathlineto{\pgfqpoint{2.820855in}{3.298543in}}%
\pgfpathlineto{\pgfqpoint{2.822914in}{3.269410in}}%
\pgfpathlineto{\pgfqpoint{2.827032in}{3.269410in}}%
\pgfpathlineto{\pgfqpoint{2.837327in}{3.220855in}}%
\pgfpathlineto{\pgfqpoint{2.839386in}{3.220855in}}%
\pgfpathlineto{\pgfqpoint{2.841445in}{3.152879in}}%
\pgfpathlineto{\pgfqpoint{2.847621in}{3.182012in}}%
\pgfpathlineto{\pgfqpoint{2.849680in}{3.152879in}}%
\pgfpathlineto{\pgfqpoint{2.851739in}{3.182012in}}%
\pgfpathlineto{\pgfqpoint{2.853798in}{3.191723in}}%
\pgfpathlineto{\pgfqpoint{2.855857in}{3.211145in}}%
\pgfpathlineto{\pgfqpoint{2.862034in}{3.211145in}}%
\pgfpathlineto{\pgfqpoint{2.864093in}{3.220855in}}%
\pgfpathlineto{\pgfqpoint{2.866152in}{3.152879in}}%
\pgfpathlineto{\pgfqpoint{2.868211in}{3.143168in}}%
\pgfpathlineto{\pgfqpoint{2.870270in}{3.143168in}}%
\pgfpathlineto{\pgfqpoint{2.876447in}{3.152879in}}%
\pgfpathlineto{\pgfqpoint{2.878506in}{3.114035in}}%
\pgfpathlineto{\pgfqpoint{2.882623in}{3.094613in}}%
\pgfpathlineto{\pgfqpoint{2.884682in}{3.104324in}}%
\pgfpathlineto{\pgfqpoint{2.890859in}{3.084902in}}%
\pgfpathlineto{\pgfqpoint{2.894977in}{3.172301in}}%
\pgfpathlineto{\pgfqpoint{2.899095in}{3.230566in}}%
\pgfpathlineto{\pgfqpoint{2.905272in}{3.269410in}}%
\pgfpathlineto{\pgfqpoint{2.909390in}{3.249988in}}%
\pgfpathlineto{\pgfqpoint{2.913508in}{3.327676in}}%
\pgfpathlineto{\pgfqpoint{2.919685in}{3.317965in}}%
\pgfpathlineto{\pgfqpoint{2.921743in}{3.308254in}}%
\pgfpathlineto{\pgfqpoint{2.923802in}{3.269410in}}%
\pgfpathlineto{\pgfqpoint{2.925861in}{3.308254in}}%
\pgfpathlineto{\pgfqpoint{2.927920in}{3.288832in}}%
\pgfpathlineto{\pgfqpoint{2.934097in}{3.269410in}}%
\pgfpathlineto{\pgfqpoint{2.936156in}{3.230566in}}%
\pgfpathlineto{\pgfqpoint{2.938215in}{3.230566in}}%
\pgfpathlineto{\pgfqpoint{2.940274in}{3.220855in}}%
\pgfpathlineto{\pgfqpoint{2.942333in}{3.191723in}}%
\pgfpathlineto{\pgfqpoint{2.948510in}{3.211145in}}%
\pgfpathlineto{\pgfqpoint{2.950569in}{3.288832in}}%
\pgfpathlineto{\pgfqpoint{2.952628in}{3.269410in}}%
\pgfpathlineto{\pgfqpoint{2.954687in}{3.298543in}}%
\pgfpathlineto{\pgfqpoint{2.956746in}{3.269410in}}%
\pgfpathlineto{\pgfqpoint{2.962922in}{3.279121in}}%
\pgfpathlineto{\pgfqpoint{2.964981in}{3.230566in}}%
\pgfpathlineto{\pgfqpoint{2.967040in}{3.220855in}}%
\pgfpathlineto{\pgfqpoint{2.969099in}{3.182012in}}%
\pgfpathlineto{\pgfqpoint{2.971158in}{3.230566in}}%
\pgfpathlineto{\pgfqpoint{2.977335in}{3.220855in}}%
\pgfpathlineto{\pgfqpoint{2.979394in}{3.249988in}}%
\pgfpathlineto{\pgfqpoint{2.983512in}{3.172301in}}%
\pgfpathlineto{\pgfqpoint{2.985571in}{3.172301in}}%
\pgfpathlineto{\pgfqpoint{2.991748in}{3.191723in}}%
\pgfpathlineto{\pgfqpoint{2.993807in}{3.220855in}}%
\pgfpathlineto{\pgfqpoint{2.995866in}{3.201434in}}%
\pgfpathlineto{\pgfqpoint{2.997924in}{3.162590in}}%
\pgfpathlineto{\pgfqpoint{2.999983in}{3.162590in}}%
\pgfpathlineto{\pgfqpoint{3.006160in}{3.143168in}}%
\pgfpathlineto{\pgfqpoint{3.008219in}{3.172301in}}%
\pgfpathlineto{\pgfqpoint{3.010278in}{3.133457in}}%
\pgfpathlineto{\pgfqpoint{3.012337in}{3.152879in}}%
\pgfpathlineto{\pgfqpoint{3.014396in}{3.133457in}}%
\pgfpathlineto{\pgfqpoint{3.020573in}{3.133457in}}%
\pgfpathlineto{\pgfqpoint{3.022632in}{3.104324in}}%
\pgfpathlineto{\pgfqpoint{3.024691in}{3.114035in}}%
\pgfpathlineto{\pgfqpoint{3.026750in}{3.094613in}}%
\pgfpathlineto{\pgfqpoint{3.028809in}{3.133457in}}%
\pgfpathlineto{\pgfqpoint{3.037044in}{3.055769in}}%
\pgfpathlineto{\pgfqpoint{3.039103in}{3.084902in}}%
\pgfpathlineto{\pgfqpoint{3.041162in}{3.026636in}}%
\pgfpathlineto{\pgfqpoint{3.043221in}{3.036347in}}%
\pgfpathlineto{\pgfqpoint{3.053516in}{3.152879in}}%
\pgfpathlineto{\pgfqpoint{3.055575in}{3.143168in}}%
\pgfpathlineto{\pgfqpoint{3.057634in}{3.143168in}}%
\pgfpathlineto{\pgfqpoint{3.067929in}{3.211145in}}%
\pgfpathlineto{\pgfqpoint{3.069988in}{3.191723in}}%
\pgfpathlineto{\pgfqpoint{3.072046in}{3.191723in}}%
\pgfpathlineto{\pgfqpoint{3.078223in}{3.152879in}}%
\pgfpathlineto{\pgfqpoint{3.080282in}{3.162590in}}%
\pgfpathlineto{\pgfqpoint{3.082341in}{3.240277in}}%
\pgfpathlineto{\pgfqpoint{3.084400in}{3.249988in}}%
\pgfpathlineto{\pgfqpoint{3.086459in}{3.249988in}}%
\pgfpathlineto{\pgfqpoint{3.092636in}{3.259699in}}%
\pgfpathlineto{\pgfqpoint{3.094695in}{3.249988in}}%
\pgfpathlineto{\pgfqpoint{3.098813in}{3.269410in}}%
\pgfpathlineto{\pgfqpoint{3.100872in}{3.298543in}}%
\pgfpathlineto{\pgfqpoint{3.111166in}{3.259699in}}%
\pgfpathlineto{\pgfqpoint{3.113225in}{3.240277in}}%
\pgfpathlineto{\pgfqpoint{3.115284in}{3.201434in}}%
\pgfpathlineto{\pgfqpoint{3.123520in}{3.201434in}}%
\pgfpathlineto{\pgfqpoint{3.125579in}{3.240277in}}%
\pgfpathlineto{\pgfqpoint{3.127638in}{3.220855in}}%
\pgfpathlineto{\pgfqpoint{3.129697in}{3.288832in}}%
\pgfpathlineto{\pgfqpoint{3.135874in}{3.279121in}}%
\pgfpathlineto{\pgfqpoint{3.139992in}{3.337387in}}%
\pgfpathlineto{\pgfqpoint{3.142051in}{3.356809in}}%
\pgfpathlineto{\pgfqpoint{3.144110in}{3.327676in}}%
\pgfpathlineto{\pgfqpoint{3.150286in}{3.279121in}}%
\pgfpathlineto{\pgfqpoint{3.152345in}{3.279121in}}%
\pgfpathlineto{\pgfqpoint{3.156463in}{3.230566in}}%
\pgfpathlineto{\pgfqpoint{3.158522in}{3.211145in}}%
\pgfpathlineto{\pgfqpoint{3.164699in}{3.201434in}}%
\pgfpathlineto{\pgfqpoint{3.166758in}{3.182012in}}%
\pgfpathlineto{\pgfqpoint{3.168817in}{3.191723in}}%
\pgfpathlineto{\pgfqpoint{3.170876in}{3.211145in}}%
\pgfpathlineto{\pgfqpoint{3.172935in}{3.288832in}}%
\pgfpathlineto{\pgfqpoint{3.179112in}{3.288832in}}%
\pgfpathlineto{\pgfqpoint{3.181171in}{3.259699in}}%
\pgfpathlineto{\pgfqpoint{3.183230in}{3.201434in}}%
\pgfpathlineto{\pgfqpoint{3.185289in}{3.240277in}}%
\pgfpathlineto{\pgfqpoint{3.187347in}{3.211145in}}%
\pgfpathlineto{\pgfqpoint{3.193524in}{3.220855in}}%
\pgfpathlineto{\pgfqpoint{3.197642in}{3.191723in}}%
\pgfpathlineto{\pgfqpoint{3.201760in}{3.201434in}}%
\pgfpathlineto{\pgfqpoint{3.207937in}{3.191723in}}%
\pgfpathlineto{\pgfqpoint{3.209996in}{3.201434in}}%
\pgfpathlineto{\pgfqpoint{3.214114in}{3.269410in}}%
\pgfpathlineto{\pgfqpoint{3.216173in}{3.201434in}}%
\pgfpathlineto{\pgfqpoint{3.222350in}{3.201434in}}%
\pgfpathlineto{\pgfqpoint{3.226467in}{3.152879in}}%
\pgfpathlineto{\pgfqpoint{3.228526in}{3.201434in}}%
\pgfpathlineto{\pgfqpoint{3.230585in}{3.211145in}}%
\pgfpathlineto{\pgfqpoint{3.236762in}{3.211145in}}%
\pgfpathlineto{\pgfqpoint{3.238821in}{3.220855in}}%
\pgfpathlineto{\pgfqpoint{3.242939in}{3.152879in}}%
\pgfpathlineto{\pgfqpoint{3.244998in}{3.143168in}}%
\pgfpathlineto{\pgfqpoint{3.251175in}{3.191723in}}%
\pgfpathlineto{\pgfqpoint{3.255293in}{3.327676in}}%
\pgfpathlineto{\pgfqpoint{3.257352in}{3.298543in}}%
\pgfpathlineto{\pgfqpoint{3.259411in}{3.298543in}}%
\pgfpathlineto{\pgfqpoint{3.267646in}{3.279121in}}%
\pgfpathlineto{\pgfqpoint{3.269705in}{3.211145in}}%
\pgfpathlineto{\pgfqpoint{3.271764in}{3.220855in}}%
\pgfpathlineto{\pgfqpoint{3.273823in}{3.201434in}}%
\pgfpathlineto{\pgfqpoint{3.282059in}{3.259699in}}%
\pgfpathlineto{\pgfqpoint{3.284118in}{3.240277in}}%
\pgfpathlineto{\pgfqpoint{3.286177in}{3.240277in}}%
\pgfpathlineto{\pgfqpoint{3.288236in}{3.259699in}}%
\pgfpathlineto{\pgfqpoint{3.294413in}{3.269410in}}%
\pgfpathlineto{\pgfqpoint{3.296472in}{3.337387in}}%
\pgfpathlineto{\pgfqpoint{3.298531in}{3.347098in}}%
\pgfpathlineto{\pgfqpoint{3.302648in}{3.327676in}}%
\pgfpathlineto{\pgfqpoint{3.310884in}{3.308254in}}%
\pgfpathlineto{\pgfqpoint{3.312943in}{3.327676in}}%
\pgfpathlineto{\pgfqpoint{3.315002in}{3.385942in}}%
\pgfpathlineto{\pgfqpoint{3.317061in}{3.395653in}}%
\pgfpathlineto{\pgfqpoint{3.323238in}{3.405364in}}%
\pgfpathlineto{\pgfqpoint{3.325297in}{3.385942in}}%
\pgfpathlineto{\pgfqpoint{3.327356in}{3.415075in}}%
\pgfpathlineto{\pgfqpoint{3.329415in}{3.376231in}}%
\pgfpathlineto{\pgfqpoint{3.331474in}{3.405364in}}%
\pgfpathlineto{\pgfqpoint{3.337650in}{3.434497in}}%
\pgfpathlineto{\pgfqpoint{3.339709in}{3.473341in}}%
\pgfpathlineto{\pgfqpoint{3.341768in}{3.444208in}}%
\pgfpathlineto{\pgfqpoint{3.345886in}{3.580162in}}%
\pgfpathlineto{\pgfqpoint{3.352063in}{3.531607in}}%
\pgfpathlineto{\pgfqpoint{3.354122in}{3.551029in}}%
\pgfpathlineto{\pgfqpoint{3.356181in}{3.619006in}}%
\pgfpathlineto{\pgfqpoint{3.358240in}{3.638428in}}%
\pgfpathlineto{\pgfqpoint{3.360299in}{3.628717in}}%
\pgfpathlineto{\pgfqpoint{3.366476in}{3.628717in}}%
\pgfpathlineto{\pgfqpoint{3.368535in}{3.599584in}}%
\pgfpathlineto{\pgfqpoint{3.370594in}{3.677272in}}%
\pgfpathlineto{\pgfqpoint{3.374711in}{3.628717in}}%
\pgfpathlineto{\pgfqpoint{3.382947in}{3.648139in}}%
\pgfpathlineto{\pgfqpoint{3.385006in}{3.716116in}}%
\pgfpathlineto{\pgfqpoint{3.387065in}{3.696694in}}%
\pgfpathlineto{\pgfqpoint{3.389124in}{3.648139in}}%
\pgfpathlineto{\pgfqpoint{3.395301in}{3.638428in}}%
\pgfpathlineto{\pgfqpoint{3.397360in}{3.667561in}}%
\pgfpathlineto{\pgfqpoint{3.401478in}{3.580162in}}%
\pgfpathlineto{\pgfqpoint{3.403537in}{3.628717in}}%
\pgfpathlineto{\pgfqpoint{3.409714in}{3.648139in}}%
\pgfpathlineto{\pgfqpoint{3.411773in}{3.638428in}}%
\pgfpathlineto{\pgfqpoint{3.413831in}{3.648139in}}%
\pgfpathlineto{\pgfqpoint{3.415890in}{3.619006in}}%
\pgfpathlineto{\pgfqpoint{3.417949in}{3.648139in}}%
\pgfpathlineto{\pgfqpoint{3.426185in}{3.589873in}}%
\pgfpathlineto{\pgfqpoint{3.428244in}{3.551029in}}%
\pgfpathlineto{\pgfqpoint{3.430303in}{3.551029in}}%
\pgfpathlineto{\pgfqpoint{3.432362in}{3.570451in}}%
\pgfpathlineto{\pgfqpoint{3.438539in}{3.580162in}}%
\pgfpathlineto{\pgfqpoint{3.440598in}{3.619006in}}%
\pgfpathlineto{\pgfqpoint{3.442657in}{3.619006in}}%
\pgfpathlineto{\pgfqpoint{3.444716in}{3.551029in}}%
\pgfpathlineto{\pgfqpoint{3.446775in}{3.551029in}}%
\pgfpathlineto{\pgfqpoint{3.452951in}{3.570451in}}%
\pgfpathlineto{\pgfqpoint{3.455010in}{3.512185in}}%
\pgfpathlineto{\pgfqpoint{3.457069in}{3.502474in}}%
\pgfpathlineto{\pgfqpoint{3.459128in}{3.463630in}}%
\pgfpathlineto{\pgfqpoint{3.467364in}{3.463630in}}%
\pgfpathlineto{\pgfqpoint{3.469423in}{3.512185in}}%
\pgfpathlineto{\pgfqpoint{3.471482in}{3.521896in}}%
\pgfpathlineto{\pgfqpoint{3.473541in}{3.560740in}}%
\pgfpathlineto{\pgfqpoint{3.475600in}{3.502474in}}%
\pgfpathlineto{\pgfqpoint{3.483836in}{3.502474in}}%
\pgfpathlineto{\pgfqpoint{3.485895in}{3.483052in}}%
\pgfpathlineto{\pgfqpoint{3.487954in}{3.531607in}}%
\pgfpathlineto{\pgfqpoint{3.490012in}{3.521896in}}%
\pgfpathlineto{\pgfqpoint{3.496189in}{3.521896in}}%
\pgfpathlineto{\pgfqpoint{3.498248in}{3.502474in}}%
\pgfpathlineto{\pgfqpoint{3.504425in}{3.648139in}}%
\pgfpathlineto{\pgfqpoint{3.510602in}{3.657850in}}%
\pgfpathlineto{\pgfqpoint{3.514720in}{3.725827in}}%
\pgfpathlineto{\pgfqpoint{3.518838in}{3.638428in}}%
\pgfpathlineto{\pgfqpoint{3.525015in}{3.619006in}}%
\pgfpathlineto{\pgfqpoint{3.529132in}{3.648139in}}%
\pgfpathlineto{\pgfqpoint{3.531191in}{3.628717in}}%
\pgfpathlineto{\pgfqpoint{3.539427in}{3.628717in}}%
\pgfpathlineto{\pgfqpoint{3.543545in}{3.677272in}}%
\pgfpathlineto{\pgfqpoint{3.545604in}{3.648139in}}%
\pgfpathlineto{\pgfqpoint{3.547663in}{3.638428in}}%
\pgfpathlineto{\pgfqpoint{3.553840in}{3.667561in}}%
\pgfpathlineto{\pgfqpoint{3.555899in}{3.745249in}}%
\pgfpathlineto{\pgfqpoint{3.560017in}{3.793803in}}%
\pgfpathlineto{\pgfqpoint{3.562076in}{3.745249in}}%
\pgfpathlineto{\pgfqpoint{3.568252in}{3.735538in}}%
\pgfpathlineto{\pgfqpoint{3.570311in}{3.745249in}}%
\pgfpathlineto{\pgfqpoint{3.576488in}{3.619006in}}%
\pgfpathlineto{\pgfqpoint{3.584724in}{3.483052in}}%
\pgfpathlineto{\pgfqpoint{3.586783in}{3.541318in}}%
\pgfpathlineto{\pgfqpoint{3.588842in}{3.521896in}}%
\pgfpathlineto{\pgfqpoint{3.590901in}{3.570451in}}%
\pgfpathlineto{\pgfqpoint{3.597078in}{3.609295in}}%
\pgfpathlineto{\pgfqpoint{3.599137in}{3.599584in}}%
\pgfpathlineto{\pgfqpoint{3.601196in}{3.657850in}}%
\pgfpathlineto{\pgfqpoint{3.603254in}{3.609295in}}%
\pgfpathlineto{\pgfqpoint{3.605313in}{3.609295in}}%
\pgfpathlineto{\pgfqpoint{3.611490in}{3.628717in}}%
\pgfpathlineto{\pgfqpoint{3.613549in}{3.628717in}}%
\pgfpathlineto{\pgfqpoint{3.615608in}{3.648139in}}%
\pgfpathlineto{\pgfqpoint{3.617667in}{3.599584in}}%
\pgfpathlineto{\pgfqpoint{3.619726in}{3.589873in}}%
\pgfpathlineto{\pgfqpoint{3.625903in}{3.589873in}}%
\pgfpathlineto{\pgfqpoint{3.627962in}{3.560740in}}%
\pgfpathlineto{\pgfqpoint{3.630021in}{3.599584in}}%
\pgfpathlineto{\pgfqpoint{3.632080in}{3.580162in}}%
\pgfpathlineto{\pgfqpoint{3.634139in}{3.580162in}}%
\pgfpathlineto{\pgfqpoint{3.640315in}{3.560740in}}%
\pgfpathlineto{\pgfqpoint{3.642374in}{3.560740in}}%
\pgfpathlineto{\pgfqpoint{3.644433in}{3.512185in}}%
\pgfpathlineto{\pgfqpoint{3.646492in}{3.521896in}}%
\pgfpathlineto{\pgfqpoint{3.648551in}{3.521896in}}%
\pgfpathlineto{\pgfqpoint{3.654728in}{3.531607in}}%
\pgfpathlineto{\pgfqpoint{3.656787in}{3.502474in}}%
\pgfpathlineto{\pgfqpoint{3.660905in}{3.492763in}}%
\pgfpathlineto{\pgfqpoint{3.662964in}{3.483052in}}%
\pgfpathlineto{\pgfqpoint{3.671200in}{3.521896in}}%
\pgfpathlineto{\pgfqpoint{3.673259in}{3.502474in}}%
\pgfpathlineto{\pgfqpoint{3.675318in}{3.502474in}}%
\pgfpathlineto{\pgfqpoint{3.677377in}{3.483052in}}%
\pgfpathlineto{\pgfqpoint{3.685612in}{3.521896in}}%
\pgfpathlineto{\pgfqpoint{3.687671in}{3.541318in}}%
\pgfpathlineto{\pgfqpoint{3.689730in}{3.512185in}}%
\pgfpathlineto{\pgfqpoint{3.691789in}{3.570451in}}%
\pgfpathlineto{\pgfqpoint{3.697966in}{3.648139in}}%
\pgfpathlineto{\pgfqpoint{3.702084in}{3.609295in}}%
\pgfpathlineto{\pgfqpoint{3.704143in}{3.657850in}}%
\pgfpathlineto{\pgfqpoint{3.706202in}{3.638428in}}%
\pgfpathlineto{\pgfqpoint{3.712379in}{3.657850in}}%
\pgfpathlineto{\pgfqpoint{3.714438in}{3.638428in}}%
\pgfpathlineto{\pgfqpoint{3.716496in}{3.677272in}}%
\pgfpathlineto{\pgfqpoint{3.718555in}{3.667561in}}%
\pgfpathlineto{\pgfqpoint{3.720614in}{3.638428in}}%
\pgfpathlineto{\pgfqpoint{3.726791in}{3.628717in}}%
\pgfpathlineto{\pgfqpoint{3.728850in}{3.667561in}}%
\pgfpathlineto{\pgfqpoint{3.730909in}{3.657850in}}%
\pgfpathlineto{\pgfqpoint{3.735027in}{3.570451in}}%
\pgfpathlineto{\pgfqpoint{3.741204in}{3.580162in}}%
\pgfpathlineto{\pgfqpoint{3.743263in}{3.589873in}}%
\pgfpathlineto{\pgfqpoint{3.745322in}{3.560740in}}%
\pgfpathlineto{\pgfqpoint{3.749440in}{3.560740in}}%
\pgfpathlineto{\pgfqpoint{3.755616in}{3.521896in}}%
\pgfpathlineto{\pgfqpoint{3.757675in}{3.541318in}}%
\pgfpathlineto{\pgfqpoint{3.761793in}{3.512185in}}%
\pgfpathlineto{\pgfqpoint{3.763852in}{3.502474in}}%
\pgfpathlineto{\pgfqpoint{3.770029in}{3.531607in}}%
\pgfpathlineto{\pgfqpoint{3.772088in}{3.570451in}}%
\pgfpathlineto{\pgfqpoint{3.774147in}{3.570451in}}%
\pgfpathlineto{\pgfqpoint{3.776206in}{3.541318in}}%
\pgfpathlineto{\pgfqpoint{3.778265in}{3.560740in}}%
\pgfpathlineto{\pgfqpoint{3.788560in}{3.609295in}}%
\pgfpathlineto{\pgfqpoint{3.790619in}{3.589873in}}%
\pgfpathlineto{\pgfqpoint{3.792677in}{3.638428in}}%
\pgfpathlineto{\pgfqpoint{3.798854in}{3.628717in}}%
\pgfpathlineto{\pgfqpoint{3.800913in}{3.667561in}}%
\pgfpathlineto{\pgfqpoint{3.802972in}{3.648139in}}%
\pgfpathlineto{\pgfqpoint{3.805031in}{3.648139in}}%
\pgfpathlineto{\pgfqpoint{3.807090in}{3.677272in}}%
\pgfpathlineto{\pgfqpoint{3.813267in}{3.677272in}}%
\pgfpathlineto{\pgfqpoint{3.815326in}{3.745249in}}%
\pgfpathlineto{\pgfqpoint{3.817385in}{3.764671in}}%
\pgfpathlineto{\pgfqpoint{3.821503in}{3.745249in}}%
\pgfpathlineto{\pgfqpoint{3.827680in}{3.754960in}}%
\pgfpathlineto{\pgfqpoint{3.829738in}{3.774382in}}%
\pgfpathlineto{\pgfqpoint{3.831797in}{3.745249in}}%
\pgfpathlineto{\pgfqpoint{3.833856in}{3.735538in}}%
\pgfpathlineto{\pgfqpoint{3.835915in}{3.735538in}}%
\pgfpathlineto{\pgfqpoint{3.842092in}{3.784092in}}%
\pgfpathlineto{\pgfqpoint{3.844151in}{3.745249in}}%
\pgfpathlineto{\pgfqpoint{3.848269in}{3.890913in}}%
\pgfpathlineto{\pgfqpoint{3.850328in}{3.939468in}}%
\pgfpathlineto{\pgfqpoint{3.858564in}{3.900624in}}%
\pgfpathlineto{\pgfqpoint{3.860623in}{3.929757in}}%
\pgfpathlineto{\pgfqpoint{3.862682in}{3.852069in}}%
\pgfpathlineto{\pgfqpoint{3.864741in}{3.852069in}}%
\pgfpathlineto{\pgfqpoint{3.870917in}{3.871491in}}%
\pgfpathlineto{\pgfqpoint{3.872976in}{3.861780in}}%
\pgfpathlineto{\pgfqpoint{3.875035in}{3.890913in}}%
\pgfpathlineto{\pgfqpoint{3.877094in}{3.881202in}}%
\pgfpathlineto{\pgfqpoint{3.879153in}{3.910335in}}%
\pgfpathlineto{\pgfqpoint{3.885330in}{3.910335in}}%
\pgfpathlineto{\pgfqpoint{3.887389in}{3.890913in}}%
\pgfpathlineto{\pgfqpoint{3.889448in}{3.842358in}}%
\pgfpathlineto{\pgfqpoint{3.891507in}{3.871491in}}%
\pgfpathlineto{\pgfqpoint{3.893566in}{3.832647in}}%
\pgfpathlineto{\pgfqpoint{3.899743in}{3.832647in}}%
\pgfpathlineto{\pgfqpoint{3.903861in}{3.900624in}}%
\pgfpathlineto{\pgfqpoint{3.905919in}{3.890913in}}%
\pgfpathlineto{\pgfqpoint{3.907978in}{3.968601in}}%
\pgfpathlineto{\pgfqpoint{3.914155in}{3.939468in}}%
\pgfpathlineto{\pgfqpoint{3.916214in}{3.949179in}}%
\pgfpathlineto{\pgfqpoint{3.918273in}{3.949179in}}%
\pgfpathlineto{\pgfqpoint{3.920332in}{3.958890in}}%
\pgfpathlineto{\pgfqpoint{3.922391in}{3.920046in}}%
\pgfpathlineto{\pgfqpoint{3.930627in}{3.881202in}}%
\pgfpathlineto{\pgfqpoint{3.932686in}{3.861780in}}%
\pgfpathlineto{\pgfqpoint{3.934745in}{3.871491in}}%
\pgfpathlineto{\pgfqpoint{3.936804in}{3.832647in}}%
\pgfpathlineto{\pgfqpoint{3.942980in}{3.822936in}}%
\pgfpathlineto{\pgfqpoint{3.947098in}{3.822936in}}%
\pgfpathlineto{\pgfqpoint{3.951216in}{3.813225in}}%
\pgfpathlineto{\pgfqpoint{3.957393in}{3.822936in}}%
\pgfpathlineto{\pgfqpoint{3.959452in}{3.822936in}}%
\pgfpathlineto{\pgfqpoint{3.961511in}{3.832647in}}%
\pgfpathlineto{\pgfqpoint{3.965629in}{3.793803in}}%
\pgfpathlineto{\pgfqpoint{3.971806in}{3.754960in}}%
\pgfpathlineto{\pgfqpoint{3.973865in}{3.657850in}}%
\pgfpathlineto{\pgfqpoint{3.977983in}{3.619006in}}%
\pgfpathlineto{\pgfqpoint{3.980042in}{3.619006in}}%
\pgfpathlineto{\pgfqpoint{3.986218in}{3.609295in}}%
\pgfpathlineto{\pgfqpoint{3.992395in}{3.657850in}}%
\pgfpathlineto{\pgfqpoint{3.994454in}{3.638428in}}%
\pgfpathlineto{\pgfqpoint{4.000631in}{3.609295in}}%
\pgfpathlineto{\pgfqpoint{4.002690in}{3.570451in}}%
\pgfpathlineto{\pgfqpoint{4.004749in}{3.502474in}}%
\pgfpathlineto{\pgfqpoint{4.006808in}{3.531607in}}%
\pgfpathlineto{\pgfqpoint{4.008867in}{3.531607in}}%
\pgfpathlineto{\pgfqpoint{4.015044in}{3.492763in}}%
\pgfpathlineto{\pgfqpoint{4.019161in}{3.551029in}}%
\pgfpathlineto{\pgfqpoint{4.023279in}{3.502474in}}%
\pgfpathlineto{\pgfqpoint{4.029456in}{3.483052in}}%
\pgfpathlineto{\pgfqpoint{4.033574in}{3.444208in}}%
\pgfpathlineto{\pgfqpoint{4.035633in}{3.366520in}}%
\pgfpathlineto{\pgfqpoint{4.037692in}{3.444208in}}%
\pgfpathlineto{\pgfqpoint{4.043869in}{3.473341in}}%
\pgfpathlineto{\pgfqpoint{4.050046in}{3.531607in}}%
\pgfpathlineto{\pgfqpoint{4.052105in}{3.512185in}}%
\pgfpathlineto{\pgfqpoint{4.058281in}{3.521896in}}%
\pgfpathlineto{\pgfqpoint{4.060340in}{3.531607in}}%
\pgfpathlineto{\pgfqpoint{4.062399in}{3.531607in}}%
\pgfpathlineto{\pgfqpoint{4.064458in}{3.541318in}}%
\pgfpathlineto{\pgfqpoint{4.066517in}{3.560740in}}%
\pgfpathlineto{\pgfqpoint{4.074753in}{3.521896in}}%
\pgfpathlineto{\pgfqpoint{4.076812in}{3.541318in}}%
\pgfpathlineto{\pgfqpoint{4.078871in}{3.502474in}}%
\pgfpathlineto{\pgfqpoint{4.080930in}{3.531607in}}%
\pgfpathlineto{\pgfqpoint{4.087107in}{3.531607in}}%
\pgfpathlineto{\pgfqpoint{4.089166in}{3.512185in}}%
\pgfpathlineto{\pgfqpoint{4.091225in}{3.512185in}}%
\pgfpathlineto{\pgfqpoint{4.093284in}{3.444208in}}%
\pgfpathlineto{\pgfqpoint{4.095342in}{3.492763in}}%
\pgfpathlineto{\pgfqpoint{4.101519in}{3.531607in}}%
\pgfpathlineto{\pgfqpoint{4.103578in}{3.502474in}}%
\pgfpathlineto{\pgfqpoint{4.105637in}{3.492763in}}%
\pgfpathlineto{\pgfqpoint{4.109755in}{3.434497in}}%
\pgfpathlineto{\pgfqpoint{4.115932in}{3.463630in}}%
\pgfpathlineto{\pgfqpoint{4.120050in}{3.502474in}}%
\pgfpathlineto{\pgfqpoint{4.122109in}{3.463630in}}%
\pgfpathlineto{\pgfqpoint{4.124168in}{3.453919in}}%
\pgfpathlineto{\pgfqpoint{4.134462in}{3.453919in}}%
\pgfpathlineto{\pgfqpoint{4.136521in}{3.502474in}}%
\pgfpathlineto{\pgfqpoint{4.138580in}{3.473341in}}%
\pgfpathlineto{\pgfqpoint{4.144757in}{3.483052in}}%
\pgfpathlineto{\pgfqpoint{4.146816in}{3.453919in}}%
\pgfpathlineto{\pgfqpoint{4.148875in}{3.521896in}}%
\pgfpathlineto{\pgfqpoint{4.152993in}{3.580162in}}%
\pgfpathlineto{\pgfqpoint{4.159170in}{3.541318in}}%
\pgfpathlineto{\pgfqpoint{4.161229in}{3.541318in}}%
\pgfpathlineto{\pgfqpoint{4.167406in}{3.444208in}}%
\pgfpathlineto{\pgfqpoint{4.173582in}{3.473341in}}%
\pgfpathlineto{\pgfqpoint{4.175641in}{3.434497in}}%
\pgfpathlineto{\pgfqpoint{4.177700in}{3.434497in}}%
\pgfpathlineto{\pgfqpoint{4.179759in}{3.473341in}}%
\pgfpathlineto{\pgfqpoint{4.181818in}{3.444208in}}%
\pgfpathlineto{\pgfqpoint{4.187995in}{3.444208in}}%
\pgfpathlineto{\pgfqpoint{4.190054in}{3.453919in}}%
\pgfpathlineto{\pgfqpoint{4.192113in}{3.405364in}}%
\pgfpathlineto{\pgfqpoint{4.194172in}{3.395653in}}%
\pgfpathlineto{\pgfqpoint{4.196231in}{3.308254in}}%
\pgfpathlineto{\pgfqpoint{4.202408in}{3.298543in}}%
\pgfpathlineto{\pgfqpoint{4.204467in}{3.288832in}}%
\pgfpathlineto{\pgfqpoint{4.206526in}{3.249988in}}%
\pgfpathlineto{\pgfqpoint{4.208584in}{3.240277in}}%
\pgfpathlineto{\pgfqpoint{4.210643in}{3.249988in}}%
\pgfpathlineto{\pgfqpoint{4.216820in}{3.327676in}}%
\pgfpathlineto{\pgfqpoint{4.218879in}{3.317965in}}%
\pgfpathlineto{\pgfqpoint{4.220938in}{3.366520in}}%
\pgfpathlineto{\pgfqpoint{4.222997in}{3.356809in}}%
\pgfpathlineto{\pgfqpoint{4.225056in}{3.337387in}}%
\pgfpathlineto{\pgfqpoint{4.231233in}{3.356809in}}%
\pgfpathlineto{\pgfqpoint{4.233292in}{3.347098in}}%
\pgfpathlineto{\pgfqpoint{4.235351in}{3.327676in}}%
\pgfpathlineto{\pgfqpoint{4.239469in}{3.395653in}}%
\pgfpathlineto{\pgfqpoint{4.245645in}{3.385942in}}%
\pgfpathlineto{\pgfqpoint{4.247704in}{3.424786in}}%
\pgfpathlineto{\pgfqpoint{4.249763in}{3.424786in}}%
\pgfpathlineto{\pgfqpoint{4.251822in}{3.395653in}}%
\pgfpathlineto{\pgfqpoint{4.260058in}{3.434497in}}%
\pgfpathlineto{\pgfqpoint{4.262117in}{3.424786in}}%
\pgfpathlineto{\pgfqpoint{4.264176in}{3.376231in}}%
\pgfpathlineto{\pgfqpoint{4.266235in}{3.376231in}}%
\pgfpathlineto{\pgfqpoint{4.268294in}{3.356809in}}%
\pgfpathlineto{\pgfqpoint{4.274471in}{3.395653in}}%
\pgfpathlineto{\pgfqpoint{4.276530in}{3.366520in}}%
\pgfpathlineto{\pgfqpoint{4.278589in}{3.356809in}}%
\pgfpathlineto{\pgfqpoint{4.280648in}{3.385942in}}%
\pgfpathlineto{\pgfqpoint{4.282707in}{3.366520in}}%
\pgfpathlineto{\pgfqpoint{4.288883in}{3.347098in}}%
\pgfpathlineto{\pgfqpoint{4.290942in}{3.298543in}}%
\pgfpathlineto{\pgfqpoint{4.293001in}{3.327676in}}%
\pgfpathlineto{\pgfqpoint{4.295060in}{3.308254in}}%
\pgfpathlineto{\pgfqpoint{4.297119in}{3.317965in}}%
\pgfpathlineto{\pgfqpoint{4.303296in}{3.269410in}}%
\pgfpathlineto{\pgfqpoint{4.305355in}{3.288832in}}%
\pgfpathlineto{\pgfqpoint{4.307414in}{3.249988in}}%
\pgfpathlineto{\pgfqpoint{4.309473in}{3.269410in}}%
\pgfpathlineto{\pgfqpoint{4.311532in}{3.249988in}}%
\pgfpathlineto{\pgfqpoint{4.317709in}{3.269410in}}%
\pgfpathlineto{\pgfqpoint{4.319768in}{3.288832in}}%
\pgfpathlineto{\pgfqpoint{4.321826in}{3.259699in}}%
\pgfpathlineto{\pgfqpoint{4.323885in}{3.182012in}}%
\pgfpathlineto{\pgfqpoint{4.325944in}{3.191723in}}%
\pgfpathlineto{\pgfqpoint{4.334180in}{3.143168in}}%
\pgfpathlineto{\pgfqpoint{4.336239in}{3.123746in}}%
\pgfpathlineto{\pgfqpoint{4.338298in}{3.084902in}}%
\pgfpathlineto{\pgfqpoint{4.340357in}{3.016925in}}%
\pgfpathlineto{\pgfqpoint{4.346534in}{2.968370in}}%
\pgfpathlineto{\pgfqpoint{4.348593in}{3.036347in}}%
\pgfpathlineto{\pgfqpoint{4.350652in}{3.046058in}}%
\pgfpathlineto{\pgfqpoint{4.352711in}{3.046058in}}%
\pgfpathlineto{\pgfqpoint{4.354770in}{2.987792in}}%
\pgfpathlineto{\pgfqpoint{4.360946in}{3.046058in}}%
\pgfpathlineto{\pgfqpoint{4.363005in}{3.046058in}}%
\pgfpathlineto{\pgfqpoint{4.365064in}{3.036347in}}%
\pgfpathlineto{\pgfqpoint{4.369182in}{3.007214in}}%
\pgfpathlineto{\pgfqpoint{4.375359in}{2.997503in}}%
\pgfpathlineto{\pgfqpoint{4.379477in}{2.948948in}}%
\pgfpathlineto{\pgfqpoint{4.381536in}{2.939237in}}%
\pgfpathlineto{\pgfqpoint{4.383595in}{2.997503in}}%
\pgfpathlineto{\pgfqpoint{4.391831in}{2.939237in}}%
\pgfpathlineto{\pgfqpoint{4.393890in}{2.978081in}}%
\pgfpathlineto{\pgfqpoint{4.395949in}{2.939237in}}%
\pgfpathlineto{\pgfqpoint{4.398007in}{2.939237in}}%
\pgfpathlineto{\pgfqpoint{4.404184in}{2.968370in}}%
\pgfpathlineto{\pgfqpoint{4.408302in}{2.880971in}}%
\pgfpathlineto{\pgfqpoint{4.412420in}{2.968370in}}%
\pgfpathlineto{\pgfqpoint{4.418597in}{2.948948in}}%
\pgfpathlineto{\pgfqpoint{4.422715in}{2.987792in}}%
\pgfpathlineto{\pgfqpoint{4.424774in}{3.046058in}}%
\pgfpathlineto{\pgfqpoint{4.426833in}{3.046058in}}%
\pgfpathlineto{\pgfqpoint{4.433010in}{3.016925in}}%
\pgfpathlineto{\pgfqpoint{4.435068in}{3.046058in}}%
\pgfpathlineto{\pgfqpoint{4.437127in}{2.978081in}}%
\pgfpathlineto{\pgfqpoint{4.439186in}{2.968370in}}%
\pgfpathlineto{\pgfqpoint{4.441245in}{2.978081in}}%
\pgfpathlineto{\pgfqpoint{4.447422in}{2.987792in}}%
\pgfpathlineto{\pgfqpoint{4.449481in}{3.007214in}}%
\pgfpathlineto{\pgfqpoint{4.451540in}{2.987792in}}%
\pgfpathlineto{\pgfqpoint{4.453599in}{3.007214in}}%
\pgfpathlineto{\pgfqpoint{4.455658in}{3.007214in}}%
\pgfpathlineto{\pgfqpoint{4.461835in}{2.997503in}}%
\pgfpathlineto{\pgfqpoint{4.463894in}{2.987792in}}%
\pgfpathlineto{\pgfqpoint{4.465953in}{2.939237in}}%
\pgfpathlineto{\pgfqpoint{4.470071in}{2.793572in}}%
\pgfpathlineto{\pgfqpoint{4.480365in}{2.647908in}}%
\pgfpathlineto{\pgfqpoint{4.484483in}{2.667329in}}%
\pgfpathlineto{\pgfqpoint{4.490660in}{2.560509in}}%
\pgfpathlineto{\pgfqpoint{4.492719in}{2.579931in}}%
\pgfpathlineto{\pgfqpoint{4.494778in}{2.482821in}}%
\pgfpathlineto{\pgfqpoint{4.496837in}{2.443977in}}%
\pgfpathlineto{\pgfqpoint{4.505073in}{2.521665in}}%
\pgfpathlineto{\pgfqpoint{4.507132in}{2.482821in}}%
\pgfpathlineto{\pgfqpoint{4.511249in}{2.541087in}}%
\pgfpathlineto{\pgfqpoint{4.513308in}{2.463399in}}%
\pgfpathlineto{\pgfqpoint{4.519485in}{2.482821in}}%
\pgfpathlineto{\pgfqpoint{4.521544in}{2.414844in}}%
\pgfpathlineto{\pgfqpoint{4.523603in}{2.405133in}}%
\pgfpathlineto{\pgfqpoint{4.525662in}{2.424555in}}%
\pgfpathlineto{\pgfqpoint{4.527721in}{2.424555in}}%
\pgfpathlineto{\pgfqpoint{4.535957in}{2.414844in}}%
\pgfpathlineto{\pgfqpoint{4.538016in}{2.414844in}}%
\pgfpathlineto{\pgfqpoint{4.540075in}{2.502243in}}%
\pgfpathlineto{\pgfqpoint{4.542134in}{2.473110in}}%
\pgfpathlineto{\pgfqpoint{4.548311in}{2.550798in}}%
\pgfpathlineto{\pgfqpoint{4.550369in}{2.638197in}}%
\pgfpathlineto{\pgfqpoint{4.552428in}{2.657618in}}%
\pgfpathlineto{\pgfqpoint{4.554487in}{2.696462in}}%
\pgfpathlineto{\pgfqpoint{4.556546in}{2.803283in}}%
\pgfpathlineto{\pgfqpoint{4.568900in}{2.677040in}}%
\pgfpathlineto{\pgfqpoint{4.570959in}{2.628486in}}%
\pgfpathlineto{\pgfqpoint{4.577136in}{2.618775in}}%
\pgfpathlineto{\pgfqpoint{4.579195in}{2.550798in}}%
\pgfpathlineto{\pgfqpoint{4.581254in}{2.628486in}}%
\pgfpathlineto{\pgfqpoint{4.583313in}{2.599353in}}%
\pgfpathlineto{\pgfqpoint{4.585372in}{2.589642in}}%
\pgfpathlineto{\pgfqpoint{4.591548in}{2.579931in}}%
\pgfpathlineto{\pgfqpoint{4.593607in}{2.570220in}}%
\pgfpathlineto{\pgfqpoint{4.595666in}{2.541087in}}%
\pgfpathlineto{\pgfqpoint{4.599784in}{2.453688in}}%
\pgfpathlineto{\pgfqpoint{4.605961in}{2.492532in}}%
\pgfpathlineto{\pgfqpoint{4.608020in}{2.482821in}}%
\pgfpathlineto{\pgfqpoint{4.610079in}{2.521665in}}%
\pgfpathlineto{\pgfqpoint{4.614197in}{2.677040in}}%
\pgfpathlineto{\pgfqpoint{4.622433in}{2.696462in}}%
\pgfpathlineto{\pgfqpoint{4.624491in}{2.686751in}}%
\pgfpathlineto{\pgfqpoint{4.626550in}{2.686751in}}%
\pgfpathlineto{\pgfqpoint{4.628609in}{2.696462in}}%
\pgfpathlineto{\pgfqpoint{4.634786in}{2.735306in}}%
\pgfpathlineto{\pgfqpoint{4.636845in}{2.706173in}}%
\pgfpathlineto{\pgfqpoint{4.638904in}{2.696462in}}%
\pgfpathlineto{\pgfqpoint{4.649199in}{2.793572in}}%
\pgfpathlineto{\pgfqpoint{4.651258in}{2.783861in}}%
\pgfpathlineto{\pgfqpoint{4.655376in}{2.638197in}}%
\pgfpathlineto{\pgfqpoint{4.657435in}{2.667329in}}%
\pgfpathlineto{\pgfqpoint{4.663611in}{2.735306in}}%
\pgfpathlineto{\pgfqpoint{4.665670in}{2.803283in}}%
\pgfpathlineto{\pgfqpoint{4.667729in}{2.764439in}}%
\pgfpathlineto{\pgfqpoint{4.669788in}{2.871260in}}%
\pgfpathlineto{\pgfqpoint{4.671847in}{2.900393in}}%
\pgfpathlineto{\pgfqpoint{4.680083in}{2.871260in}}%
\pgfpathlineto{\pgfqpoint{4.684201in}{2.783861in}}%
\pgfpathlineto{\pgfqpoint{4.686260in}{2.793572in}}%
\pgfpathlineto{\pgfqpoint{4.692437in}{2.774150in}}%
\pgfpathlineto{\pgfqpoint{4.694496in}{2.745017in}}%
\pgfpathlineto{\pgfqpoint{4.696555in}{2.686751in}}%
\pgfpathlineto{\pgfqpoint{4.698614in}{2.725595in}}%
\pgfpathlineto{\pgfqpoint{4.700672in}{2.715884in}}%
\pgfpathlineto{\pgfqpoint{4.706849in}{2.706173in}}%
\pgfpathlineto{\pgfqpoint{4.708908in}{2.677040in}}%
\pgfpathlineto{\pgfqpoint{4.710967in}{2.696462in}}%
\pgfpathlineto{\pgfqpoint{4.715085in}{2.706173in}}%
\pgfpathlineto{\pgfqpoint{4.721262in}{2.783861in}}%
\pgfpathlineto{\pgfqpoint{4.723321in}{2.667329in}}%
\pgfpathlineto{\pgfqpoint{4.727439in}{2.745017in}}%
\pgfpathlineto{\pgfqpoint{4.729498in}{2.774150in}}%
\pgfpathlineto{\pgfqpoint{4.735675in}{2.764439in}}%
\pgfpathlineto{\pgfqpoint{4.737733in}{2.754728in}}%
\pgfpathlineto{\pgfqpoint{4.739792in}{2.715884in}}%
\pgfpathlineto{\pgfqpoint{4.741851in}{2.812994in}}%
\pgfpathlineto{\pgfqpoint{4.743910in}{2.745017in}}%
\pgfpathlineto{\pgfqpoint{4.754205in}{2.851838in}}%
\pgfpathlineto{\pgfqpoint{4.756264in}{2.842127in}}%
\pgfpathlineto{\pgfqpoint{4.758323in}{2.842127in}}%
\pgfpathlineto{\pgfqpoint{4.764500in}{2.851838in}}%
\pgfpathlineto{\pgfqpoint{4.766559in}{2.832416in}}%
\pgfpathlineto{\pgfqpoint{4.770677in}{2.822705in}}%
\pgfpathlineto{\pgfqpoint{4.772736in}{2.812994in}}%
\pgfpathlineto{\pgfqpoint{4.778912in}{2.842127in}}%
\pgfpathlineto{\pgfqpoint{4.780971in}{2.880971in}}%
\pgfpathlineto{\pgfqpoint{4.785089in}{2.822705in}}%
\pgfpathlineto{\pgfqpoint{4.787148in}{2.745017in}}%
\pgfpathlineto{\pgfqpoint{4.793325in}{2.764439in}}%
\pgfpathlineto{\pgfqpoint{4.795384in}{2.793572in}}%
\pgfpathlineto{\pgfqpoint{4.797443in}{2.842127in}}%
\pgfpathlineto{\pgfqpoint{4.801561in}{2.774150in}}%
\pgfpathlineto{\pgfqpoint{4.807738in}{2.793572in}}%
\pgfpathlineto{\pgfqpoint{4.811856in}{2.725595in}}%
\pgfpathlineto{\pgfqpoint{4.813914in}{2.745017in}}%
\pgfpathlineto{\pgfqpoint{4.815973in}{2.793572in}}%
\pgfpathlineto{\pgfqpoint{4.824209in}{2.735306in}}%
\pgfpathlineto{\pgfqpoint{4.836563in}{2.550798in}}%
\pgfpathlineto{\pgfqpoint{4.838622in}{2.589642in}}%
\pgfpathlineto{\pgfqpoint{4.840681in}{2.531376in}}%
\pgfpathlineto{\pgfqpoint{4.842740in}{2.521665in}}%
\pgfpathlineto{\pgfqpoint{4.844799in}{2.473110in}}%
\pgfpathlineto{\pgfqpoint{4.850976in}{2.482821in}}%
\pgfpathlineto{\pgfqpoint{4.855093in}{2.609064in}}%
\pgfpathlineto{\pgfqpoint{4.857152in}{2.579931in}}%
\pgfpathlineto{\pgfqpoint{4.859211in}{2.531376in}}%
\pgfpathlineto{\pgfqpoint{4.865388in}{2.511954in}}%
\pgfpathlineto{\pgfqpoint{4.867447in}{2.531376in}}%
\pgfpathlineto{\pgfqpoint{4.869506in}{2.570220in}}%
\pgfpathlineto{\pgfqpoint{4.873624in}{2.531376in}}%
\pgfpathlineto{\pgfqpoint{4.881860in}{2.492532in}}%
\pgfpathlineto{\pgfqpoint{4.883919in}{2.502243in}}%
\pgfpathlineto{\pgfqpoint{4.888037in}{2.395422in}}%
\pgfpathlineto{\pgfqpoint{4.894213in}{2.327445in}}%
\pgfpathlineto{\pgfqpoint{4.896272in}{2.278890in}}%
\pgfpathlineto{\pgfqpoint{4.898331in}{2.288601in}}%
\pgfpathlineto{\pgfqpoint{4.900390in}{2.259468in}}%
\pgfpathlineto{\pgfqpoint{4.902449in}{2.113803in}}%
\pgfpathlineto{\pgfqpoint{4.908626in}{2.113803in}}%
\pgfpathlineto{\pgfqpoint{4.910685in}{2.094382in}}%
\pgfpathlineto{\pgfqpoint{4.912744in}{2.104092in}}%
\pgfpathlineto{\pgfqpoint{4.914803in}{1.997272in}}%
\pgfpathlineto{\pgfqpoint{4.916862in}{1.754497in}}%
\pgfpathlineto{\pgfqpoint{4.923039in}{1.540855in}}%
\pgfpathlineto{\pgfqpoint{4.925098in}{1.822474in}}%
\pgfpathlineto{\pgfqpoint{4.927156in}{1.793341in}}%
\pgfpathlineto{\pgfqpoint{4.929215in}{1.929295in}}%
\pgfpathlineto{\pgfqpoint{4.931274in}{1.968139in}}%
\pgfpathlineto{\pgfqpoint{4.937451in}{1.764208in}}%
\pgfpathlineto{\pgfqpoint{4.939510in}{2.104092in}}%
\pgfpathlineto{\pgfqpoint{4.941569in}{2.249757in}}%
\pgfpathlineto{\pgfqpoint{4.943628in}{2.210913in}}%
\pgfpathlineto{\pgfqpoint{4.945687in}{2.006983in}}%
\pgfpathlineto{\pgfqpoint{4.951864in}{1.783630in}}%
\pgfpathlineto{\pgfqpoint{4.955982in}{1.890451in}}%
\pgfpathlineto{\pgfqpoint{4.958041in}{1.861318in}}%
\pgfpathlineto{\pgfqpoint{4.960100in}{1.754497in}}%
\pgfpathlineto{\pgfqpoint{4.966276in}{1.764208in}}%
\pgfpathlineto{\pgfqpoint{4.968335in}{1.812763in}}%
\pgfpathlineto{\pgfqpoint{4.970394in}{1.705942in}}%
\pgfpathlineto{\pgfqpoint{4.972453in}{1.705942in}}%
\pgfpathlineto{\pgfqpoint{4.980689in}{1.744786in}}%
\pgfpathlineto{\pgfqpoint{4.984807in}{1.841896in}}%
\pgfpathlineto{\pgfqpoint{4.986866in}{1.812763in}}%
\pgfpathlineto{\pgfqpoint{4.995102in}{1.851607in}}%
\pgfpathlineto{\pgfqpoint{4.997161in}{1.851607in}}%
\pgfpathlineto{\pgfqpoint{4.999220in}{1.725364in}}%
\pgfpathlineto{\pgfqpoint{5.001279in}{1.676809in}}%
\pgfpathlineto{\pgfqpoint{5.003337in}{1.744786in}}%
\pgfpathlineto{\pgfqpoint{5.009514in}{1.705942in}}%
\pgfpathlineto{\pgfqpoint{5.011573in}{1.647676in}}%
\pgfpathlineto{\pgfqpoint{5.013632in}{1.696231in}}%
\pgfpathlineto{\pgfqpoint{5.015691in}{1.657387in}}%
\pgfpathlineto{\pgfqpoint{5.017750in}{1.647676in}}%
\pgfpathlineto{\pgfqpoint{5.023927in}{1.725364in}}%
\pgfpathlineto{\pgfqpoint{5.025986in}{1.667098in}}%
\pgfpathlineto{\pgfqpoint{5.030104in}{1.715653in}}%
\pgfpathlineto{\pgfqpoint{5.032163in}{1.705942in}}%
\pgfpathlineto{\pgfqpoint{5.038340in}{1.715653in}}%
\pgfpathlineto{\pgfqpoint{5.040399in}{1.735075in}}%
\pgfpathlineto{\pgfqpoint{5.042457in}{1.822474in}}%
\pgfpathlineto{\pgfqpoint{5.044516in}{1.715653in}}%
\pgfpathlineto{\pgfqpoint{5.046575in}{1.783630in}}%
\pgfpathlineto{\pgfqpoint{5.052752in}{1.822474in}}%
\pgfpathlineto{\pgfqpoint{5.058929in}{1.696231in}}%
\pgfpathlineto{\pgfqpoint{5.060988in}{1.715653in}}%
\pgfpathlineto{\pgfqpoint{5.067165in}{1.832185in}}%
\pgfpathlineto{\pgfqpoint{5.069224in}{1.812763in}}%
\pgfpathlineto{\pgfqpoint{5.071283in}{1.851607in}}%
\pgfpathlineto{\pgfqpoint{5.075401in}{1.783630in}}%
\pgfpathlineto{\pgfqpoint{5.083636in}{1.851607in}}%
\pgfpathlineto{\pgfqpoint{5.085695in}{1.851607in}}%
\pgfpathlineto{\pgfqpoint{5.087754in}{1.890451in}}%
\pgfpathlineto{\pgfqpoint{5.089813in}{1.841896in}}%
\pgfpathlineto{\pgfqpoint{5.098049in}{1.900162in}}%
\pgfpathlineto{\pgfqpoint{5.104226in}{2.113803in}}%
\pgfpathlineto{\pgfqpoint{5.110403in}{2.084671in}}%
\pgfpathlineto{\pgfqpoint{5.114521in}{1.968139in}}%
\pgfpathlineto{\pgfqpoint{5.116579in}{1.851607in}}%
\pgfpathlineto{\pgfqpoint{5.118638in}{1.900162in}}%
\pgfpathlineto{\pgfqpoint{5.124815in}{1.900162in}}%
\pgfpathlineto{\pgfqpoint{5.126874in}{1.968139in}}%
\pgfpathlineto{\pgfqpoint{5.128933in}{1.958428in}}%
\pgfpathlineto{\pgfqpoint{5.130992in}{1.900162in}}%
\pgfpathlineto{\pgfqpoint{5.133051in}{1.890451in}}%
\pgfpathlineto{\pgfqpoint{5.139228in}{1.890451in}}%
\pgfpathlineto{\pgfqpoint{5.141287in}{1.909873in}}%
\pgfpathlineto{\pgfqpoint{5.143346in}{1.871029in}}%
\pgfpathlineto{\pgfqpoint{5.145405in}{1.861318in}}%
\pgfpathlineto{\pgfqpoint{5.147464in}{1.812763in}}%
\pgfpathlineto{\pgfqpoint{5.153641in}{1.822474in}}%
\pgfpathlineto{\pgfqpoint{5.157758in}{1.861318in}}%
\pgfpathlineto{\pgfqpoint{5.159817in}{1.861318in}}%
\pgfpathlineto{\pgfqpoint{5.168053in}{1.871029in}}%
\pgfpathlineto{\pgfqpoint{5.170112in}{1.812763in}}%
\pgfpathlineto{\pgfqpoint{5.172171in}{1.822474in}}%
\pgfpathlineto{\pgfqpoint{5.174230in}{1.754497in}}%
\pgfpathlineto{\pgfqpoint{5.176289in}{1.783630in}}%
\pgfpathlineto{\pgfqpoint{5.182466in}{1.773919in}}%
\pgfpathlineto{\pgfqpoint{5.184525in}{1.754497in}}%
\pgfpathlineto{\pgfqpoint{5.186584in}{1.773919in}}%
\pgfpathlineto{\pgfqpoint{5.188643in}{1.754497in}}%
\pgfpathlineto{\pgfqpoint{5.190702in}{1.773919in}}%
\pgfpathlineto{\pgfqpoint{5.196878in}{1.764208in}}%
\pgfpathlineto{\pgfqpoint{5.200996in}{1.744786in}}%
\pgfpathlineto{\pgfqpoint{5.203055in}{1.696231in}}%
\pgfpathlineto{\pgfqpoint{5.205114in}{1.696231in}}%
\pgfpathlineto{\pgfqpoint{5.211291in}{1.715653in}}%
\pgfpathlineto{\pgfqpoint{5.213350in}{1.676809in}}%
\pgfpathlineto{\pgfqpoint{5.215409in}{1.696231in}}%
\pgfpathlineto{\pgfqpoint{5.217468in}{1.647676in}}%
\pgfpathlineto{\pgfqpoint{5.219527in}{1.647676in}}%
\pgfpathlineto{\pgfqpoint{5.225704in}{1.676809in}}%
\pgfpathlineto{\pgfqpoint{5.227763in}{1.628254in}}%
\pgfpathlineto{\pgfqpoint{5.229822in}{1.667098in}}%
\pgfpathlineto{\pgfqpoint{5.231880in}{1.647676in}}%
\pgfpathlineto{\pgfqpoint{5.233939in}{1.676809in}}%
\pgfpathlineto{\pgfqpoint{5.240116in}{1.705942in}}%
\pgfpathlineto{\pgfqpoint{5.246293in}{1.861318in}}%
\pgfpathlineto{\pgfqpoint{5.248352in}{1.871029in}}%
\pgfpathlineto{\pgfqpoint{5.254529in}{1.851607in}}%
\pgfpathlineto{\pgfqpoint{5.256588in}{1.822474in}}%
\pgfpathlineto{\pgfqpoint{5.258647in}{1.861318in}}%
\pgfpathlineto{\pgfqpoint{5.260706in}{1.812763in}}%
\pgfpathlineto{\pgfqpoint{5.262765in}{1.793341in}}%
\pgfpathlineto{\pgfqpoint{5.268941in}{1.803052in}}%
\pgfpathlineto{\pgfqpoint{5.277177in}{1.948717in}}%
\pgfpathlineto{\pgfqpoint{5.283354in}{1.919584in}}%
\pgfpathlineto{\pgfqpoint{5.287472in}{1.822474in}}%
\pgfpathlineto{\pgfqpoint{5.289531in}{1.793341in}}%
\pgfpathlineto{\pgfqpoint{5.291590in}{1.909873in}}%
\pgfpathlineto{\pgfqpoint{5.299826in}{1.880740in}}%
\pgfpathlineto{\pgfqpoint{5.301885in}{1.909873in}}%
\pgfpathlineto{\pgfqpoint{5.303944in}{1.880740in}}%
\pgfpathlineto{\pgfqpoint{5.306002in}{1.871029in}}%
\pgfpathlineto{\pgfqpoint{5.314238in}{1.871029in}}%
\pgfpathlineto{\pgfqpoint{5.316297in}{1.890451in}}%
\pgfpathlineto{\pgfqpoint{5.318356in}{1.880740in}}%
\pgfpathlineto{\pgfqpoint{5.320415in}{1.900162in}}%
\pgfpathlineto{\pgfqpoint{5.330710in}{1.871029in}}%
\pgfpathlineto{\pgfqpoint{5.332769in}{1.851607in}}%
\pgfpathlineto{\pgfqpoint{5.334828in}{1.851607in}}%
\pgfpathlineto{\pgfqpoint{5.341005in}{1.861318in}}%
\pgfpathlineto{\pgfqpoint{5.343064in}{1.851607in}}%
\pgfpathlineto{\pgfqpoint{5.345122in}{1.890451in}}%
\pgfpathlineto{\pgfqpoint{5.347181in}{1.890451in}}%
\pgfpathlineto{\pgfqpoint{5.349240in}{1.909873in}}%
\pgfpathlineto{\pgfqpoint{5.355417in}{1.997272in}}%
\pgfpathlineto{\pgfqpoint{5.357476in}{1.987561in}}%
\pgfpathlineto{\pgfqpoint{5.359535in}{2.026405in}}%
\pgfpathlineto{\pgfqpoint{5.361594in}{1.997272in}}%
\pgfpathlineto{\pgfqpoint{5.363653in}{1.997272in}}%
\pgfpathlineto{\pgfqpoint{5.373948in}{1.939006in}}%
\pgfpathlineto{\pgfqpoint{5.378066in}{1.958428in}}%
\pgfpathlineto{\pgfqpoint{5.384242in}{1.977850in}}%
\pgfpathlineto{\pgfqpoint{5.388360in}{2.055538in}}%
\pgfpathlineto{\pgfqpoint{5.390419in}{2.084671in}}%
\pgfpathlineto{\pgfqpoint{5.392478in}{2.065249in}}%
\pgfpathlineto{\pgfqpoint{5.398655in}{2.026405in}}%
\pgfpathlineto{\pgfqpoint{5.400714in}{1.997272in}}%
\pgfpathlineto{\pgfqpoint{5.402773in}{1.997272in}}%
\pgfpathlineto{\pgfqpoint{5.406891in}{2.084671in}}%
\pgfpathlineto{\pgfqpoint{5.413068in}{2.065249in}}%
\pgfpathlineto{\pgfqpoint{5.415127in}{2.094382in}}%
\pgfpathlineto{\pgfqpoint{5.417186in}{1.987561in}}%
\pgfpathlineto{\pgfqpoint{5.419244in}{1.977850in}}%
\pgfpathlineto{\pgfqpoint{5.427480in}{2.162358in}}%
\pgfpathlineto{\pgfqpoint{5.429539in}{2.181780in}}%
\pgfpathlineto{\pgfqpoint{5.433657in}{2.074960in}}%
\pgfpathlineto{\pgfqpoint{5.435716in}{2.084671in}}%
\pgfpathlineto{\pgfqpoint{5.441893in}{2.094382in}}%
\pgfpathlineto{\pgfqpoint{5.443952in}{2.055538in}}%
\pgfpathlineto{\pgfqpoint{5.446011in}{2.074960in}}%
\pgfpathlineto{\pgfqpoint{5.450129in}{1.987561in}}%
\pgfpathlineto{\pgfqpoint{5.456306in}{2.016694in}}%
\pgfpathlineto{\pgfqpoint{5.458364in}{2.055538in}}%
\pgfpathlineto{\pgfqpoint{5.460423in}{2.065249in}}%
\pgfpathlineto{\pgfqpoint{5.464541in}{2.016694in}}%
\pgfpathlineto{\pgfqpoint{5.470718in}{2.026405in}}%
\pgfpathlineto{\pgfqpoint{5.472777in}{2.113803in}}%
\pgfpathlineto{\pgfqpoint{5.474836in}{2.152647in}}%
\pgfpathlineto{\pgfqpoint{5.476895in}{2.113803in}}%
\pgfpathlineto{\pgfqpoint{5.478954in}{2.181780in}}%
\pgfpathlineto{\pgfqpoint{5.485131in}{2.142936in}}%
\pgfpathlineto{\pgfqpoint{5.487190in}{2.113803in}}%
\pgfpathlineto{\pgfqpoint{5.489249in}{2.133225in}}%
\pgfpathlineto{\pgfqpoint{5.493367in}{2.074960in}}%
\pgfpathlineto{\pgfqpoint{5.499543in}{2.084671in}}%
\pgfpathlineto{\pgfqpoint{5.503661in}{2.113803in}}%
\pgfpathlineto{\pgfqpoint{5.505720in}{2.123514in}}%
\pgfpathlineto{\pgfqpoint{5.507779in}{2.142936in}}%
\pgfpathlineto{\pgfqpoint{5.513956in}{2.133225in}}%
\pgfpathlineto{\pgfqpoint{5.516015in}{2.104092in}}%
\pgfpathlineto{\pgfqpoint{5.518074in}{2.142936in}}%
\pgfpathlineto{\pgfqpoint{5.520133in}{2.113803in}}%
\pgfpathlineto{\pgfqpoint{5.528369in}{2.113803in}}%
\pgfpathlineto{\pgfqpoint{5.530428in}{2.123514in}}%
\pgfpathlineto{\pgfqpoint{5.534545in}{2.104092in}}%
\pgfpathlineto{\pgfqpoint{5.534545in}{2.104092in}}%
\pgfusepath{stroke}%
\end{pgfscope}%
\begin{pgfscope}%
\pgfpathrectangle{\pgfqpoint{0.800000in}{0.528000in}}{\pgfqpoint{4.960000in}{3.696000in}}%
\pgfusepath{clip}%
\pgfsetrectcap%
\pgfsetroundjoin%
\pgfsetlinewidth{1.003750pt}%
\definecolor{currentstroke}{rgb}{1.000000,0.000000,0.000000}%
\pgfsetstrokecolor{currentstroke}%
\pgfsetdash{}{0pt}%
\pgfpathmoveto{\pgfqpoint{1.025455in}{0.715422in}}%
\pgfpathlineto{\pgfqpoint{1.035749in}{0.715422in}}%
\pgfpathlineto{\pgfqpoint{1.037808in}{0.705711in}}%
\pgfpathlineto{\pgfqpoint{1.039867in}{0.715422in}}%
\pgfpathlineto{\pgfqpoint{1.050162in}{0.715422in}}%
\pgfpathlineto{\pgfqpoint{1.052221in}{0.725133in}}%
\pgfpathlineto{\pgfqpoint{1.054280in}{0.715422in}}%
\pgfpathlineto{\pgfqpoint{1.062516in}{0.705711in}}%
\pgfpathlineto{\pgfqpoint{1.064575in}{0.705711in}}%
\pgfpathlineto{\pgfqpoint{1.066633in}{0.715422in}}%
\pgfpathlineto{\pgfqpoint{1.078987in}{0.715422in}}%
\pgfpathlineto{\pgfqpoint{1.081046in}{0.705711in}}%
\pgfpathlineto{\pgfqpoint{1.089282in}{0.705711in}}%
\pgfpathlineto{\pgfqpoint{1.091341in}{0.715422in}}%
\pgfpathlineto{\pgfqpoint{1.093400in}{0.705711in}}%
\pgfpathlineto{\pgfqpoint{1.095459in}{0.715422in}}%
\pgfpathlineto{\pgfqpoint{1.097518in}{0.705711in}}%
\pgfpathlineto{\pgfqpoint{1.103694in}{0.715422in}}%
\pgfpathlineto{\pgfqpoint{1.126343in}{0.715422in}}%
\pgfpathlineto{\pgfqpoint{1.132520in}{0.705711in}}%
\pgfpathlineto{\pgfqpoint{1.134579in}{0.715422in}}%
\pgfpathlineto{\pgfqpoint{1.146932in}{0.715422in}}%
\pgfpathlineto{\pgfqpoint{1.148991in}{0.705711in}}%
\pgfpathlineto{\pgfqpoint{1.151050in}{0.705711in}}%
\pgfpathlineto{\pgfqpoint{1.153109in}{0.715422in}}%
\pgfpathlineto{\pgfqpoint{1.155168in}{0.715422in}}%
\pgfpathlineto{\pgfqpoint{1.161345in}{0.705711in}}%
\pgfpathlineto{\pgfqpoint{1.163404in}{0.725133in}}%
\pgfpathlineto{\pgfqpoint{1.169581in}{0.725133in}}%
\pgfpathlineto{\pgfqpoint{1.175758in}{0.715422in}}%
\pgfpathlineto{\pgfqpoint{1.177817in}{0.744555in}}%
\pgfpathlineto{\pgfqpoint{1.179875in}{0.715422in}}%
\pgfpathlineto{\pgfqpoint{1.181934in}{0.715422in}}%
\pgfpathlineto{\pgfqpoint{1.183993in}{0.696000in}}%
\pgfpathlineto{\pgfqpoint{1.192229in}{0.725133in}}%
\pgfpathlineto{\pgfqpoint{1.194288in}{0.715422in}}%
\pgfpathlineto{\pgfqpoint{1.204583in}{0.715422in}}%
\pgfpathlineto{\pgfqpoint{1.206642in}{0.744555in}}%
\pgfpathlineto{\pgfqpoint{1.208701in}{0.715422in}}%
\pgfpathlineto{\pgfqpoint{1.210760in}{0.715422in}}%
\pgfpathlineto{\pgfqpoint{1.212819in}{0.734844in}}%
\pgfpathlineto{\pgfqpoint{1.218995in}{0.715422in}}%
\pgfpathlineto{\pgfqpoint{1.225172in}{0.715422in}}%
\pgfpathlineto{\pgfqpoint{1.227231in}{0.705711in}}%
\pgfpathlineto{\pgfqpoint{1.233408in}{0.715422in}}%
\pgfpathlineto{\pgfqpoint{1.237526in}{0.715422in}}%
\pgfpathlineto{\pgfqpoint{1.239585in}{0.725133in}}%
\pgfpathlineto{\pgfqpoint{1.247821in}{0.725133in}}%
\pgfpathlineto{\pgfqpoint{1.251939in}{0.705711in}}%
\pgfpathlineto{\pgfqpoint{1.253998in}{0.705711in}}%
\pgfpathlineto{\pgfqpoint{1.256056in}{0.725133in}}%
\pgfpathlineto{\pgfqpoint{1.266351in}{0.696000in}}%
\pgfpathlineto{\pgfqpoint{1.270469in}{0.696000in}}%
\pgfpathlineto{\pgfqpoint{1.276646in}{0.705711in}}%
\pgfpathlineto{\pgfqpoint{1.278705in}{0.705711in}}%
\pgfpathlineto{\pgfqpoint{1.280764in}{0.715422in}}%
\pgfpathlineto{\pgfqpoint{1.282823in}{0.705711in}}%
\pgfpathlineto{\pgfqpoint{1.284882in}{0.705711in}}%
\pgfpathlineto{\pgfqpoint{1.291059in}{0.715422in}}%
\pgfpathlineto{\pgfqpoint{1.293117in}{0.705711in}}%
\pgfpathlineto{\pgfqpoint{1.295176in}{0.715422in}}%
\pgfpathlineto{\pgfqpoint{1.297235in}{0.696000in}}%
\pgfpathlineto{\pgfqpoint{1.299294in}{0.715422in}}%
\pgfpathlineto{\pgfqpoint{1.305471in}{0.705711in}}%
\pgfpathlineto{\pgfqpoint{1.307530in}{0.715422in}}%
\pgfpathlineto{\pgfqpoint{1.311648in}{0.715422in}}%
\pgfpathlineto{\pgfqpoint{1.313707in}{0.705711in}}%
\pgfpathlineto{\pgfqpoint{1.321943in}{0.705711in}}%
\pgfpathlineto{\pgfqpoint{1.324002in}{0.715422in}}%
\pgfpathlineto{\pgfqpoint{1.326061in}{0.705711in}}%
\pgfpathlineto{\pgfqpoint{1.328120in}{0.705711in}}%
\pgfpathlineto{\pgfqpoint{1.334296in}{0.715422in}}%
\pgfpathlineto{\pgfqpoint{1.342532in}{0.715422in}}%
\pgfpathlineto{\pgfqpoint{1.348709in}{0.705711in}}%
\pgfpathlineto{\pgfqpoint{1.356945in}{0.705711in}}%
\pgfpathlineto{\pgfqpoint{1.363122in}{0.696000in}}%
\pgfpathlineto{\pgfqpoint{1.391947in}{0.696000in}}%
\pgfpathlineto{\pgfqpoint{1.394006in}{0.715422in}}%
\pgfpathlineto{\pgfqpoint{1.396065in}{0.705711in}}%
\pgfpathlineto{\pgfqpoint{1.406359in}{0.705711in}}%
\pgfpathlineto{\pgfqpoint{1.408418in}{0.715422in}}%
\pgfpathlineto{\pgfqpoint{1.412536in}{0.715422in}}%
\pgfpathlineto{\pgfqpoint{1.414595in}{0.705711in}}%
\pgfpathlineto{\pgfqpoint{1.420772in}{0.715422in}}%
\pgfpathlineto{\pgfqpoint{1.422831in}{0.725133in}}%
\pgfpathlineto{\pgfqpoint{1.435185in}{0.725133in}}%
\pgfpathlineto{\pgfqpoint{1.437244in}{0.734844in}}%
\pgfpathlineto{\pgfqpoint{1.449597in}{0.734844in}}%
\pgfpathlineto{\pgfqpoint{1.451656in}{0.744555in}}%
\pgfpathlineto{\pgfqpoint{1.455774in}{0.744555in}}%
\pgfpathlineto{\pgfqpoint{1.457833in}{0.734844in}}%
\pgfpathlineto{\pgfqpoint{1.464010in}{0.725133in}}%
\pgfpathlineto{\pgfqpoint{1.466069in}{0.744555in}}%
\pgfpathlineto{\pgfqpoint{1.468128in}{0.744555in}}%
\pgfpathlineto{\pgfqpoint{1.472246in}{0.725133in}}%
\pgfpathlineto{\pgfqpoint{1.478423in}{0.725133in}}%
\pgfpathlineto{\pgfqpoint{1.480482in}{0.754266in}}%
\pgfpathlineto{\pgfqpoint{1.484599in}{0.754266in}}%
\pgfpathlineto{\pgfqpoint{1.486658in}{0.734844in}}%
\pgfpathlineto{\pgfqpoint{1.492835in}{0.725133in}}%
\pgfpathlineto{\pgfqpoint{1.494894in}{0.744555in}}%
\pgfpathlineto{\pgfqpoint{1.496953in}{0.744555in}}%
\pgfpathlineto{\pgfqpoint{1.499012in}{0.705711in}}%
\pgfpathlineto{\pgfqpoint{1.501071in}{0.715422in}}%
\pgfpathlineto{\pgfqpoint{1.507248in}{0.715422in}}%
\pgfpathlineto{\pgfqpoint{1.509307in}{0.734844in}}%
\pgfpathlineto{\pgfqpoint{1.513425in}{0.715422in}}%
\pgfpathlineto{\pgfqpoint{1.515484in}{0.715422in}}%
\pgfpathlineto{\pgfqpoint{1.521660in}{0.696000in}}%
\pgfpathlineto{\pgfqpoint{1.523719in}{0.705711in}}%
\pgfpathlineto{\pgfqpoint{1.525778in}{0.705711in}}%
\pgfpathlineto{\pgfqpoint{1.527837in}{0.715422in}}%
\pgfpathlineto{\pgfqpoint{1.542250in}{0.715422in}}%
\pgfpathlineto{\pgfqpoint{1.544309in}{0.734844in}}%
\pgfpathlineto{\pgfqpoint{1.550486in}{0.715422in}}%
\pgfpathlineto{\pgfqpoint{1.552545in}{0.696000in}}%
\pgfpathlineto{\pgfqpoint{1.599900in}{0.696000in}}%
\pgfpathlineto{\pgfqpoint{1.601959in}{0.705711in}}%
\pgfpathlineto{\pgfqpoint{1.610195in}{0.696000in}}%
\pgfpathlineto{\pgfqpoint{1.614313in}{0.696000in}}%
\pgfpathlineto{\pgfqpoint{1.616372in}{0.725133in}}%
\pgfpathlineto{\pgfqpoint{1.622549in}{0.754266in}}%
\pgfpathlineto{\pgfqpoint{1.624608in}{0.773688in}}%
\pgfpathlineto{\pgfqpoint{1.626667in}{0.705711in}}%
\pgfpathlineto{\pgfqpoint{1.628726in}{0.715422in}}%
\pgfpathlineto{\pgfqpoint{1.630785in}{0.705711in}}%
\pgfpathlineto{\pgfqpoint{1.639020in}{0.705711in}}%
\pgfpathlineto{\pgfqpoint{1.641079in}{0.725133in}}%
\pgfpathlineto{\pgfqpoint{1.645197in}{0.705711in}}%
\pgfpathlineto{\pgfqpoint{1.651374in}{0.705711in}}%
\pgfpathlineto{\pgfqpoint{1.653433in}{0.763977in}}%
\pgfpathlineto{\pgfqpoint{1.655492in}{0.763977in}}%
\pgfpathlineto{\pgfqpoint{1.657551in}{0.734844in}}%
\pgfpathlineto{\pgfqpoint{1.659610in}{0.734844in}}%
\pgfpathlineto{\pgfqpoint{1.665787in}{0.754266in}}%
\pgfpathlineto{\pgfqpoint{1.667846in}{0.773688in}}%
\pgfpathlineto{\pgfqpoint{1.671963in}{0.773688in}}%
\pgfpathlineto{\pgfqpoint{1.674022in}{0.725133in}}%
\pgfpathlineto{\pgfqpoint{1.680199in}{0.744555in}}%
\pgfpathlineto{\pgfqpoint{1.682258in}{0.763977in}}%
\pgfpathlineto{\pgfqpoint{1.684317in}{0.763977in}}%
\pgfpathlineto{\pgfqpoint{1.688435in}{0.744555in}}%
\pgfpathlineto{\pgfqpoint{1.694612in}{0.754266in}}%
\pgfpathlineto{\pgfqpoint{1.696671in}{0.822243in}}%
\pgfpathlineto{\pgfqpoint{1.698730in}{0.831954in}}%
\pgfpathlineto{\pgfqpoint{1.702848in}{0.822243in}}%
\pgfpathlineto{\pgfqpoint{1.709024in}{0.802821in}}%
\pgfpathlineto{\pgfqpoint{1.711083in}{0.880509in}}%
\pgfpathlineto{\pgfqpoint{1.713142in}{0.880509in}}%
\pgfpathlineto{\pgfqpoint{1.717260in}{0.861087in}}%
\pgfpathlineto{\pgfqpoint{1.725496in}{0.919353in}}%
\pgfpathlineto{\pgfqpoint{1.727555in}{0.880509in}}%
\pgfpathlineto{\pgfqpoint{1.729614in}{0.880509in}}%
\pgfpathlineto{\pgfqpoint{1.731673in}{0.841665in}}%
\pgfpathlineto{\pgfqpoint{1.737850in}{0.822243in}}%
\pgfpathlineto{\pgfqpoint{1.739909in}{0.899931in}}%
\pgfpathlineto{\pgfqpoint{1.741968in}{0.890220in}}%
\pgfpathlineto{\pgfqpoint{1.746086in}{0.851376in}}%
\pgfpathlineto{\pgfqpoint{1.752262in}{0.831954in}}%
\pgfpathlineto{\pgfqpoint{1.754321in}{0.880509in}}%
\pgfpathlineto{\pgfqpoint{1.756380in}{0.880509in}}%
\pgfpathlineto{\pgfqpoint{1.758439in}{0.841665in}}%
\pgfpathlineto{\pgfqpoint{1.766675in}{0.822243in}}%
\pgfpathlineto{\pgfqpoint{1.768734in}{0.870798in}}%
\pgfpathlineto{\pgfqpoint{1.770793in}{0.773688in}}%
\pgfpathlineto{\pgfqpoint{1.772852in}{0.831954in}}%
\pgfpathlineto{\pgfqpoint{1.781088in}{0.861087in}}%
\pgfpathlineto{\pgfqpoint{1.783147in}{0.890220in}}%
\pgfpathlineto{\pgfqpoint{1.785205in}{0.899931in}}%
\pgfpathlineto{\pgfqpoint{1.787264in}{0.890220in}}%
\pgfpathlineto{\pgfqpoint{1.789323in}{0.890220in}}%
\pgfpathlineto{\pgfqpoint{1.795500in}{0.880509in}}%
\pgfpathlineto{\pgfqpoint{1.797559in}{0.909642in}}%
\pgfpathlineto{\pgfqpoint{1.801677in}{0.909642in}}%
\pgfpathlineto{\pgfqpoint{1.803736in}{0.880509in}}%
\pgfpathlineto{\pgfqpoint{1.811972in}{0.899931in}}%
\pgfpathlineto{\pgfqpoint{1.814031in}{0.948486in}}%
\pgfpathlineto{\pgfqpoint{1.816090in}{0.958197in}}%
\pgfpathlineto{\pgfqpoint{1.818149in}{0.948486in}}%
\pgfpathlineto{\pgfqpoint{1.824325in}{0.938775in}}%
\pgfpathlineto{\pgfqpoint{1.826384in}{0.977618in}}%
\pgfpathlineto{\pgfqpoint{1.828443in}{0.967908in}}%
\pgfpathlineto{\pgfqpoint{1.830502in}{0.948486in}}%
\pgfpathlineto{\pgfqpoint{1.832561in}{0.909642in}}%
\pgfpathlineto{\pgfqpoint{1.838738in}{0.880509in}}%
\pgfpathlineto{\pgfqpoint{1.840797in}{0.948486in}}%
\pgfpathlineto{\pgfqpoint{1.842856in}{0.958197in}}%
\pgfpathlineto{\pgfqpoint{1.844915in}{0.929064in}}%
\pgfpathlineto{\pgfqpoint{1.846974in}{0.919353in}}%
\pgfpathlineto{\pgfqpoint{1.853151in}{0.899931in}}%
\pgfpathlineto{\pgfqpoint{1.855210in}{0.958197in}}%
\pgfpathlineto{\pgfqpoint{1.859328in}{0.958197in}}%
\pgfpathlineto{\pgfqpoint{1.861386in}{0.948486in}}%
\pgfpathlineto{\pgfqpoint{1.869622in}{0.919353in}}%
\pgfpathlineto{\pgfqpoint{1.871681in}{0.967908in}}%
\pgfpathlineto{\pgfqpoint{1.873740in}{0.967908in}}%
\pgfpathlineto{\pgfqpoint{1.875799in}{0.948486in}}%
\pgfpathlineto{\pgfqpoint{1.881976in}{0.967908in}}%
\pgfpathlineto{\pgfqpoint{1.886094in}{0.967908in}}%
\pgfpathlineto{\pgfqpoint{1.896389in}{0.919353in}}%
\pgfpathlineto{\pgfqpoint{1.898447in}{0.977618in}}%
\pgfpathlineto{\pgfqpoint{1.900506in}{0.967908in}}%
\pgfpathlineto{\pgfqpoint{1.902565in}{0.938775in}}%
\pgfpathlineto{\pgfqpoint{1.904624in}{0.938775in}}%
\pgfpathlineto{\pgfqpoint{1.910801in}{0.958197in}}%
\pgfpathlineto{\pgfqpoint{1.919037in}{0.958197in}}%
\pgfpathlineto{\pgfqpoint{1.925214in}{0.967908in}}%
\pgfpathlineto{\pgfqpoint{1.927273in}{0.977618in}}%
\pgfpathlineto{\pgfqpoint{1.929332in}{0.967908in}}%
\pgfpathlineto{\pgfqpoint{1.931391in}{0.977618in}}%
\pgfpathlineto{\pgfqpoint{1.933450in}{0.958197in}}%
\pgfpathlineto{\pgfqpoint{1.939626in}{0.948486in}}%
\pgfpathlineto{\pgfqpoint{1.941685in}{0.967908in}}%
\pgfpathlineto{\pgfqpoint{1.943744in}{0.958197in}}%
\pgfpathlineto{\pgfqpoint{1.945803in}{0.929064in}}%
\pgfpathlineto{\pgfqpoint{1.956098in}{0.870798in}}%
\pgfpathlineto{\pgfqpoint{1.958157in}{0.831954in}}%
\pgfpathlineto{\pgfqpoint{1.962275in}{0.890220in}}%
\pgfpathlineto{\pgfqpoint{1.968452in}{0.870798in}}%
\pgfpathlineto{\pgfqpoint{1.970511in}{0.880509in}}%
\pgfpathlineto{\pgfqpoint{1.972570in}{0.880509in}}%
\pgfpathlineto{\pgfqpoint{1.974628in}{0.890220in}}%
\pgfpathlineto{\pgfqpoint{1.976687in}{0.890220in}}%
\pgfpathlineto{\pgfqpoint{1.982864in}{0.880509in}}%
\pgfpathlineto{\pgfqpoint{1.984923in}{0.899931in}}%
\pgfpathlineto{\pgfqpoint{1.989041in}{0.899931in}}%
\pgfpathlineto{\pgfqpoint{1.991100in}{0.880509in}}%
\pgfpathlineto{\pgfqpoint{1.997277in}{0.851376in}}%
\pgfpathlineto{\pgfqpoint{1.999336in}{0.870798in}}%
\pgfpathlineto{\pgfqpoint{2.001395in}{0.870798in}}%
\pgfpathlineto{\pgfqpoint{2.003454in}{0.880509in}}%
\pgfpathlineto{\pgfqpoint{2.005513in}{0.880509in}}%
\pgfpathlineto{\pgfqpoint{2.011689in}{0.861087in}}%
\pgfpathlineto{\pgfqpoint{2.013748in}{0.880509in}}%
\pgfpathlineto{\pgfqpoint{2.019925in}{0.851376in}}%
\pgfpathlineto{\pgfqpoint{2.026102in}{0.802821in}}%
\pgfpathlineto{\pgfqpoint{2.028161in}{0.870798in}}%
\pgfpathlineto{\pgfqpoint{2.030220in}{0.870798in}}%
\pgfpathlineto{\pgfqpoint{2.032279in}{0.890220in}}%
\pgfpathlineto{\pgfqpoint{2.034338in}{0.890220in}}%
\pgfpathlineto{\pgfqpoint{2.040515in}{0.899931in}}%
\pgfpathlineto{\pgfqpoint{2.042574in}{0.938775in}}%
\pgfpathlineto{\pgfqpoint{2.048751in}{0.938775in}}%
\pgfpathlineto{\pgfqpoint{2.054927in}{0.899931in}}%
\pgfpathlineto{\pgfqpoint{2.056986in}{0.938775in}}%
\pgfpathlineto{\pgfqpoint{2.061104in}{0.938775in}}%
\pgfpathlineto{\pgfqpoint{2.063163in}{0.948486in}}%
\pgfpathlineto{\pgfqpoint{2.069340in}{0.948486in}}%
\pgfpathlineto{\pgfqpoint{2.071399in}{0.967908in}}%
\pgfpathlineto{\pgfqpoint{2.073458in}{0.929064in}}%
\pgfpathlineto{\pgfqpoint{2.075517in}{0.861087in}}%
\pgfpathlineto{\pgfqpoint{2.077576in}{0.919353in}}%
\pgfpathlineto{\pgfqpoint{2.085812in}{0.958197in}}%
\pgfpathlineto{\pgfqpoint{2.087870in}{0.958197in}}%
\pgfpathlineto{\pgfqpoint{2.089929in}{0.880509in}}%
\pgfpathlineto{\pgfqpoint{2.098165in}{0.880509in}}%
\pgfpathlineto{\pgfqpoint{2.100224in}{0.890220in}}%
\pgfpathlineto{\pgfqpoint{2.102283in}{0.890220in}}%
\pgfpathlineto{\pgfqpoint{2.104342in}{0.899931in}}%
\pgfpathlineto{\pgfqpoint{2.106401in}{0.870798in}}%
\pgfpathlineto{\pgfqpoint{2.114637in}{0.929064in}}%
\pgfpathlineto{\pgfqpoint{2.116696in}{0.919353in}}%
\pgfpathlineto{\pgfqpoint{2.118755in}{0.919353in}}%
\pgfpathlineto{\pgfqpoint{2.120814in}{0.909642in}}%
\pgfpathlineto{\pgfqpoint{2.126990in}{0.919353in}}%
\pgfpathlineto{\pgfqpoint{2.129049in}{0.938775in}}%
\pgfpathlineto{\pgfqpoint{2.131108in}{0.938775in}}%
\pgfpathlineto{\pgfqpoint{2.133167in}{0.958197in}}%
\pgfpathlineto{\pgfqpoint{2.135226in}{0.929064in}}%
\pgfpathlineto{\pgfqpoint{2.141403in}{0.909642in}}%
\pgfpathlineto{\pgfqpoint{2.143462in}{0.938775in}}%
\pgfpathlineto{\pgfqpoint{2.145521in}{0.870798in}}%
\pgfpathlineto{\pgfqpoint{2.147580in}{0.890220in}}%
\pgfpathlineto{\pgfqpoint{2.149639in}{0.929064in}}%
\pgfpathlineto{\pgfqpoint{2.157875in}{0.958197in}}%
\pgfpathlineto{\pgfqpoint{2.159934in}{0.948486in}}%
\pgfpathlineto{\pgfqpoint{2.161993in}{0.958197in}}%
\pgfpathlineto{\pgfqpoint{2.164051in}{0.948486in}}%
\pgfpathlineto{\pgfqpoint{2.174346in}{0.977618in}}%
\pgfpathlineto{\pgfqpoint{2.176405in}{0.977618in}}%
\pgfpathlineto{\pgfqpoint{2.178464in}{0.958197in}}%
\pgfpathlineto{\pgfqpoint{2.184641in}{0.948486in}}%
\pgfpathlineto{\pgfqpoint{2.186700in}{0.977618in}}%
\pgfpathlineto{\pgfqpoint{2.188759in}{0.967908in}}%
\pgfpathlineto{\pgfqpoint{2.190818in}{0.967908in}}%
\pgfpathlineto{\pgfqpoint{2.192877in}{0.977618in}}%
\pgfpathlineto{\pgfqpoint{2.199054in}{0.967908in}}%
\pgfpathlineto{\pgfqpoint{2.201112in}{0.929064in}}%
\pgfpathlineto{\pgfqpoint{2.203171in}{0.938775in}}%
\pgfpathlineto{\pgfqpoint{2.205230in}{0.880509in}}%
\pgfpathlineto{\pgfqpoint{2.207289in}{0.880509in}}%
\pgfpathlineto{\pgfqpoint{2.213466in}{0.890220in}}%
\pgfpathlineto{\pgfqpoint{2.215525in}{0.967908in}}%
\pgfpathlineto{\pgfqpoint{2.221702in}{0.919353in}}%
\pgfpathlineto{\pgfqpoint{2.227879in}{0.958197in}}%
\pgfpathlineto{\pgfqpoint{2.236115in}{0.958197in}}%
\pgfpathlineto{\pgfqpoint{2.242291in}{0.948486in}}%
\pgfpathlineto{\pgfqpoint{2.244350in}{0.958197in}}%
\pgfpathlineto{\pgfqpoint{2.250527in}{0.958197in}}%
\pgfpathlineto{\pgfqpoint{2.256704in}{0.929064in}}%
\pgfpathlineto{\pgfqpoint{2.258763in}{0.967908in}}%
\pgfpathlineto{\pgfqpoint{2.264940in}{0.967908in}}%
\pgfpathlineto{\pgfqpoint{2.271117in}{0.938775in}}%
\pgfpathlineto{\pgfqpoint{2.273176in}{0.919353in}}%
\pgfpathlineto{\pgfqpoint{2.275235in}{0.948486in}}%
\pgfpathlineto{\pgfqpoint{2.277293in}{0.958197in}}%
\pgfpathlineto{\pgfqpoint{2.279352in}{0.938775in}}%
\pgfpathlineto{\pgfqpoint{2.287588in}{0.929064in}}%
\pgfpathlineto{\pgfqpoint{2.291706in}{0.948486in}}%
\pgfpathlineto{\pgfqpoint{2.293765in}{0.929064in}}%
\pgfpathlineto{\pgfqpoint{2.299942in}{0.929064in}}%
\pgfpathlineto{\pgfqpoint{2.302001in}{0.938775in}}%
\pgfpathlineto{\pgfqpoint{2.304060in}{0.929064in}}%
\pgfpathlineto{\pgfqpoint{2.306119in}{0.890220in}}%
\pgfpathlineto{\pgfqpoint{2.308178in}{0.890220in}}%
\pgfpathlineto{\pgfqpoint{2.314355in}{0.851376in}}%
\pgfpathlineto{\pgfqpoint{2.316413in}{0.861087in}}%
\pgfpathlineto{\pgfqpoint{2.318472in}{0.812532in}}%
\pgfpathlineto{\pgfqpoint{2.322590in}{0.783399in}}%
\pgfpathlineto{\pgfqpoint{2.328767in}{0.812532in}}%
\pgfpathlineto{\pgfqpoint{2.330826in}{0.851376in}}%
\pgfpathlineto{\pgfqpoint{2.334944in}{0.812532in}}%
\pgfpathlineto{\pgfqpoint{2.337003in}{0.890220in}}%
\pgfpathlineto{\pgfqpoint{2.343180in}{0.938775in}}%
\pgfpathlineto{\pgfqpoint{2.345239in}{0.967908in}}%
\pgfpathlineto{\pgfqpoint{2.347298in}{0.948486in}}%
\pgfpathlineto{\pgfqpoint{2.351416in}{0.948486in}}%
\pgfpathlineto{\pgfqpoint{2.359651in}{0.958197in}}%
\pgfpathlineto{\pgfqpoint{2.361710in}{0.948486in}}%
\pgfpathlineto{\pgfqpoint{2.365828in}{0.948486in}}%
\pgfpathlineto{\pgfqpoint{2.372005in}{0.929064in}}%
\pgfpathlineto{\pgfqpoint{2.374064in}{0.948486in}}%
\pgfpathlineto{\pgfqpoint{2.376123in}{0.938775in}}%
\pgfpathlineto{\pgfqpoint{2.380241in}{0.938775in}}%
\pgfpathlineto{\pgfqpoint{2.388477in}{0.909642in}}%
\pgfpathlineto{\pgfqpoint{2.390535in}{0.938775in}}%
\pgfpathlineto{\pgfqpoint{2.392594in}{0.880509in}}%
\pgfpathlineto{\pgfqpoint{2.394653in}{0.870798in}}%
\pgfpathlineto{\pgfqpoint{2.400830in}{0.890220in}}%
\pgfpathlineto{\pgfqpoint{2.402889in}{0.929064in}}%
\pgfpathlineto{\pgfqpoint{2.407007in}{0.929064in}}%
\pgfpathlineto{\pgfqpoint{2.415243in}{0.967908in}}%
\pgfpathlineto{\pgfqpoint{2.417302in}{0.967908in}}%
\pgfpathlineto{\pgfqpoint{2.419361in}{0.987329in}}%
\pgfpathlineto{\pgfqpoint{2.421420in}{0.987329in}}%
\pgfpathlineto{\pgfqpoint{2.429655in}{1.006751in}}%
\pgfpathlineto{\pgfqpoint{2.431714in}{0.987329in}}%
\pgfpathlineto{\pgfqpoint{2.433773in}{1.006751in}}%
\pgfpathlineto{\pgfqpoint{2.437891in}{0.967908in}}%
\pgfpathlineto{\pgfqpoint{2.444068in}{0.967908in}}%
\pgfpathlineto{\pgfqpoint{2.446127in}{1.026173in}}%
\pgfpathlineto{\pgfqpoint{2.448186in}{1.035884in}}%
\pgfpathlineto{\pgfqpoint{2.452304in}{1.026173in}}%
\pgfpathlineto{\pgfqpoint{2.458481in}{1.006751in}}%
\pgfpathlineto{\pgfqpoint{2.460540in}{1.026173in}}%
\pgfpathlineto{\pgfqpoint{2.462599in}{1.065017in}}%
\pgfpathlineto{\pgfqpoint{2.464658in}{1.006751in}}%
\pgfpathlineto{\pgfqpoint{2.466716in}{1.026173in}}%
\pgfpathlineto{\pgfqpoint{2.472893in}{1.026173in}}%
\pgfpathlineto{\pgfqpoint{2.474952in}{1.035884in}}%
\pgfpathlineto{\pgfqpoint{2.477011in}{1.065017in}}%
\pgfpathlineto{\pgfqpoint{2.479070in}{1.074728in}}%
\pgfpathlineto{\pgfqpoint{2.481129in}{1.094150in}}%
\pgfpathlineto{\pgfqpoint{2.487306in}{1.103861in}}%
\pgfpathlineto{\pgfqpoint{2.489365in}{1.152416in}}%
\pgfpathlineto{\pgfqpoint{2.491424in}{1.171838in}}%
\pgfpathlineto{\pgfqpoint{2.493483in}{1.162127in}}%
\pgfpathlineto{\pgfqpoint{2.495542in}{1.142705in}}%
\pgfpathlineto{\pgfqpoint{2.501719in}{1.132994in}}%
\pgfpathlineto{\pgfqpoint{2.503778in}{1.162127in}}%
\pgfpathlineto{\pgfqpoint{2.505836in}{1.142705in}}%
\pgfpathlineto{\pgfqpoint{2.507895in}{1.103861in}}%
\pgfpathlineto{\pgfqpoint{2.509954in}{1.103861in}}%
\pgfpathlineto{\pgfqpoint{2.518190in}{1.181549in}}%
\pgfpathlineto{\pgfqpoint{2.520249in}{1.162127in}}%
\pgfpathlineto{\pgfqpoint{2.522308in}{1.074728in}}%
\pgfpathlineto{\pgfqpoint{2.524367in}{1.123283in}}%
\pgfpathlineto{\pgfqpoint{2.532603in}{1.200971in}}%
\pgfpathlineto{\pgfqpoint{2.534662in}{1.171838in}}%
\pgfpathlineto{\pgfqpoint{2.536721in}{1.191260in}}%
\pgfpathlineto{\pgfqpoint{2.538780in}{1.181549in}}%
\pgfpathlineto{\pgfqpoint{2.544956in}{1.181549in}}%
\pgfpathlineto{\pgfqpoint{2.547015in}{1.191260in}}%
\pgfpathlineto{\pgfqpoint{2.549074in}{1.191260in}}%
\pgfpathlineto{\pgfqpoint{2.551133in}{1.200971in}}%
\pgfpathlineto{\pgfqpoint{2.561428in}{1.200971in}}%
\pgfpathlineto{\pgfqpoint{2.563487in}{1.162127in}}%
\pgfpathlineto{\pgfqpoint{2.567605in}{1.142705in}}%
\pgfpathlineto{\pgfqpoint{2.573782in}{1.142705in}}%
\pgfpathlineto{\pgfqpoint{2.575841in}{1.181549in}}%
\pgfpathlineto{\pgfqpoint{2.577900in}{1.162127in}}%
\pgfpathlineto{\pgfqpoint{2.579958in}{1.171838in}}%
\pgfpathlineto{\pgfqpoint{2.588194in}{1.171838in}}%
\pgfpathlineto{\pgfqpoint{2.590253in}{1.181549in}}%
\pgfpathlineto{\pgfqpoint{2.594371in}{1.181549in}}%
\pgfpathlineto{\pgfqpoint{2.596430in}{1.171838in}}%
\pgfpathlineto{\pgfqpoint{2.602607in}{1.162127in}}%
\pgfpathlineto{\pgfqpoint{2.604666in}{1.191260in}}%
\pgfpathlineto{\pgfqpoint{2.606725in}{1.200971in}}%
\pgfpathlineto{\pgfqpoint{2.608784in}{1.191260in}}%
\pgfpathlineto{\pgfqpoint{2.610843in}{1.191260in}}%
\pgfpathlineto{\pgfqpoint{2.617020in}{1.181549in}}%
\pgfpathlineto{\pgfqpoint{2.619078in}{1.191260in}}%
\pgfpathlineto{\pgfqpoint{2.621137in}{1.210682in}}%
\pgfpathlineto{\pgfqpoint{2.625255in}{1.181549in}}%
\pgfpathlineto{\pgfqpoint{2.633491in}{1.171838in}}%
\pgfpathlineto{\pgfqpoint{2.635550in}{1.152416in}}%
\pgfpathlineto{\pgfqpoint{2.637609in}{1.074728in}}%
\pgfpathlineto{\pgfqpoint{2.639668in}{1.084439in}}%
\pgfpathlineto{\pgfqpoint{2.645845in}{1.123283in}}%
\pgfpathlineto{\pgfqpoint{2.647904in}{1.084439in}}%
\pgfpathlineto{\pgfqpoint{2.654081in}{1.239815in}}%
\pgfpathlineto{\pgfqpoint{2.660257in}{1.239815in}}%
\pgfpathlineto{\pgfqpoint{2.664375in}{1.220393in}}%
\pgfpathlineto{\pgfqpoint{2.666434in}{1.181549in}}%
\pgfpathlineto{\pgfqpoint{2.668493in}{1.278659in}}%
\pgfpathlineto{\pgfqpoint{2.674670in}{1.366058in}}%
\pgfpathlineto{\pgfqpoint{2.676729in}{1.443746in}}%
\pgfpathlineto{\pgfqpoint{2.680847in}{1.356347in}}%
\pgfpathlineto{\pgfqpoint{2.682906in}{1.385480in}}%
\pgfpathlineto{\pgfqpoint{2.689083in}{1.375769in}}%
\pgfpathlineto{\pgfqpoint{2.691142in}{1.434035in}}%
\pgfpathlineto{\pgfqpoint{2.695259in}{1.404902in}}%
\pgfpathlineto{\pgfqpoint{2.703495in}{1.404902in}}%
\pgfpathlineto{\pgfqpoint{2.707613in}{1.434035in}}%
\pgfpathlineto{\pgfqpoint{2.711731in}{1.414613in}}%
\pgfpathlineto{\pgfqpoint{2.717908in}{1.404902in}}%
\pgfpathlineto{\pgfqpoint{2.719967in}{1.443746in}}%
\pgfpathlineto{\pgfqpoint{2.722026in}{1.443746in}}%
\pgfpathlineto{\pgfqpoint{2.724085in}{1.453457in}}%
\pgfpathlineto{\pgfqpoint{2.726144in}{1.443746in}}%
\pgfpathlineto{\pgfqpoint{2.732320in}{1.443746in}}%
\pgfpathlineto{\pgfqpoint{2.734379in}{1.414613in}}%
\pgfpathlineto{\pgfqpoint{2.736438in}{1.443746in}}%
\pgfpathlineto{\pgfqpoint{2.738497in}{1.434035in}}%
\pgfpathlineto{\pgfqpoint{2.748792in}{1.434035in}}%
\pgfpathlineto{\pgfqpoint{2.750851in}{1.424324in}}%
\pgfpathlineto{\pgfqpoint{2.754969in}{1.395191in}}%
\pgfpathlineto{\pgfqpoint{2.761146in}{1.414613in}}%
\pgfpathlineto{\pgfqpoint{2.763205in}{1.404902in}}%
\pgfpathlineto{\pgfqpoint{2.765264in}{1.414613in}}%
\pgfpathlineto{\pgfqpoint{2.769381in}{1.356347in}}%
\pgfpathlineto{\pgfqpoint{2.775558in}{1.346636in}}%
\pgfpathlineto{\pgfqpoint{2.777617in}{1.395191in}}%
\pgfpathlineto{\pgfqpoint{2.779676in}{1.404902in}}%
\pgfpathlineto{\pgfqpoint{2.781735in}{1.385480in}}%
\pgfpathlineto{\pgfqpoint{2.783794in}{1.385480in}}%
\pgfpathlineto{\pgfqpoint{2.792030in}{1.414613in}}%
\pgfpathlineto{\pgfqpoint{2.796148in}{1.356347in}}%
\pgfpathlineto{\pgfqpoint{2.798207in}{1.366058in}}%
\pgfpathlineto{\pgfqpoint{2.804384in}{1.404902in}}%
\pgfpathlineto{\pgfqpoint{2.806443in}{1.395191in}}%
\pgfpathlineto{\pgfqpoint{2.808501in}{1.395191in}}%
\pgfpathlineto{\pgfqpoint{2.810560in}{1.404902in}}%
\pgfpathlineto{\pgfqpoint{2.812619in}{1.385480in}}%
\pgfpathlineto{\pgfqpoint{2.818796in}{1.375769in}}%
\pgfpathlineto{\pgfqpoint{2.820855in}{1.434035in}}%
\pgfpathlineto{\pgfqpoint{2.822914in}{1.434035in}}%
\pgfpathlineto{\pgfqpoint{2.824973in}{1.395191in}}%
\pgfpathlineto{\pgfqpoint{2.827032in}{1.424324in}}%
\pgfpathlineto{\pgfqpoint{2.835268in}{1.443746in}}%
\pgfpathlineto{\pgfqpoint{2.837327in}{1.531145in}}%
\pgfpathlineto{\pgfqpoint{2.839386in}{1.492301in}}%
\pgfpathlineto{\pgfqpoint{2.841445in}{1.492301in}}%
\pgfpathlineto{\pgfqpoint{2.847621in}{1.502012in}}%
\pgfpathlineto{\pgfqpoint{2.849680in}{1.502012in}}%
\pgfpathlineto{\pgfqpoint{2.851739in}{1.511723in}}%
\pgfpathlineto{\pgfqpoint{2.853798in}{1.472879in}}%
\pgfpathlineto{\pgfqpoint{2.855857in}{1.472879in}}%
\pgfpathlineto{\pgfqpoint{2.862034in}{1.492301in}}%
\pgfpathlineto{\pgfqpoint{2.864093in}{1.560277in}}%
\pgfpathlineto{\pgfqpoint{2.866152in}{1.569988in}}%
\pgfpathlineto{\pgfqpoint{2.868211in}{1.531145in}}%
\pgfpathlineto{\pgfqpoint{2.870270in}{1.521434in}}%
\pgfpathlineto{\pgfqpoint{2.876447in}{1.521434in}}%
\pgfpathlineto{\pgfqpoint{2.878506in}{1.550566in}}%
\pgfpathlineto{\pgfqpoint{2.880565in}{1.521434in}}%
\pgfpathlineto{\pgfqpoint{2.884682in}{1.434035in}}%
\pgfpathlineto{\pgfqpoint{2.890859in}{1.482590in}}%
\pgfpathlineto{\pgfqpoint{2.892918in}{1.560277in}}%
\pgfpathlineto{\pgfqpoint{2.894977in}{1.560277in}}%
\pgfpathlineto{\pgfqpoint{2.897036in}{1.550566in}}%
\pgfpathlineto{\pgfqpoint{2.899095in}{1.511723in}}%
\pgfpathlineto{\pgfqpoint{2.905272in}{1.628254in}}%
\pgfpathlineto{\pgfqpoint{2.909390in}{1.637965in}}%
\pgfpathlineto{\pgfqpoint{2.913508in}{1.608832in}}%
\pgfpathlineto{\pgfqpoint{2.919685in}{1.618543in}}%
\pgfpathlineto{\pgfqpoint{2.921743in}{1.637965in}}%
\pgfpathlineto{\pgfqpoint{2.923802in}{1.608832in}}%
\pgfpathlineto{\pgfqpoint{2.925861in}{1.618543in}}%
\pgfpathlineto{\pgfqpoint{2.927920in}{1.599121in}}%
\pgfpathlineto{\pgfqpoint{2.934097in}{1.618543in}}%
\pgfpathlineto{\pgfqpoint{2.936156in}{1.618543in}}%
\pgfpathlineto{\pgfqpoint{2.938215in}{1.657387in}}%
\pgfpathlineto{\pgfqpoint{2.940274in}{1.667098in}}%
\pgfpathlineto{\pgfqpoint{2.948510in}{1.667098in}}%
\pgfpathlineto{\pgfqpoint{2.950569in}{1.628254in}}%
\pgfpathlineto{\pgfqpoint{2.952628in}{1.686520in}}%
\pgfpathlineto{\pgfqpoint{2.956746in}{1.667098in}}%
\pgfpathlineto{\pgfqpoint{2.964981in}{1.667098in}}%
\pgfpathlineto{\pgfqpoint{2.967040in}{1.686520in}}%
\pgfpathlineto{\pgfqpoint{2.969099in}{1.667098in}}%
\pgfpathlineto{\pgfqpoint{2.971158in}{1.667098in}}%
\pgfpathlineto{\pgfqpoint{2.977335in}{1.657387in}}%
\pgfpathlineto{\pgfqpoint{2.983512in}{1.686520in}}%
\pgfpathlineto{\pgfqpoint{2.985571in}{1.657387in}}%
\pgfpathlineto{\pgfqpoint{2.991748in}{1.618543in}}%
\pgfpathlineto{\pgfqpoint{2.993807in}{1.637965in}}%
\pgfpathlineto{\pgfqpoint{2.995866in}{1.637965in}}%
\pgfpathlineto{\pgfqpoint{2.997924in}{1.618543in}}%
\pgfpathlineto{\pgfqpoint{2.999983in}{1.637965in}}%
\pgfpathlineto{\pgfqpoint{3.006160in}{1.618543in}}%
\pgfpathlineto{\pgfqpoint{3.008219in}{1.599121in}}%
\pgfpathlineto{\pgfqpoint{3.010278in}{1.647676in}}%
\pgfpathlineto{\pgfqpoint{3.012337in}{1.647676in}}%
\pgfpathlineto{\pgfqpoint{3.014396in}{1.657387in}}%
\pgfpathlineto{\pgfqpoint{3.020573in}{1.657387in}}%
\pgfpathlineto{\pgfqpoint{3.022632in}{1.628254in}}%
\pgfpathlineto{\pgfqpoint{3.024691in}{1.628254in}}%
\pgfpathlineto{\pgfqpoint{3.026750in}{1.618543in}}%
\pgfpathlineto{\pgfqpoint{3.028809in}{1.628254in}}%
\pgfpathlineto{\pgfqpoint{3.037044in}{1.958428in}}%
\pgfpathlineto{\pgfqpoint{3.039103in}{1.705942in}}%
\pgfpathlineto{\pgfqpoint{3.041162in}{1.647676in}}%
\pgfpathlineto{\pgfqpoint{3.043221in}{1.628254in}}%
\pgfpathlineto{\pgfqpoint{3.049398in}{1.637965in}}%
\pgfpathlineto{\pgfqpoint{3.051457in}{1.657387in}}%
\pgfpathlineto{\pgfqpoint{3.055575in}{1.657387in}}%
\pgfpathlineto{\pgfqpoint{3.057634in}{1.647676in}}%
\pgfpathlineto{\pgfqpoint{3.063811in}{1.628254in}}%
\pgfpathlineto{\pgfqpoint{3.069988in}{1.657387in}}%
\pgfpathlineto{\pgfqpoint{3.072046in}{1.637965in}}%
\pgfpathlineto{\pgfqpoint{3.078223in}{1.637965in}}%
\pgfpathlineto{\pgfqpoint{3.080282in}{1.628254in}}%
\pgfpathlineto{\pgfqpoint{3.082341in}{1.657387in}}%
\pgfpathlineto{\pgfqpoint{3.086459in}{1.628254in}}%
\pgfpathlineto{\pgfqpoint{3.092636in}{1.618543in}}%
\pgfpathlineto{\pgfqpoint{3.094695in}{1.676809in}}%
\pgfpathlineto{\pgfqpoint{3.096754in}{1.667098in}}%
\pgfpathlineto{\pgfqpoint{3.100872in}{1.696231in}}%
\pgfpathlineto{\pgfqpoint{3.109108in}{1.696231in}}%
\pgfpathlineto{\pgfqpoint{3.111166in}{1.705942in}}%
\pgfpathlineto{\pgfqpoint{3.113225in}{1.657387in}}%
\pgfpathlineto{\pgfqpoint{3.115284in}{1.637965in}}%
\pgfpathlineto{\pgfqpoint{3.121461in}{1.637965in}}%
\pgfpathlineto{\pgfqpoint{3.123520in}{1.657387in}}%
\pgfpathlineto{\pgfqpoint{3.129697in}{1.657387in}}%
\pgfpathlineto{\pgfqpoint{3.135874in}{1.667098in}}%
\pgfpathlineto{\pgfqpoint{3.137933in}{1.667098in}}%
\pgfpathlineto{\pgfqpoint{3.139992in}{1.676809in}}%
\pgfpathlineto{\pgfqpoint{3.144110in}{1.647676in}}%
\pgfpathlineto{\pgfqpoint{3.150286in}{1.637965in}}%
\pgfpathlineto{\pgfqpoint{3.152345in}{1.657387in}}%
\pgfpathlineto{\pgfqpoint{3.154404in}{1.725364in}}%
\pgfpathlineto{\pgfqpoint{3.156463in}{1.686520in}}%
\pgfpathlineto{\pgfqpoint{3.158522in}{1.686520in}}%
\pgfpathlineto{\pgfqpoint{3.164699in}{1.696231in}}%
\pgfpathlineto{\pgfqpoint{3.166758in}{1.715653in}}%
\pgfpathlineto{\pgfqpoint{3.168817in}{1.715653in}}%
\pgfpathlineto{\pgfqpoint{3.170876in}{1.735075in}}%
\pgfpathlineto{\pgfqpoint{3.172935in}{1.725364in}}%
\pgfpathlineto{\pgfqpoint{3.179112in}{1.735075in}}%
\pgfpathlineto{\pgfqpoint{3.181171in}{1.725364in}}%
\pgfpathlineto{\pgfqpoint{3.183230in}{1.744786in}}%
\pgfpathlineto{\pgfqpoint{3.187347in}{1.744786in}}%
\pgfpathlineto{\pgfqpoint{3.193524in}{1.754497in}}%
\pgfpathlineto{\pgfqpoint{3.195583in}{1.812763in}}%
\pgfpathlineto{\pgfqpoint{3.197642in}{1.822474in}}%
\pgfpathlineto{\pgfqpoint{3.201760in}{1.803052in}}%
\pgfpathlineto{\pgfqpoint{3.207937in}{1.812763in}}%
\pgfpathlineto{\pgfqpoint{3.212055in}{1.832185in}}%
\pgfpathlineto{\pgfqpoint{3.214114in}{1.803052in}}%
\pgfpathlineto{\pgfqpoint{3.216173in}{1.803052in}}%
\pgfpathlineto{\pgfqpoint{3.222350in}{1.822474in}}%
\pgfpathlineto{\pgfqpoint{3.224408in}{1.871029in}}%
\pgfpathlineto{\pgfqpoint{3.228526in}{1.822474in}}%
\pgfpathlineto{\pgfqpoint{3.230585in}{1.803052in}}%
\pgfpathlineto{\pgfqpoint{3.236762in}{1.841896in}}%
\pgfpathlineto{\pgfqpoint{3.238821in}{1.919584in}}%
\pgfpathlineto{\pgfqpoint{3.240880in}{1.880740in}}%
\pgfpathlineto{\pgfqpoint{3.242939in}{1.871029in}}%
\pgfpathlineto{\pgfqpoint{3.244998in}{1.900162in}}%
\pgfpathlineto{\pgfqpoint{3.251175in}{1.919584in}}%
\pgfpathlineto{\pgfqpoint{3.253234in}{1.909873in}}%
\pgfpathlineto{\pgfqpoint{3.255293in}{1.880740in}}%
\pgfpathlineto{\pgfqpoint{3.257352in}{1.871029in}}%
\pgfpathlineto{\pgfqpoint{3.259411in}{1.812763in}}%
\pgfpathlineto{\pgfqpoint{3.267646in}{1.900162in}}%
\pgfpathlineto{\pgfqpoint{3.269705in}{1.841896in}}%
\pgfpathlineto{\pgfqpoint{3.271764in}{1.851607in}}%
\pgfpathlineto{\pgfqpoint{3.273823in}{1.939006in}}%
\pgfpathlineto{\pgfqpoint{3.282059in}{1.948717in}}%
\pgfpathlineto{\pgfqpoint{3.284118in}{1.948717in}}%
\pgfpathlineto{\pgfqpoint{3.288236in}{1.929295in}}%
\pgfpathlineto{\pgfqpoint{3.294413in}{1.958428in}}%
\pgfpathlineto{\pgfqpoint{3.296472in}{1.929295in}}%
\pgfpathlineto{\pgfqpoint{3.298531in}{1.968139in}}%
\pgfpathlineto{\pgfqpoint{3.300589in}{1.977850in}}%
\pgfpathlineto{\pgfqpoint{3.302648in}{1.968139in}}%
\pgfpathlineto{\pgfqpoint{3.310884in}{1.987561in}}%
\pgfpathlineto{\pgfqpoint{3.317061in}{1.939006in}}%
\pgfpathlineto{\pgfqpoint{3.323238in}{1.929295in}}%
\pgfpathlineto{\pgfqpoint{3.327356in}{1.909873in}}%
\pgfpathlineto{\pgfqpoint{3.329415in}{1.890451in}}%
\pgfpathlineto{\pgfqpoint{3.331474in}{1.900162in}}%
\pgfpathlineto{\pgfqpoint{3.337650in}{1.939006in}}%
\pgfpathlineto{\pgfqpoint{3.339709in}{2.142936in}}%
\pgfpathlineto{\pgfqpoint{3.341768in}{2.084671in}}%
\pgfpathlineto{\pgfqpoint{3.345886in}{2.055538in}}%
\pgfpathlineto{\pgfqpoint{3.352063in}{2.055538in}}%
\pgfpathlineto{\pgfqpoint{3.354122in}{2.133225in}}%
\pgfpathlineto{\pgfqpoint{3.356181in}{2.016694in}}%
\pgfpathlineto{\pgfqpoint{3.358240in}{1.977850in}}%
\pgfpathlineto{\pgfqpoint{3.360299in}{1.968139in}}%
\pgfpathlineto{\pgfqpoint{3.366476in}{2.006983in}}%
\pgfpathlineto{\pgfqpoint{3.368535in}{1.997272in}}%
\pgfpathlineto{\pgfqpoint{3.372653in}{1.958428in}}%
\pgfpathlineto{\pgfqpoint{3.374711in}{2.006983in}}%
\pgfpathlineto{\pgfqpoint{3.385006in}{2.055538in}}%
\pgfpathlineto{\pgfqpoint{3.387065in}{1.997272in}}%
\pgfpathlineto{\pgfqpoint{3.389124in}{2.036116in}}%
\pgfpathlineto{\pgfqpoint{3.395301in}{2.045827in}}%
\pgfpathlineto{\pgfqpoint{3.397360in}{2.142936in}}%
\pgfpathlineto{\pgfqpoint{3.399419in}{2.152647in}}%
\pgfpathlineto{\pgfqpoint{3.403537in}{2.152647in}}%
\pgfpathlineto{\pgfqpoint{3.409714in}{2.181780in}}%
\pgfpathlineto{\pgfqpoint{3.411773in}{2.210913in}}%
\pgfpathlineto{\pgfqpoint{3.413831in}{2.220624in}}%
\pgfpathlineto{\pgfqpoint{3.417949in}{2.220624in}}%
\pgfpathlineto{\pgfqpoint{3.424126in}{2.249757in}}%
\pgfpathlineto{\pgfqpoint{3.426185in}{2.288601in}}%
\pgfpathlineto{\pgfqpoint{3.428244in}{2.356578in}}%
\pgfpathlineto{\pgfqpoint{3.430303in}{2.346867in}}%
\pgfpathlineto{\pgfqpoint{3.432362in}{2.356578in}}%
\pgfpathlineto{\pgfqpoint{3.438539in}{2.346867in}}%
\pgfpathlineto{\pgfqpoint{3.440598in}{2.405133in}}%
\pgfpathlineto{\pgfqpoint{3.444716in}{2.317734in}}%
\pgfpathlineto{\pgfqpoint{3.446775in}{2.337156in}}%
\pgfpathlineto{\pgfqpoint{3.452951in}{2.356578in}}%
\pgfpathlineto{\pgfqpoint{3.455010in}{2.337156in}}%
\pgfpathlineto{\pgfqpoint{3.459128in}{2.278890in}}%
\pgfpathlineto{\pgfqpoint{3.467364in}{2.327445in}}%
\pgfpathlineto{\pgfqpoint{3.469423in}{2.346867in}}%
\pgfpathlineto{\pgfqpoint{3.471482in}{2.317734in}}%
\pgfpathlineto{\pgfqpoint{3.473541in}{2.317734in}}%
\pgfpathlineto{\pgfqpoint{3.475600in}{2.327445in}}%
\pgfpathlineto{\pgfqpoint{3.481777in}{2.317734in}}%
\pgfpathlineto{\pgfqpoint{3.483836in}{2.278890in}}%
\pgfpathlineto{\pgfqpoint{3.487954in}{2.298312in}}%
\pgfpathlineto{\pgfqpoint{3.490012in}{2.288601in}}%
\pgfpathlineto{\pgfqpoint{3.496189in}{2.288601in}}%
\pgfpathlineto{\pgfqpoint{3.498248in}{2.327445in}}%
\pgfpathlineto{\pgfqpoint{3.500307in}{2.317734in}}%
\pgfpathlineto{\pgfqpoint{3.502366in}{2.317734in}}%
\pgfpathlineto{\pgfqpoint{3.504425in}{2.298312in}}%
\pgfpathlineto{\pgfqpoint{3.512661in}{2.346867in}}%
\pgfpathlineto{\pgfqpoint{3.516779in}{2.269179in}}%
\pgfpathlineto{\pgfqpoint{3.518838in}{2.269179in}}%
\pgfpathlineto{\pgfqpoint{3.525015in}{2.298312in}}%
\pgfpathlineto{\pgfqpoint{3.527073in}{2.327445in}}%
\pgfpathlineto{\pgfqpoint{3.529132in}{2.337156in}}%
\pgfpathlineto{\pgfqpoint{3.533250in}{2.317734in}}%
\pgfpathlineto{\pgfqpoint{3.539427in}{2.337156in}}%
\pgfpathlineto{\pgfqpoint{3.541486in}{2.337156in}}%
\pgfpathlineto{\pgfqpoint{3.543545in}{2.327445in}}%
\pgfpathlineto{\pgfqpoint{3.545604in}{2.337156in}}%
\pgfpathlineto{\pgfqpoint{3.547663in}{2.327445in}}%
\pgfpathlineto{\pgfqpoint{3.553840in}{2.346867in}}%
\pgfpathlineto{\pgfqpoint{3.555899in}{2.337156in}}%
\pgfpathlineto{\pgfqpoint{3.557958in}{2.337156in}}%
\pgfpathlineto{\pgfqpoint{3.560017in}{2.346867in}}%
\pgfpathlineto{\pgfqpoint{3.562076in}{2.327445in}}%
\pgfpathlineto{\pgfqpoint{3.568252in}{2.356578in}}%
\pgfpathlineto{\pgfqpoint{3.572370in}{2.405133in}}%
\pgfpathlineto{\pgfqpoint{3.574429in}{2.385711in}}%
\pgfpathlineto{\pgfqpoint{3.576488in}{2.346867in}}%
\pgfpathlineto{\pgfqpoint{3.584724in}{2.414844in}}%
\pgfpathlineto{\pgfqpoint{3.586783in}{2.414844in}}%
\pgfpathlineto{\pgfqpoint{3.588842in}{2.405133in}}%
\pgfpathlineto{\pgfqpoint{3.590901in}{2.385711in}}%
\pgfpathlineto{\pgfqpoint{3.597078in}{2.414844in}}%
\pgfpathlineto{\pgfqpoint{3.599137in}{2.463399in}}%
\pgfpathlineto{\pgfqpoint{3.601196in}{2.453688in}}%
\pgfpathlineto{\pgfqpoint{3.603254in}{2.424555in}}%
\pgfpathlineto{\pgfqpoint{3.605313in}{2.424555in}}%
\pgfpathlineto{\pgfqpoint{3.611490in}{2.463399in}}%
\pgfpathlineto{\pgfqpoint{3.613549in}{2.453688in}}%
\pgfpathlineto{\pgfqpoint{3.615608in}{2.463399in}}%
\pgfpathlineto{\pgfqpoint{3.617667in}{2.453688in}}%
\pgfpathlineto{\pgfqpoint{3.619726in}{2.463399in}}%
\pgfpathlineto{\pgfqpoint{3.630021in}{2.492532in}}%
\pgfpathlineto{\pgfqpoint{3.632080in}{2.492532in}}%
\pgfpathlineto{\pgfqpoint{3.634139in}{2.473110in}}%
\pgfpathlineto{\pgfqpoint{3.642374in}{2.434266in}}%
\pgfpathlineto{\pgfqpoint{3.644433in}{2.434266in}}%
\pgfpathlineto{\pgfqpoint{3.646492in}{2.405133in}}%
\pgfpathlineto{\pgfqpoint{3.648551in}{2.414844in}}%
\pgfpathlineto{\pgfqpoint{3.654728in}{2.541087in}}%
\pgfpathlineto{\pgfqpoint{3.656787in}{2.550798in}}%
\pgfpathlineto{\pgfqpoint{3.662964in}{2.502243in}}%
\pgfpathlineto{\pgfqpoint{3.669141in}{2.511954in}}%
\pgfpathlineto{\pgfqpoint{3.673259in}{2.531376in}}%
\pgfpathlineto{\pgfqpoint{3.675318in}{2.531376in}}%
\pgfpathlineto{\pgfqpoint{3.677377in}{2.511954in}}%
\pgfpathlineto{\pgfqpoint{3.683553in}{2.541087in}}%
\pgfpathlineto{\pgfqpoint{3.685612in}{2.570220in}}%
\pgfpathlineto{\pgfqpoint{3.687671in}{2.541087in}}%
\pgfpathlineto{\pgfqpoint{3.689730in}{2.531376in}}%
\pgfpathlineto{\pgfqpoint{3.691789in}{2.502243in}}%
\pgfpathlineto{\pgfqpoint{3.697966in}{2.521665in}}%
\pgfpathlineto{\pgfqpoint{3.700025in}{2.560509in}}%
\pgfpathlineto{\pgfqpoint{3.704143in}{2.531376in}}%
\pgfpathlineto{\pgfqpoint{3.706202in}{2.541087in}}%
\pgfpathlineto{\pgfqpoint{3.712379in}{2.550798in}}%
\pgfpathlineto{\pgfqpoint{3.714438in}{2.579931in}}%
\pgfpathlineto{\pgfqpoint{3.716496in}{2.570220in}}%
\pgfpathlineto{\pgfqpoint{3.718555in}{2.531376in}}%
\pgfpathlineto{\pgfqpoint{3.720614in}{2.541087in}}%
\pgfpathlineto{\pgfqpoint{3.726791in}{2.560509in}}%
\pgfpathlineto{\pgfqpoint{3.728850in}{2.599353in}}%
\pgfpathlineto{\pgfqpoint{3.732968in}{2.550798in}}%
\pgfpathlineto{\pgfqpoint{3.735027in}{2.560509in}}%
\pgfpathlineto{\pgfqpoint{3.741204in}{2.570220in}}%
\pgfpathlineto{\pgfqpoint{3.743263in}{2.599353in}}%
\pgfpathlineto{\pgfqpoint{3.747381in}{2.599353in}}%
\pgfpathlineto{\pgfqpoint{3.749440in}{2.589642in}}%
\pgfpathlineto{\pgfqpoint{3.755616in}{2.570220in}}%
\pgfpathlineto{\pgfqpoint{3.759734in}{2.589642in}}%
\pgfpathlineto{\pgfqpoint{3.761793in}{2.579931in}}%
\pgfpathlineto{\pgfqpoint{3.763852in}{2.589642in}}%
\pgfpathlineto{\pgfqpoint{3.770029in}{2.599353in}}%
\pgfpathlineto{\pgfqpoint{3.772088in}{2.599353in}}%
\pgfpathlineto{\pgfqpoint{3.774147in}{2.609064in}}%
\pgfpathlineto{\pgfqpoint{3.776206in}{2.609064in}}%
\pgfpathlineto{\pgfqpoint{3.778265in}{2.589642in}}%
\pgfpathlineto{\pgfqpoint{3.786501in}{2.638197in}}%
\pgfpathlineto{\pgfqpoint{3.788560in}{2.638197in}}%
\pgfpathlineto{\pgfqpoint{3.790619in}{2.618775in}}%
\pgfpathlineto{\pgfqpoint{3.798854in}{2.618775in}}%
\pgfpathlineto{\pgfqpoint{3.800913in}{2.657618in}}%
\pgfpathlineto{\pgfqpoint{3.802972in}{2.647908in}}%
\pgfpathlineto{\pgfqpoint{3.805031in}{2.647908in}}%
\pgfpathlineto{\pgfqpoint{3.807090in}{2.657618in}}%
\pgfpathlineto{\pgfqpoint{3.813267in}{2.657618in}}%
\pgfpathlineto{\pgfqpoint{3.815326in}{2.686751in}}%
\pgfpathlineto{\pgfqpoint{3.817385in}{2.667329in}}%
\pgfpathlineto{\pgfqpoint{3.819444in}{2.667329in}}%
\pgfpathlineto{\pgfqpoint{3.821503in}{2.686751in}}%
\pgfpathlineto{\pgfqpoint{3.827680in}{2.706173in}}%
\pgfpathlineto{\pgfqpoint{3.829738in}{2.754728in}}%
\pgfpathlineto{\pgfqpoint{3.833856in}{2.735306in}}%
\pgfpathlineto{\pgfqpoint{3.835915in}{2.754728in}}%
\pgfpathlineto{\pgfqpoint{3.842092in}{2.764439in}}%
\pgfpathlineto{\pgfqpoint{3.848269in}{2.793572in}}%
\pgfpathlineto{\pgfqpoint{3.850328in}{2.783861in}}%
\pgfpathlineto{\pgfqpoint{3.858564in}{2.803283in}}%
\pgfpathlineto{\pgfqpoint{3.860623in}{2.812994in}}%
\pgfpathlineto{\pgfqpoint{3.862682in}{2.774150in}}%
\pgfpathlineto{\pgfqpoint{3.864741in}{2.774150in}}%
\pgfpathlineto{\pgfqpoint{3.870917in}{2.803283in}}%
\pgfpathlineto{\pgfqpoint{3.875035in}{2.832416in}}%
\pgfpathlineto{\pgfqpoint{3.877094in}{2.822705in}}%
\pgfpathlineto{\pgfqpoint{3.879153in}{2.822705in}}%
\pgfpathlineto{\pgfqpoint{3.885330in}{2.812994in}}%
\pgfpathlineto{\pgfqpoint{3.887389in}{2.842127in}}%
\pgfpathlineto{\pgfqpoint{3.891507in}{2.822705in}}%
\pgfpathlineto{\pgfqpoint{3.893566in}{2.793572in}}%
\pgfpathlineto{\pgfqpoint{3.899743in}{2.803283in}}%
\pgfpathlineto{\pgfqpoint{3.901802in}{2.842127in}}%
\pgfpathlineto{\pgfqpoint{3.903861in}{2.832416in}}%
\pgfpathlineto{\pgfqpoint{3.905919in}{2.842127in}}%
\pgfpathlineto{\pgfqpoint{3.907978in}{2.822705in}}%
\pgfpathlineto{\pgfqpoint{3.914155in}{2.832416in}}%
\pgfpathlineto{\pgfqpoint{3.916214in}{2.861549in}}%
\pgfpathlineto{\pgfqpoint{3.918273in}{2.861549in}}%
\pgfpathlineto{\pgfqpoint{3.920332in}{2.842127in}}%
\pgfpathlineto{\pgfqpoint{3.922391in}{2.842127in}}%
\pgfpathlineto{\pgfqpoint{3.930627in}{2.871260in}}%
\pgfpathlineto{\pgfqpoint{3.932686in}{2.871260in}}%
\pgfpathlineto{\pgfqpoint{3.934745in}{2.832416in}}%
\pgfpathlineto{\pgfqpoint{3.936804in}{2.842127in}}%
\pgfpathlineto{\pgfqpoint{3.942980in}{2.861549in}}%
\pgfpathlineto{\pgfqpoint{3.945039in}{2.861549in}}%
\pgfpathlineto{\pgfqpoint{3.947098in}{2.880971in}}%
\pgfpathlineto{\pgfqpoint{3.951216in}{2.880971in}}%
\pgfpathlineto{\pgfqpoint{3.957393in}{2.871260in}}%
\pgfpathlineto{\pgfqpoint{3.959452in}{2.939237in}}%
\pgfpathlineto{\pgfqpoint{3.965629in}{2.939237in}}%
\pgfpathlineto{\pgfqpoint{3.971806in}{2.929526in}}%
\pgfpathlineto{\pgfqpoint{3.973865in}{2.997503in}}%
\pgfpathlineto{\pgfqpoint{3.977983in}{2.987792in}}%
\pgfpathlineto{\pgfqpoint{3.980042in}{2.948948in}}%
\pgfpathlineto{\pgfqpoint{3.986218in}{2.948948in}}%
\pgfpathlineto{\pgfqpoint{3.988277in}{2.910104in}}%
\pgfpathlineto{\pgfqpoint{3.990336in}{2.929526in}}%
\pgfpathlineto{\pgfqpoint{3.992395in}{2.987792in}}%
\pgfpathlineto{\pgfqpoint{4.000631in}{2.987792in}}%
\pgfpathlineto{\pgfqpoint{4.002690in}{2.978081in}}%
\pgfpathlineto{\pgfqpoint{4.004749in}{2.978081in}}%
\pgfpathlineto{\pgfqpoint{4.006808in}{3.046058in}}%
\pgfpathlineto{\pgfqpoint{4.008867in}{3.036347in}}%
\pgfpathlineto{\pgfqpoint{4.015044in}{3.046058in}}%
\pgfpathlineto{\pgfqpoint{4.019161in}{3.036347in}}%
\pgfpathlineto{\pgfqpoint{4.021220in}{3.055769in}}%
\pgfpathlineto{\pgfqpoint{4.023279in}{3.016925in}}%
\pgfpathlineto{\pgfqpoint{4.029456in}{3.065480in}}%
\pgfpathlineto{\pgfqpoint{4.033574in}{3.026636in}}%
\pgfpathlineto{\pgfqpoint{4.035633in}{3.046058in}}%
\pgfpathlineto{\pgfqpoint{4.037692in}{3.026636in}}%
\pgfpathlineto{\pgfqpoint{4.043869in}{3.046058in}}%
\pgfpathlineto{\pgfqpoint{4.045928in}{3.026636in}}%
\pgfpathlineto{\pgfqpoint{4.047987in}{3.026636in}}%
\pgfpathlineto{\pgfqpoint{4.050046in}{3.046058in}}%
\pgfpathlineto{\pgfqpoint{4.052105in}{3.036347in}}%
\pgfpathlineto{\pgfqpoint{4.058281in}{3.046058in}}%
\pgfpathlineto{\pgfqpoint{4.060340in}{3.036347in}}%
\pgfpathlineto{\pgfqpoint{4.064458in}{3.036347in}}%
\pgfpathlineto{\pgfqpoint{4.066517in}{3.026636in}}%
\pgfpathlineto{\pgfqpoint{4.074753in}{3.007214in}}%
\pgfpathlineto{\pgfqpoint{4.076812in}{2.997503in}}%
\pgfpathlineto{\pgfqpoint{4.078871in}{3.007214in}}%
\pgfpathlineto{\pgfqpoint{4.080930in}{2.987792in}}%
\pgfpathlineto{\pgfqpoint{4.087107in}{3.016925in}}%
\pgfpathlineto{\pgfqpoint{4.089166in}{3.016925in}}%
\pgfpathlineto{\pgfqpoint{4.091225in}{3.026636in}}%
\pgfpathlineto{\pgfqpoint{4.093284in}{3.046058in}}%
\pgfpathlineto{\pgfqpoint{4.095342in}{3.036347in}}%
\pgfpathlineto{\pgfqpoint{4.101519in}{3.036347in}}%
\pgfpathlineto{\pgfqpoint{4.103578in}{3.016925in}}%
\pgfpathlineto{\pgfqpoint{4.105637in}{3.026636in}}%
\pgfpathlineto{\pgfqpoint{4.107696in}{3.055769in}}%
\pgfpathlineto{\pgfqpoint{4.109755in}{3.055769in}}%
\pgfpathlineto{\pgfqpoint{4.115932in}{3.065480in}}%
\pgfpathlineto{\pgfqpoint{4.117991in}{3.046058in}}%
\pgfpathlineto{\pgfqpoint{4.120050in}{3.046058in}}%
\pgfpathlineto{\pgfqpoint{4.122109in}{3.075191in}}%
\pgfpathlineto{\pgfqpoint{4.124168in}{3.055769in}}%
\pgfpathlineto{\pgfqpoint{4.132403in}{3.065480in}}%
\pgfpathlineto{\pgfqpoint{4.134462in}{3.036347in}}%
\pgfpathlineto{\pgfqpoint{4.136521in}{3.055769in}}%
\pgfpathlineto{\pgfqpoint{4.138580in}{3.055769in}}%
\pgfpathlineto{\pgfqpoint{4.144757in}{3.065480in}}%
\pgfpathlineto{\pgfqpoint{4.146816in}{3.055769in}}%
\pgfpathlineto{\pgfqpoint{4.148875in}{3.055769in}}%
\pgfpathlineto{\pgfqpoint{4.150934in}{3.065480in}}%
\pgfpathlineto{\pgfqpoint{4.152993in}{3.065480in}}%
\pgfpathlineto{\pgfqpoint{4.159170in}{3.075191in}}%
\pgfpathlineto{\pgfqpoint{4.163288in}{3.055769in}}%
\pgfpathlineto{\pgfqpoint{4.165347in}{3.075191in}}%
\pgfpathlineto{\pgfqpoint{4.167406in}{3.075191in}}%
\pgfpathlineto{\pgfqpoint{4.173582in}{3.065480in}}%
\pgfpathlineto{\pgfqpoint{4.175641in}{3.065480in}}%
\pgfpathlineto{\pgfqpoint{4.177700in}{3.055769in}}%
\pgfpathlineto{\pgfqpoint{4.179759in}{3.104324in}}%
\pgfpathlineto{\pgfqpoint{4.181818in}{3.084902in}}%
\pgfpathlineto{\pgfqpoint{4.187995in}{3.094613in}}%
\pgfpathlineto{\pgfqpoint{4.192113in}{3.075191in}}%
\pgfpathlineto{\pgfqpoint{4.194172in}{3.133457in}}%
\pgfpathlineto{\pgfqpoint{4.196231in}{3.114035in}}%
\pgfpathlineto{\pgfqpoint{4.204467in}{3.084902in}}%
\pgfpathlineto{\pgfqpoint{4.210643in}{3.055769in}}%
\pgfpathlineto{\pgfqpoint{4.216820in}{3.046058in}}%
\pgfpathlineto{\pgfqpoint{4.218879in}{3.055769in}}%
\pgfpathlineto{\pgfqpoint{4.220938in}{3.046058in}}%
\pgfpathlineto{\pgfqpoint{4.222997in}{3.055769in}}%
\pgfpathlineto{\pgfqpoint{4.225056in}{3.046058in}}%
\pgfpathlineto{\pgfqpoint{4.231233in}{3.055769in}}%
\pgfpathlineto{\pgfqpoint{4.233292in}{3.046058in}}%
\pgfpathlineto{\pgfqpoint{4.235351in}{3.026636in}}%
\pgfpathlineto{\pgfqpoint{4.237410in}{3.046058in}}%
\pgfpathlineto{\pgfqpoint{4.239469in}{3.036347in}}%
\pgfpathlineto{\pgfqpoint{4.245645in}{3.046058in}}%
\pgfpathlineto{\pgfqpoint{4.247704in}{3.055769in}}%
\pgfpathlineto{\pgfqpoint{4.249763in}{3.055769in}}%
\pgfpathlineto{\pgfqpoint{4.251822in}{3.065480in}}%
\pgfpathlineto{\pgfqpoint{4.260058in}{3.065480in}}%
\pgfpathlineto{\pgfqpoint{4.264176in}{3.046058in}}%
\pgfpathlineto{\pgfqpoint{4.266235in}{3.055769in}}%
\pgfpathlineto{\pgfqpoint{4.268294in}{3.046058in}}%
\pgfpathlineto{\pgfqpoint{4.274471in}{3.075191in}}%
\pgfpathlineto{\pgfqpoint{4.278589in}{3.046058in}}%
\pgfpathlineto{\pgfqpoint{4.280648in}{3.065480in}}%
\pgfpathlineto{\pgfqpoint{4.282707in}{3.046058in}}%
\pgfpathlineto{\pgfqpoint{4.288883in}{3.055769in}}%
\pgfpathlineto{\pgfqpoint{4.290942in}{3.065480in}}%
\pgfpathlineto{\pgfqpoint{4.293001in}{3.046058in}}%
\pgfpathlineto{\pgfqpoint{4.295060in}{3.055769in}}%
\pgfpathlineto{\pgfqpoint{4.297119in}{3.046058in}}%
\pgfpathlineto{\pgfqpoint{4.303296in}{3.036347in}}%
\pgfpathlineto{\pgfqpoint{4.305355in}{3.036347in}}%
\pgfpathlineto{\pgfqpoint{4.307414in}{3.026636in}}%
\pgfpathlineto{\pgfqpoint{4.309473in}{3.026636in}}%
\pgfpathlineto{\pgfqpoint{4.311532in}{3.016925in}}%
\pgfpathlineto{\pgfqpoint{4.317709in}{3.016925in}}%
\pgfpathlineto{\pgfqpoint{4.321826in}{2.987792in}}%
\pgfpathlineto{\pgfqpoint{4.323885in}{3.007214in}}%
\pgfpathlineto{\pgfqpoint{4.325944in}{2.997503in}}%
\pgfpathlineto{\pgfqpoint{4.334180in}{2.978081in}}%
\pgfpathlineto{\pgfqpoint{4.336239in}{2.978081in}}%
\pgfpathlineto{\pgfqpoint{4.338298in}{2.997503in}}%
\pgfpathlineto{\pgfqpoint{4.340357in}{2.978081in}}%
\pgfpathlineto{\pgfqpoint{4.346534in}{2.987792in}}%
\pgfpathlineto{\pgfqpoint{4.350652in}{2.939237in}}%
\pgfpathlineto{\pgfqpoint{4.352711in}{2.948948in}}%
\pgfpathlineto{\pgfqpoint{4.354770in}{2.929526in}}%
\pgfpathlineto{\pgfqpoint{4.360946in}{2.929526in}}%
\pgfpathlineto{\pgfqpoint{4.363005in}{2.900393in}}%
\pgfpathlineto{\pgfqpoint{4.365064in}{2.822705in}}%
\pgfpathlineto{\pgfqpoint{4.367123in}{2.871260in}}%
\pgfpathlineto{\pgfqpoint{4.369182in}{2.861549in}}%
\pgfpathlineto{\pgfqpoint{4.375359in}{2.851838in}}%
\pgfpathlineto{\pgfqpoint{4.379477in}{2.774150in}}%
\pgfpathlineto{\pgfqpoint{4.381536in}{2.803283in}}%
\pgfpathlineto{\pgfqpoint{4.383595in}{2.793572in}}%
\pgfpathlineto{\pgfqpoint{4.391831in}{2.745017in}}%
\pgfpathlineto{\pgfqpoint{4.393890in}{2.745017in}}%
\pgfpathlineto{\pgfqpoint{4.395949in}{2.822705in}}%
\pgfpathlineto{\pgfqpoint{4.398007in}{2.812994in}}%
\pgfpathlineto{\pgfqpoint{4.404184in}{2.803283in}}%
\pgfpathlineto{\pgfqpoint{4.408302in}{2.880971in}}%
\pgfpathlineto{\pgfqpoint{4.412420in}{2.890682in}}%
\pgfpathlineto{\pgfqpoint{4.420656in}{2.851838in}}%
\pgfpathlineto{\pgfqpoint{4.422715in}{2.812994in}}%
\pgfpathlineto{\pgfqpoint{4.426833in}{2.793572in}}%
\pgfpathlineto{\pgfqpoint{4.433010in}{2.803283in}}%
\pgfpathlineto{\pgfqpoint{4.435068in}{2.793572in}}%
\pgfpathlineto{\pgfqpoint{4.439186in}{2.745017in}}%
\pgfpathlineto{\pgfqpoint{4.441245in}{2.745017in}}%
\pgfpathlineto{\pgfqpoint{4.447422in}{2.764439in}}%
\pgfpathlineto{\pgfqpoint{4.449481in}{2.754728in}}%
\pgfpathlineto{\pgfqpoint{4.451540in}{2.735306in}}%
\pgfpathlineto{\pgfqpoint{4.453599in}{2.793572in}}%
\pgfpathlineto{\pgfqpoint{4.455658in}{2.764439in}}%
\pgfpathlineto{\pgfqpoint{4.461835in}{2.754728in}}%
\pgfpathlineto{\pgfqpoint{4.465953in}{2.647908in}}%
\pgfpathlineto{\pgfqpoint{4.468012in}{2.745017in}}%
\pgfpathlineto{\pgfqpoint{4.470071in}{2.745017in}}%
\pgfpathlineto{\pgfqpoint{4.478306in}{2.686751in}}%
\pgfpathlineto{\pgfqpoint{4.480365in}{2.657618in}}%
\pgfpathlineto{\pgfqpoint{4.482424in}{2.725595in}}%
\pgfpathlineto{\pgfqpoint{4.484483in}{2.686751in}}%
\pgfpathlineto{\pgfqpoint{4.490660in}{2.725595in}}%
\pgfpathlineto{\pgfqpoint{4.492719in}{2.686751in}}%
\pgfpathlineto{\pgfqpoint{4.494778in}{2.618775in}}%
\pgfpathlineto{\pgfqpoint{4.496837in}{2.715884in}}%
\pgfpathlineto{\pgfqpoint{4.498896in}{2.686751in}}%
\pgfpathlineto{\pgfqpoint{4.505073in}{2.696462in}}%
\pgfpathlineto{\pgfqpoint{4.507132in}{2.686751in}}%
\pgfpathlineto{\pgfqpoint{4.509191in}{2.667329in}}%
\pgfpathlineto{\pgfqpoint{4.511249in}{2.735306in}}%
\pgfpathlineto{\pgfqpoint{4.513308in}{2.706173in}}%
\pgfpathlineto{\pgfqpoint{4.519485in}{2.725595in}}%
\pgfpathlineto{\pgfqpoint{4.521544in}{2.706173in}}%
\pgfpathlineto{\pgfqpoint{4.523603in}{2.706173in}}%
\pgfpathlineto{\pgfqpoint{4.525662in}{2.735306in}}%
\pgfpathlineto{\pgfqpoint{4.527721in}{2.735306in}}%
\pgfpathlineto{\pgfqpoint{4.538016in}{2.686751in}}%
\pgfpathlineto{\pgfqpoint{4.542134in}{2.686751in}}%
\pgfpathlineto{\pgfqpoint{4.548311in}{2.677040in}}%
\pgfpathlineto{\pgfqpoint{4.550369in}{2.677040in}}%
\pgfpathlineto{\pgfqpoint{4.554487in}{2.628486in}}%
\pgfpathlineto{\pgfqpoint{4.556546in}{2.628486in}}%
\pgfpathlineto{\pgfqpoint{4.564782in}{2.735306in}}%
\pgfpathlineto{\pgfqpoint{4.566841in}{2.579931in}}%
\pgfpathlineto{\pgfqpoint{4.568900in}{2.647908in}}%
\pgfpathlineto{\pgfqpoint{4.570959in}{2.589642in}}%
\pgfpathlineto{\pgfqpoint{4.577136in}{2.579931in}}%
\pgfpathlineto{\pgfqpoint{4.579195in}{2.541087in}}%
\pgfpathlineto{\pgfqpoint{4.581254in}{2.443977in}}%
\pgfpathlineto{\pgfqpoint{4.583313in}{2.550798in}}%
\pgfpathlineto{\pgfqpoint{4.585372in}{2.541087in}}%
\pgfpathlineto{\pgfqpoint{4.591548in}{2.550798in}}%
\pgfpathlineto{\pgfqpoint{4.593607in}{2.434266in}}%
\pgfpathlineto{\pgfqpoint{4.595666in}{2.395422in}}%
\pgfpathlineto{\pgfqpoint{4.597725in}{2.424555in}}%
\pgfpathlineto{\pgfqpoint{4.599784in}{2.376000in}}%
\pgfpathlineto{\pgfqpoint{4.605961in}{2.405133in}}%
\pgfpathlineto{\pgfqpoint{4.608020in}{2.337156in}}%
\pgfpathlineto{\pgfqpoint{4.610079in}{2.337156in}}%
\pgfpathlineto{\pgfqpoint{4.612138in}{2.385711in}}%
\pgfpathlineto{\pgfqpoint{4.614197in}{2.405133in}}%
\pgfpathlineto{\pgfqpoint{4.622433in}{2.385711in}}%
\pgfpathlineto{\pgfqpoint{4.624491in}{2.356578in}}%
\pgfpathlineto{\pgfqpoint{4.626550in}{2.385711in}}%
\pgfpathlineto{\pgfqpoint{4.628609in}{2.395422in}}%
\pgfpathlineto{\pgfqpoint{4.634786in}{2.405133in}}%
\pgfpathlineto{\pgfqpoint{4.638904in}{2.385711in}}%
\pgfpathlineto{\pgfqpoint{4.640963in}{2.395422in}}%
\pgfpathlineto{\pgfqpoint{4.643022in}{2.376000in}}%
\pgfpathlineto{\pgfqpoint{4.649199in}{2.385711in}}%
\pgfpathlineto{\pgfqpoint{4.653317in}{2.259468in}}%
\pgfpathlineto{\pgfqpoint{4.657435in}{2.230335in}}%
\pgfpathlineto{\pgfqpoint{4.663611in}{2.230335in}}%
\pgfpathlineto{\pgfqpoint{4.667729in}{2.201202in}}%
\pgfpathlineto{\pgfqpoint{4.669788in}{2.220624in}}%
\pgfpathlineto{\pgfqpoint{4.671847in}{2.210913in}}%
\pgfpathlineto{\pgfqpoint{4.682142in}{2.210913in}}%
\pgfpathlineto{\pgfqpoint{4.684201in}{2.240046in}}%
\pgfpathlineto{\pgfqpoint{4.692437in}{2.240046in}}%
\pgfpathlineto{\pgfqpoint{4.696555in}{2.220624in}}%
\pgfpathlineto{\pgfqpoint{4.698614in}{2.220624in}}%
\pgfpathlineto{\pgfqpoint{4.700672in}{2.230335in}}%
\pgfpathlineto{\pgfqpoint{4.706849in}{2.249757in}}%
\pgfpathlineto{\pgfqpoint{4.710967in}{2.298312in}}%
\pgfpathlineto{\pgfqpoint{4.715085in}{2.269179in}}%
\pgfpathlineto{\pgfqpoint{4.721262in}{2.249757in}}%
\pgfpathlineto{\pgfqpoint{4.723321in}{2.210913in}}%
\pgfpathlineto{\pgfqpoint{4.725380in}{2.240046in}}%
\pgfpathlineto{\pgfqpoint{4.727439in}{2.172069in}}%
\pgfpathlineto{\pgfqpoint{4.729498in}{2.172069in}}%
\pgfpathlineto{\pgfqpoint{4.735675in}{2.191491in}}%
\pgfpathlineto{\pgfqpoint{4.737733in}{2.181780in}}%
\pgfpathlineto{\pgfqpoint{4.739792in}{2.191491in}}%
\pgfpathlineto{\pgfqpoint{4.741851in}{2.220624in}}%
\pgfpathlineto{\pgfqpoint{4.743910in}{2.201202in}}%
\pgfpathlineto{\pgfqpoint{4.750087in}{2.220624in}}%
\pgfpathlineto{\pgfqpoint{4.752146in}{2.210913in}}%
\pgfpathlineto{\pgfqpoint{4.754205in}{2.210913in}}%
\pgfpathlineto{\pgfqpoint{4.756264in}{2.191491in}}%
\pgfpathlineto{\pgfqpoint{4.758323in}{2.220624in}}%
\pgfpathlineto{\pgfqpoint{4.764500in}{2.220624in}}%
\pgfpathlineto{\pgfqpoint{4.766559in}{2.201202in}}%
\pgfpathlineto{\pgfqpoint{4.770677in}{2.240046in}}%
\pgfpathlineto{\pgfqpoint{4.772736in}{2.210913in}}%
\pgfpathlineto{\pgfqpoint{4.778912in}{2.162358in}}%
\pgfpathlineto{\pgfqpoint{4.780971in}{2.133225in}}%
\pgfpathlineto{\pgfqpoint{4.785089in}{2.181780in}}%
\pgfpathlineto{\pgfqpoint{4.787148in}{2.172069in}}%
\pgfpathlineto{\pgfqpoint{4.793325in}{2.191491in}}%
\pgfpathlineto{\pgfqpoint{4.797443in}{2.152647in}}%
\pgfpathlineto{\pgfqpoint{4.799502in}{2.181780in}}%
\pgfpathlineto{\pgfqpoint{4.801561in}{2.172069in}}%
\pgfpathlineto{\pgfqpoint{4.807738in}{2.191491in}}%
\pgfpathlineto{\pgfqpoint{4.809797in}{2.181780in}}%
\pgfpathlineto{\pgfqpoint{4.811856in}{2.181780in}}%
\pgfpathlineto{\pgfqpoint{4.813914in}{2.191491in}}%
\pgfpathlineto{\pgfqpoint{4.815973in}{2.191491in}}%
\pgfpathlineto{\pgfqpoint{4.824209in}{2.172069in}}%
\pgfpathlineto{\pgfqpoint{4.826268in}{2.172069in}}%
\pgfpathlineto{\pgfqpoint{4.828327in}{2.201202in}}%
\pgfpathlineto{\pgfqpoint{4.830386in}{2.191491in}}%
\pgfpathlineto{\pgfqpoint{4.836563in}{2.181780in}}%
\pgfpathlineto{\pgfqpoint{4.838622in}{2.181780in}}%
\pgfpathlineto{\pgfqpoint{4.840681in}{2.172069in}}%
\pgfpathlineto{\pgfqpoint{4.842740in}{2.240046in}}%
\pgfpathlineto{\pgfqpoint{4.844799in}{2.210913in}}%
\pgfpathlineto{\pgfqpoint{4.850976in}{2.210913in}}%
\pgfpathlineto{\pgfqpoint{4.853034in}{2.201202in}}%
\pgfpathlineto{\pgfqpoint{4.855093in}{2.201202in}}%
\pgfpathlineto{\pgfqpoint{4.857152in}{2.240046in}}%
\pgfpathlineto{\pgfqpoint{4.859211in}{2.220624in}}%
\pgfpathlineto{\pgfqpoint{4.865388in}{2.230335in}}%
\pgfpathlineto{\pgfqpoint{4.867447in}{2.220624in}}%
\pgfpathlineto{\pgfqpoint{4.869506in}{2.220624in}}%
\pgfpathlineto{\pgfqpoint{4.873624in}{2.249757in}}%
\pgfpathlineto{\pgfqpoint{4.881860in}{2.259468in}}%
\pgfpathlineto{\pgfqpoint{4.885978in}{2.259468in}}%
\pgfpathlineto{\pgfqpoint{4.888037in}{2.249757in}}%
\pgfpathlineto{\pgfqpoint{4.894213in}{2.249757in}}%
\pgfpathlineto{\pgfqpoint{4.896272in}{2.240046in}}%
\pgfpathlineto{\pgfqpoint{4.898331in}{2.240046in}}%
\pgfpathlineto{\pgfqpoint{4.900390in}{2.210913in}}%
\pgfpathlineto{\pgfqpoint{4.902449in}{2.104092in}}%
\pgfpathlineto{\pgfqpoint{4.908626in}{2.065249in}}%
\pgfpathlineto{\pgfqpoint{4.910685in}{1.773919in}}%
\pgfpathlineto{\pgfqpoint{4.916862in}{1.463168in}}%
\pgfpathlineto{\pgfqpoint{4.923039in}{1.249526in}}%
\pgfpathlineto{\pgfqpoint{4.925098in}{1.249526in}}%
\pgfpathlineto{\pgfqpoint{4.927156in}{1.103861in}}%
\pgfpathlineto{\pgfqpoint{4.929215in}{1.094150in}}%
\pgfpathlineto{\pgfqpoint{4.931274in}{1.016462in}}%
\pgfpathlineto{\pgfqpoint{4.937451in}{0.938775in}}%
\pgfpathlineto{\pgfqpoint{4.941569in}{0.734844in}}%
\pgfpathlineto{\pgfqpoint{4.945687in}{0.734844in}}%
\pgfpathlineto{\pgfqpoint{4.951864in}{0.705711in}}%
\pgfpathlineto{\pgfqpoint{4.953923in}{0.705711in}}%
\pgfpathlineto{\pgfqpoint{4.955982in}{0.696000in}}%
\pgfpathlineto{\pgfqpoint{4.958041in}{0.705711in}}%
\pgfpathlineto{\pgfqpoint{4.960100in}{0.705711in}}%
\pgfpathlineto{\pgfqpoint{4.968335in}{0.744555in}}%
\pgfpathlineto{\pgfqpoint{4.970394in}{0.725133in}}%
\pgfpathlineto{\pgfqpoint{4.972453in}{0.783399in}}%
\pgfpathlineto{\pgfqpoint{4.980689in}{0.783399in}}%
\pgfpathlineto{\pgfqpoint{4.982748in}{0.793110in}}%
\pgfpathlineto{\pgfqpoint{4.986866in}{0.890220in}}%
\pgfpathlineto{\pgfqpoint{4.995102in}{0.861087in}}%
\pgfpathlineto{\pgfqpoint{4.997161in}{0.861087in}}%
\pgfpathlineto{\pgfqpoint{4.999220in}{0.831954in}}%
\pgfpathlineto{\pgfqpoint{5.001279in}{0.831954in}}%
\pgfpathlineto{\pgfqpoint{5.003337in}{0.812532in}}%
\pgfpathlineto{\pgfqpoint{5.009514in}{0.793110in}}%
\pgfpathlineto{\pgfqpoint{5.011573in}{0.773688in}}%
\pgfpathlineto{\pgfqpoint{5.013632in}{0.783399in}}%
\pgfpathlineto{\pgfqpoint{5.015691in}{0.783399in}}%
\pgfpathlineto{\pgfqpoint{5.017750in}{0.793110in}}%
\pgfpathlineto{\pgfqpoint{5.023927in}{0.783399in}}%
\pgfpathlineto{\pgfqpoint{5.025986in}{0.773688in}}%
\pgfpathlineto{\pgfqpoint{5.028045in}{0.793110in}}%
\pgfpathlineto{\pgfqpoint{5.038340in}{0.793110in}}%
\pgfpathlineto{\pgfqpoint{5.042457in}{0.773688in}}%
\pgfpathlineto{\pgfqpoint{5.044516in}{0.793110in}}%
\pgfpathlineto{\pgfqpoint{5.046575in}{0.793110in}}%
\pgfpathlineto{\pgfqpoint{5.052752in}{0.783399in}}%
\pgfpathlineto{\pgfqpoint{5.054811in}{0.793110in}}%
\pgfpathlineto{\pgfqpoint{5.056870in}{0.793110in}}%
\pgfpathlineto{\pgfqpoint{5.058929in}{0.783399in}}%
\pgfpathlineto{\pgfqpoint{5.060988in}{0.783399in}}%
\pgfpathlineto{\pgfqpoint{5.067165in}{0.793110in}}%
\pgfpathlineto{\pgfqpoint{5.071283in}{0.773688in}}%
\pgfpathlineto{\pgfqpoint{5.073342in}{0.783399in}}%
\pgfpathlineto{\pgfqpoint{5.075401in}{0.783399in}}%
\pgfpathlineto{\pgfqpoint{5.083636in}{0.793110in}}%
\pgfpathlineto{\pgfqpoint{5.085695in}{0.802821in}}%
\pgfpathlineto{\pgfqpoint{5.087754in}{0.831954in}}%
\pgfpathlineto{\pgfqpoint{5.089813in}{0.822243in}}%
\pgfpathlineto{\pgfqpoint{5.095990in}{0.812532in}}%
\pgfpathlineto{\pgfqpoint{5.100108in}{0.812532in}}%
\pgfpathlineto{\pgfqpoint{5.102167in}{0.822243in}}%
\pgfpathlineto{\pgfqpoint{5.104226in}{0.822243in}}%
\pgfpathlineto{\pgfqpoint{5.110403in}{0.841665in}}%
\pgfpathlineto{\pgfqpoint{5.114521in}{0.822243in}}%
\pgfpathlineto{\pgfqpoint{5.116579in}{0.831954in}}%
\pgfpathlineto{\pgfqpoint{5.118638in}{0.831954in}}%
\pgfpathlineto{\pgfqpoint{5.124815in}{0.841665in}}%
\pgfpathlineto{\pgfqpoint{5.128933in}{0.822243in}}%
\pgfpathlineto{\pgfqpoint{5.133051in}{0.822243in}}%
\pgfpathlineto{\pgfqpoint{5.139228in}{0.831954in}}%
\pgfpathlineto{\pgfqpoint{5.143346in}{0.802821in}}%
\pgfpathlineto{\pgfqpoint{5.145405in}{0.822243in}}%
\pgfpathlineto{\pgfqpoint{5.147464in}{0.812532in}}%
\pgfpathlineto{\pgfqpoint{5.153641in}{0.802821in}}%
\pgfpathlineto{\pgfqpoint{5.155699in}{0.822243in}}%
\pgfpathlineto{\pgfqpoint{5.157758in}{0.812532in}}%
\pgfpathlineto{\pgfqpoint{5.159817in}{0.822243in}}%
\pgfpathlineto{\pgfqpoint{5.168053in}{0.812532in}}%
\pgfpathlineto{\pgfqpoint{5.170112in}{0.812532in}}%
\pgfpathlineto{\pgfqpoint{5.172171in}{0.802821in}}%
\pgfpathlineto{\pgfqpoint{5.174230in}{0.802821in}}%
\pgfpathlineto{\pgfqpoint{5.176289in}{0.793110in}}%
\pgfpathlineto{\pgfqpoint{5.182466in}{0.802821in}}%
\pgfpathlineto{\pgfqpoint{5.184525in}{0.802821in}}%
\pgfpathlineto{\pgfqpoint{5.186584in}{0.812532in}}%
\pgfpathlineto{\pgfqpoint{5.188643in}{0.812532in}}%
\pgfpathlineto{\pgfqpoint{5.190702in}{0.802821in}}%
\pgfpathlineto{\pgfqpoint{5.196878in}{0.802821in}}%
\pgfpathlineto{\pgfqpoint{5.198937in}{0.783399in}}%
\pgfpathlineto{\pgfqpoint{5.203055in}{0.783399in}}%
\pgfpathlineto{\pgfqpoint{5.205114in}{0.793110in}}%
\pgfpathlineto{\pgfqpoint{5.211291in}{0.793110in}}%
\pgfpathlineto{\pgfqpoint{5.213350in}{0.783399in}}%
\pgfpathlineto{\pgfqpoint{5.215409in}{0.783399in}}%
\pgfpathlineto{\pgfqpoint{5.217468in}{0.793110in}}%
\pgfpathlineto{\pgfqpoint{5.219527in}{0.783399in}}%
\pgfpathlineto{\pgfqpoint{5.227763in}{0.783399in}}%
\pgfpathlineto{\pgfqpoint{5.231880in}{0.763977in}}%
\pgfpathlineto{\pgfqpoint{5.233939in}{0.773688in}}%
\pgfpathlineto{\pgfqpoint{5.240116in}{0.783399in}}%
\pgfpathlineto{\pgfqpoint{5.242175in}{0.773688in}}%
\pgfpathlineto{\pgfqpoint{5.246293in}{0.773688in}}%
\pgfpathlineto{\pgfqpoint{5.248352in}{0.783399in}}%
\pgfpathlineto{\pgfqpoint{5.254529in}{0.783399in}}%
\pgfpathlineto{\pgfqpoint{5.258647in}{0.763977in}}%
\pgfpathlineto{\pgfqpoint{5.260706in}{0.773688in}}%
\pgfpathlineto{\pgfqpoint{5.262765in}{0.763977in}}%
\pgfpathlineto{\pgfqpoint{5.268941in}{0.783399in}}%
\pgfpathlineto{\pgfqpoint{5.271000in}{0.773688in}}%
\pgfpathlineto{\pgfqpoint{5.273059in}{0.773688in}}%
\pgfpathlineto{\pgfqpoint{5.275118in}{0.783399in}}%
\pgfpathlineto{\pgfqpoint{5.277177in}{0.783399in}}%
\pgfpathlineto{\pgfqpoint{5.283354in}{0.773688in}}%
\pgfpathlineto{\pgfqpoint{5.287472in}{0.793110in}}%
\pgfpathlineto{\pgfqpoint{5.289531in}{0.793110in}}%
\pgfpathlineto{\pgfqpoint{5.291590in}{0.783399in}}%
\pgfpathlineto{\pgfqpoint{5.299826in}{0.793110in}}%
\pgfpathlineto{\pgfqpoint{5.312179in}{0.793110in}}%
\pgfpathlineto{\pgfqpoint{5.316297in}{0.773688in}}%
\pgfpathlineto{\pgfqpoint{5.318356in}{0.783399in}}%
\pgfpathlineto{\pgfqpoint{5.326592in}{0.783399in}}%
\pgfpathlineto{\pgfqpoint{5.328651in}{0.773688in}}%
\pgfpathlineto{\pgfqpoint{5.334828in}{0.773688in}}%
\pgfpathlineto{\pgfqpoint{5.341005in}{0.783399in}}%
\pgfpathlineto{\pgfqpoint{5.343064in}{0.763977in}}%
\pgfpathlineto{\pgfqpoint{5.349240in}{0.793110in}}%
\pgfpathlineto{\pgfqpoint{5.355417in}{0.783399in}}%
\pgfpathlineto{\pgfqpoint{5.357476in}{0.773688in}}%
\pgfpathlineto{\pgfqpoint{5.359535in}{0.773688in}}%
\pgfpathlineto{\pgfqpoint{5.363653in}{0.793110in}}%
\pgfpathlineto{\pgfqpoint{5.371889in}{0.783399in}}%
\pgfpathlineto{\pgfqpoint{5.373948in}{0.793110in}}%
\pgfpathlineto{\pgfqpoint{5.376007in}{0.793110in}}%
\pgfpathlineto{\pgfqpoint{5.378066in}{0.783399in}}%
\pgfpathlineto{\pgfqpoint{5.384242in}{0.783399in}}%
\pgfpathlineto{\pgfqpoint{5.386301in}{0.773688in}}%
\pgfpathlineto{\pgfqpoint{5.388360in}{0.773688in}}%
\pgfpathlineto{\pgfqpoint{5.390419in}{0.783399in}}%
\pgfpathlineto{\pgfqpoint{5.392478in}{0.773688in}}%
\pgfpathlineto{\pgfqpoint{5.400714in}{0.773688in}}%
\pgfpathlineto{\pgfqpoint{5.402773in}{0.763977in}}%
\pgfpathlineto{\pgfqpoint{5.404832in}{0.773688in}}%
\pgfpathlineto{\pgfqpoint{5.406891in}{0.773688in}}%
\pgfpathlineto{\pgfqpoint{5.413068in}{0.783399in}}%
\pgfpathlineto{\pgfqpoint{5.415127in}{0.783399in}}%
\pgfpathlineto{\pgfqpoint{5.417186in}{0.773688in}}%
\pgfpathlineto{\pgfqpoint{5.421303in}{0.793110in}}%
\pgfpathlineto{\pgfqpoint{5.427480in}{0.793110in}}%
\pgfpathlineto{\pgfqpoint{5.429539in}{0.783399in}}%
\pgfpathlineto{\pgfqpoint{5.433657in}{0.793110in}}%
\pgfpathlineto{\pgfqpoint{5.435716in}{0.793110in}}%
\pgfpathlineto{\pgfqpoint{5.441893in}{0.783399in}}%
\pgfpathlineto{\pgfqpoint{5.446011in}{0.763977in}}%
\pgfpathlineto{\pgfqpoint{5.450129in}{0.783399in}}%
\pgfpathlineto{\pgfqpoint{5.456306in}{0.773688in}}%
\pgfpathlineto{\pgfqpoint{5.458364in}{0.773688in}}%
\pgfpathlineto{\pgfqpoint{5.460423in}{0.763977in}}%
\pgfpathlineto{\pgfqpoint{5.464541in}{0.783399in}}%
\pgfpathlineto{\pgfqpoint{5.470718in}{0.773688in}}%
\pgfpathlineto{\pgfqpoint{5.472777in}{0.763977in}}%
\pgfpathlineto{\pgfqpoint{5.474836in}{0.763977in}}%
\pgfpathlineto{\pgfqpoint{5.476895in}{0.773688in}}%
\pgfpathlineto{\pgfqpoint{5.478954in}{0.763977in}}%
\pgfpathlineto{\pgfqpoint{5.485131in}{0.783399in}}%
\pgfpathlineto{\pgfqpoint{5.489249in}{0.763977in}}%
\pgfpathlineto{\pgfqpoint{5.491308in}{0.763977in}}%
\pgfpathlineto{\pgfqpoint{5.493367in}{0.773688in}}%
\pgfpathlineto{\pgfqpoint{5.499543in}{0.763977in}}%
\pgfpathlineto{\pgfqpoint{5.501602in}{0.763977in}}%
\pgfpathlineto{\pgfqpoint{5.503661in}{0.773688in}}%
\pgfpathlineto{\pgfqpoint{5.513956in}{0.773688in}}%
\pgfpathlineto{\pgfqpoint{5.516015in}{0.763977in}}%
\pgfpathlineto{\pgfqpoint{5.518074in}{0.763977in}}%
\pgfpathlineto{\pgfqpoint{5.520133in}{0.783399in}}%
\pgfpathlineto{\pgfqpoint{5.528369in}{0.783399in}}%
\pgfpathlineto{\pgfqpoint{5.530428in}{0.773688in}}%
\pgfpathlineto{\pgfqpoint{5.532487in}{0.754266in}}%
\pgfpathlineto{\pgfqpoint{5.534545in}{0.773688in}}%
\pgfpathlineto{\pgfqpoint{5.534545in}{0.773688in}}%
\pgfusepath{stroke}%
\end{pgfscope}%
\begin{pgfscope}%
\pgfpathrectangle{\pgfqpoint{0.800000in}{0.528000in}}{\pgfqpoint{4.960000in}{3.696000in}}%
\pgfusepath{clip}%
\pgfsetrectcap%
\pgfsetroundjoin%
\pgfsetlinewidth{1.003750pt}%
\definecolor{currentstroke}{rgb}{0.000000,0.000000,1.000000}%
\pgfsetstrokecolor{currentstroke}%
\pgfsetdash{}{0pt}%
\pgfpathmoveto{\pgfqpoint{1.025455in}{3.308254in}}%
\pgfpathlineto{\pgfqpoint{1.031631in}{3.220855in}}%
\pgfpathlineto{\pgfqpoint{1.033690in}{3.143168in}}%
\pgfpathlineto{\pgfqpoint{1.035749in}{3.143168in}}%
\pgfpathlineto{\pgfqpoint{1.037808in}{3.211145in}}%
\pgfpathlineto{\pgfqpoint{1.039867in}{3.172301in}}%
\pgfpathlineto{\pgfqpoint{1.046044in}{3.114035in}}%
\pgfpathlineto{\pgfqpoint{1.048103in}{3.114035in}}%
\pgfpathlineto{\pgfqpoint{1.050162in}{3.094613in}}%
\pgfpathlineto{\pgfqpoint{1.052221in}{3.026636in}}%
\pgfpathlineto{\pgfqpoint{1.054280in}{3.065480in}}%
\pgfpathlineto{\pgfqpoint{1.062516in}{3.016925in}}%
\pgfpathlineto{\pgfqpoint{1.064575in}{3.065480in}}%
\pgfpathlineto{\pgfqpoint{1.066633in}{3.084902in}}%
\pgfpathlineto{\pgfqpoint{1.068692in}{3.007214in}}%
\pgfpathlineto{\pgfqpoint{1.074869in}{3.026636in}}%
\pgfpathlineto{\pgfqpoint{1.076928in}{3.026636in}}%
\pgfpathlineto{\pgfqpoint{1.078987in}{2.919815in}}%
\pgfpathlineto{\pgfqpoint{1.081046in}{2.958659in}}%
\pgfpathlineto{\pgfqpoint{1.083105in}{2.880971in}}%
\pgfpathlineto{\pgfqpoint{1.089282in}{2.880971in}}%
\pgfpathlineto{\pgfqpoint{1.091341in}{2.997503in}}%
\pgfpathlineto{\pgfqpoint{1.095459in}{3.046058in}}%
\pgfpathlineto{\pgfqpoint{1.097518in}{3.133457in}}%
\pgfpathlineto{\pgfqpoint{1.103694in}{3.143168in}}%
\pgfpathlineto{\pgfqpoint{1.105753in}{3.201434in}}%
\pgfpathlineto{\pgfqpoint{1.107812in}{3.191723in}}%
\pgfpathlineto{\pgfqpoint{1.109871in}{3.201434in}}%
\pgfpathlineto{\pgfqpoint{1.111930in}{3.249988in}}%
\pgfpathlineto{\pgfqpoint{1.120166in}{3.347098in}}%
\pgfpathlineto{\pgfqpoint{1.122225in}{3.317965in}}%
\pgfpathlineto{\pgfqpoint{1.124284in}{3.347098in}}%
\pgfpathlineto{\pgfqpoint{1.126343in}{3.347098in}}%
\pgfpathlineto{\pgfqpoint{1.132520in}{3.279121in}}%
\pgfpathlineto{\pgfqpoint{1.136638in}{3.182012in}}%
\pgfpathlineto{\pgfqpoint{1.138697in}{3.249988in}}%
\pgfpathlineto{\pgfqpoint{1.140756in}{3.220855in}}%
\pgfpathlineto{\pgfqpoint{1.151050in}{3.337387in}}%
\pgfpathlineto{\pgfqpoint{1.153109in}{3.327676in}}%
\pgfpathlineto{\pgfqpoint{1.155168in}{3.444208in}}%
\pgfpathlineto{\pgfqpoint{1.161345in}{3.415075in}}%
\pgfpathlineto{\pgfqpoint{1.165463in}{3.308254in}}%
\pgfpathlineto{\pgfqpoint{1.167522in}{3.308254in}}%
\pgfpathlineto{\pgfqpoint{1.169581in}{3.317965in}}%
\pgfpathlineto{\pgfqpoint{1.175758in}{3.288832in}}%
\pgfpathlineto{\pgfqpoint{1.177817in}{3.230566in}}%
\pgfpathlineto{\pgfqpoint{1.179875in}{3.133457in}}%
\pgfpathlineto{\pgfqpoint{1.181934in}{3.162590in}}%
\pgfpathlineto{\pgfqpoint{1.183993in}{3.123746in}}%
\pgfpathlineto{\pgfqpoint{1.190170in}{3.133457in}}%
\pgfpathlineto{\pgfqpoint{1.192229in}{3.084902in}}%
\pgfpathlineto{\pgfqpoint{1.194288in}{3.123746in}}%
\pgfpathlineto{\pgfqpoint{1.196347in}{3.220855in}}%
\pgfpathlineto{\pgfqpoint{1.198406in}{3.152879in}}%
\pgfpathlineto{\pgfqpoint{1.204583in}{3.172301in}}%
\pgfpathlineto{\pgfqpoint{1.206642in}{3.162590in}}%
\pgfpathlineto{\pgfqpoint{1.208701in}{3.094613in}}%
\pgfpathlineto{\pgfqpoint{1.210760in}{3.152879in}}%
\pgfpathlineto{\pgfqpoint{1.212819in}{3.114035in}}%
\pgfpathlineto{\pgfqpoint{1.218995in}{3.191723in}}%
\pgfpathlineto{\pgfqpoint{1.221054in}{3.143168in}}%
\pgfpathlineto{\pgfqpoint{1.223113in}{3.152879in}}%
\pgfpathlineto{\pgfqpoint{1.225172in}{3.230566in}}%
\pgfpathlineto{\pgfqpoint{1.227231in}{3.201434in}}%
\pgfpathlineto{\pgfqpoint{1.233408in}{3.201434in}}%
\pgfpathlineto{\pgfqpoint{1.235467in}{3.162590in}}%
\pgfpathlineto{\pgfqpoint{1.239585in}{3.182012in}}%
\pgfpathlineto{\pgfqpoint{1.241644in}{3.133457in}}%
\pgfpathlineto{\pgfqpoint{1.249880in}{3.201434in}}%
\pgfpathlineto{\pgfqpoint{1.251939in}{3.279121in}}%
\pgfpathlineto{\pgfqpoint{1.253998in}{3.249988in}}%
\pgfpathlineto{\pgfqpoint{1.256056in}{3.240277in}}%
\pgfpathlineto{\pgfqpoint{1.262233in}{3.230566in}}%
\pgfpathlineto{\pgfqpoint{1.266351in}{3.376231in}}%
\pgfpathlineto{\pgfqpoint{1.268410in}{3.366520in}}%
\pgfpathlineto{\pgfqpoint{1.270469in}{3.434497in}}%
\pgfpathlineto{\pgfqpoint{1.278705in}{3.512185in}}%
\pgfpathlineto{\pgfqpoint{1.280764in}{3.589873in}}%
\pgfpathlineto{\pgfqpoint{1.282823in}{3.512185in}}%
\pgfpathlineto{\pgfqpoint{1.284882in}{3.512185in}}%
\pgfpathlineto{\pgfqpoint{1.291059in}{3.638428in}}%
\pgfpathlineto{\pgfqpoint{1.293117in}{3.628717in}}%
\pgfpathlineto{\pgfqpoint{1.295176in}{3.677272in}}%
\pgfpathlineto{\pgfqpoint{1.297235in}{3.638428in}}%
\pgfpathlineto{\pgfqpoint{1.299294in}{3.541318in}}%
\pgfpathlineto{\pgfqpoint{1.307530in}{3.657850in}}%
\pgfpathlineto{\pgfqpoint{1.309589in}{3.667561in}}%
\pgfpathlineto{\pgfqpoint{1.311648in}{3.589873in}}%
\pgfpathlineto{\pgfqpoint{1.313707in}{3.599584in}}%
\pgfpathlineto{\pgfqpoint{1.321943in}{3.502474in}}%
\pgfpathlineto{\pgfqpoint{1.324002in}{3.492763in}}%
\pgfpathlineto{\pgfqpoint{1.326061in}{3.502474in}}%
\pgfpathlineto{\pgfqpoint{1.328120in}{3.492763in}}%
\pgfpathlineto{\pgfqpoint{1.334296in}{3.551029in}}%
\pgfpathlineto{\pgfqpoint{1.338414in}{3.716116in}}%
\pgfpathlineto{\pgfqpoint{1.340473in}{3.638428in}}%
\pgfpathlineto{\pgfqpoint{1.342532in}{3.716116in}}%
\pgfpathlineto{\pgfqpoint{1.348709in}{3.716116in}}%
\pgfpathlineto{\pgfqpoint{1.350768in}{3.754960in}}%
\pgfpathlineto{\pgfqpoint{1.352827in}{3.822936in}}%
\pgfpathlineto{\pgfqpoint{1.354886in}{3.716116in}}%
\pgfpathlineto{\pgfqpoint{1.356945in}{3.706405in}}%
\pgfpathlineto{\pgfqpoint{1.363122in}{3.696694in}}%
\pgfpathlineto{\pgfqpoint{1.365181in}{3.667561in}}%
\pgfpathlineto{\pgfqpoint{1.367240in}{3.696694in}}%
\pgfpathlineto{\pgfqpoint{1.369298in}{3.745249in}}%
\pgfpathlineto{\pgfqpoint{1.371357in}{3.657850in}}%
\pgfpathlineto{\pgfqpoint{1.379593in}{3.803514in}}%
\pgfpathlineto{\pgfqpoint{1.381652in}{3.764671in}}%
\pgfpathlineto{\pgfqpoint{1.383711in}{3.764671in}}%
\pgfpathlineto{\pgfqpoint{1.385770in}{3.852069in}}%
\pgfpathlineto{\pgfqpoint{1.391947in}{3.696694in}}%
\pgfpathlineto{\pgfqpoint{1.394006in}{3.716116in}}%
\pgfpathlineto{\pgfqpoint{1.396065in}{3.803514in}}%
\pgfpathlineto{\pgfqpoint{1.398124in}{3.793803in}}%
\pgfpathlineto{\pgfqpoint{1.408418in}{3.648139in}}%
\pgfpathlineto{\pgfqpoint{1.410477in}{3.599584in}}%
\pgfpathlineto{\pgfqpoint{1.414595in}{3.803514in}}%
\pgfpathlineto{\pgfqpoint{1.420772in}{3.813225in}}%
\pgfpathlineto{\pgfqpoint{1.422831in}{3.803514in}}%
\pgfpathlineto{\pgfqpoint{1.424890in}{3.735538in}}%
\pgfpathlineto{\pgfqpoint{1.429008in}{3.686983in}}%
\pgfpathlineto{\pgfqpoint{1.435185in}{3.706405in}}%
\pgfpathlineto{\pgfqpoint{1.437244in}{3.686983in}}%
\pgfpathlineto{\pgfqpoint{1.443421in}{3.570451in}}%
\pgfpathlineto{\pgfqpoint{1.449597in}{3.541318in}}%
\pgfpathlineto{\pgfqpoint{1.453715in}{3.599584in}}%
\pgfpathlineto{\pgfqpoint{1.457833in}{3.531607in}}%
\pgfpathlineto{\pgfqpoint{1.464010in}{3.473341in}}%
\pgfpathlineto{\pgfqpoint{1.468128in}{3.551029in}}%
\pgfpathlineto{\pgfqpoint{1.470187in}{3.512185in}}%
\pgfpathlineto{\pgfqpoint{1.472246in}{3.444208in}}%
\pgfpathlineto{\pgfqpoint{1.478423in}{3.502474in}}%
\pgfpathlineto{\pgfqpoint{1.480482in}{3.424786in}}%
\pgfpathlineto{\pgfqpoint{1.484599in}{3.473341in}}%
\pgfpathlineto{\pgfqpoint{1.486658in}{3.453919in}}%
\pgfpathlineto{\pgfqpoint{1.492835in}{3.424786in}}%
\pgfpathlineto{\pgfqpoint{1.494894in}{3.483052in}}%
\pgfpathlineto{\pgfqpoint{1.499012in}{3.376231in}}%
\pgfpathlineto{\pgfqpoint{1.501071in}{3.356809in}}%
\pgfpathlineto{\pgfqpoint{1.507248in}{3.347098in}}%
\pgfpathlineto{\pgfqpoint{1.511366in}{3.551029in}}%
\pgfpathlineto{\pgfqpoint{1.515484in}{3.531607in}}%
\pgfpathlineto{\pgfqpoint{1.521660in}{3.560740in}}%
\pgfpathlineto{\pgfqpoint{1.523719in}{3.541318in}}%
\pgfpathlineto{\pgfqpoint{1.525778in}{3.580162in}}%
\pgfpathlineto{\pgfqpoint{1.527837in}{3.560740in}}%
\pgfpathlineto{\pgfqpoint{1.529896in}{3.502474in}}%
\pgfpathlineto{\pgfqpoint{1.538132in}{3.580162in}}%
\pgfpathlineto{\pgfqpoint{1.540191in}{3.570451in}}%
\pgfpathlineto{\pgfqpoint{1.542250in}{3.589873in}}%
\pgfpathlineto{\pgfqpoint{1.544309in}{3.560740in}}%
\pgfpathlineto{\pgfqpoint{1.550486in}{3.560740in}}%
\pgfpathlineto{\pgfqpoint{1.552545in}{3.667561in}}%
\pgfpathlineto{\pgfqpoint{1.554604in}{3.686983in}}%
\pgfpathlineto{\pgfqpoint{1.558721in}{3.541318in}}%
\pgfpathlineto{\pgfqpoint{1.564898in}{3.628717in}}%
\pgfpathlineto{\pgfqpoint{1.566957in}{3.551029in}}%
\pgfpathlineto{\pgfqpoint{1.569016in}{3.560740in}}%
\pgfpathlineto{\pgfqpoint{1.571075in}{3.521896in}}%
\pgfpathlineto{\pgfqpoint{1.573134in}{3.570451in}}%
\pgfpathlineto{\pgfqpoint{1.581370in}{3.463630in}}%
\pgfpathlineto{\pgfqpoint{1.583429in}{3.483052in}}%
\pgfpathlineto{\pgfqpoint{1.587547in}{3.434497in}}%
\pgfpathlineto{\pgfqpoint{1.593724in}{3.512185in}}%
\pgfpathlineto{\pgfqpoint{1.595782in}{3.492763in}}%
\pgfpathlineto{\pgfqpoint{1.597841in}{3.502474in}}%
\pgfpathlineto{\pgfqpoint{1.599900in}{3.570451in}}%
\pgfpathlineto{\pgfqpoint{1.601959in}{3.551029in}}%
\pgfpathlineto{\pgfqpoint{1.610195in}{3.502474in}}%
\pgfpathlineto{\pgfqpoint{1.612254in}{3.453919in}}%
\pgfpathlineto{\pgfqpoint{1.614313in}{3.483052in}}%
\pgfpathlineto{\pgfqpoint{1.616372in}{3.483052in}}%
\pgfpathlineto{\pgfqpoint{1.622549in}{3.502474in}}%
\pgfpathlineto{\pgfqpoint{1.624608in}{3.531607in}}%
\pgfpathlineto{\pgfqpoint{1.626667in}{3.483052in}}%
\pgfpathlineto{\pgfqpoint{1.628726in}{3.483052in}}%
\pgfpathlineto{\pgfqpoint{1.630785in}{3.512185in}}%
\pgfpathlineto{\pgfqpoint{1.639020in}{3.473341in}}%
\pgfpathlineto{\pgfqpoint{1.641079in}{3.483052in}}%
\pgfpathlineto{\pgfqpoint{1.643138in}{3.570451in}}%
\pgfpathlineto{\pgfqpoint{1.645197in}{3.541318in}}%
\pgfpathlineto{\pgfqpoint{1.651374in}{3.560740in}}%
\pgfpathlineto{\pgfqpoint{1.653433in}{3.609295in}}%
\pgfpathlineto{\pgfqpoint{1.655492in}{3.609295in}}%
\pgfpathlineto{\pgfqpoint{1.657551in}{3.619006in}}%
\pgfpathlineto{\pgfqpoint{1.659610in}{3.696694in}}%
\pgfpathlineto{\pgfqpoint{1.665787in}{3.725827in}}%
\pgfpathlineto{\pgfqpoint{1.667846in}{3.706405in}}%
\pgfpathlineto{\pgfqpoint{1.671963in}{3.696694in}}%
\pgfpathlineto{\pgfqpoint{1.674022in}{3.667561in}}%
\pgfpathlineto{\pgfqpoint{1.680199in}{3.677272in}}%
\pgfpathlineto{\pgfqpoint{1.682258in}{3.648139in}}%
\pgfpathlineto{\pgfqpoint{1.684317in}{3.648139in}}%
\pgfpathlineto{\pgfqpoint{1.686376in}{3.609295in}}%
\pgfpathlineto{\pgfqpoint{1.688435in}{3.628717in}}%
\pgfpathlineto{\pgfqpoint{1.694612in}{3.609295in}}%
\pgfpathlineto{\pgfqpoint{1.702848in}{3.609295in}}%
\pgfpathlineto{\pgfqpoint{1.709024in}{3.589873in}}%
\pgfpathlineto{\pgfqpoint{1.711083in}{3.521896in}}%
\pgfpathlineto{\pgfqpoint{1.713142in}{3.521896in}}%
\pgfpathlineto{\pgfqpoint{1.715201in}{3.677272in}}%
\pgfpathlineto{\pgfqpoint{1.717260in}{3.619006in}}%
\pgfpathlineto{\pgfqpoint{1.723437in}{3.560740in}}%
\pgfpathlineto{\pgfqpoint{1.725496in}{3.580162in}}%
\pgfpathlineto{\pgfqpoint{1.727555in}{3.580162in}}%
\pgfpathlineto{\pgfqpoint{1.729614in}{3.589873in}}%
\pgfpathlineto{\pgfqpoint{1.731673in}{3.483052in}}%
\pgfpathlineto{\pgfqpoint{1.741968in}{3.628717in}}%
\pgfpathlineto{\pgfqpoint{1.746086in}{3.512185in}}%
\pgfpathlineto{\pgfqpoint{1.752262in}{3.531607in}}%
\pgfpathlineto{\pgfqpoint{1.756380in}{3.609295in}}%
\pgfpathlineto{\pgfqpoint{1.758439in}{3.570451in}}%
\pgfpathlineto{\pgfqpoint{1.766675in}{3.560740in}}%
\pgfpathlineto{\pgfqpoint{1.768734in}{3.648139in}}%
\pgfpathlineto{\pgfqpoint{1.770793in}{3.648139in}}%
\pgfpathlineto{\pgfqpoint{1.772852in}{3.619006in}}%
\pgfpathlineto{\pgfqpoint{1.781088in}{3.589873in}}%
\pgfpathlineto{\pgfqpoint{1.783147in}{3.619006in}}%
\pgfpathlineto{\pgfqpoint{1.785205in}{3.551029in}}%
\pgfpathlineto{\pgfqpoint{1.789323in}{3.521896in}}%
\pgfpathlineto{\pgfqpoint{1.795500in}{3.570451in}}%
\pgfpathlineto{\pgfqpoint{1.799618in}{3.463630in}}%
\pgfpathlineto{\pgfqpoint{1.801677in}{3.512185in}}%
\pgfpathlineto{\pgfqpoint{1.803736in}{3.424786in}}%
\pgfpathlineto{\pgfqpoint{1.811972in}{3.434497in}}%
\pgfpathlineto{\pgfqpoint{1.814031in}{3.385942in}}%
\pgfpathlineto{\pgfqpoint{1.816090in}{3.405364in}}%
\pgfpathlineto{\pgfqpoint{1.818149in}{3.444208in}}%
\pgfpathlineto{\pgfqpoint{1.826384in}{3.405364in}}%
\pgfpathlineto{\pgfqpoint{1.828443in}{3.415075in}}%
\pgfpathlineto{\pgfqpoint{1.830502in}{3.405364in}}%
\pgfpathlineto{\pgfqpoint{1.832561in}{3.366520in}}%
\pgfpathlineto{\pgfqpoint{1.838738in}{3.385942in}}%
\pgfpathlineto{\pgfqpoint{1.840797in}{3.288832in}}%
\pgfpathlineto{\pgfqpoint{1.842856in}{3.317965in}}%
\pgfpathlineto{\pgfqpoint{1.844915in}{3.317965in}}%
\pgfpathlineto{\pgfqpoint{1.846974in}{3.298543in}}%
\pgfpathlineto{\pgfqpoint{1.853151in}{3.182012in}}%
\pgfpathlineto{\pgfqpoint{1.855210in}{3.172301in}}%
\pgfpathlineto{\pgfqpoint{1.859328in}{3.123746in}}%
\pgfpathlineto{\pgfqpoint{1.861386in}{3.220855in}}%
\pgfpathlineto{\pgfqpoint{1.869622in}{3.259699in}}%
\pgfpathlineto{\pgfqpoint{1.871681in}{3.298543in}}%
\pgfpathlineto{\pgfqpoint{1.873740in}{3.240277in}}%
\pgfpathlineto{\pgfqpoint{1.875799in}{3.230566in}}%
\pgfpathlineto{\pgfqpoint{1.881976in}{3.240277in}}%
\pgfpathlineto{\pgfqpoint{1.884035in}{3.220855in}}%
\pgfpathlineto{\pgfqpoint{1.886094in}{3.230566in}}%
\pgfpathlineto{\pgfqpoint{1.888153in}{3.201434in}}%
\pgfpathlineto{\pgfqpoint{1.890212in}{3.249988in}}%
\pgfpathlineto{\pgfqpoint{1.896389in}{3.230566in}}%
\pgfpathlineto{\pgfqpoint{1.898447in}{3.317965in}}%
\pgfpathlineto{\pgfqpoint{1.900506in}{3.308254in}}%
\pgfpathlineto{\pgfqpoint{1.902565in}{3.269410in}}%
\pgfpathlineto{\pgfqpoint{1.904624in}{3.317965in}}%
\pgfpathlineto{\pgfqpoint{1.910801in}{3.327676in}}%
\pgfpathlineto{\pgfqpoint{1.912860in}{3.249988in}}%
\pgfpathlineto{\pgfqpoint{1.914919in}{3.298543in}}%
\pgfpathlineto{\pgfqpoint{1.916978in}{3.317965in}}%
\pgfpathlineto{\pgfqpoint{1.919037in}{3.366520in}}%
\pgfpathlineto{\pgfqpoint{1.925214in}{3.356809in}}%
\pgfpathlineto{\pgfqpoint{1.927273in}{3.347098in}}%
\pgfpathlineto{\pgfqpoint{1.929332in}{3.347098in}}%
\pgfpathlineto{\pgfqpoint{1.931391in}{3.308254in}}%
\pgfpathlineto{\pgfqpoint{1.933450in}{3.298543in}}%
\pgfpathlineto{\pgfqpoint{1.939626in}{3.337387in}}%
\pgfpathlineto{\pgfqpoint{1.941685in}{3.337387in}}%
\pgfpathlineto{\pgfqpoint{1.943744in}{3.269410in}}%
\pgfpathlineto{\pgfqpoint{1.945803in}{3.288832in}}%
\pgfpathlineto{\pgfqpoint{1.954039in}{3.279121in}}%
\pgfpathlineto{\pgfqpoint{1.956098in}{3.220855in}}%
\pgfpathlineto{\pgfqpoint{1.958157in}{3.269410in}}%
\pgfpathlineto{\pgfqpoint{1.960216in}{3.230566in}}%
\pgfpathlineto{\pgfqpoint{1.962275in}{3.240277in}}%
\pgfpathlineto{\pgfqpoint{1.968452in}{3.220855in}}%
\pgfpathlineto{\pgfqpoint{1.970511in}{3.162590in}}%
\pgfpathlineto{\pgfqpoint{1.972570in}{3.201434in}}%
\pgfpathlineto{\pgfqpoint{1.974628in}{3.143168in}}%
\pgfpathlineto{\pgfqpoint{1.976687in}{3.172301in}}%
\pgfpathlineto{\pgfqpoint{1.982864in}{3.182012in}}%
\pgfpathlineto{\pgfqpoint{1.984923in}{3.230566in}}%
\pgfpathlineto{\pgfqpoint{1.986982in}{3.201434in}}%
\pgfpathlineto{\pgfqpoint{1.989041in}{3.230566in}}%
\pgfpathlineto{\pgfqpoint{1.991100in}{3.182012in}}%
\pgfpathlineto{\pgfqpoint{1.997277in}{3.201434in}}%
\pgfpathlineto{\pgfqpoint{1.999336in}{3.220855in}}%
\pgfpathlineto{\pgfqpoint{2.003454in}{3.308254in}}%
\pgfpathlineto{\pgfqpoint{2.005513in}{3.317965in}}%
\pgfpathlineto{\pgfqpoint{2.011689in}{3.337387in}}%
\pgfpathlineto{\pgfqpoint{2.013748in}{3.376231in}}%
\pgfpathlineto{\pgfqpoint{2.017866in}{3.298543in}}%
\pgfpathlineto{\pgfqpoint{2.019925in}{3.279121in}}%
\pgfpathlineto{\pgfqpoint{2.026102in}{3.327676in}}%
\pgfpathlineto{\pgfqpoint{2.028161in}{3.279121in}}%
\pgfpathlineto{\pgfqpoint{2.030220in}{3.259699in}}%
\pgfpathlineto{\pgfqpoint{2.032279in}{3.220855in}}%
\pgfpathlineto{\pgfqpoint{2.034338in}{3.240277in}}%
\pgfpathlineto{\pgfqpoint{2.040515in}{3.230566in}}%
\pgfpathlineto{\pgfqpoint{2.042574in}{3.230566in}}%
\pgfpathlineto{\pgfqpoint{2.044633in}{3.201434in}}%
\pgfpathlineto{\pgfqpoint{2.046692in}{3.220855in}}%
\pgfpathlineto{\pgfqpoint{2.048751in}{3.172301in}}%
\pgfpathlineto{\pgfqpoint{2.054927in}{3.211145in}}%
\pgfpathlineto{\pgfqpoint{2.056986in}{3.211145in}}%
\pgfpathlineto{\pgfqpoint{2.059045in}{3.288832in}}%
\pgfpathlineto{\pgfqpoint{2.061104in}{3.259699in}}%
\pgfpathlineto{\pgfqpoint{2.063163in}{3.249988in}}%
\pgfpathlineto{\pgfqpoint{2.069340in}{3.249988in}}%
\pgfpathlineto{\pgfqpoint{2.073458in}{3.288832in}}%
\pgfpathlineto{\pgfqpoint{2.075517in}{3.259699in}}%
\pgfpathlineto{\pgfqpoint{2.077576in}{3.269410in}}%
\pgfpathlineto{\pgfqpoint{2.085812in}{3.259699in}}%
\pgfpathlineto{\pgfqpoint{2.087870in}{3.249988in}}%
\pgfpathlineto{\pgfqpoint{2.091988in}{3.143168in}}%
\pgfpathlineto{\pgfqpoint{2.098165in}{3.172301in}}%
\pgfpathlineto{\pgfqpoint{2.100224in}{3.162590in}}%
\pgfpathlineto{\pgfqpoint{2.106401in}{3.065480in}}%
\pgfpathlineto{\pgfqpoint{2.112578in}{3.055769in}}%
\pgfpathlineto{\pgfqpoint{2.116696in}{3.055769in}}%
\pgfpathlineto{\pgfqpoint{2.118755in}{3.016925in}}%
\pgfpathlineto{\pgfqpoint{2.120814in}{3.055769in}}%
\pgfpathlineto{\pgfqpoint{2.126990in}{3.094613in}}%
\pgfpathlineto{\pgfqpoint{2.129049in}{3.123746in}}%
\pgfpathlineto{\pgfqpoint{2.131108in}{3.123746in}}%
\pgfpathlineto{\pgfqpoint{2.133167in}{3.172301in}}%
\pgfpathlineto{\pgfqpoint{2.135226in}{3.046058in}}%
\pgfpathlineto{\pgfqpoint{2.141403in}{2.910104in}}%
\pgfpathlineto{\pgfqpoint{2.143462in}{2.900393in}}%
\pgfpathlineto{\pgfqpoint{2.145521in}{2.929526in}}%
\pgfpathlineto{\pgfqpoint{2.147580in}{2.929526in}}%
\pgfpathlineto{\pgfqpoint{2.149639in}{2.871260in}}%
\pgfpathlineto{\pgfqpoint{2.157875in}{2.774150in}}%
\pgfpathlineto{\pgfqpoint{2.161993in}{2.774150in}}%
\pgfpathlineto{\pgfqpoint{2.164051in}{2.745017in}}%
\pgfpathlineto{\pgfqpoint{2.170228in}{2.774150in}}%
\pgfpathlineto{\pgfqpoint{2.172287in}{2.871260in}}%
\pgfpathlineto{\pgfqpoint{2.174346in}{2.812994in}}%
\pgfpathlineto{\pgfqpoint{2.178464in}{2.929526in}}%
\pgfpathlineto{\pgfqpoint{2.184641in}{2.929526in}}%
\pgfpathlineto{\pgfqpoint{2.186700in}{2.900393in}}%
\pgfpathlineto{\pgfqpoint{2.188759in}{2.929526in}}%
\pgfpathlineto{\pgfqpoint{2.190818in}{2.919815in}}%
\pgfpathlineto{\pgfqpoint{2.199054in}{2.919815in}}%
\pgfpathlineto{\pgfqpoint{2.201112in}{2.910104in}}%
\pgfpathlineto{\pgfqpoint{2.203171in}{2.861549in}}%
\pgfpathlineto{\pgfqpoint{2.205230in}{2.861549in}}%
\pgfpathlineto{\pgfqpoint{2.207289in}{2.812994in}}%
\pgfpathlineto{\pgfqpoint{2.213466in}{2.871260in}}%
\pgfpathlineto{\pgfqpoint{2.215525in}{2.919815in}}%
\pgfpathlineto{\pgfqpoint{2.217584in}{2.919815in}}%
\pgfpathlineto{\pgfqpoint{2.219643in}{2.880971in}}%
\pgfpathlineto{\pgfqpoint{2.221702in}{2.948948in}}%
\pgfpathlineto{\pgfqpoint{2.227879in}{2.929526in}}%
\pgfpathlineto{\pgfqpoint{2.229938in}{2.880971in}}%
\pgfpathlineto{\pgfqpoint{2.231997in}{2.861549in}}%
\pgfpathlineto{\pgfqpoint{2.234056in}{2.910104in}}%
\pgfpathlineto{\pgfqpoint{2.236115in}{2.861549in}}%
\pgfpathlineto{\pgfqpoint{2.244350in}{2.919815in}}%
\pgfpathlineto{\pgfqpoint{2.248468in}{2.890682in}}%
\pgfpathlineto{\pgfqpoint{2.250527in}{2.919815in}}%
\pgfpathlineto{\pgfqpoint{2.256704in}{2.871260in}}%
\pgfpathlineto{\pgfqpoint{2.260822in}{2.871260in}}%
\pgfpathlineto{\pgfqpoint{2.264940in}{2.919815in}}%
\pgfpathlineto{\pgfqpoint{2.271117in}{2.851838in}}%
\pgfpathlineto{\pgfqpoint{2.273176in}{2.861549in}}%
\pgfpathlineto{\pgfqpoint{2.277293in}{2.861549in}}%
\pgfpathlineto{\pgfqpoint{2.279352in}{2.910104in}}%
\pgfpathlineto{\pgfqpoint{2.289647in}{2.861549in}}%
\pgfpathlineto{\pgfqpoint{2.293765in}{3.016925in}}%
\pgfpathlineto{\pgfqpoint{2.299942in}{3.026636in}}%
\pgfpathlineto{\pgfqpoint{2.302001in}{3.094613in}}%
\pgfpathlineto{\pgfqpoint{2.304060in}{3.065480in}}%
\pgfpathlineto{\pgfqpoint{2.306119in}{3.104324in}}%
\pgfpathlineto{\pgfqpoint{2.308178in}{3.065480in}}%
\pgfpathlineto{\pgfqpoint{2.314355in}{3.075191in}}%
\pgfpathlineto{\pgfqpoint{2.316413in}{3.055769in}}%
\pgfpathlineto{\pgfqpoint{2.320531in}{2.968370in}}%
\pgfpathlineto{\pgfqpoint{2.322590in}{2.968370in}}%
\pgfpathlineto{\pgfqpoint{2.328767in}{2.948948in}}%
\pgfpathlineto{\pgfqpoint{2.330826in}{2.910104in}}%
\pgfpathlineto{\pgfqpoint{2.332885in}{2.919815in}}%
\pgfpathlineto{\pgfqpoint{2.334944in}{2.910104in}}%
\pgfpathlineto{\pgfqpoint{2.337003in}{2.948948in}}%
\pgfpathlineto{\pgfqpoint{2.343180in}{2.968370in}}%
\pgfpathlineto{\pgfqpoint{2.347298in}{3.065480in}}%
\pgfpathlineto{\pgfqpoint{2.349357in}{3.084902in}}%
\pgfpathlineto{\pgfqpoint{2.351416in}{3.084902in}}%
\pgfpathlineto{\pgfqpoint{2.361710in}{3.133457in}}%
\pgfpathlineto{\pgfqpoint{2.363769in}{3.104324in}}%
\pgfpathlineto{\pgfqpoint{2.365828in}{3.172301in}}%
\pgfpathlineto{\pgfqpoint{2.374064in}{3.133457in}}%
\pgfpathlineto{\pgfqpoint{2.376123in}{3.133457in}}%
\pgfpathlineto{\pgfqpoint{2.378182in}{3.123746in}}%
\pgfpathlineto{\pgfqpoint{2.380241in}{3.104324in}}%
\pgfpathlineto{\pgfqpoint{2.386418in}{3.143168in}}%
\pgfpathlineto{\pgfqpoint{2.388477in}{3.123746in}}%
\pgfpathlineto{\pgfqpoint{2.390535in}{3.152879in}}%
\pgfpathlineto{\pgfqpoint{2.392594in}{3.220855in}}%
\pgfpathlineto{\pgfqpoint{2.394653in}{3.240277in}}%
\pgfpathlineto{\pgfqpoint{2.400830in}{3.201434in}}%
\pgfpathlineto{\pgfqpoint{2.402889in}{3.201434in}}%
\pgfpathlineto{\pgfqpoint{2.404948in}{3.182012in}}%
\pgfpathlineto{\pgfqpoint{2.407007in}{3.220855in}}%
\pgfpathlineto{\pgfqpoint{2.409066in}{3.182012in}}%
\pgfpathlineto{\pgfqpoint{2.415243in}{3.220855in}}%
\pgfpathlineto{\pgfqpoint{2.417302in}{3.249988in}}%
\pgfpathlineto{\pgfqpoint{2.419361in}{3.492763in}}%
\pgfpathlineto{\pgfqpoint{2.421420in}{3.551029in}}%
\pgfpathlineto{\pgfqpoint{2.429655in}{3.599584in}}%
\pgfpathlineto{\pgfqpoint{2.431714in}{3.580162in}}%
\pgfpathlineto{\pgfqpoint{2.433773in}{3.531607in}}%
\pgfpathlineto{\pgfqpoint{2.435832in}{3.619006in}}%
\pgfpathlineto{\pgfqpoint{2.437891in}{3.619006in}}%
\pgfpathlineto{\pgfqpoint{2.444068in}{3.609295in}}%
\pgfpathlineto{\pgfqpoint{2.446127in}{3.609295in}}%
\pgfpathlineto{\pgfqpoint{2.448186in}{3.628717in}}%
\pgfpathlineto{\pgfqpoint{2.452304in}{3.619006in}}%
\pgfpathlineto{\pgfqpoint{2.458481in}{3.599584in}}%
\pgfpathlineto{\pgfqpoint{2.460540in}{3.560740in}}%
\pgfpathlineto{\pgfqpoint{2.464658in}{3.706405in}}%
\pgfpathlineto{\pgfqpoint{2.466716in}{3.686983in}}%
\pgfpathlineto{\pgfqpoint{2.472893in}{3.657850in}}%
\pgfpathlineto{\pgfqpoint{2.474952in}{3.686983in}}%
\pgfpathlineto{\pgfqpoint{2.477011in}{3.628717in}}%
\pgfpathlineto{\pgfqpoint{2.481129in}{3.764671in}}%
\pgfpathlineto{\pgfqpoint{2.487306in}{3.764671in}}%
\pgfpathlineto{\pgfqpoint{2.489365in}{3.745249in}}%
\pgfpathlineto{\pgfqpoint{2.491424in}{3.745249in}}%
\pgfpathlineto{\pgfqpoint{2.495542in}{3.793803in}}%
\pgfpathlineto{\pgfqpoint{2.501719in}{3.725827in}}%
\pgfpathlineto{\pgfqpoint{2.503778in}{3.754960in}}%
\pgfpathlineto{\pgfqpoint{2.505836in}{3.725827in}}%
\pgfpathlineto{\pgfqpoint{2.509954in}{3.725827in}}%
\pgfpathlineto{\pgfqpoint{2.518190in}{3.745249in}}%
\pgfpathlineto{\pgfqpoint{2.520249in}{3.696694in}}%
\pgfpathlineto{\pgfqpoint{2.522308in}{3.686983in}}%
\pgfpathlineto{\pgfqpoint{2.524367in}{3.667561in}}%
\pgfpathlineto{\pgfqpoint{2.532603in}{3.648139in}}%
\pgfpathlineto{\pgfqpoint{2.534662in}{3.657850in}}%
\pgfpathlineto{\pgfqpoint{2.536721in}{3.570451in}}%
\pgfpathlineto{\pgfqpoint{2.538780in}{3.609295in}}%
\pgfpathlineto{\pgfqpoint{2.544956in}{3.580162in}}%
\pgfpathlineto{\pgfqpoint{2.547015in}{3.580162in}}%
\pgfpathlineto{\pgfqpoint{2.549074in}{3.570451in}}%
\pgfpathlineto{\pgfqpoint{2.551133in}{3.619006in}}%
\pgfpathlineto{\pgfqpoint{2.553192in}{3.599584in}}%
\pgfpathlineto{\pgfqpoint{2.561428in}{3.541318in}}%
\pgfpathlineto{\pgfqpoint{2.565546in}{3.648139in}}%
\pgfpathlineto{\pgfqpoint{2.567605in}{3.657850in}}%
\pgfpathlineto{\pgfqpoint{2.573782in}{3.599584in}}%
\pgfpathlineto{\pgfqpoint{2.577900in}{3.706405in}}%
\pgfpathlineto{\pgfqpoint{2.582017in}{3.667561in}}%
\pgfpathlineto{\pgfqpoint{2.588194in}{3.686983in}}%
\pgfpathlineto{\pgfqpoint{2.590253in}{3.657850in}}%
\pgfpathlineto{\pgfqpoint{2.592312in}{3.686983in}}%
\pgfpathlineto{\pgfqpoint{2.594371in}{3.696694in}}%
\pgfpathlineto{\pgfqpoint{2.596430in}{3.716116in}}%
\pgfpathlineto{\pgfqpoint{2.604666in}{3.628717in}}%
\pgfpathlineto{\pgfqpoint{2.606725in}{3.570451in}}%
\pgfpathlineto{\pgfqpoint{2.608784in}{3.628717in}}%
\pgfpathlineto{\pgfqpoint{2.610843in}{3.619006in}}%
\pgfpathlineto{\pgfqpoint{2.617020in}{3.638428in}}%
\pgfpathlineto{\pgfqpoint{2.621137in}{3.696694in}}%
\pgfpathlineto{\pgfqpoint{2.625255in}{3.638428in}}%
\pgfpathlineto{\pgfqpoint{2.633491in}{3.648139in}}%
\pgfpathlineto{\pgfqpoint{2.635550in}{3.648139in}}%
\pgfpathlineto{\pgfqpoint{2.637609in}{3.628717in}}%
\pgfpathlineto{\pgfqpoint{2.639668in}{3.560740in}}%
\pgfpathlineto{\pgfqpoint{2.645845in}{3.589873in}}%
\pgfpathlineto{\pgfqpoint{2.647904in}{3.580162in}}%
\pgfpathlineto{\pgfqpoint{2.649963in}{3.667561in}}%
\pgfpathlineto{\pgfqpoint{2.652022in}{3.696694in}}%
\pgfpathlineto{\pgfqpoint{2.654081in}{3.686983in}}%
\pgfpathlineto{\pgfqpoint{2.662316in}{3.716116in}}%
\pgfpathlineto{\pgfqpoint{2.666434in}{3.793803in}}%
\pgfpathlineto{\pgfqpoint{2.668493in}{3.764671in}}%
\pgfpathlineto{\pgfqpoint{2.674670in}{3.803514in}}%
\pgfpathlineto{\pgfqpoint{2.676729in}{3.774382in}}%
\pgfpathlineto{\pgfqpoint{2.678788in}{3.716116in}}%
\pgfpathlineto{\pgfqpoint{2.680847in}{3.745249in}}%
\pgfpathlineto{\pgfqpoint{2.682906in}{3.716116in}}%
\pgfpathlineto{\pgfqpoint{2.689083in}{3.686983in}}%
\pgfpathlineto{\pgfqpoint{2.693200in}{3.628717in}}%
\pgfpathlineto{\pgfqpoint{2.695259in}{3.628717in}}%
\pgfpathlineto{\pgfqpoint{2.697318in}{3.609295in}}%
\pgfpathlineto{\pgfqpoint{2.703495in}{3.589873in}}%
\pgfpathlineto{\pgfqpoint{2.705554in}{3.628717in}}%
\pgfpathlineto{\pgfqpoint{2.707613in}{3.599584in}}%
\pgfpathlineto{\pgfqpoint{2.709672in}{3.638428in}}%
\pgfpathlineto{\pgfqpoint{2.711731in}{3.628717in}}%
\pgfpathlineto{\pgfqpoint{2.717908in}{3.589873in}}%
\pgfpathlineto{\pgfqpoint{2.719967in}{3.599584in}}%
\pgfpathlineto{\pgfqpoint{2.722026in}{3.589873in}}%
\pgfpathlineto{\pgfqpoint{2.726144in}{3.609295in}}%
\pgfpathlineto{\pgfqpoint{2.732320in}{3.599584in}}%
\pgfpathlineto{\pgfqpoint{2.734379in}{3.541318in}}%
\pgfpathlineto{\pgfqpoint{2.736438in}{3.531607in}}%
\pgfpathlineto{\pgfqpoint{2.738497in}{3.502474in}}%
\pgfpathlineto{\pgfqpoint{2.746733in}{3.531607in}}%
\pgfpathlineto{\pgfqpoint{2.748792in}{3.453919in}}%
\pgfpathlineto{\pgfqpoint{2.752910in}{3.502474in}}%
\pgfpathlineto{\pgfqpoint{2.754969in}{3.502474in}}%
\pgfpathlineto{\pgfqpoint{2.761146in}{3.541318in}}%
\pgfpathlineto{\pgfqpoint{2.763205in}{3.599584in}}%
\pgfpathlineto{\pgfqpoint{2.767323in}{3.570451in}}%
\pgfpathlineto{\pgfqpoint{2.769381in}{3.570451in}}%
\pgfpathlineto{\pgfqpoint{2.775558in}{3.609295in}}%
\pgfpathlineto{\pgfqpoint{2.777617in}{3.580162in}}%
\pgfpathlineto{\pgfqpoint{2.779676in}{3.580162in}}%
\pgfpathlineto{\pgfqpoint{2.781735in}{3.609295in}}%
\pgfpathlineto{\pgfqpoint{2.783794in}{3.599584in}}%
\pgfpathlineto{\pgfqpoint{2.789971in}{3.628717in}}%
\pgfpathlineto{\pgfqpoint{2.792030in}{3.648139in}}%
\pgfpathlineto{\pgfqpoint{2.794089in}{3.638428in}}%
\pgfpathlineto{\pgfqpoint{2.796148in}{3.638428in}}%
\pgfpathlineto{\pgfqpoint{2.798207in}{3.589873in}}%
\pgfpathlineto{\pgfqpoint{2.804384in}{3.609295in}}%
\pgfpathlineto{\pgfqpoint{2.806443in}{3.599584in}}%
\pgfpathlineto{\pgfqpoint{2.808501in}{3.521896in}}%
\pgfpathlineto{\pgfqpoint{2.810560in}{3.512185in}}%
\pgfpathlineto{\pgfqpoint{2.812619in}{3.512185in}}%
\pgfpathlineto{\pgfqpoint{2.818796in}{3.521896in}}%
\pgfpathlineto{\pgfqpoint{2.820855in}{3.560740in}}%
\pgfpathlineto{\pgfqpoint{2.822914in}{3.531607in}}%
\pgfpathlineto{\pgfqpoint{2.827032in}{3.531607in}}%
\pgfpathlineto{\pgfqpoint{2.837327in}{3.483052in}}%
\pgfpathlineto{\pgfqpoint{2.839386in}{3.483052in}}%
\pgfpathlineto{\pgfqpoint{2.841445in}{3.415075in}}%
\pgfpathlineto{\pgfqpoint{2.847621in}{3.453919in}}%
\pgfpathlineto{\pgfqpoint{2.849680in}{3.424786in}}%
\pgfpathlineto{\pgfqpoint{2.851739in}{3.453919in}}%
\pgfpathlineto{\pgfqpoint{2.855857in}{3.473341in}}%
\pgfpathlineto{\pgfqpoint{2.862034in}{3.473341in}}%
\pgfpathlineto{\pgfqpoint{2.864093in}{3.483052in}}%
\pgfpathlineto{\pgfqpoint{2.866152in}{3.405364in}}%
\pgfpathlineto{\pgfqpoint{2.868211in}{3.395653in}}%
\pgfpathlineto{\pgfqpoint{2.870270in}{3.395653in}}%
\pgfpathlineto{\pgfqpoint{2.876447in}{3.405364in}}%
\pgfpathlineto{\pgfqpoint{2.878506in}{3.356809in}}%
\pgfpathlineto{\pgfqpoint{2.884682in}{3.327676in}}%
\pgfpathlineto{\pgfqpoint{2.890859in}{3.317965in}}%
\pgfpathlineto{\pgfqpoint{2.892918in}{3.366520in}}%
\pgfpathlineto{\pgfqpoint{2.894977in}{3.385942in}}%
\pgfpathlineto{\pgfqpoint{2.897036in}{3.434497in}}%
\pgfpathlineto{\pgfqpoint{2.899095in}{3.453919in}}%
\pgfpathlineto{\pgfqpoint{2.905272in}{3.473341in}}%
\pgfpathlineto{\pgfqpoint{2.909390in}{3.463630in}}%
\pgfpathlineto{\pgfqpoint{2.913508in}{3.541318in}}%
\pgfpathlineto{\pgfqpoint{2.919685in}{3.541318in}}%
\pgfpathlineto{\pgfqpoint{2.921743in}{3.531607in}}%
\pgfpathlineto{\pgfqpoint{2.923802in}{3.502474in}}%
\pgfpathlineto{\pgfqpoint{2.925861in}{3.531607in}}%
\pgfpathlineto{\pgfqpoint{2.927920in}{3.521896in}}%
\pgfpathlineto{\pgfqpoint{2.934097in}{3.502474in}}%
\pgfpathlineto{\pgfqpoint{2.936156in}{3.463630in}}%
\pgfpathlineto{\pgfqpoint{2.938215in}{3.463630in}}%
\pgfpathlineto{\pgfqpoint{2.942333in}{3.424786in}}%
\pgfpathlineto{\pgfqpoint{2.948510in}{3.444208in}}%
\pgfpathlineto{\pgfqpoint{2.950569in}{3.521896in}}%
\pgfpathlineto{\pgfqpoint{2.952628in}{3.502474in}}%
\pgfpathlineto{\pgfqpoint{2.954687in}{3.541318in}}%
\pgfpathlineto{\pgfqpoint{2.956746in}{3.502474in}}%
\pgfpathlineto{\pgfqpoint{2.962922in}{3.502474in}}%
\pgfpathlineto{\pgfqpoint{2.964981in}{3.473341in}}%
\pgfpathlineto{\pgfqpoint{2.967040in}{3.463630in}}%
\pgfpathlineto{\pgfqpoint{2.969099in}{3.424786in}}%
\pgfpathlineto{\pgfqpoint{2.971158in}{3.453919in}}%
\pgfpathlineto{\pgfqpoint{2.977335in}{3.453919in}}%
\pgfpathlineto{\pgfqpoint{2.979394in}{3.473341in}}%
\pgfpathlineto{\pgfqpoint{2.983512in}{3.405364in}}%
\pgfpathlineto{\pgfqpoint{2.985571in}{3.405364in}}%
\pgfpathlineto{\pgfqpoint{2.991748in}{3.424786in}}%
\pgfpathlineto{\pgfqpoint{2.993807in}{3.453919in}}%
\pgfpathlineto{\pgfqpoint{2.997924in}{3.395653in}}%
\pgfpathlineto{\pgfqpoint{2.999983in}{3.395653in}}%
\pgfpathlineto{\pgfqpoint{3.006160in}{3.385942in}}%
\pgfpathlineto{\pgfqpoint{3.008219in}{3.405364in}}%
\pgfpathlineto{\pgfqpoint{3.010278in}{3.366520in}}%
\pgfpathlineto{\pgfqpoint{3.012337in}{3.385942in}}%
\pgfpathlineto{\pgfqpoint{3.014396in}{3.366520in}}%
\pgfpathlineto{\pgfqpoint{3.020573in}{3.376231in}}%
\pgfpathlineto{\pgfqpoint{3.022632in}{3.356809in}}%
\pgfpathlineto{\pgfqpoint{3.024691in}{3.366520in}}%
\pgfpathlineto{\pgfqpoint{3.026750in}{3.347098in}}%
\pgfpathlineto{\pgfqpoint{3.028809in}{3.385942in}}%
\pgfpathlineto{\pgfqpoint{3.037044in}{3.308254in}}%
\pgfpathlineto{\pgfqpoint{3.039103in}{3.337387in}}%
\pgfpathlineto{\pgfqpoint{3.041162in}{3.279121in}}%
\pgfpathlineto{\pgfqpoint{3.043221in}{3.288832in}}%
\pgfpathlineto{\pgfqpoint{3.053516in}{3.405364in}}%
\pgfpathlineto{\pgfqpoint{3.055575in}{3.385942in}}%
\pgfpathlineto{\pgfqpoint{3.057634in}{3.385942in}}%
\pgfpathlineto{\pgfqpoint{3.067929in}{3.434497in}}%
\pgfpathlineto{\pgfqpoint{3.069988in}{3.415075in}}%
\pgfpathlineto{\pgfqpoint{3.072046in}{3.415075in}}%
\pgfpathlineto{\pgfqpoint{3.078223in}{3.376231in}}%
\pgfpathlineto{\pgfqpoint{3.080282in}{3.395653in}}%
\pgfpathlineto{\pgfqpoint{3.082341in}{3.473341in}}%
\pgfpathlineto{\pgfqpoint{3.084400in}{3.483052in}}%
\pgfpathlineto{\pgfqpoint{3.086459in}{3.473341in}}%
\pgfpathlineto{\pgfqpoint{3.092636in}{3.483052in}}%
\pgfpathlineto{\pgfqpoint{3.096754in}{3.483052in}}%
\pgfpathlineto{\pgfqpoint{3.100872in}{3.521896in}}%
\pgfpathlineto{\pgfqpoint{3.109108in}{3.492763in}}%
\pgfpathlineto{\pgfqpoint{3.111166in}{3.492763in}}%
\pgfpathlineto{\pgfqpoint{3.113225in}{3.473341in}}%
\pgfpathlineto{\pgfqpoint{3.115284in}{3.424786in}}%
\pgfpathlineto{\pgfqpoint{3.121461in}{3.434497in}}%
\pgfpathlineto{\pgfqpoint{3.123520in}{3.415075in}}%
\pgfpathlineto{\pgfqpoint{3.125579in}{3.463630in}}%
\pgfpathlineto{\pgfqpoint{3.127638in}{3.444208in}}%
\pgfpathlineto{\pgfqpoint{3.129697in}{3.502474in}}%
\pgfpathlineto{\pgfqpoint{3.135874in}{3.502474in}}%
\pgfpathlineto{\pgfqpoint{3.139992in}{3.560740in}}%
\pgfpathlineto{\pgfqpoint{3.142051in}{3.570451in}}%
\pgfpathlineto{\pgfqpoint{3.144110in}{3.541318in}}%
\pgfpathlineto{\pgfqpoint{3.150286in}{3.492763in}}%
\pgfpathlineto{\pgfqpoint{3.152345in}{3.492763in}}%
\pgfpathlineto{\pgfqpoint{3.156463in}{3.444208in}}%
\pgfpathlineto{\pgfqpoint{3.158522in}{3.434497in}}%
\pgfpathlineto{\pgfqpoint{3.164699in}{3.415075in}}%
\pgfpathlineto{\pgfqpoint{3.166758in}{3.385942in}}%
\pgfpathlineto{\pgfqpoint{3.170876in}{3.424786in}}%
\pgfpathlineto{\pgfqpoint{3.172935in}{3.492763in}}%
\pgfpathlineto{\pgfqpoint{3.179112in}{3.483052in}}%
\pgfpathlineto{\pgfqpoint{3.181171in}{3.453919in}}%
\pgfpathlineto{\pgfqpoint{3.183230in}{3.385942in}}%
\pgfpathlineto{\pgfqpoint{3.185289in}{3.424786in}}%
\pgfpathlineto{\pgfqpoint{3.187347in}{3.395653in}}%
\pgfpathlineto{\pgfqpoint{3.193524in}{3.395653in}}%
\pgfpathlineto{\pgfqpoint{3.197642in}{3.366520in}}%
\pgfpathlineto{\pgfqpoint{3.201760in}{3.376231in}}%
\pgfpathlineto{\pgfqpoint{3.207937in}{3.376231in}}%
\pgfpathlineto{\pgfqpoint{3.209996in}{3.385942in}}%
\pgfpathlineto{\pgfqpoint{3.214114in}{3.444208in}}%
\pgfpathlineto{\pgfqpoint{3.216173in}{3.376231in}}%
\pgfpathlineto{\pgfqpoint{3.222350in}{3.385942in}}%
\pgfpathlineto{\pgfqpoint{3.226467in}{3.327676in}}%
\pgfpathlineto{\pgfqpoint{3.228526in}{3.376231in}}%
\pgfpathlineto{\pgfqpoint{3.230585in}{3.385942in}}%
\pgfpathlineto{\pgfqpoint{3.236762in}{3.385942in}}%
\pgfpathlineto{\pgfqpoint{3.238821in}{3.405364in}}%
\pgfpathlineto{\pgfqpoint{3.242939in}{3.327676in}}%
\pgfpathlineto{\pgfqpoint{3.244998in}{3.298543in}}%
\pgfpathlineto{\pgfqpoint{3.251175in}{3.356809in}}%
\pgfpathlineto{\pgfqpoint{3.255293in}{3.492763in}}%
\pgfpathlineto{\pgfqpoint{3.257352in}{3.453919in}}%
\pgfpathlineto{\pgfqpoint{3.259411in}{3.444208in}}%
\pgfpathlineto{\pgfqpoint{3.267646in}{3.434497in}}%
\pgfpathlineto{\pgfqpoint{3.269705in}{3.366520in}}%
\pgfpathlineto{\pgfqpoint{3.271764in}{3.366520in}}%
\pgfpathlineto{\pgfqpoint{3.273823in}{3.356809in}}%
\pgfpathlineto{\pgfqpoint{3.282059in}{3.424786in}}%
\pgfpathlineto{\pgfqpoint{3.284118in}{3.395653in}}%
\pgfpathlineto{\pgfqpoint{3.286177in}{3.405364in}}%
\pgfpathlineto{\pgfqpoint{3.288236in}{3.424786in}}%
\pgfpathlineto{\pgfqpoint{3.294413in}{3.424786in}}%
\pgfpathlineto{\pgfqpoint{3.296472in}{3.492763in}}%
\pgfpathlineto{\pgfqpoint{3.298531in}{3.492763in}}%
\pgfpathlineto{\pgfqpoint{3.300589in}{3.521896in}}%
\pgfpathlineto{\pgfqpoint{3.302648in}{3.463630in}}%
\pgfpathlineto{\pgfqpoint{3.310884in}{3.444208in}}%
\pgfpathlineto{\pgfqpoint{3.312943in}{3.453919in}}%
\pgfpathlineto{\pgfqpoint{3.315002in}{3.512185in}}%
\pgfpathlineto{\pgfqpoint{3.317061in}{3.521896in}}%
\pgfpathlineto{\pgfqpoint{3.323238in}{3.541318in}}%
\pgfpathlineto{\pgfqpoint{3.325297in}{3.512185in}}%
\pgfpathlineto{\pgfqpoint{3.327356in}{3.541318in}}%
\pgfpathlineto{\pgfqpoint{3.329415in}{3.502474in}}%
\pgfpathlineto{\pgfqpoint{3.331474in}{3.521896in}}%
\pgfpathlineto{\pgfqpoint{3.337650in}{3.551029in}}%
\pgfpathlineto{\pgfqpoint{3.339709in}{3.589873in}}%
\pgfpathlineto{\pgfqpoint{3.341768in}{3.560740in}}%
\pgfpathlineto{\pgfqpoint{3.345886in}{3.686983in}}%
\pgfpathlineto{\pgfqpoint{3.352063in}{3.648139in}}%
\pgfpathlineto{\pgfqpoint{3.354122in}{3.667561in}}%
\pgfpathlineto{\pgfqpoint{3.356181in}{3.725827in}}%
\pgfpathlineto{\pgfqpoint{3.358240in}{3.745249in}}%
\pgfpathlineto{\pgfqpoint{3.366476in}{3.745249in}}%
\pgfpathlineto{\pgfqpoint{3.368535in}{3.716116in}}%
\pgfpathlineto{\pgfqpoint{3.370594in}{3.784092in}}%
\pgfpathlineto{\pgfqpoint{3.374711in}{3.735538in}}%
\pgfpathlineto{\pgfqpoint{3.382947in}{3.754960in}}%
\pgfpathlineto{\pgfqpoint{3.385006in}{3.822936in}}%
\pgfpathlineto{\pgfqpoint{3.387065in}{3.813225in}}%
\pgfpathlineto{\pgfqpoint{3.389124in}{3.764671in}}%
\pgfpathlineto{\pgfqpoint{3.395301in}{3.754960in}}%
\pgfpathlineto{\pgfqpoint{3.397360in}{3.774382in}}%
\pgfpathlineto{\pgfqpoint{3.401478in}{3.696694in}}%
\pgfpathlineto{\pgfqpoint{3.403537in}{3.745249in}}%
\pgfpathlineto{\pgfqpoint{3.409714in}{3.764671in}}%
\pgfpathlineto{\pgfqpoint{3.411773in}{3.745249in}}%
\pgfpathlineto{\pgfqpoint{3.413831in}{3.754960in}}%
\pgfpathlineto{\pgfqpoint{3.415890in}{3.735538in}}%
\pgfpathlineto{\pgfqpoint{3.417949in}{3.764671in}}%
\pgfpathlineto{\pgfqpoint{3.424126in}{3.735538in}}%
\pgfpathlineto{\pgfqpoint{3.426185in}{3.706405in}}%
\pgfpathlineto{\pgfqpoint{3.428244in}{3.657850in}}%
\pgfpathlineto{\pgfqpoint{3.430303in}{3.657850in}}%
\pgfpathlineto{\pgfqpoint{3.432362in}{3.686983in}}%
\pgfpathlineto{\pgfqpoint{3.438539in}{3.696694in}}%
\pgfpathlineto{\pgfqpoint{3.440598in}{3.725827in}}%
\pgfpathlineto{\pgfqpoint{3.442657in}{3.725827in}}%
\pgfpathlineto{\pgfqpoint{3.444716in}{3.667561in}}%
\pgfpathlineto{\pgfqpoint{3.446775in}{3.667561in}}%
\pgfpathlineto{\pgfqpoint{3.452951in}{3.686983in}}%
\pgfpathlineto{\pgfqpoint{3.455010in}{3.638428in}}%
\pgfpathlineto{\pgfqpoint{3.457069in}{3.619006in}}%
\pgfpathlineto{\pgfqpoint{3.459128in}{3.580162in}}%
\pgfpathlineto{\pgfqpoint{3.467364in}{3.580162in}}%
\pgfpathlineto{\pgfqpoint{3.469423in}{3.628717in}}%
\pgfpathlineto{\pgfqpoint{3.471482in}{3.638428in}}%
\pgfpathlineto{\pgfqpoint{3.473541in}{3.677272in}}%
\pgfpathlineto{\pgfqpoint{3.475600in}{3.619006in}}%
\pgfpathlineto{\pgfqpoint{3.481777in}{3.628717in}}%
\pgfpathlineto{\pgfqpoint{3.483836in}{3.628717in}}%
\pgfpathlineto{\pgfqpoint{3.485895in}{3.599584in}}%
\pgfpathlineto{\pgfqpoint{3.487954in}{3.657850in}}%
\pgfpathlineto{\pgfqpoint{3.490012in}{3.638428in}}%
\pgfpathlineto{\pgfqpoint{3.496189in}{3.638428in}}%
\pgfpathlineto{\pgfqpoint{3.498248in}{3.609295in}}%
\pgfpathlineto{\pgfqpoint{3.502366in}{3.716116in}}%
\pgfpathlineto{\pgfqpoint{3.504425in}{3.745249in}}%
\pgfpathlineto{\pgfqpoint{3.510602in}{3.754960in}}%
\pgfpathlineto{\pgfqpoint{3.514720in}{3.813225in}}%
\pgfpathlineto{\pgfqpoint{3.516779in}{3.784092in}}%
\pgfpathlineto{\pgfqpoint{3.518838in}{3.735538in}}%
\pgfpathlineto{\pgfqpoint{3.525015in}{3.716116in}}%
\pgfpathlineto{\pgfqpoint{3.529132in}{3.745249in}}%
\pgfpathlineto{\pgfqpoint{3.531191in}{3.725827in}}%
\pgfpathlineto{\pgfqpoint{3.539427in}{3.725827in}}%
\pgfpathlineto{\pgfqpoint{3.541486in}{3.735538in}}%
\pgfpathlineto{\pgfqpoint{3.543545in}{3.764671in}}%
\pgfpathlineto{\pgfqpoint{3.547663in}{3.706405in}}%
\pgfpathlineto{\pgfqpoint{3.553840in}{3.735538in}}%
\pgfpathlineto{\pgfqpoint{3.555899in}{3.803514in}}%
\pgfpathlineto{\pgfqpoint{3.557958in}{3.813225in}}%
\pgfpathlineto{\pgfqpoint{3.560017in}{3.852069in}}%
\pgfpathlineto{\pgfqpoint{3.562076in}{3.803514in}}%
\pgfpathlineto{\pgfqpoint{3.568252in}{3.803514in}}%
\pgfpathlineto{\pgfqpoint{3.570311in}{3.813225in}}%
\pgfpathlineto{\pgfqpoint{3.576488in}{3.696694in}}%
\pgfpathlineto{\pgfqpoint{3.584724in}{3.570451in}}%
\pgfpathlineto{\pgfqpoint{3.586783in}{3.619006in}}%
\pgfpathlineto{\pgfqpoint{3.588842in}{3.609295in}}%
\pgfpathlineto{\pgfqpoint{3.590901in}{3.648139in}}%
\pgfpathlineto{\pgfqpoint{3.597078in}{3.686983in}}%
\pgfpathlineto{\pgfqpoint{3.599137in}{3.677272in}}%
\pgfpathlineto{\pgfqpoint{3.601196in}{3.735538in}}%
\pgfpathlineto{\pgfqpoint{3.603254in}{3.686983in}}%
\pgfpathlineto{\pgfqpoint{3.605313in}{3.686983in}}%
\pgfpathlineto{\pgfqpoint{3.611490in}{3.706405in}}%
\pgfpathlineto{\pgfqpoint{3.613549in}{3.696694in}}%
\pgfpathlineto{\pgfqpoint{3.615608in}{3.706405in}}%
\pgfpathlineto{\pgfqpoint{3.617667in}{3.657850in}}%
\pgfpathlineto{\pgfqpoint{3.625903in}{3.657850in}}%
\pgfpathlineto{\pgfqpoint{3.627962in}{3.628717in}}%
\pgfpathlineto{\pgfqpoint{3.630021in}{3.667561in}}%
\pgfpathlineto{\pgfqpoint{3.632080in}{3.648139in}}%
\pgfpathlineto{\pgfqpoint{3.634139in}{3.648139in}}%
\pgfpathlineto{\pgfqpoint{3.640315in}{3.628717in}}%
\pgfpathlineto{\pgfqpoint{3.642374in}{3.638428in}}%
\pgfpathlineto{\pgfqpoint{3.644433in}{3.580162in}}%
\pgfpathlineto{\pgfqpoint{3.646492in}{3.580162in}}%
\pgfpathlineto{\pgfqpoint{3.648551in}{3.589873in}}%
\pgfpathlineto{\pgfqpoint{3.654728in}{3.599584in}}%
\pgfpathlineto{\pgfqpoint{3.656787in}{3.570451in}}%
\pgfpathlineto{\pgfqpoint{3.660905in}{3.560740in}}%
\pgfpathlineto{\pgfqpoint{3.662964in}{3.551029in}}%
\pgfpathlineto{\pgfqpoint{3.671200in}{3.580162in}}%
\pgfpathlineto{\pgfqpoint{3.673259in}{3.560740in}}%
\pgfpathlineto{\pgfqpoint{3.675318in}{3.560740in}}%
\pgfpathlineto{\pgfqpoint{3.677377in}{3.551029in}}%
\pgfpathlineto{\pgfqpoint{3.685612in}{3.580162in}}%
\pgfpathlineto{\pgfqpoint{3.687671in}{3.599584in}}%
\pgfpathlineto{\pgfqpoint{3.689730in}{3.570451in}}%
\pgfpathlineto{\pgfqpoint{3.691789in}{3.638428in}}%
\pgfpathlineto{\pgfqpoint{3.697966in}{3.706405in}}%
\pgfpathlineto{\pgfqpoint{3.702084in}{3.667561in}}%
\pgfpathlineto{\pgfqpoint{3.704143in}{3.706405in}}%
\pgfpathlineto{\pgfqpoint{3.706202in}{3.696694in}}%
\pgfpathlineto{\pgfqpoint{3.712379in}{3.716116in}}%
\pgfpathlineto{\pgfqpoint{3.714438in}{3.686983in}}%
\pgfpathlineto{\pgfqpoint{3.716496in}{3.735538in}}%
\pgfpathlineto{\pgfqpoint{3.718555in}{3.725827in}}%
\pgfpathlineto{\pgfqpoint{3.720614in}{3.696694in}}%
\pgfpathlineto{\pgfqpoint{3.726791in}{3.686983in}}%
\pgfpathlineto{\pgfqpoint{3.728850in}{3.725827in}}%
\pgfpathlineto{\pgfqpoint{3.730909in}{3.725827in}}%
\pgfpathlineto{\pgfqpoint{3.735027in}{3.638428in}}%
\pgfpathlineto{\pgfqpoint{3.743263in}{3.667561in}}%
\pgfpathlineto{\pgfqpoint{3.745322in}{3.638428in}}%
\pgfpathlineto{\pgfqpoint{3.749440in}{3.638428in}}%
\pgfpathlineto{\pgfqpoint{3.755616in}{3.599584in}}%
\pgfpathlineto{\pgfqpoint{3.757675in}{3.609295in}}%
\pgfpathlineto{\pgfqpoint{3.759734in}{3.599584in}}%
\pgfpathlineto{\pgfqpoint{3.761793in}{3.580162in}}%
\pgfpathlineto{\pgfqpoint{3.763852in}{3.580162in}}%
\pgfpathlineto{\pgfqpoint{3.770029in}{3.609295in}}%
\pgfpathlineto{\pgfqpoint{3.772088in}{3.638428in}}%
\pgfpathlineto{\pgfqpoint{3.774147in}{3.628717in}}%
\pgfpathlineto{\pgfqpoint{3.776206in}{3.609295in}}%
\pgfpathlineto{\pgfqpoint{3.778265in}{3.628717in}}%
\pgfpathlineto{\pgfqpoint{3.788560in}{3.686983in}}%
\pgfpathlineto{\pgfqpoint{3.790619in}{3.667561in}}%
\pgfpathlineto{\pgfqpoint{3.792677in}{3.716116in}}%
\pgfpathlineto{\pgfqpoint{3.798854in}{3.696694in}}%
\pgfpathlineto{\pgfqpoint{3.800913in}{3.735538in}}%
\pgfpathlineto{\pgfqpoint{3.802972in}{3.716116in}}%
\pgfpathlineto{\pgfqpoint{3.805031in}{3.716116in}}%
\pgfpathlineto{\pgfqpoint{3.807090in}{3.735538in}}%
\pgfpathlineto{\pgfqpoint{3.813267in}{3.735538in}}%
\pgfpathlineto{\pgfqpoint{3.815326in}{3.803514in}}%
\pgfpathlineto{\pgfqpoint{3.817385in}{3.832647in}}%
\pgfpathlineto{\pgfqpoint{3.821503in}{3.803514in}}%
\pgfpathlineto{\pgfqpoint{3.827680in}{3.813225in}}%
\pgfpathlineto{\pgfqpoint{3.829738in}{3.832647in}}%
\pgfpathlineto{\pgfqpoint{3.831797in}{3.793803in}}%
\pgfpathlineto{\pgfqpoint{3.835915in}{3.793803in}}%
\pgfpathlineto{\pgfqpoint{3.842092in}{3.842358in}}%
\pgfpathlineto{\pgfqpoint{3.844151in}{3.803514in}}%
\pgfpathlineto{\pgfqpoint{3.848269in}{3.949179in}}%
\pgfpathlineto{\pgfqpoint{3.850328in}{3.997734in}}%
\pgfpathlineto{\pgfqpoint{3.858564in}{3.968601in}}%
\pgfpathlineto{\pgfqpoint{3.860623in}{3.988023in}}%
\pgfpathlineto{\pgfqpoint{3.862682in}{3.920046in}}%
\pgfpathlineto{\pgfqpoint{3.864741in}{3.920046in}}%
\pgfpathlineto{\pgfqpoint{3.870917in}{3.939468in}}%
\pgfpathlineto{\pgfqpoint{3.872976in}{3.920046in}}%
\pgfpathlineto{\pgfqpoint{3.875035in}{3.949179in}}%
\pgfpathlineto{\pgfqpoint{3.877094in}{3.958890in}}%
\pgfpathlineto{\pgfqpoint{3.879153in}{3.978312in}}%
\pgfpathlineto{\pgfqpoint{3.885330in}{3.978312in}}%
\pgfpathlineto{\pgfqpoint{3.887389in}{3.968601in}}%
\pgfpathlineto{\pgfqpoint{3.889448in}{3.929757in}}%
\pgfpathlineto{\pgfqpoint{3.891507in}{3.949179in}}%
\pgfpathlineto{\pgfqpoint{3.893566in}{3.920046in}}%
\pgfpathlineto{\pgfqpoint{3.899743in}{3.929757in}}%
\pgfpathlineto{\pgfqpoint{3.903861in}{3.988023in}}%
\pgfpathlineto{\pgfqpoint{3.905919in}{3.978312in}}%
\pgfpathlineto{\pgfqpoint{3.907978in}{4.056000in}}%
\pgfpathlineto{\pgfqpoint{3.914155in}{4.026867in}}%
\pgfpathlineto{\pgfqpoint{3.920332in}{4.026867in}}%
\pgfpathlineto{\pgfqpoint{3.922391in}{3.997734in}}%
\pgfpathlineto{\pgfqpoint{3.932686in}{3.949179in}}%
\pgfpathlineto{\pgfqpoint{3.934745in}{3.958890in}}%
\pgfpathlineto{\pgfqpoint{3.936804in}{3.929757in}}%
\pgfpathlineto{\pgfqpoint{3.942980in}{3.920046in}}%
\pgfpathlineto{\pgfqpoint{3.945039in}{3.910335in}}%
\pgfpathlineto{\pgfqpoint{3.951216in}{3.910335in}}%
\pgfpathlineto{\pgfqpoint{3.957393in}{3.920046in}}%
\pgfpathlineto{\pgfqpoint{3.959452in}{3.920046in}}%
\pgfpathlineto{\pgfqpoint{3.961511in}{3.939468in}}%
\pgfpathlineto{\pgfqpoint{3.963570in}{3.929757in}}%
\pgfpathlineto{\pgfqpoint{3.965629in}{3.900624in}}%
\pgfpathlineto{\pgfqpoint{3.971806in}{3.871491in}}%
\pgfpathlineto{\pgfqpoint{3.973865in}{3.764671in}}%
\pgfpathlineto{\pgfqpoint{3.977983in}{3.745249in}}%
\pgfpathlineto{\pgfqpoint{3.980042in}{3.745249in}}%
\pgfpathlineto{\pgfqpoint{3.986218in}{3.735538in}}%
\pgfpathlineto{\pgfqpoint{3.988277in}{3.735538in}}%
\pgfpathlineto{\pgfqpoint{3.992395in}{3.764671in}}%
\pgfpathlineto{\pgfqpoint{3.994454in}{3.745249in}}%
\pgfpathlineto{\pgfqpoint{4.000631in}{3.716116in}}%
\pgfpathlineto{\pgfqpoint{4.002690in}{3.677272in}}%
\pgfpathlineto{\pgfqpoint{4.004749in}{3.609295in}}%
\pgfpathlineto{\pgfqpoint{4.008867in}{3.638428in}}%
\pgfpathlineto{\pgfqpoint{4.015044in}{3.609295in}}%
\pgfpathlineto{\pgfqpoint{4.019161in}{3.667561in}}%
\pgfpathlineto{\pgfqpoint{4.023279in}{3.648139in}}%
\pgfpathlineto{\pgfqpoint{4.029456in}{3.628717in}}%
\pgfpathlineto{\pgfqpoint{4.033574in}{3.580162in}}%
\pgfpathlineto{\pgfqpoint{4.035633in}{3.531607in}}%
\pgfpathlineto{\pgfqpoint{4.037692in}{3.589873in}}%
\pgfpathlineto{\pgfqpoint{4.043869in}{3.599584in}}%
\pgfpathlineto{\pgfqpoint{4.045928in}{3.609295in}}%
\pgfpathlineto{\pgfqpoint{4.050046in}{3.667561in}}%
\pgfpathlineto{\pgfqpoint{4.052105in}{3.648139in}}%
\pgfpathlineto{\pgfqpoint{4.058281in}{3.667561in}}%
\pgfpathlineto{\pgfqpoint{4.060340in}{3.686983in}}%
\pgfpathlineto{\pgfqpoint{4.062399in}{3.677272in}}%
\pgfpathlineto{\pgfqpoint{4.064458in}{3.677272in}}%
\pgfpathlineto{\pgfqpoint{4.066517in}{3.696694in}}%
\pgfpathlineto{\pgfqpoint{4.074753in}{3.667561in}}%
\pgfpathlineto{\pgfqpoint{4.076812in}{3.677272in}}%
\pgfpathlineto{\pgfqpoint{4.078871in}{3.648139in}}%
\pgfpathlineto{\pgfqpoint{4.080930in}{3.667561in}}%
\pgfpathlineto{\pgfqpoint{4.087107in}{3.667561in}}%
\pgfpathlineto{\pgfqpoint{4.089166in}{3.648139in}}%
\pgfpathlineto{\pgfqpoint{4.091225in}{3.667561in}}%
\pgfpathlineto{\pgfqpoint{4.093284in}{3.599584in}}%
\pgfpathlineto{\pgfqpoint{4.095342in}{3.638428in}}%
\pgfpathlineto{\pgfqpoint{4.101519in}{3.667561in}}%
\pgfpathlineto{\pgfqpoint{4.103578in}{3.638428in}}%
\pgfpathlineto{\pgfqpoint{4.105637in}{3.638428in}}%
\pgfpathlineto{\pgfqpoint{4.109755in}{3.580162in}}%
\pgfpathlineto{\pgfqpoint{4.115932in}{3.609295in}}%
\pgfpathlineto{\pgfqpoint{4.120050in}{3.648139in}}%
\pgfpathlineto{\pgfqpoint{4.122109in}{3.619006in}}%
\pgfpathlineto{\pgfqpoint{4.124168in}{3.609295in}}%
\pgfpathlineto{\pgfqpoint{4.132403in}{3.599584in}}%
\pgfpathlineto{\pgfqpoint{4.134462in}{3.609295in}}%
\pgfpathlineto{\pgfqpoint{4.136521in}{3.657850in}}%
\pgfpathlineto{\pgfqpoint{4.138580in}{3.628717in}}%
\pgfpathlineto{\pgfqpoint{4.144757in}{3.638428in}}%
\pgfpathlineto{\pgfqpoint{4.146816in}{3.619006in}}%
\pgfpathlineto{\pgfqpoint{4.148875in}{3.677272in}}%
\pgfpathlineto{\pgfqpoint{4.150934in}{3.696694in}}%
\pgfpathlineto{\pgfqpoint{4.152993in}{3.735538in}}%
\pgfpathlineto{\pgfqpoint{4.163288in}{3.667561in}}%
\pgfpathlineto{\pgfqpoint{4.167406in}{3.609295in}}%
\pgfpathlineto{\pgfqpoint{4.173582in}{3.638428in}}%
\pgfpathlineto{\pgfqpoint{4.175641in}{3.609295in}}%
\pgfpathlineto{\pgfqpoint{4.179759in}{3.648139in}}%
\pgfpathlineto{\pgfqpoint{4.181818in}{3.628717in}}%
\pgfpathlineto{\pgfqpoint{4.187995in}{3.619006in}}%
\pgfpathlineto{\pgfqpoint{4.190054in}{3.628717in}}%
\pgfpathlineto{\pgfqpoint{4.196231in}{3.492763in}}%
\pgfpathlineto{\pgfqpoint{4.202408in}{3.483052in}}%
\pgfpathlineto{\pgfqpoint{4.204467in}{3.473341in}}%
\pgfpathlineto{\pgfqpoint{4.208584in}{3.424786in}}%
\pgfpathlineto{\pgfqpoint{4.210643in}{3.424786in}}%
\pgfpathlineto{\pgfqpoint{4.216820in}{3.502474in}}%
\pgfpathlineto{\pgfqpoint{4.218879in}{3.492763in}}%
\pgfpathlineto{\pgfqpoint{4.220938in}{3.541318in}}%
\pgfpathlineto{\pgfqpoint{4.225056in}{3.521896in}}%
\pgfpathlineto{\pgfqpoint{4.231233in}{3.541318in}}%
\pgfpathlineto{\pgfqpoint{4.233292in}{3.531607in}}%
\pgfpathlineto{\pgfqpoint{4.235351in}{3.512185in}}%
\pgfpathlineto{\pgfqpoint{4.239469in}{3.580162in}}%
\pgfpathlineto{\pgfqpoint{4.245645in}{3.570451in}}%
\pgfpathlineto{\pgfqpoint{4.247704in}{3.599584in}}%
\pgfpathlineto{\pgfqpoint{4.249763in}{3.599584in}}%
\pgfpathlineto{\pgfqpoint{4.251822in}{3.570451in}}%
\pgfpathlineto{\pgfqpoint{4.260058in}{3.599584in}}%
\pgfpathlineto{\pgfqpoint{4.262117in}{3.589873in}}%
\pgfpathlineto{\pgfqpoint{4.264176in}{3.551029in}}%
\pgfpathlineto{\pgfqpoint{4.266235in}{3.551029in}}%
\pgfpathlineto{\pgfqpoint{4.268294in}{3.531607in}}%
\pgfpathlineto{\pgfqpoint{4.274471in}{3.570451in}}%
\pgfpathlineto{\pgfqpoint{4.276530in}{3.541318in}}%
\pgfpathlineto{\pgfqpoint{4.278589in}{3.531607in}}%
\pgfpathlineto{\pgfqpoint{4.280648in}{3.551029in}}%
\pgfpathlineto{\pgfqpoint{4.282707in}{3.541318in}}%
\pgfpathlineto{\pgfqpoint{4.288883in}{3.521896in}}%
\pgfpathlineto{\pgfqpoint{4.290942in}{3.473341in}}%
\pgfpathlineto{\pgfqpoint{4.293001in}{3.502474in}}%
\pgfpathlineto{\pgfqpoint{4.295060in}{3.483052in}}%
\pgfpathlineto{\pgfqpoint{4.297119in}{3.502474in}}%
\pgfpathlineto{\pgfqpoint{4.303296in}{3.444208in}}%
\pgfpathlineto{\pgfqpoint{4.305355in}{3.473341in}}%
\pgfpathlineto{\pgfqpoint{4.307414in}{3.434497in}}%
\pgfpathlineto{\pgfqpoint{4.309473in}{3.453919in}}%
\pgfpathlineto{\pgfqpoint{4.311532in}{3.434497in}}%
\pgfpathlineto{\pgfqpoint{4.317709in}{3.444208in}}%
\pgfpathlineto{\pgfqpoint{4.319768in}{3.453919in}}%
\pgfpathlineto{\pgfqpoint{4.321826in}{3.434497in}}%
\pgfpathlineto{\pgfqpoint{4.323885in}{3.366520in}}%
\pgfpathlineto{\pgfqpoint{4.325944in}{3.366520in}}%
\pgfpathlineto{\pgfqpoint{4.336239in}{3.308254in}}%
\pgfpathlineto{\pgfqpoint{4.338298in}{3.269410in}}%
\pgfpathlineto{\pgfqpoint{4.340357in}{3.201434in}}%
\pgfpathlineto{\pgfqpoint{4.346534in}{3.152879in}}%
\pgfpathlineto{\pgfqpoint{4.348593in}{3.220855in}}%
\pgfpathlineto{\pgfqpoint{4.350652in}{3.249988in}}%
\pgfpathlineto{\pgfqpoint{4.352711in}{3.240277in}}%
\pgfpathlineto{\pgfqpoint{4.354770in}{3.191723in}}%
\pgfpathlineto{\pgfqpoint{4.360946in}{3.240277in}}%
\pgfpathlineto{\pgfqpoint{4.365064in}{3.240277in}}%
\pgfpathlineto{\pgfqpoint{4.367123in}{3.230566in}}%
\pgfpathlineto{\pgfqpoint{4.369182in}{3.211145in}}%
\pgfpathlineto{\pgfqpoint{4.375359in}{3.201434in}}%
\pgfpathlineto{\pgfqpoint{4.377418in}{3.172301in}}%
\pgfpathlineto{\pgfqpoint{4.381536in}{3.152879in}}%
\pgfpathlineto{\pgfqpoint{4.383595in}{3.211145in}}%
\pgfpathlineto{\pgfqpoint{4.391831in}{3.152879in}}%
\pgfpathlineto{\pgfqpoint{4.393890in}{3.191723in}}%
\pgfpathlineto{\pgfqpoint{4.395949in}{3.143168in}}%
\pgfpathlineto{\pgfqpoint{4.398007in}{3.143168in}}%
\pgfpathlineto{\pgfqpoint{4.404184in}{3.172301in}}%
\pgfpathlineto{\pgfqpoint{4.408302in}{3.094613in}}%
\pgfpathlineto{\pgfqpoint{4.412420in}{3.162590in}}%
\pgfpathlineto{\pgfqpoint{4.418597in}{3.152879in}}%
\pgfpathlineto{\pgfqpoint{4.420656in}{3.162590in}}%
\pgfpathlineto{\pgfqpoint{4.422715in}{3.191723in}}%
\pgfpathlineto{\pgfqpoint{4.424774in}{3.269410in}}%
\pgfpathlineto{\pgfqpoint{4.433010in}{3.230566in}}%
\pgfpathlineto{\pgfqpoint{4.435068in}{3.249988in}}%
\pgfpathlineto{\pgfqpoint{4.437127in}{3.191723in}}%
\pgfpathlineto{\pgfqpoint{4.439186in}{3.182012in}}%
\pgfpathlineto{\pgfqpoint{4.441245in}{3.191723in}}%
\pgfpathlineto{\pgfqpoint{4.447422in}{3.201434in}}%
\pgfpathlineto{\pgfqpoint{4.449481in}{3.230566in}}%
\pgfpathlineto{\pgfqpoint{4.451540in}{3.201434in}}%
\pgfpathlineto{\pgfqpoint{4.453599in}{3.220855in}}%
\pgfpathlineto{\pgfqpoint{4.455658in}{3.211145in}}%
\pgfpathlineto{\pgfqpoint{4.461835in}{3.211145in}}%
\pgfpathlineto{\pgfqpoint{4.463894in}{3.201434in}}%
\pgfpathlineto{\pgfqpoint{4.465953in}{3.152879in}}%
\pgfpathlineto{\pgfqpoint{4.470071in}{3.016925in}}%
\pgfpathlineto{\pgfqpoint{4.476247in}{2.929526in}}%
\pgfpathlineto{\pgfqpoint{4.480365in}{2.851838in}}%
\pgfpathlineto{\pgfqpoint{4.482424in}{2.880971in}}%
\pgfpathlineto{\pgfqpoint{4.484483in}{2.890682in}}%
\pgfpathlineto{\pgfqpoint{4.490660in}{2.774150in}}%
\pgfpathlineto{\pgfqpoint{4.492719in}{2.783861in}}%
\pgfpathlineto{\pgfqpoint{4.494778in}{2.667329in}}%
\pgfpathlineto{\pgfqpoint{4.496837in}{2.618775in}}%
\pgfpathlineto{\pgfqpoint{4.498896in}{2.647908in}}%
\pgfpathlineto{\pgfqpoint{4.505073in}{2.715884in}}%
\pgfpathlineto{\pgfqpoint{4.507132in}{2.677040in}}%
\pgfpathlineto{\pgfqpoint{4.511249in}{2.745017in}}%
\pgfpathlineto{\pgfqpoint{4.513308in}{2.657618in}}%
\pgfpathlineto{\pgfqpoint{4.519485in}{2.677040in}}%
\pgfpathlineto{\pgfqpoint{4.521544in}{2.609064in}}%
\pgfpathlineto{\pgfqpoint{4.523603in}{2.579931in}}%
\pgfpathlineto{\pgfqpoint{4.525662in}{2.609064in}}%
\pgfpathlineto{\pgfqpoint{4.527721in}{2.599353in}}%
\pgfpathlineto{\pgfqpoint{4.535957in}{2.589642in}}%
\pgfpathlineto{\pgfqpoint{4.538016in}{2.609064in}}%
\pgfpathlineto{\pgfqpoint{4.540075in}{2.696462in}}%
\pgfpathlineto{\pgfqpoint{4.542134in}{2.657618in}}%
\pgfpathlineto{\pgfqpoint{4.548311in}{2.745017in}}%
\pgfpathlineto{\pgfqpoint{4.550369in}{2.822705in}}%
\pgfpathlineto{\pgfqpoint{4.552428in}{2.851838in}}%
\pgfpathlineto{\pgfqpoint{4.554487in}{2.851838in}}%
\pgfpathlineto{\pgfqpoint{4.556546in}{2.997503in}}%
\pgfpathlineto{\pgfqpoint{4.562723in}{2.939237in}}%
\pgfpathlineto{\pgfqpoint{4.566841in}{2.880971in}}%
\pgfpathlineto{\pgfqpoint{4.568900in}{2.851838in}}%
\pgfpathlineto{\pgfqpoint{4.570959in}{2.803283in}}%
\pgfpathlineto{\pgfqpoint{4.577136in}{2.793572in}}%
\pgfpathlineto{\pgfqpoint{4.579195in}{2.725595in}}%
\pgfpathlineto{\pgfqpoint{4.581254in}{2.812994in}}%
\pgfpathlineto{\pgfqpoint{4.585372in}{2.764439in}}%
\pgfpathlineto{\pgfqpoint{4.591548in}{2.754728in}}%
\pgfpathlineto{\pgfqpoint{4.593607in}{2.745017in}}%
\pgfpathlineto{\pgfqpoint{4.595666in}{2.725595in}}%
\pgfpathlineto{\pgfqpoint{4.599784in}{2.647908in}}%
\pgfpathlineto{\pgfqpoint{4.605961in}{2.686751in}}%
\pgfpathlineto{\pgfqpoint{4.608020in}{2.677040in}}%
\pgfpathlineto{\pgfqpoint{4.610079in}{2.715884in}}%
\pgfpathlineto{\pgfqpoint{4.614197in}{2.851838in}}%
\pgfpathlineto{\pgfqpoint{4.622433in}{2.861549in}}%
\pgfpathlineto{\pgfqpoint{4.624491in}{2.861549in}}%
\pgfpathlineto{\pgfqpoint{4.634786in}{2.910104in}}%
\pgfpathlineto{\pgfqpoint{4.638904in}{2.880971in}}%
\pgfpathlineto{\pgfqpoint{4.640963in}{2.890682in}}%
\pgfpathlineto{\pgfqpoint{4.643022in}{2.919815in}}%
\pgfpathlineto{\pgfqpoint{4.649199in}{2.968370in}}%
\pgfpathlineto{\pgfqpoint{4.651258in}{2.958659in}}%
\pgfpathlineto{\pgfqpoint{4.655376in}{2.803283in}}%
\pgfpathlineto{\pgfqpoint{4.657435in}{2.842127in}}%
\pgfpathlineto{\pgfqpoint{4.663611in}{2.900393in}}%
\pgfpathlineto{\pgfqpoint{4.665670in}{2.968370in}}%
\pgfpathlineto{\pgfqpoint{4.667729in}{2.929526in}}%
\pgfpathlineto{\pgfqpoint{4.669788in}{3.026636in}}%
\pgfpathlineto{\pgfqpoint{4.671847in}{3.055769in}}%
\pgfpathlineto{\pgfqpoint{4.680083in}{3.016925in}}%
\pgfpathlineto{\pgfqpoint{4.682142in}{2.987792in}}%
\pgfpathlineto{\pgfqpoint{4.684201in}{2.939237in}}%
\pgfpathlineto{\pgfqpoint{4.686260in}{2.939237in}}%
\pgfpathlineto{\pgfqpoint{4.692437in}{2.929526in}}%
\pgfpathlineto{\pgfqpoint{4.696555in}{2.832416in}}%
\pgfpathlineto{\pgfqpoint{4.698614in}{2.871260in}}%
\pgfpathlineto{\pgfqpoint{4.700672in}{2.851838in}}%
\pgfpathlineto{\pgfqpoint{4.706849in}{2.842127in}}%
\pgfpathlineto{\pgfqpoint{4.708908in}{2.812994in}}%
\pgfpathlineto{\pgfqpoint{4.715085in}{2.842127in}}%
\pgfpathlineto{\pgfqpoint{4.721262in}{2.910104in}}%
\pgfpathlineto{\pgfqpoint{4.723321in}{2.803283in}}%
\pgfpathlineto{\pgfqpoint{4.725380in}{2.851838in}}%
\pgfpathlineto{\pgfqpoint{4.727439in}{2.871260in}}%
\pgfpathlineto{\pgfqpoint{4.729498in}{2.919815in}}%
\pgfpathlineto{\pgfqpoint{4.737733in}{2.890682in}}%
\pgfpathlineto{\pgfqpoint{4.739792in}{2.861549in}}%
\pgfpathlineto{\pgfqpoint{4.741851in}{2.948948in}}%
\pgfpathlineto{\pgfqpoint{4.743910in}{2.890682in}}%
\pgfpathlineto{\pgfqpoint{4.752146in}{2.939237in}}%
\pgfpathlineto{\pgfqpoint{4.754205in}{2.978081in}}%
\pgfpathlineto{\pgfqpoint{4.756264in}{2.978081in}}%
\pgfpathlineto{\pgfqpoint{4.758323in}{2.968370in}}%
\pgfpathlineto{\pgfqpoint{4.764500in}{2.978081in}}%
\pgfpathlineto{\pgfqpoint{4.766559in}{2.958659in}}%
\pgfpathlineto{\pgfqpoint{4.770677in}{2.958659in}}%
\pgfpathlineto{\pgfqpoint{4.772736in}{2.948948in}}%
\pgfpathlineto{\pgfqpoint{4.778912in}{2.968370in}}%
\pgfpathlineto{\pgfqpoint{4.780971in}{3.016925in}}%
\pgfpathlineto{\pgfqpoint{4.785089in}{2.958659in}}%
\pgfpathlineto{\pgfqpoint{4.787148in}{2.890682in}}%
\pgfpathlineto{\pgfqpoint{4.793325in}{2.910104in}}%
\pgfpathlineto{\pgfqpoint{4.799502in}{3.007214in}}%
\pgfpathlineto{\pgfqpoint{4.801561in}{2.910104in}}%
\pgfpathlineto{\pgfqpoint{4.807738in}{2.929526in}}%
\pgfpathlineto{\pgfqpoint{4.811856in}{2.861549in}}%
\pgfpathlineto{\pgfqpoint{4.815973in}{2.919815in}}%
\pgfpathlineto{\pgfqpoint{4.826268in}{2.851838in}}%
\pgfpathlineto{\pgfqpoint{4.830386in}{2.774150in}}%
\pgfpathlineto{\pgfqpoint{4.836563in}{2.686751in}}%
\pgfpathlineto{\pgfqpoint{4.838622in}{2.735306in}}%
\pgfpathlineto{\pgfqpoint{4.840681in}{2.686751in}}%
\pgfpathlineto{\pgfqpoint{4.842740in}{2.677040in}}%
\pgfpathlineto{\pgfqpoint{4.844799in}{2.628486in}}%
\pgfpathlineto{\pgfqpoint{4.850976in}{2.647908in}}%
\pgfpathlineto{\pgfqpoint{4.855093in}{2.774150in}}%
\pgfpathlineto{\pgfqpoint{4.857152in}{2.745017in}}%
\pgfpathlineto{\pgfqpoint{4.859211in}{2.686751in}}%
\pgfpathlineto{\pgfqpoint{4.865388in}{2.667329in}}%
\pgfpathlineto{\pgfqpoint{4.867447in}{2.686751in}}%
\pgfpathlineto{\pgfqpoint{4.869506in}{2.725595in}}%
\pgfpathlineto{\pgfqpoint{4.873624in}{2.677040in}}%
\pgfpathlineto{\pgfqpoint{4.881860in}{2.638197in}}%
\pgfpathlineto{\pgfqpoint{4.883919in}{2.647908in}}%
\pgfpathlineto{\pgfqpoint{4.885978in}{2.609064in}}%
\pgfpathlineto{\pgfqpoint{4.888037in}{2.541087in}}%
\pgfpathlineto{\pgfqpoint{4.894213in}{2.482821in}}%
\pgfpathlineto{\pgfqpoint{4.896272in}{2.443977in}}%
\pgfpathlineto{\pgfqpoint{4.898331in}{2.453688in}}%
\pgfpathlineto{\pgfqpoint{4.900390in}{2.434266in}}%
\pgfpathlineto{\pgfqpoint{4.902449in}{2.298312in}}%
\pgfpathlineto{\pgfqpoint{4.908626in}{2.308023in}}%
\pgfpathlineto{\pgfqpoint{4.910685in}{2.288601in}}%
\pgfpathlineto{\pgfqpoint{4.912744in}{2.317734in}}%
\pgfpathlineto{\pgfqpoint{4.914803in}{2.210913in}}%
\pgfpathlineto{\pgfqpoint{4.916862in}{1.909873in}}%
\pgfpathlineto{\pgfqpoint{4.923039in}{1.657387in}}%
\pgfpathlineto{\pgfqpoint{4.925098in}{1.939006in}}%
\pgfpathlineto{\pgfqpoint{4.927156in}{1.958428in}}%
\pgfpathlineto{\pgfqpoint{4.929215in}{2.142936in}}%
\pgfpathlineto{\pgfqpoint{4.931274in}{2.210913in}}%
\pgfpathlineto{\pgfqpoint{4.937451in}{1.997272in}}%
\pgfpathlineto{\pgfqpoint{4.941569in}{2.414844in}}%
\pgfpathlineto{\pgfqpoint{4.943628in}{2.424555in}}%
\pgfpathlineto{\pgfqpoint{4.945687in}{2.201202in}}%
\pgfpathlineto{\pgfqpoint{4.951864in}{1.987561in}}%
\pgfpathlineto{\pgfqpoint{4.955982in}{2.104092in}}%
\pgfpathlineto{\pgfqpoint{4.958041in}{2.074960in}}%
\pgfpathlineto{\pgfqpoint{4.960100in}{1.948717in}}%
\pgfpathlineto{\pgfqpoint{4.966276in}{1.968139in}}%
\pgfpathlineto{\pgfqpoint{4.968335in}{2.006983in}}%
\pgfpathlineto{\pgfqpoint{4.970394in}{1.929295in}}%
\pgfpathlineto{\pgfqpoint{4.972453in}{1.919584in}}%
\pgfpathlineto{\pgfqpoint{4.974512in}{1.900162in}}%
\pgfpathlineto{\pgfqpoint{4.980689in}{1.929295in}}%
\pgfpathlineto{\pgfqpoint{4.984807in}{2.026405in}}%
\pgfpathlineto{\pgfqpoint{4.986866in}{2.006983in}}%
\pgfpathlineto{\pgfqpoint{4.995102in}{2.045827in}}%
\pgfpathlineto{\pgfqpoint{4.997161in}{2.065249in}}%
\pgfpathlineto{\pgfqpoint{4.999220in}{1.929295in}}%
\pgfpathlineto{\pgfqpoint{5.001279in}{1.871029in}}%
\pgfpathlineto{\pgfqpoint{5.003337in}{1.929295in}}%
\pgfpathlineto{\pgfqpoint{5.009514in}{1.890451in}}%
\pgfpathlineto{\pgfqpoint{5.011573in}{1.832185in}}%
\pgfpathlineto{\pgfqpoint{5.013632in}{1.880740in}}%
\pgfpathlineto{\pgfqpoint{5.015691in}{1.841896in}}%
\pgfpathlineto{\pgfqpoint{5.017750in}{1.832185in}}%
\pgfpathlineto{\pgfqpoint{5.023927in}{1.909873in}}%
\pgfpathlineto{\pgfqpoint{5.025986in}{1.861318in}}%
\pgfpathlineto{\pgfqpoint{5.030104in}{1.939006in}}%
\pgfpathlineto{\pgfqpoint{5.032163in}{1.929295in}}%
\pgfpathlineto{\pgfqpoint{5.038340in}{1.948717in}}%
\pgfpathlineto{\pgfqpoint{5.040399in}{1.977850in}}%
\pgfpathlineto{\pgfqpoint{5.042457in}{2.065249in}}%
\pgfpathlineto{\pgfqpoint{5.044516in}{1.968139in}}%
\pgfpathlineto{\pgfqpoint{5.046575in}{2.045827in}}%
\pgfpathlineto{\pgfqpoint{5.052752in}{2.084671in}}%
\pgfpathlineto{\pgfqpoint{5.058929in}{1.958428in}}%
\pgfpathlineto{\pgfqpoint{5.060988in}{1.977850in}}%
\pgfpathlineto{\pgfqpoint{5.067165in}{2.094382in}}%
\pgfpathlineto{\pgfqpoint{5.069224in}{2.084671in}}%
\pgfpathlineto{\pgfqpoint{5.071283in}{2.055538in}}%
\pgfpathlineto{\pgfqpoint{5.073342in}{2.055538in}}%
\pgfpathlineto{\pgfqpoint{5.075401in}{2.026405in}}%
\pgfpathlineto{\pgfqpoint{5.085695in}{2.094382in}}%
\pgfpathlineto{\pgfqpoint{5.087754in}{2.123514in}}%
\pgfpathlineto{\pgfqpoint{5.089813in}{2.065249in}}%
\pgfpathlineto{\pgfqpoint{5.098049in}{2.133225in}}%
\pgfpathlineto{\pgfqpoint{5.104226in}{2.327445in}}%
\pgfpathlineto{\pgfqpoint{5.110403in}{2.298312in}}%
\pgfpathlineto{\pgfqpoint{5.114521in}{2.181780in}}%
\pgfpathlineto{\pgfqpoint{5.116579in}{2.065249in}}%
\pgfpathlineto{\pgfqpoint{5.118638in}{2.104092in}}%
\pgfpathlineto{\pgfqpoint{5.124815in}{2.104092in}}%
\pgfpathlineto{\pgfqpoint{5.126874in}{2.191491in}}%
\pgfpathlineto{\pgfqpoint{5.128933in}{2.172069in}}%
\pgfpathlineto{\pgfqpoint{5.130992in}{2.123514in}}%
\pgfpathlineto{\pgfqpoint{5.133051in}{2.123514in}}%
\pgfpathlineto{\pgfqpoint{5.139228in}{2.113803in}}%
\pgfpathlineto{\pgfqpoint{5.141287in}{2.142936in}}%
\pgfpathlineto{\pgfqpoint{5.143346in}{2.094382in}}%
\pgfpathlineto{\pgfqpoint{5.145405in}{2.084671in}}%
\pgfpathlineto{\pgfqpoint{5.147464in}{2.026405in}}%
\pgfpathlineto{\pgfqpoint{5.153641in}{2.045827in}}%
\pgfpathlineto{\pgfqpoint{5.157758in}{2.084671in}}%
\pgfpathlineto{\pgfqpoint{5.159817in}{2.084671in}}%
\pgfpathlineto{\pgfqpoint{5.168053in}{2.104092in}}%
\pgfpathlineto{\pgfqpoint{5.170112in}{2.036116in}}%
\pgfpathlineto{\pgfqpoint{5.172171in}{2.045827in}}%
\pgfpathlineto{\pgfqpoint{5.174230in}{1.977850in}}%
\pgfpathlineto{\pgfqpoint{5.176289in}{1.987561in}}%
\pgfpathlineto{\pgfqpoint{5.182466in}{1.987561in}}%
\pgfpathlineto{\pgfqpoint{5.184525in}{1.958428in}}%
\pgfpathlineto{\pgfqpoint{5.186584in}{1.987561in}}%
\pgfpathlineto{\pgfqpoint{5.188643in}{1.968139in}}%
\pgfpathlineto{\pgfqpoint{5.190702in}{1.987561in}}%
\pgfpathlineto{\pgfqpoint{5.196878in}{1.977850in}}%
\pgfpathlineto{\pgfqpoint{5.198937in}{1.968139in}}%
\pgfpathlineto{\pgfqpoint{5.200996in}{1.948717in}}%
\pgfpathlineto{\pgfqpoint{5.203055in}{1.900162in}}%
\pgfpathlineto{\pgfqpoint{5.205114in}{1.890451in}}%
\pgfpathlineto{\pgfqpoint{5.211291in}{1.909873in}}%
\pgfpathlineto{\pgfqpoint{5.213350in}{1.880740in}}%
\pgfpathlineto{\pgfqpoint{5.215409in}{1.900162in}}%
\pgfpathlineto{\pgfqpoint{5.217468in}{1.861318in}}%
\pgfpathlineto{\pgfqpoint{5.219527in}{1.861318in}}%
\pgfpathlineto{\pgfqpoint{5.225704in}{1.890451in}}%
\pgfpathlineto{\pgfqpoint{5.227763in}{1.851607in}}%
\pgfpathlineto{\pgfqpoint{5.229822in}{1.880740in}}%
\pgfpathlineto{\pgfqpoint{5.231880in}{1.861318in}}%
\pgfpathlineto{\pgfqpoint{5.233939in}{1.890451in}}%
\pgfpathlineto{\pgfqpoint{5.240116in}{1.909873in}}%
\pgfpathlineto{\pgfqpoint{5.244234in}{2.026405in}}%
\pgfpathlineto{\pgfqpoint{5.248352in}{2.104092in}}%
\pgfpathlineto{\pgfqpoint{5.254529in}{2.084671in}}%
\pgfpathlineto{\pgfqpoint{5.256588in}{2.055538in}}%
\pgfpathlineto{\pgfqpoint{5.258647in}{2.074960in}}%
\pgfpathlineto{\pgfqpoint{5.262765in}{2.006983in}}%
\pgfpathlineto{\pgfqpoint{5.268941in}{2.006983in}}%
\pgfpathlineto{\pgfqpoint{5.273059in}{2.065249in}}%
\pgfpathlineto{\pgfqpoint{5.275118in}{2.152647in}}%
\pgfpathlineto{\pgfqpoint{5.277177in}{2.172069in}}%
\pgfpathlineto{\pgfqpoint{5.283354in}{2.142936in}}%
\pgfpathlineto{\pgfqpoint{5.289531in}{1.997272in}}%
\pgfpathlineto{\pgfqpoint{5.291590in}{2.113803in}}%
\pgfpathlineto{\pgfqpoint{5.299826in}{2.084671in}}%
\pgfpathlineto{\pgfqpoint{5.301885in}{2.104092in}}%
\pgfpathlineto{\pgfqpoint{5.306002in}{2.074960in}}%
\pgfpathlineto{\pgfqpoint{5.312179in}{2.074960in}}%
\pgfpathlineto{\pgfqpoint{5.314238in}{2.084671in}}%
\pgfpathlineto{\pgfqpoint{5.316297in}{2.104092in}}%
\pgfpathlineto{\pgfqpoint{5.318356in}{2.084671in}}%
\pgfpathlineto{\pgfqpoint{5.320415in}{2.104092in}}%
\pgfpathlineto{\pgfqpoint{5.330710in}{2.074960in}}%
\pgfpathlineto{\pgfqpoint{5.332769in}{2.055538in}}%
\pgfpathlineto{\pgfqpoint{5.334828in}{2.055538in}}%
\pgfpathlineto{\pgfqpoint{5.341005in}{2.074960in}}%
\pgfpathlineto{\pgfqpoint{5.343064in}{2.065249in}}%
\pgfpathlineto{\pgfqpoint{5.345122in}{2.113803in}}%
\pgfpathlineto{\pgfqpoint{5.347181in}{2.104092in}}%
\pgfpathlineto{\pgfqpoint{5.355417in}{2.220624in}}%
\pgfpathlineto{\pgfqpoint{5.357476in}{2.210913in}}%
\pgfpathlineto{\pgfqpoint{5.359535in}{2.249757in}}%
\pgfpathlineto{\pgfqpoint{5.361594in}{2.220624in}}%
\pgfpathlineto{\pgfqpoint{5.363653in}{2.230335in}}%
\pgfpathlineto{\pgfqpoint{5.373948in}{2.152647in}}%
\pgfpathlineto{\pgfqpoint{5.376007in}{2.172069in}}%
\pgfpathlineto{\pgfqpoint{5.378066in}{2.172069in}}%
\pgfpathlineto{\pgfqpoint{5.384242in}{2.201202in}}%
\pgfpathlineto{\pgfqpoint{5.386301in}{2.249757in}}%
\pgfpathlineto{\pgfqpoint{5.388360in}{2.269179in}}%
\pgfpathlineto{\pgfqpoint{5.390419in}{2.317734in}}%
\pgfpathlineto{\pgfqpoint{5.392478in}{2.288601in}}%
\pgfpathlineto{\pgfqpoint{5.402773in}{2.210913in}}%
\pgfpathlineto{\pgfqpoint{5.406891in}{2.298312in}}%
\pgfpathlineto{\pgfqpoint{5.413068in}{2.278890in}}%
\pgfpathlineto{\pgfqpoint{5.415127in}{2.308023in}}%
\pgfpathlineto{\pgfqpoint{5.417186in}{2.201202in}}%
\pgfpathlineto{\pgfqpoint{5.419244in}{2.191491in}}%
\pgfpathlineto{\pgfqpoint{5.421303in}{2.249757in}}%
\pgfpathlineto{\pgfqpoint{5.429539in}{2.395422in}}%
\pgfpathlineto{\pgfqpoint{5.433657in}{2.288601in}}%
\pgfpathlineto{\pgfqpoint{5.435716in}{2.298312in}}%
\pgfpathlineto{\pgfqpoint{5.441893in}{2.308023in}}%
\pgfpathlineto{\pgfqpoint{5.443952in}{2.269179in}}%
\pgfpathlineto{\pgfqpoint{5.446011in}{2.269179in}}%
\pgfpathlineto{\pgfqpoint{5.450129in}{2.181780in}}%
\pgfpathlineto{\pgfqpoint{5.456306in}{2.210913in}}%
\pgfpathlineto{\pgfqpoint{5.460423in}{2.269179in}}%
\pgfpathlineto{\pgfqpoint{5.464541in}{2.220624in}}%
\pgfpathlineto{\pgfqpoint{5.470718in}{2.230335in}}%
\pgfpathlineto{\pgfqpoint{5.474836in}{2.346867in}}%
\pgfpathlineto{\pgfqpoint{5.476895in}{2.317734in}}%
\pgfpathlineto{\pgfqpoint{5.478954in}{2.376000in}}%
\pgfpathlineto{\pgfqpoint{5.487190in}{2.317734in}}%
\pgfpathlineto{\pgfqpoint{5.489249in}{2.337156in}}%
\pgfpathlineto{\pgfqpoint{5.493367in}{2.278890in}}%
\pgfpathlineto{\pgfqpoint{5.499543in}{2.278890in}}%
\pgfpathlineto{\pgfqpoint{5.507779in}{2.346867in}}%
\pgfpathlineto{\pgfqpoint{5.513956in}{2.327445in}}%
\pgfpathlineto{\pgfqpoint{5.516015in}{2.298312in}}%
\pgfpathlineto{\pgfqpoint{5.518074in}{2.346867in}}%
\pgfpathlineto{\pgfqpoint{5.520133in}{2.308023in}}%
\pgfpathlineto{\pgfqpoint{5.528369in}{2.317734in}}%
\pgfpathlineto{\pgfqpoint{5.530428in}{2.317734in}}%
\pgfpathlineto{\pgfqpoint{5.534545in}{2.298312in}}%
\pgfpathlineto{\pgfqpoint{5.534545in}{2.298312in}}%
\pgfusepath{stroke}%
\end{pgfscope}%
\begin{pgfscope}%
\pgfsetrectcap%
\pgfsetmiterjoin%
\pgfsetlinewidth{0.803000pt}%
\definecolor{currentstroke}{rgb}{0.000000,0.000000,0.000000}%
\pgfsetstrokecolor{currentstroke}%
\pgfsetdash{}{0pt}%
\pgfpathmoveto{\pgfqpoint{0.800000in}{0.528000in}}%
\pgfpathlineto{\pgfqpoint{0.800000in}{4.224000in}}%
\pgfusepath{stroke}%
\end{pgfscope}%
\begin{pgfscope}%
\pgfsetrectcap%
\pgfsetmiterjoin%
\pgfsetlinewidth{0.803000pt}%
\definecolor{currentstroke}{rgb}{0.000000,0.000000,0.000000}%
\pgfsetstrokecolor{currentstroke}%
\pgfsetdash{}{0pt}%
\pgfpathmoveto{\pgfqpoint{5.760000in}{0.528000in}}%
\pgfpathlineto{\pgfqpoint{5.760000in}{4.224000in}}%
\pgfusepath{stroke}%
\end{pgfscope}%
\begin{pgfscope}%
\pgfsetrectcap%
\pgfsetmiterjoin%
\pgfsetlinewidth{0.803000pt}%
\definecolor{currentstroke}{rgb}{0.000000,0.000000,0.000000}%
\pgfsetstrokecolor{currentstroke}%
\pgfsetdash{}{0pt}%
\pgfpathmoveto{\pgfqpoint{0.800000in}{0.528000in}}%
\pgfpathlineto{\pgfqpoint{5.760000in}{0.528000in}}%
\pgfusepath{stroke}%
\end{pgfscope}%
\begin{pgfscope}%
\pgfsetrectcap%
\pgfsetmiterjoin%
\pgfsetlinewidth{0.803000pt}%
\definecolor{currentstroke}{rgb}{0.000000,0.000000,0.000000}%
\pgfsetstrokecolor{currentstroke}%
\pgfsetdash{}{0pt}%
\pgfpathmoveto{\pgfqpoint{0.800000in}{4.224000in}}%
\pgfpathlineto{\pgfqpoint{5.760000in}{4.224000in}}%
\pgfusepath{stroke}%
\end{pgfscope}%
\begin{pgfscope}%
\pgfsetbuttcap%
\pgfsetmiterjoin%
\definecolor{currentfill}{rgb}{1.000000,1.000000,1.000000}%
\pgfsetfillcolor{currentfill}%
\pgfsetfillopacity{0.800000}%
\pgfsetlinewidth{1.003750pt}%
\definecolor{currentstroke}{rgb}{0.800000,0.800000,0.800000}%
\pgfsetstrokecolor{currentstroke}%
\pgfsetstrokeopacity{0.800000}%
\pgfsetdash{}{0pt}%
\pgfpathmoveto{\pgfqpoint{4.886157in}{3.725543in}}%
\pgfpathlineto{\pgfqpoint{5.662778in}{3.725543in}}%
\pgfpathquadraticcurveto{\pgfqpoint{5.690556in}{3.725543in}}{\pgfqpoint{5.690556in}{3.753321in}}%
\pgfpathlineto{\pgfqpoint{5.690556in}{4.126778in}}%
\pgfpathquadraticcurveto{\pgfqpoint{5.690556in}{4.154556in}}{\pgfqpoint{5.662778in}{4.154556in}}%
\pgfpathlineto{\pgfqpoint{4.886157in}{4.154556in}}%
\pgfpathquadraticcurveto{\pgfqpoint{4.858379in}{4.154556in}}{\pgfqpoint{4.858379in}{4.126778in}}%
\pgfpathlineto{\pgfqpoint{4.858379in}{3.753321in}}%
\pgfpathquadraticcurveto{\pgfqpoint{4.858379in}{3.725543in}}{\pgfqpoint{4.886157in}{3.725543in}}%
\pgfpathclose%
\pgfusepath{stroke,fill}%
\end{pgfscope}%
\begin{pgfscope}%
\pgfsetrectcap%
\pgfsetroundjoin%
\pgfsetlinewidth{1.003750pt}%
\definecolor{currentstroke}{rgb}{1.000000,0.000000,0.000000}%
\pgfsetstrokecolor{currentstroke}%
\pgfsetdash{}{0pt}%
\pgfpathmoveto{\pgfqpoint{4.913935in}{4.050389in}}%
\pgfpathlineto{\pgfqpoint{5.191713in}{4.050389in}}%
\pgfusepath{stroke}%
\end{pgfscope}%
\begin{pgfscope}%
\definecolor{textcolor}{rgb}{0.000000,0.000000,0.000000}%
\pgfsetstrokecolor{textcolor}%
\pgfsetfillcolor{textcolor}%
\pgftext[x=5.302824in,y=4.001778in,left,base]{\color{textcolor}\rmfamily\fontsize{10.000000}{12.000000}\selectfont 1 Mo}%
\end{pgfscope}%
\begin{pgfscope}%
\pgfsetrectcap%
\pgfsetroundjoin%
\pgfsetlinewidth{1.003750pt}%
\definecolor{currentstroke}{rgb}{0.000000,0.000000,1.000000}%
\pgfsetstrokecolor{currentstroke}%
\pgfsetdash{}{0pt}%
\pgfpathmoveto{\pgfqpoint{4.913935in}{3.856716in}}%
\pgfpathlineto{\pgfqpoint{5.191713in}{3.856716in}}%
\pgfusepath{stroke}%
\end{pgfscope}%
\begin{pgfscope}%
\definecolor{textcolor}{rgb}{0.000000,0.000000,0.000000}%
\pgfsetstrokecolor{textcolor}%
\pgfsetfillcolor{textcolor}%
\pgftext[x=5.302824in,y=3.808105in,left,base]{\color{textcolor}\rmfamily\fontsize{10.000000}{12.000000}\selectfont 30 Yr}%
\end{pgfscope}%
\end{pgfpicture}%
\makeatother%
\endgroup%
}
	\caption{Treasury Yield Curve Rates 2015-2020}
	\label{fig:linePlotRates}
\end{figure}

% DESCRIPTION OF DATA / SOURCES

% VARIETY OF VISUALS EXPLAINING ITS STRUCTURE

% TIME SERIES FEATURES: AUTOCORRELATION, STATIONARITY, ETC. 

\section{Results}
\label{res}

% DESCRIPTION OF A WALKTHROUGH OF THE PIPELINE PROCESS
% DISCUSS FINDINGS ALONG THE WAY

% SUMMARY OF KEY FINDINGS WITH EVIDENCE

\section{Conclusion}
\label{conc}

% WHAT DID WE LEARN? 

% WHAT ARE THE IMPLICATIONS? 

% WHAT ARE DIRECTIONS FOR FUTURE RESEARCH? 

\clearpage
\bibliographystyle{plain}
\bibliography{refs}

\end{document}
